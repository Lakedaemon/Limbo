\catcode`@=11\relax
\def\LD@Catcodes@Change{%
	\catcode`¤=0\relax
	\catcode`\#=11\relax
	\catcode`\\=11\relax
}%
{%
\newlinechar=`|\relax
	\LD@Catcodes@Change
	¤long¤gdef¤LD@Lilypond@Template@Pre{%
righthand={}|
lefthand={}|
rightfoot={}|
leftfoot={}
	}%
	¤long¤gdef¤LD@Lilypond@Template@Post{%
\version "2.10.33" |
\paper {|
indent = 0\mm|
oddFooterMarkup=##f
oddHeaderMarkup=##f
bookTitleMarkup = ##f
scoreTitleMarkup = ##f
#(define dump-extents #t)|
force-assignment = #""|
printallheaders=##f|
}|
\score {|
\new DrumStaff <<|
\new DrumVoice {\stemDown\rightfoot}|
\new DrumVoice {\stemDown\lefthand}|
\new DrumVoice {\stemUp\righthand}|
\new DrumVoice {\stemUp\leftfoot}|
>>|
}|
\header {|
  tagline = ##f|
  composer = ##f|
}|
\layout {|
indent = 0\mm|
line-width=¤LD@Option@@LineWidth \mm|
ragged-right=##t|
}|
	}%
}%
\newwrite\LD@Lilypond@Output
\newcount\LD@Lilypond@Count\global\LD@Lilypond@Count=0\relax
\def\Lilypond{%
	\@getoptionalarg
	\LD@Lilypond
}%
\def\LD@Lilypond{%
	\LD@Catcodes@Change
	\LD@Lilypond@Process
}%
\def\LD@Lilypond@Process#1\par{%
	\def\LD@Option@@LineWidth{120}%
	\LD@Option@Process
	\global\advance\LD@Lilypond@Count by1\relax
	\edef\LD@Lilypond@File{Lilypond\LD@Drums@Thème\the\LD@Lilypond@Count}%
	\if Y\LD@Drums@Lilypond@Process
		\EA\immediate\EA\openout\EA\LD@Lilypond@Output\LD@Lilypond@File.ly\relax
		{\newlinechar=`|\relax
			\immediate\write\LD@Lilypond@Output{\LD@Lilypond@Template@Pre}%
			\immediate\write\LD@Lilypond@Output{#1}%
			\immediate\write\LD@Lilypond@Output{\LD@Lilypond@Template@Post}%
		}%
		\immediate\closeout\LD@Lilypond@Output
		\immediate\write18{"F:/Program Files/Lilypond/usr/bin/Lilypond-windows.exe" -b eps -dgui  -f pdf G:/TeX/Output/\LD@Lilypond@File.ly}%
	\fi
	\LD@Catcodes@Restore
	\global\advance\LD@Drums@Pattern by1\relax
	\noindent
	\llap{%
		\bf\the\LD@Drums@Pattern.\quad
	}%
	\noindent
	\setbox\LD@Box=\hbox{\XeTeXpdffile "G:/TeX/Output/\LD@Lilypond@File.pdf" }%
	\LD@Dimen=\ht\LD@Box\relax
	\advance\LD@Dimen by\dp\LD@Box
	\divide\LD@Dimen by2\relax
	\raise-\LD@Dimen\copy\LD@Box
	\qquad\LD@Drums@Process
	\medskip
}%

\def\LD@Catcodes@Restore{%
	\catcode`¤=11\relax
	\catcode`\#=6\relax
	\catcode`\\=0\relax
}%
\newcount\LD@Drums@Pattern
\def\Thème#1.{%
	\def\LD@Drums@Thème{#1}%
	\global\LD@Drums@Pattern=0\relax
	\goodbreak
	\centerline{\seventeenbf #1}%
	\bigskip
}%
\def\Session{%
	\@getoptionalarg
	\LD@Drums@Session
}%
\def\LD@Drums@Session #1/#2/#3.{%
	\def\LD@Option@@{}%
	\LD@Option@Process
	\let\LD@Drums@Tempo@Session\LD@Option@@
	\def\LD@Drums@Date{#3/#2/#1}%
}%
\def\Drums{%
	\@getoptionalarg
	\LD@Drums
}%
\def\LD@Drums#1, #2 : #3.{%
	\let\LD@Option@@\LD@Drums@Tempo@Session
	\LD@Option@Process
	\let\LD@Drums@Tempo\LD@Option@@
	\def\LD@Drums@String{}%
	\LD@Loop@FFor\LD@Loop@Token=#3\WithSeparator ,\Do{%
			\LD@Strings@Options@Separate\LD@Loop@Token
			\def\LD@Option@@{}%
			\LD@Option@Process
			\ifx\LD@Option@@\LD@Empty
				\unless\ifx\LD@Drums@Tempo\LD@Empty
					\ifx\@optionalarg\LD@Empty
						\edef\@optionalarg{\@optionalarg\LD@Drums@Tempo}%
					\else
						\edef\@optionalarg{\@optionalarg,\LD@Drums@Tempo}%
					\fi
				\fi
			\fi
			\unless\ifx\LD@Drums@String\LD@Empty
				\edef\LD@Drums@String{\LD@Drums@String ,}%
			\fi
			\ifx\@optionalarg\LD@Empty
				\edef\LD@Drums@String{\LD@Drums@String\LD@Loop@Token}%
			\else
				\edef\LD@Drums@String{\LD@Drums@String[\@optionalarg]\LD@Loop@Token}%
			\fi
	 }%
	\ifcsname LD@Drums@@#1-#2\endcsname
		\EA\let\EA\LD@Trash\CS LD@Drums@@#1-#2\EC
		\EA\edef\CS LD@Drums@@#1-#2\EC{\LD@Trash,\LD@Drums@String}%
	\else
		\EA\let\CS LD@Drums@@#1-#2\EC\LD@Drums@String
	\fi
}%

\def\LD@Strings@Options@Separate#1{%
	\unless\ifx#1\LD@Empty
		\def\LD@Strings@Value@Extract##1\LD@End{%
			\def#1{##1}%
		}%
		\def\@optionalarg{}%
		\EA\@getoptionalarg\EA\LD@Strings@Value@Extract #1\LD@End
	\else
		\def#1{}%
		\def\@optionalarg{}%
	\fi
}%
\newcount\LD@Drums@Count
\def\LD@Loop@FFor#1=#2\WithSeparator#3\Do#4{%
	\EA\def\EA\LD@Loop@FFor@Tail\EA{#2}%
	\unless\ifx\LD@Loop@FFor@Tail\LD@Empty
		\EA\def\EA\LD@Loop@FFor@Tail\EA{\LD@Loop@FFor@Tail #3}%
		\def\LD@Loop@FGobble##1#3##2\LD@Loop@FEmpty{%
			\def#1{##1}%
			\def\LD@Loop@FFor@Tail{##2}%
		}%
		\def\LD@Loop@FFor@Iterate{%
			\ifx\LD@Loop@FFor@Tail\LD@Empty
				\let\LD@Loop@FFor@Next\relax
			\else	
				\EA\LD@Loop@FGobble\LD@Loop@FFor@Tail\LD@Loop@FEmpty
				#4\relax
				\let\LD@Loop@FFor@Next\LD@Loop@FFor@Iterate
			\fi
			\LD@Loop@FFor@Next
		}%
		\LD@Loop@FFor@Iterate
	\fi
}%
\def\LD@Drums@Process{%
	\LD@Loop@For\LD@Loop@Token=01,02,03,11,12,13,21,22,23,31,32,33\WithSeparator ,\Do{%
		\EA\def\CS LD@Drums@@Variant\LD@Loop@Token\EC{0}%
		\EA\def\CS LD@Drums@@Variant\LD@Loop@Token @Tempo\EC{0}%
	}%
	\ifcsname LD@Drums@@\LD@Drums@Thème-\the\LD@Drums@Pattern\endcsname
		\EA\let\EA\LD@Temp\CS LD@Drums@@\LD@Drums@Thème-\the\LD@Drums@Pattern\EC
		\LD@Loop@FFor\LD@Loop@Token=\LD@Temp\WithSeparator ,\Do{%
			\LD@Strings@Options@Separate\LD@Loop@Token
			\def\LD@Option@@{}%
			\LD@Option@Process
			\unless\ifx\LD@Option@@\LD@Empty
				\EA\ifnum\CS LD@Drums@@Variant\LD@Loop@Token @Tempo\EC<\LD@Option@@\relax
					\EA\edef\CS LD@Drums@@Variant\LD@Loop@Token @Tempo\EC{\LD@Option@@}%
				\fi
			\fi
			\LD@Drums@Count=\CS LD@Drums@@Variant\LD@Loop@Token\EC
			\advance\LD@Drums@Count by1\relax
			\EA\edef\CS LD@Drums@@Variant\LD@Loop@Token\EC{\the\LD@Drums@Count}%
		}%
		\hfil
		{\parindent0mm%
			\tikzstyle{Case}=[rectangle,text width=1cm,text height=11pt,text depth=3pt,inner sep=0pt]
			\tikzpicture [baseline=(current bounding box.west)]
				\foreach \x in{1,2,3}
					\foreach \y in{0,1,2,3}
						\node[Case,draw,text centered] at (\x*1.2cm,-\y*17pt) {\noindent\EA\ifnum\CS LD@Drums@@Variant\y\x\EC>0\relax$\CS LD@Drums@@Variant\y\x @Tempo\EC^{\CS LD@Drums@@Variant\y\x\EC}$\fi};
				\foreach \x in{1,2,3}
					\node[rectangle,text width=1.2cm,text height=15pt,text depth=2pt,inner sep=0pt,text centered] at (\x*1.2cm,17pt) (a) {\LD@Drums@Cymbals\x};
				\foreach \y in{0,1,2,3}
					\node[Case] at (4.8cm,-\y*17pt) {\LD@Drums@HiHat\y};
			\endtikzpicture
		}%
	\fi
}%
\endinput