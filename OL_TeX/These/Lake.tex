%%%%%%%%%%%%% 3eme article 10/11/1998-10/03/2000

\magnification 1100
\input eplaingt
\input epsf
\input Macrols
\SecLabelEqtrue             % le numero de section prefixe le label des equations
\NumReftrue                   % les references bibliographiques sont numerotees automatiquement

\def\olop_#1{\ifinner\prec\!\!\prec_{#1}\else\ \mathop{\prec\!\!\prec}_{#1}\ \fi}
\def\olopd_#1{\ifinner\prec\!\!\prec^\diamond_{#1}\else\ \mathop{\prec\!\!\prec}^\diamond_{#1}\ \fi}
\def\olops_#1{\ifinner\prec\!\!\prec^\star_{#1}\else\ \mathop{\prec\!\!\prec}^\star_{#1}\ \fi}
\def\sco_#1{{\scal O}_{#1}}
\def\scod_#1{{\scal O}^{\diamond}_{#1}}
\def\scos_#1{{\scal O}^\star_{#1}}
\def\diamond{{\star\star}}

\def\sc#1{{\scal #1}}
\def\bno{\big|\!\big|}
\def\Bno{\Big|\!\Big|}
\def\bgno{\bigg|\!\bigg|}
\def\Bgno{\Bigg|\!\Bigg|}
\let\ub=\underbrace


\hautspages{O. Binda}{Suite auto-d\'ecrite de Golomb et \'equations fonctionnelles associ\'ees}


\titrecentre{\seventeenbf Suite auto-d\'ecrite de Golomb et~\'equations fonctionnelles associ\'ees} 
\bigskip
\bigskip

\centerline{Olivier Binda}
\bigskip
\bigskip

\noindent
{\qquad\quad \bf Sommaire}
\smallskip
{\eightpts\leftskip.6cm\rightskip.8cm
\readtocfile}


\Sect Intro, Introduction. 


\Secti Golomb, Suite auto-d\'ecrite de Golomb $\{u_n\}_{n=1}^\infty$. 

Dans cet article, nous nous proposons de pr\'eciser le comportement asymptotique de la suite de Golomb, d\'efinie formellement
comme l'unique suite croissante d'entiers $\{u_n\}_{n=1}^\infty$ v\'erifiant
$u_1=1$ et 
$$
u_n=\b|\{k\ge1:u_k=n\}\b|\qquad(n\ge1), 
\eqdef{defautode}
$$ 
et de faire la synth\`ese des r\'esultats obtenus dans la litt\'erature \`a son propos.  
\bigskip

Nous commen\c{c}ons par \'etablir qu'il existe effectivement une unique suite croissante d'entiers $\{u_n\}_{n=1}^\infty$ v\'erifiant
$u_1=1$ et \eqref{defautode}, que l'on construit par r\'ecurrence  en posant  
$$
u_{n+1}:=\Q\{\eqalign{
&u_n+1\qquad\hbox{ si } \sum_{1\le k\le u_n}u_k=n
\cr
&u_n\qquad\qquad\hbox{sinon}
}
\W.\qquad\qquad(n\ge1). \eqdef{defF}
$$
\'Etant donn\'ee une telle suite, nous posons $U_n:=\{k\ge1:u_k=n\}$ pour chaque $n\ge1$ et nous remarquons d'une part que 
$$
|U_n|=u_n\ge 1\qquad(n\ge1)
$$
et d'autre part que les ensembles $U_n\ \,(n\ge1)$ forment une partition de l'ensemble $\ob N^*$, 
pour~laquelle les \'el\'ements de $U_1$ sont inf\'erieurs aux \'el\'ements de $U_2$, qui sont  inf\'erieurs aux \'el\'ements de $U_3$ et ainsi de suite. 
Comme l'\'egalit\'e $|U_1|=u_1=1$ induit que $U_1=\{1\}$, l'entier $2$ appartient alors n\'ecessairement \`a l'ensemble $U_2$, ce qui impose que $u_2=2$. 
La~relation $|U_2|=u_2=2$ implique alors que $U_2=\{2,3\}$, ce qui d\'etermine $u_2=u_3=2$. 
De m\^eme, il r\'esulte de l'\'egalit\'e $|U_3|=u_3=2$ que $U_3=\{4,5\}$ et par suite que $u_4=u_5=3$. 
Nous  proc\'edons ainsi de suite par r\'ecurrence, selon le sch\'ema 
$$
\{u_k\}_{k=1}^\infty=\{
\ub{1,}_{1,}
\ub{2,2,}_{2,}
\ub{3,3,}_{2,}
\ub{4,4,4,}_{3,}
\ub{5,5,5,}_{3,}
\ub{6,6,6,6,}_{4,}
7,
\cdots,n-1,\ub{n,n,\cdots,n}_{|U_n|=u_n,}
,n+1,\cdots
\}.
$$
Supposons que $u_1,\cdots,u_N$ sont construits de sorte que $u_N=m\le N$ et $|U_n|=u_n\ \,(n<m)$. 
Nous d\'efinissons alors $u_{N+1}$ par 
$$
u_{N+1}:=\Q\{\eqalign{
&m+1\qquad\hbox{ si } \b|\{j\le N:u_j=m\}\b|=u_m,
\cr
&m\qquad\qquad\hbox{sinon}.
}
\W.
$$
Il est imm\'ediat que cette construction garantit l'existence et l'unicit\'e de la suite $\{u_n\}_{n=1}^\infty$. Notant  $\{v_m\}_{m=0}^\infty$ la suite 
d\'efinie~par 
$$
v_m:=\sum_{1\le k\le m}u_k
\qquad(m\ge0),
\eqdef{defvm}
$$
nous avons \'evidemment $v_m=|U_1\cup U_2\cdots\cup U_m|\ \,(m\ge 1)$ et donc que 
$$
u_n=m\quad\Longleftrightarrow \quad n\in U_m\quad\Longleftrightarrow\quad v_{m-1}<n\le v_m
\qquad\quad(m\ge1,n\ge1). 
\eqdef{fond}
$$
En particulier, la suite $\{u_n\}_{n=1}^\infty$ satisfait les relations $u_1=1$, \eqref{defautode} et \eqref{defF}. 
\bigskip

Dans le paragraphe pr\'ec\'edent, nous avons \'etabli que la suite de Golomb est bien d\'efinie. 
Comme cette suite est croissante et auto-d\'ecrite, i.e.  le nombre d'occurrence de l'entier $n$ dans la suite $\{u_k\}_{k=1}^\infty$ est $u_n$,  
elle satisfait la relation
$$
\lim_{n\to\infty}u_n=+\infty \eqdef{liminf}
$$ 
et il est alors naturel de chercher \`a pr\'eciser son comportement \`a l'infini. 
C'est pourquoi, Golomb \CitRef{Golomb} demande, en 1966, une formule asymptotique pour la suite de Golomb. 
\bigskip

La premi\`ere famille d'\'equivalents asymptotiques \`a consid\'erer pour la suite de Golomb \'etant celle des mon\^omes, nous prouvons 
qu'une condition n\'ecessaire pour que l'estimation  
$$
u_n\sim cn^{\varphi-1}
\qquad(n\to\infty)
\eqdef{Fecpn}
$$
soit satisfaite est que 
$$
\varphi={1+\sqrt5\F2}\qquad\hbox{et}\qquad c=\varphi^{2-\varphi}. \eqdef{La4}
$$
En effet, \'etant donn\'e l'ensemble $\sc V:=\{v_m:m\ge1\}$, nous d\'eduisons de \eqref{fond} pour $m=u_n$ 
que la suite de Golomb satisfait l'\'equation fonctionnelle 
$$
n=\sum_{1\le k\le u_n}u_k\qquad(n\in\sc V). 
\eqdef{La6}
$$
En supposant que la suite de Golomb satisfait la relation \eqref{Fecpn}, il r\'esulte alors de \eqref{liminf} que $c>0$ et $\varphi>1$.
De plus, comme l'identit\'e \eqref{La6} implique que 
$$
n\sim\sum_{1\le k\le cn^{\varphi-1}}ck^{\varphi-1}\sim
{c^{\varphi+1}\F\varphi}n^{\varphi(\varphi-1)}
\qquad(n\in\sc V,n\to\infty), 
$$
nous remarquons que ces nombres $c>0$ et $\varphi>1$ v\'erifient n\'ecessairement les relations 
$$
\eqalignno{
\varphi(\varphi-1)&=1,&
\eqdef{Phi1}
\cr
c^{\varphi+1}=&\varphi&
\eqdef{Phi2}
}
$$ 
et donc qu'ils sont compl\`etement caract\'eris\'es par \eqref{La4}. 
\bigskip


En~1967, Fine \CitRef{Fine} prouve que la suite de Golomb satisfait effectivement l'estimation~\eqref{Fecpn} pour les nombres \eqref{La4} 
et il est alors naturel de chercher \`a pr\'eciser l'approximation~\eqref{Fecpn}. C'est pourquoi,  
Knuth \CitRef{Knuth} demande en 1988 l'ordre de grandeur de la suite $\{r_n\}_{n=1}^\infty$ des~restes assoc\'ies \`a la suite de Golomb  par
$$
u_n=cn^{\varphi-1}+r_n\qquad(n\ge1).
\eqdef{La7}
$$
Cette \'etude du comportement \`a l'infini de la suite $\{r_n\}_{n=1}^\infty$ est r\'ealis\'ee au paragraphe \CitSec{Restes}, 
o\`u nous \'enon\c{c}ons un d\'eveloppement asymptotique pour les suites $\{u_n\}_{n=1}^\infty$ et $\{r_n\}_{n=1}^\infty$. 
\bigskip
 

Les paragraphes pr\'ec\'edents ont r\'ev\'el\'es que les nombres \eqref{La4} sont li\'es 
au comportement asymptotique de la suite de Golomb.  La fonction $t\mapsto\log(\log t)$ et le nombre $\log\varphi$ 
\'etant similairement li\'es \`a ce comportement \`a l'infini, comme nous le verrons au paragraphe~\CitSec{Restes}, 
dans~toute~la~suite~de~l'article, nous notons $[t]$ la partie enti\`ere du nombre r\'eel $t$, nous~posons 
$$
\log_2(t):=\log(\log t)\qquad(t>1)
$$
et nous d\'esignons par $\varphi$, $c$ et $\lambda$ les nombres d\'efinis par \eqref{La4} et 
$$
\lambda:=\log\varphi,\eqdef{deflambda}
$$
qui valent approximativement $\varphi\approx 1,618$,  $c\approx 1,202$ et $\lambda\approx0,481$. 
Au passage, remarquons que~$\varphi$ est le~nombre d'or et que   
$$
\eqalignno{
\varphi^2&=\varphi+1,&\eqdef{Phi3}\cr
(\varphi-1)^2&=2-\varphi.&\eqdef{Phi4}\cr
}
$$


\Secti Restes, Estimation asymptotique de la suite des restes $\{r_n\}_{n=1}^\infty$.

Dans cette sous-section, nous cherchons \`a pr\'eciser le comportement de la suite  $\{r_n\}_{n=1}^\infty$ des restes associ\'es 
\`a la suite de Golomb via l'identit\'e \eqref{La7}. D'apr\`es l'estimation~\eqref{Fecpn}, 
nous~pouvons d'ores et d\'ej\`a affirmer que 
$$
r_n=o\b(n^{\varphi-1}\b)\qquad(n\to\infty)
$$
et nous pr\'esentons dans le paragraphe suivant une am\'elioration de cette estimation 
ainsi que deux~conjectures dues \`a Vardi \CitRef{Vardi}, que nous justifions sommairement. 
\bigskip
 

En 1992, Vardi \CitRef{Vardi} prouve que
$$
r_n\ll {n^{\varphi-1}\F\log n}\qquad(n\ge2)  
\eqdef{Vardi}
$$
et il conjecture non seulement que cette estimation est optimale, c'est-\`a-dire que
$$
r_n=\Omega_\pm\Q({n^{\varphi-1}\F\log n}\W)
\qquad(n\to\infty), 
\leqno{(c)}
$$
mais \'egalement que l'estimation 
$$
r_n=\nu(\log_2 n){n^{\varphi-1}\F\log n}+O\Q({n^{\varphi-1}\F(\log n)^2}\W)
\qquad(n\ge2)
\leqno{(C)}
$$
est satisfaite par une fonction continue $\nu:\ob R\to\ob R$ v\'erifiant le syst\`eme 
$$
\eqalignno{
\nu(x)+\nu(x+\lambda)=0\qquad(x\in\ob R),&
&\eqdef{n+tn}
\cr
\nu(x)\neq0\qquad\quad(0<x<\lambda)^{\strut}.&
&\eqdef{condeb}
}
$$
Les raisons heuristiques pour lesquelles Vardi propose ces conjectures sont les suivantes : comme $(C)$ et \eqref{condeb}
induisent $(c)$, il remarque d'une part que sa seconde conjecture implique sa premi\`ere conjecture 
et d'autre part que sa seconde conjecture a un~sens si la~suite des~restes $\{r_n\}_{n=1}^\infty$ satisfait 
$$
r_n=-c^{-\varphi} n^{\varphi-2}\sum_{\ss 1\le k\le
cn^{\varphi-1}}r_k+O\Q({n^{\varphi-1}\F(\log n)^2}\W)
\qquad(n\ge2). 
\eqdef{eqrn}
$$
En effet, Vardi sugg\`ere que l'\'equation approch\'ee \eqref{eqrn} est satisfaite par la suite  
$$
n\mapsto \nu(\log_2 n){n^{\varphi-1}\F\log n}
$$ 
pour~chaque application $\nu\in\sc C^1(\ob R)$ v\'erifiant \eqref{n+tn}. Vardi reste sur le plan heuristique et ne~prouve pas ces deux derni\`eres affirmations. 
Cependant, au~paragraphe~\CitSec{Demons}, nous~d\'eduisons des relations \eqref{La6}, \eqref{La7} et \eqref{Vardi}  que la suite de Golomb satisfait 
$$
n=\sum_{1\le k\le u_n}u_k+O\b(n^{2-\varphi}\b)
\qquad(n\ge1)
\eqdef{La10}
$$
puis que la suite $\{r_n\}_{n=1}^\infty$ est effectivement une solution de  l'\'equation approch\'ee \eqref{eqrn}. 
\bigskip


Les deux conjectures de Vardi permettent d'appr\'ehender le comportement asymptotique de la suite $\{r_n\}_{n=1}^\infty$ et ont motiv\'e quelques publications. 
Dans les paragraphes suivants, nous rappelons l'historique des r\'esultats publi\'es et 
nous discutons de leurs implications vis \`a vis des deux conjectures de Vardi.  
\bigskip
 

En 1995, P\'etermann \CitRef{Petermann1} \'etablit la conjecture $(c)$ sous une
forme l\'eg\`erement plus faible : pour~$\epsilon>0$, 
il prouve que
$$
r_n=\Omega_\pm\Q(n^{\varphi-1-\epsilon}\W)
\qquad(n\to\infty).
$$
Deux ans plus tard, R\'emy \CitRef{Remy1} prouve la conjecture
$(c)$ et il  trace le graphe sur l'intervalle~$[6\lambda,11\lambda]$  de l'application $\aleph$  
implicitement d\'efinie par  
$$
r_n=\aleph(\log_2n){n^{\varphi-1}\F \log n}\qquad(n\ge2).  
$$


Cette confirmation de la premi\`ere conjecture de Vardi donne plus de poids \`a sa seconde~conjecture, qui est plus forte. 
De plus, le trac\'e du graphe de l'application $\aleph$ vient encore la renforcer. 
En~effet, nous observons que l'estimation $(C)$ est \'equivalente \`a 
$$
\nu(u)=\aleph(u)+O\b(\e^{-u}\b)\qquad(u=\log_2n,n\ge2)\eqdef{oulala}
$$
et nous constatons sur la Figure 1 que la fonction $\aleph$ para\^\i t effectivement converger 
vers une fonction r\'eguli\`ere $\nu$ v\'erifiant \eqref{n+tn}. Par contre, la relation \eqref{condeb} semble injustifi\'ee 
dans la mesure o\`u la~fonction $\nu$ s'annule visiblement sur les intervalles $]k\lambda,(k+1)\lambda[\ \,(k\in\ob Z)$. 

\medskip
\epsfysize=8.3cm
\hfill\epsfbox{K6L100.eps}\hfill\null\par
%\hfill\epsfbox{K7L15000.eps}\hfill\null\par
%\hfill\epsfbox{K7L30000.eps}\hfill\null\par
\centerline{Figure 1. --- Graphe de la fonction $\aleph$}
\bigskip

Pour tracer le graphe de $\aleph$ sur l'intervalle $[6\lambda,13\lambda]$, 
nous avons utilis\'e le programme calculant le nombre~$u_n$ 
pour de grandes valeurs de l'entier $n$, d\'etaill\'e au paragraphe~\CitSec{Source}, dont~l'algorithme repose sur une m\'ethode d'acc\'el\'eration de calcul 
sugg\'er\'ee par Vardi \CitRef{Vardi} et sp\'ecifique \`a la suite $\{u_n\}_{n=1}^\infty$,  que nous pr\'esentons au paragraphe \CitSec{Algo}. 
Pour~information, l'entier $n$ le plus grand pour lequel nous avons du calculer $u_n$ afin d'en d\'eduire le~graphe pr\'ec\'edent 
vaut approximativement $$
n\approx 4\times10^{276}.
$$ 
Maintenant que nous avons de bonnes raisons d'adh\'erer \`a la seconde conjecture de Vardi, 
sous~r\'eserve de modifier \eqref{condeb}, 
examinons les progr\`es r\'ealis\'es en vue de sa confirmation. 
En 1998, P\'etermann et R\'emy \CitRef{PetermannRemy}  prouvent la conjecture $(C)$ 
sous une forme l\'eg\`erement plus faible : 
ils \'etablissent l'existence d'une unique fonction $\nu\in\sc C(\ob R)$ v\'erifiant \eqref{n+tn} et
$$
r_n=\nu\Q(\log_2 n\W){n^{\varphi-1}\F\log n}+O\Q({n^{\varphi-1}\F(\log n)^2}\log_2n\W)
\qquad(n\ge 3).
\eqdef{Pet}
$$
De plus, ils prouvent que cette fonction $\nu$ satisfait $\nu(0)=0$. 
\bigskip

Le sch\'ema sur lequel repose leur preuve m\'erite que l'on s'attarde pour le pr\'esenter. 
En~effet, il justifie l'\'etude des \'equations diff\'erentielles associ\'ees \`a la~suite de Golomb, men\'ee~plus~loin, 
d'autant plus que nous l'utilisons pour \'etablir l'existence d'un d\'e\-ve\-lop\-pe\-ment asymptotique pour la suite $\{r_n\}_{n=1}^\infty$ 
et {\it a fortiori} la seconde conjecture~de~Vardi. 
Pour~simplifier, P\'etermann et R\'emy approchent la suite $\{r_n\}_{n=1}^\infty$ par une fonction~$r$ 
v\'erifiant une certaine \'equation approch\'ee, dont ils estiment les solutions \`a l'infini. 
Dans~les~faits, P\'etermann et R\'emy d\'eduisent de l'estimation \eqref{eqrn} que la fonction $t\mapsto r_{[t]}$ satisfait l'\'equation int\'egrale approch\'ee
$$
f(t)=-c^{-\varphi}t^{\varphi-2}\int_3^{ct^{\varphi-1}}f(x)\d
x+O\Q({t^{\varphi-1}\F(\log t)^2}\W)
\qquad(t\ge 3).
\eqdef{eqR}
$$
Puis, pour chaque solution $f\in L_{\hbox{\sevenrm loc}}^1\b([3,\infty[\b)$ 
de l'\'equation int\'egrale approch\'ee \eqref{eqR}, 
ils~\'etablissent  
l'existence d'une unique
fonction $h\in\sc C(\ob R)$ v\'erifiant \eqref{n+tn} et
$$
f(t)=h(\log_2 t){t^{\varphi-1}\F\log t}+O\Q({t^{\varphi-1}\F(\log t)^2}\log_2 t\W)
\qquad(t\ge 3). 
$$
Enfin, en appliquant cette propri\'et\'e \`a la fonction $t\mapsto r_{[t]}$, ils en d\'eduisent \eqref{Pet}. 
\bigskip


P\'etermann et R\'emy d\'emontrent qu'une fonction $\nu\in\sc C(\ob R)$  ne peut v\'erifier \`a la fois les relations \eqref{n+tn},~\eqref{condeb}~et~$(C)$, 
ils en d\'eduisent que la seconde conjecture de Vardi est fausse sous sa forme actuelle et sugg\`erent, au~vu~du graphe de la fonction $\aleph$, 
de la modifier en rempla\c{c}ant~la relation \eqref{condeb} par l'assertion 
$$
\hbox{Il existe un nombre }t_0\in]0,\lambda[\hbox{ tel que }\quad\nu(t)\neq0\quad(t_0<t<t_0+\lambda).
\eqdef{condec}
$$ 
En effet, \'etant donn\'ee une fonction $\nu\in\sc C(\ob R)$ v\'erifiant \eqref{n+tn}, \eqref{condeb} et $(C)$, 
ils remarquent que l'estimation \eqref{Pet} est satisfaite et en d\'eduisent que $\nu(0)\neq0$. 
Comme la fonction~$\nu$ est continue et admet l'anti-p\'eriode $\lambda$, ils observent que $\nu(k\lambda)\neq0\ \,(k\in\ob Z)$ 
puis qu'il existe un~z\'ero de~$\nu$ dans l'intervalle $]k\lambda,(k+1)\lambda[$ pour chaque $k\in\ob Z$, ce qui contredit~\eqref{condeb}. 
P\'etermann et R\'emy concluent alors qu'une fonction $\nu\in\sc C(\ob R)$  ne peut satisfaire \`a la fois les relations \eqref{n+tn},~\eqref{condeb}~et~$(C)$, 
ce qui infirme la seconde conjecture de Vardi. 
\bigskip

Le fait que la seconde conjecture de Vardi soit fausse n'emp\^eche pas qu'une version plus faible de cette conjecture soit v\'erifi\'ee. 
Ainsi, le  th\'eor\`eme suivant, que~nous prouvons au~paragraphe~\CitSec{Demons}, implique non seulement la seconde conjecture de Vardi priv\'ee de~\eqref{condeb} 
mais aussi l'existence 
d'un d\'eveloppement asymptotique pour $\{u_n\}_{n=1}^\infty$ et $\{r_n\}_{n=1}^\infty$. 

\theo T1. Il existe une unique suite $\{\nu_k\}_{k\ge1}\in\sc C^\infty(\ob
R)^{\ob N^*}$ de fonctions
$2\lambda$-p\'eriodiques telle que
$$
r_n=\sum_{1\le k\le K}\nu_k(\log_2 n)
{n^{\varphi-1}\F(\log n)^k}
+O_K\Q({n^{\varphi-1}\F(\log n)^{K+1}}\W)
\qquad(K\ge1,n\ge2).
\eqdef{DaC2}
$$
De plus, l'application $\nu_1$ satisfait \eqref{n+tn} et $\nu_1(0)\neq0$.
\par
\bigskip


En effet, la seconde conjecture de Vardi priv\'ee de \eqref{condeb} est une cons\'equence imm\'ediate du th\'eor\`eme \eqrefn{T1} pour $K=2$. Par ailleurs, 
\'etant~donn\'ee l'unique suite $\{\nu_k\}_{k\ge1}\in\sc C^\infty(\ob R)^{\ob N^*}$ de fonctions $2\lambda$-p\'eriodiques v\'erifiant \eqref{DaC2}, il r\'esulte de \eqref{La7}
que la suite de Golomb admet le d\'eveloppement asymptotique
$$
u_n=cn^{\varphi-1}+\sum_{1\le k\le K}\nu_k(\log_2 n)
{n^{\varphi-1}\F(\log n)^k}
+O_K\Q({n^{\varphi-1}\F(\log n)^{K+1}}\W)
\quad\ (K\ge1,n\ge2).\!\!\!\!\!\!
\eqdef{DGol}
$$

Maintenant que sont \'etablis la seconde conjecture de Vardi priv\'ee de \eqref{condeb} et 
l'existence des d\'eveloppements asymptotiques \eqref{DaC2} et \eqref{DGol} pour les suites $\{r_n\}_{n=1}^\infty$~et~$\{u_n\}_{n=1}^\infty$, 
il serait souhaitable de disposer de formules effectives pour les coefficients de ces deux d\'eveloppements asymptotiques, c'est-\`a-dire pour les~fonctions~$\nu_k\ \,(k\ge1)$. 
Dans~les~paragraphes suivant, nous expliquons sans d\'emontrer  
comment obtenir de deux mani\`eres diff\'erentes un d\'eveloppement asymptotique de la suite $\{v_m\}_{m=1}^\infty$ pour en d\'eduire des relations polyn\^omiales d\'eterminant compl\`etement $\nu_k$ 
en fonction des applications $\nu_1,\cdots,\nu_{k-1}$ et de leurs d\'eriv\'ees pour chaque entier $k\ge2$ et permettant {\it a fortiori} 
d'exprimer les applications $\{\nu_k\}_{k\ge2}$ comme des polyn\^omes de la fonction $\nu_1$ et de ses d\'eriv\'ees. 
\bigskip


Comme $\{v_m\}_{m=1}^\infty$ est la suite sommatoire de la suite $\{u_n\}_{n=1}^\infty$ d'apr\`es la d\'efinition~\eqref{defvm}, on peut d\'eduire de \eqref{DGol} que la suite $\{v_m\}_{m=1}^\infty$ 
admet le d\'eveloppement asymptotique 
$$
v_m={c\F\varphi}m^\varphi+
\sum_{1\le k\le K}\mu_k(\log_2 m){m^\varphi\F(\log m)^k}+O_K\Q({m^\varphi\F(\log m)^{K+1}}\W)
\quad(K\ge1,m\ge2).\!\!\!\!\!\!
\eqdef{DFSGol}
$$
pour la suite $\{\mu_k\}_{k=1}^\infty\in\sc C^\infty(\ob R)$ de fonctions $2\lambda$-p\'eriodiques d\'efinie par 
$$
\mu_k(u)={1\F\varphi}\sum_{m+n<k}c_{k,m,n}\,\nu_{m+1}^{(n)}(u)\qquad(k\ge1,u\in\ob R)
\eqdef{mu1}
$$
et associ\'ee \`a la famille $\{c_{k,m,n}\}_{m+n<k}$ de coefficients uniquement d\'etermin\'ee par l'identit\'e
$$
\sum_{m+n<k}c_{k,m,n}{s^{k-m-1}u^mv^n\F(k-m-1)!}={(1-s/\varphi)^v\F1-s/\varphi-u}\qquad\b(|s|+|u|<1,v\in\ob C\b).\eqdef{cmn}
$$

De m\^eme, comme la relation \eqref{fond} implique que l'application $m\mapsto v_m$ est l'inverse \`a droite pour la composition 
de la fonction $n\mapsto u_n$, c'est-\`a-dire que 
$$
m=u_{v_m}\qquad(m\ge1), 
$$  
on peut montrer que  la suite $\{v_m\}_{m=1}^\infty$ admet le d\'eveloppement asymptotique 
$$
v_m={c\F\varphi}m^\varphi+
\sum_{1\le k\le K}\mu_k^*(\log_2 m){m^\varphi\F(\log m)^k}+O_K\Q({m^\varphi\F(\log m)^{K+1}}\W)
\quad(K\ge1,m\ge2)\!\!\!\!\!\!
\eqdef{DFSGol2}
$$
pour la famille d'applications $\{\mu_k^*\}_{k\ge1}$ d\'efinie par  
$$
\mu_k^*(u)={1\F 2\pi i}\oint\limits_{|s|=1}{1\F k!}{\partial^k\F\partial x^k}\Q(
{\partial_sF_{k,u}\F F_{k,u}}\W)(0,s)\e^{-\varphi s}\d s\qquad(k\ge1,u\in\ob R) \eqdef{mu2}
$$
et associ\'ee \`a la famille $\{F_{k,u}\}_{k\ge1}$ de fonctions holomorphes uniquement d\'etermin\'ee par 
$$
F_{K,u}(x,s):=\e^s-c-\!\!\!\!\!\!\sum_{1\le k\le K-n}\!\!\!\!\!\!{\nu_k^{(n)}(u+\lambda)\log^n(1-xs)\F n!\varphi^k(1-xs)^k}x^k
\quad\b(K\ge1,u\in\ob R, |xs|<1\b).\!\!\!\!\!\!\eqdef{komoku}
$$


Les deux d\'eveloppement asymptotiques \eqref{DFSGol} et \eqref{DFSGol2} \'etant n\'ecessairement identique, pour chaque entier $k\ge2$, 
il est possible d'exprimer la fonction $\nu_k$ comme une combinaison lin\'eaire de produits dont les facteurs sont pris parmi les fonctions 
$$
u\mapsto\nu_{m+1}^{(n)}(u)\qquad\hbox{et}\qquad u\mapsto\nu_{m+1}^{(n)}(u+\lambda)\qquad(m+n<k,m+1<k).\eqdef{oomph}
$$
En effet, pour chaque entier $k\ge2$,  on peut montrer que l'application d\'efinie par 
$$
G_k(u)=\mu_k^*(u)+{\nu_k(u+\lambda)\F\varphi^k}+{\nu_k(u)\F\varphi}-\mu_k(u)\qquad(u\in\ob R) \eqdef{IcedEarth}
$$
est une combinaison lin\'eaire de produits dont les facteurs sont pris parmi les fonctions~\eqref{oomph}.  Il r\'esulte alors de l'\'egalit\'e $\mu_k^*=\mu_k\ \,(k\ge1)$ que 
$$
G_k(u)={\nu_k(u+\lambda)\F\varphi^k}+{\nu_k(u)\F\varphi}\qquad(u\in\ob R)\eqdef{metal}
$$
et de la $2\lambda$-p\'eriodicit\'e de l'application $\nu_k$ que 
$$
\nu_k(u)={\varphi^kG_k(u)-\varphi G_k(u+\lambda)\F\varphi^{k-1}-\varphi^{1-k}}
\qquad(u\in\ob R). 
\eqdef{rec}
$$
En particulier, la fonction $\nu_k$ est pour $k\ge2$ une combinaisons lin\'eaire de produits dont les facteurs sont pris parmi les fonctions \eqref{oomph}. 
\bigskip

Comme $\nu_1$ satisfait l'identit\'e \eqref{n+tn}, pour chaque entier $k\ge2$, 
il est \'egalement possible d'exprimer la fonction $\nu_k$ comme une combinaison lin\'eaire de produits dont les facteurs sont pris parmi les fonctions $\nu_1, \cdots, \nu_1^{(k-1)}$.  
A l'aide du th\'eor\`eme suivant, que~nous~prouvons au paragraphe \CitSec{Demons}, nous~exprimons~ainsi $\nu_2$ en fonction de $\nu_1$ et $\nu_1'$. 
\bigskip


\theo T2. Pour chaque entier $k\ge2$, l'application $G_k$ d\'efinie par \eqref{IcedEarth} satisfait 
$$
\nu_k(u)={\varphi^kG_k(u)-\varphi G_k(u+\lambda)\F\varphi^{k-1}-\varphi^{1-k}}
\qquad(u\in\ob R).
\eqdef{rec}
$$
De plus, les applications $\nu_k$ et $G_k$ sont des combinaisons lin\'eaires de produits dont les facteurs sont pris parmi les fonctions \eqref{oomph} 
ou parmi les applications $\nu_1, \cdots, \nu_1^{(k-1)}$. 
\par
\bigskip


En~calculant, pour $n=2$, le membre de droite de l'identit\'e \eqref{rec}, nous obtenons que 
$$
\eqalign{
\nu_2(u)=&{1+\log\varphi\F\varphi}\nu_1'(u+\lambda)-{\varphi^2+\log\varphi\F\varphi^2}\nu_1'(u)
+{1-\log\varphi\F\varphi}\nu_1(u+\lambda)
\cr
&+{\log\varphi-\varphi^2\F\varphi^2}\nu_1(u)
+{\varphi^2\F2c}\nu_1(u+\lambda)^2-{\varphi\F2c}
\nu_1(u)^2\qquad(u\in\ob R).
\cr
}
$$
Comme $\nu_1$ satisfait l'identit\'e  \eqref{n+tn}, nous d\'eduisons alors de l'\'egalit\'e $\varphi^2=\varphi+1$ que 
$$
\nu_2(u)=-(\varphi+\log\varphi)\nu_1'(u)+(\log\varphi-\varphi)\nu_1(u)+{1\F2c}\nu_1(u)^2\qquad(u\in\ob R).
$$
\bigskip


En conclusion, les d\'eveloppements asymptotiques \eqref{DaC2} et \eqref{DGol} sont compl\`etement d\'etermin\'e par la fonction $\nu_1$ dont, 
malheureusement, nous ne disposons d'aucune formule. 
En~l'\'etat~actuel de nos connaissances, nous pouvons au mieux l'approcher num\'eriquement par la~fonction~$\aleph$ de la Figure 1 
sans ma\^{\i}triser la pr\'ecision de cette approximation, la~constante implicite au reste de l'estimation~\eqref{oulala} n'\'etant pas effective.
\bigskip


\Secti Equadiffs, \'Equations diff\'erentielles associ\'ees \`a la suite de Golomb. 


L'\'etude des \'equations fonctionnelles  \eqref{La6}, \eqref{La10}, \eqref{eqrn}~et~\eqref{eqR} 
a~permis de pr\'eciser le comportement \`a l'infini de la suite $\{u_n\}_{n=1}^\infty$. Certaines variantes de l'\'equation diff\'erentielle 
$$
f'(t)={1\F f\circ f(t)}
\eqdef{ol2}
$$
\'etant similairement li\'ees \`a la suite de Golomb, nous \'etudions 
dans les paragraphes suivants le comportement asymptotique de leurs solutions et nous consid\'erons au paragraphe \CitSec{Point fixe} 
leur existence, leur unicit\'e et leur caract\'erisation par une condition initiale.  
\bigskip

Montrons d'abord le lien entre l'\'equation diff\'erentielle~\eqref{ol2} et la suite de Golomb. 
Pour~$n\ge1$, nous notons $w_n=u_{u_n}$ le nombre d'occurrences
de $u_n$ dans cette suite. 
Comme les entiers $u_n$ et $u_n+1$ apparaissent respectivement $w_n$ fois et
$u_{u_n+1}\ge w_n$ fois dans la~suite~$\{u_k\}_{k=1}^\infty$,il r\'esulte de \eqref{fond} que $u_{n+w_n}=u_n+1$ et par suite que 
$$
{u_{n+w_n}-u_n\F w_n}={1\F u_{u_n}}\qquad(n\ge1).\eqdef{eqdis}
$$
Le membre de gauche \'etant un taux
d'accroissement de la suite de~Golomb, il~est~naturel de comparer la suite $\{u_n\}_{n=1}^\infty$
aux solutions positives de~l'\'equation diff\'erentielle~\eqref{ol2}. 
Cette~analogie est encore renforc\'ee par le~fait que \eqref{Fecpn} induit que $w_n=o(n)\ \,(n\to\infty)$, par le fait que la suite $\{u_n\}_{n=1}^\infty$ est \`a~croissance
polyn\^omiale et par le fait que l'\'equivalent asymptotique $t\mapsto ct^{\varphi-1}$ de la suite de Golomb
satisfait l'\'equation diff\'erentielle \eqref{ol2}. 
\bigskip


En 1996, P\'etermann \CitRef{Petermann1} retrouve l'estimation \eqref{Fecpn} en utilisant une variante de~\eqref{ol2}.  
Pour ce faire, il~construit une solution
approch\'ee $\xi$ de l'\'equation~\eqref{ol2} se comportant
\`a l'infini comme la suite de Golomb, il~montre que $\xi(t)\sim
ct^{\varphi-1}$ et en d\'eduit d'une part~\eqref{Fecpn} et d'autre part que la fonction $\xi$ est une solution de
l'\'equation diff\'erentielle approch\'ee 
$$
f'(t)={1+O\big(t^{\varphi-2}\big)\F f\circ f(t)}\qquad(t\ge1).
\eqdef{nimp}
$$
Plus pr\'ecis\'ement, il~construit une
fonction croissante $\xi\in\sc C^1\b(\ob R_+^*,\ob R_+^*\b)$ v\'erifiant
$$
\eqalignno{
u_n=\xi(n)+O(1)\quad\qquad\qquad(n\ge1)_{\strut}&,&\eqdef{xi2}
\cr
\xi'(t)={1\F\xi\circ\xi(t)}+O\Bg({1\F \xi\circ\xi(t)^2}\Bg)\qquad(t\ge1)&&\eqdef{xi1}
}
$$
et d\'eduit de \eqref{xi2} que $\lim_{t\to\infty}\xi(t)=+\infty$ et de \eqref{xi1}
que $\xi$ est solution de
l'\'equation~approch\'ee
$$
f'(t)={1+o(1)\F f\circ f(t)}\qquad(t\to\infty).\eqdef{eqP}
$$
Pour chaque solution $f\in\sc C^1(\ob R_+^*,\ob R_+^*)$ de l'\'equation
approch\'ee \eqref{eqP}, il prouve que l'\'equation~$f(t)=ct^{\varphi-1}$ 
poss\`ede des
racines arbitrairement grandes et aussi que
$$
f(t)\sim ct^{\varphi-1}\qquad(t\to\infty).\eqdef{fsim}
$$
{\it A fortiori}, cela implique que $\xi(t)\sim ct^{\varphi-1}$.
En reportant dans \eqref{xi2}, P\'etermann en d\'eduit~\eqref{Fecpn}.
Par~ailleurs, il d\'eduit  de \eqref{Phi4}  et de \eqref{xi1} 
que la fonction $\xi$ satisfait \eqref{nimp}.
\bigskip

Construisons une telle fonction croissante $\xi\in\sc C^1\b([1,\infty[,[1,\infty[\b)$ v\'erifiant~\eqref{xi2}~et~\eqref{xi1}, 
c'est-\`a-dire faisant le lien entre l'\'equation diff\'erentielle \eqref{ol2} et la suite de Golomb. 
Pour~chaque fonction croissante $\psi\in\sc C^\infty(\ob R)$ v\'erifiant $\psi(t)=0\ (t\le0)$ et $\psi(t)=1\ (t\ge1)$, 
nous posons  
$$
\rho(t):=u_1+\sum_{k\ge1}(u_{k+1}-u_k)\psi(t-k)\qquad(t\ge1)
$$
et nous observons que $\rho$ est une fonction croissante de $\sc C^\infty\b([1,\infty[,[1,\infty[\b)$ v\'erifiant 
$$
u_n=\rho(n)
\qquad(n\ge1).
\eqdef{arc}
$$
Par cons\'equent, l'application d\'efinie par
$$
\xi(t):=1+\int_1^t{\d u\F\rho\circ\rho(u)}
\qquad(t\ge1)
\eqdef{defxi}
$$
est une fonction croissante de l'ensemble $\sc C^\infty\b([1,\infty[,[1,\infty[\b)$ et 
nous prouvons au paragraphe~\CitSec{Demons} qu'elle satisfait \'egalement les relations \eqref{xi2} et \eqref{xi1}. 
\bigskip

En 1999, P\'etermann, R\'emy et Vardi \CitRef{PetermannRemyVardi} conjecturent
que chaque solution maximale et positive $f$ de~l'\'equation diff\'erentielle~\eqref{ol2}
se comporte \`a l'infini comme la~suite de~Golomb,
c'est-\`a-dire qu'il existe une
fonction $h\in\sc C(\ob R)$ v\'erifiant \eqref{n+tn} et
$$
f(t)=ct^{\varphi-1}+h(\log_2t){t^{\varphi-1}\F\log t}+O\Q({t^{\varphi-1}\F(\log t)^2}\W)
\qquad(t\to\infty).
\leqno{(C')}
$$
Lorsque $f$ n'est pas la solution triviale $t\mapsto ct^{\varphi-1}$ de
l'\'equation diff\'erentielle \eqref{ol2}, ils~conjecturent de plus que la fonction $h$
s'annule exactement une fois sur l'intervalle $[0,1[$.
\bigskip

La premi\`ere de ces conjectures est une cons\'equence imm\'ediate du th\'eor\`eme suivant, que nous prouvons au paragraphe \CitSec{Theo} 
et que nous utilisons au paragraphe \CitSec{Demons} pour \'etablir les th\'eor\`emes \eqrefn{T1} et \eqrefn{T2}. 
\bigskip


\theo T3. Soient  $p\ge\e$, $q\in\ob R$, $\theta>1$ et $\Theta$ des constantes telles que  
$$
\Theta:=\max\{k\in\ob Z:k<\theta\}=[\theta]-\1_{\ob Z}(\theta). \eqdef{Theta1}
$$
Pour~chaque~solution $f\in\sc C^1\b([p,\infty[,[q,\infty[\b)\cap\sc C\b([q,\infty[,\ob R_+^*\b)$ 
de~l'\'equation diff\'erentielle 
$$
f'(t)={1+O\b({1/(\log t)^\theta}\b)\F f\circ f(t)}
\qquad(t\ge p),
\eqdef{T1eq1}
$$ 
il existe une~unique~suite $\nu_k\in\sc C^{\Theta-k}(\ob R)\ \,(1\le k\le\Theta)$
de~fonctions $2\lambda$-p\'eriodiques  v\'erifiant 
$$
f(t)=ct^{\varphi-1}+\!\!\!\sum_{1\le k\le\Theta}\!\!\!
\nu_k(\log_2t){t^{\varphi-1}\F(\log t)^k}
+O\!\Q(\!{t^{\varphi-1}\F(\log t)^\theta}+\1_{\ob Z}(\theta) {t^{\varphi-1}\log_2t\F(\log t)^\theta}\!\W)
\quad(t\ge p).\!\!\!\!
\eqdef{res}
$$
L'application $\nu_1$ satisfait \eqref{n+tn} et, pour~$2\le k\le \Theta$, l'application $G_k$ d\'efinie par \eqref{IcedEarth} et la fonction $\nu_k$ v\'erifient l'identit\'e~\eqref{rec} 
et sont des combinaisons lin\'eaires de produits dont les facteurs sont pris parmi les fonctions \eqref{oomph} 
ou~parmi les fonctions $\nu_1, \cdots, \nu_1^{(k-1)}$. 
\par
\bigskip


\Sect Notations, Notations et conventions. 

Dans cet article, nous devons \'etablir des estimations pour certaines d\'eriv\'ees successives.  
C'est pourquoi, nous introduisons trois nouvelles notations inspir\'ees celle de Landau. 
Soient~$n\ge0$ un entier, $I$ un intervalle, $f\in\sc C^n(I)$ et $g:I\to[0,+\infty[$ des applications. 
\medskip

Nous~notons indiff\'eremment $f(u)\olop_n g(u)\ \,(u\in I)$ ou
$$
f(u)\olop_n g(u)\quad(u\in I)\qquad\hbox{ou}\qquad f(u)=\sco_n\b(g(u)\b)\quad(u\in I)
$$
si et seulement s'il existe une constante $\alpha$ pour laquelle  
$$
\Q|f^{(k)}(u)\W|\le \alpha g(u)\qquad(0\le k\le n,u\in I). 
$$
Cette notation mesure si une fonction $f$ et ses d\'eriv\'ees se comportent \`a l'infini comme une application p\'eriodique, 
c'est-\`a-dire si d\'eriver $f(u)$ ne change pas son ordre de grandeur asymptotique. 
\medskip

Si $I\subset[0,+\infty[$, si $\theta>1$ et si $\Theta$ v\'erifie \eqref{Theta1}. nous notons $f(u)\olops_n g(u)\ \,(u\in I)$ ou 
$$
f(u)\olops_n g(u)\quad(u\in I)\qquad\hbox{ou}\qquad f(u)=\scos_n\b(g(u)\b)\quad(u\in I)
$$
si et seulement s'il existe une constante $\alpha$ pour laquelle  
$$
\eqalignno{
\Q|f^{(k)}(u)\W|&\le \alpha g(u)\qquad\qquad\quad(0\le k< n,u\in I). &\eqdef{var3a} \cr
&\le \alpha g(u)\Q(1+{u^{\1_{\ob Z}(\theta)}\e^u\F\e^{(\theta-\Theta)u}}\W)\qquad(k=n,u\in I)&\eqdef{var3b}
}
$$
Nous observons que cette notation est identique \`a la notation pr\'ec\'edente 
sauf pour la $n^{\hbox{\sevenrm i\`eme}}$ d\'eriv\'ee de la fonction $f$ dans le cas o\`u $\theta$ est un nombre entier. 
\bigskip


Si $I\subset]0,+\infty[$, nous notons indiff\'eremment $f(t)\olopd_n g(t)\ \,(t\in I)$ ou 
$$
f(t)\olopd_n g(t)\quad(t\in I)\qquad\hbox{ou}\qquad f(t)=\scod_n\b(g(t)\b)\quad(t\in I)
$$
si et seulement s'il existe une constante $\alpha$ pour laquelle  
$$
\Q|f^{(k)}(t)\W|\le\alpha {g(t)\F t^k}\qquad(0\le k\le n,t\in I). 
$$
Cette notation mesure si une fonction $f$ et ses d\'eriv\'ees se~comportent \`a l'infini comme un mon\^ome, 
c'est-\`a-dire si d\'eriver $n$ fois $f(t)$ revient  \`a~diviser $f(t)$ par $t^n$ pour l'ordre de grandeur asymptotique. 
\medskip


\Sect Demons, D\'emonstrations \'el\'ementaires. 


Dans les paragraphes suivants, nous prouvons toutes les r\'esultats non-prouv\'es de  l'introduction, 
\`a l'exception du th\'eor\`eme \eqrefn{T3} que nous d\'emontrons au paragraphe \CitSec{Theo}. 
Ainsi, nous \'etablissons les estimations \eqref{eqrn}, \eqref{La10}, \eqref{xi2}, \eqref{xi1} et les th\'eor\`emes \eqrefn{T1} et \eqrefn{T2}. 
\bigskip

\proof{ de l'estimation \eqref{La10}}.Comme les deux membres de l'\'equation \eqref{La6} sont croissants et \'egaux pour $n=v_m\ (m\ge1)$, 
leur~variation sur l'ensemble $\{v_{m-1},\cdots,v_m\}$ 
est inf\'erieure \`a~$v_m-v_{m-1}$ et nous obtenons alors que 
$$
\sum_{1\le k\le u_n}u_k=n+O\b(v_m-v_{m-1}\b)
\qquad(m\ge2,v_{m-1}\le n\le v_m).
$$
En remarquant que  \eqref{defvm} et
\eqref{Fecpn} impliquent d'une part que $v_m-v_{m-1}\sim cm^{\varphi-1}$
et~d'autre part que $v_m\sim(c/\varphi)m^\varphi$, 
nous d\'eduisons alors de l'\'egalit\'e $\varphi(\varphi-1)=1$ que 
$$
v_m-v_{m-1}\ll n^{2-\varphi}
\qquad(m\ge2,v_{m-1}\le n\le v_m).
$$ 
En particulier, l'estimation \eqref{La10} est satisfaite. 
\hfill\qed
\bigskip


\proof{ de l'estimation \eqref{eqrn}}. La fonction $x\mapsto cx^{\varphi-1}$ \'etant~croissante, nous d\'eduisons de l'\'egalit\'e~\eqref{Phi4}~que 
$$
\sum_{1\le k\le
cn^{\varphi-1}}ck^{\varphi-1}=\int_1^{cn^{\varphi-1}}\!\!\!\!cx^{\varphi-1}\d x+O\Q(n^{(\varphi-1)^2}\W)=n+O\b(n^{2-\varphi}\b)\qquad(n\ge1).
$$
Pour chaque $n\ge1$, nous posons $E_n:=[u_n,cn^{\varphi-1}[$ si $r_n\ge0$ 
et $E_n:=]cn^{\varphi-1}, u_n]$ sinon. 
En~soustrayant l'estimation pr\'ec\'edente \`a~\eqref{La10}, 
nous d\'eduisons de \eqref{La7} que
$$
\sum_{1\le k\le cn^{\varphi-1}}r_k
+\sgn(r_n)\sum_{k\in E_n}u_k
\ll n^{2-\varphi}
\qquad(n\ge1).
\eqdef{La9}
$$
Comme les relations \eqref{La7} et \eqref{Vardi} impliquent que  
$$
u_k=ck^{\varphi-1}+O\Q({k^{\varphi-1}\F\log k}\W)
=c^\varphi n^{(\varphi-1)^2}+O\Q({n^{(\varphi-1)^2}\F\log n}\W)\qquad(n\ge2,k\in
E_n)
$$
et comme $|E_n|=|r_n|+O(1)\ll n^{\varphi-1}/\log n$, nous obtenons que 
$$
\sum_{k\in E_n}u_k
=\Q(c^\varphi |r_n|+O\Q({n^{\varphi-1}\F(\log n)^2}\W)\W)n^{(\varphi-1)^2}
\qquad(n\ge2).
$$
En reportant dans \eqref{La9}, nous d\'eduisons alors  \eqref{eqrn} de l'\'egalit\'e \eqref{Phi4}.
\hfill\qed
\bigskip


\proof{ de l'estimation \eqref{xi2}}. Comme $u_2=2$ et $v_1=1$, nous d\'eduisons de \eqref{fond} que $$
v_{u_n-1}<n\le v_{u_n}\qquad(n\ge2)
$$ 
et nous d\'eduisons de \eqref{defxi} que 
$$
\xi(n)=1+\sum_{2\le m\le u_n}\int_{v_{m-1}}^{v_m}{\d u\F
\rho\circ \rho(u)}+O\Q(\int_{v_{u_n-1}}^{v_{u_n}}{\d u\F
\rho\circ \rho(u)}\W)
\qquad(n\ge2).
\eqdef{kol}
$$
D'apr\`es les relations \eqref{fond} et \eqref{arc}, la fonction croissante $\rho\circ\rho$ vaut
$u_{m-1}$ en $v_{m-1}$ et $u_m$ sur le segment $[v_{m-1}+1,v_m]$ pour chaque entier $m\ge2$. 
{\it A fortiori}, nous obtenons que 
$$
\int_{v_{m-1}}^{v_m}{\d u\F\rho\circ\rho(u)}={v_m-v_{m-1}\F u_m}
+O\Q({1\F u_m}-{1\F u_{m-1}}\W)
\qquad(m\ge2).
$$
Comme \eqref{defvm} implique que $v_m-v_{m-1}=u_m\ \,(m\ge1)$, il suit 
$$
\int_{v_{m-1}}^{v_m}{\d u\F\rho\circ\rho(u)}=1+O\Q({1\F u_m}-{1\F u_{m-1}}\W)
\qquad(m\ge2).
$$
En reportant dans \eqref{kol}, nous obtenons alors l'estimation \eqref{xi2}. 
\hfill\qed
\bigskip


\proof{ de l'estimation \eqref{xi1}}. En d\'erivant \eqref{defxi}, nous obtenons que   
$$
\xi'(t)={1\F\rho\circ\rho(t)}={1\F\xi\circ\xi(t)}\Q(1+{\rho\circ\rho(t)-\xi\circ\xi(t)\F\xi\circ\xi(t)}\W)^{-1}
\qquad(t\ge1). 
\eqdef{intero}
$$
Comme l'in\'egalit\'e $\rho(t)\ge 1\ \,(t\ge1)$ induit que $\xi'(t)\ll1\ \,(t>0)$, nous observons alors que 
$$
\rho\circ\rho(t)-\xi\circ\xi(t)=\rho\b(\rho(t)\b)-\xi\b(\rho(t)\b)+\int_{\xi(t)}^{\rho(t)}\xi'(x)\d x
\qquad(t\ge1).
$$
Comme $\rho$ et $\xi$ sont des applications croissantes v\'erifiant \eqref{xi2}, 
nous en d\'eduisons alors d'une~part~que $\xi(t)=\rho(t)+O(1)\ \,(t\ge1)$ 
et d'autre part que 
$$
\rho\circ\rho(t)-\xi\circ\xi(t)\ll 1\qquad(t\ge1). 
$$
En reportant dans \eqref{intero}, nous obtenons enfin l'estimation \eqref{xi1}. \hfill\qed
\bigskip


\proof{ des th\'eor\`emes \eqrefn{T1} et \eqrefn{T2}}. L'application $\xi\in\sc C^\infty\b([1,+\infty[,[1,+\infty[\b)$ 
d\'efinie par \eqref{defxi} esrt croissante  et satisfait \eqref{xi2} et \eqref{xi1}. 
En proc\'edant comme au paragraphe \CitSec{Equadiffs}, nous~d\'eduisons alors des estimations~\eqref{Fecpn}~et~\eqref{xi2} 
que $\xi(t)\sim ct^{\varphi}\ \,(t\to+\infty)$ et nous d\'eduisons des relations  \eqref{Phi2} et \eqref{xi1} 
que la fonction $\xi$ satisfait l'\'equation~\eqref{nimp}. 
D'apr\`es~le th\'eor\`eme \eqrefn{T3} appliqu\'e \`a $\xi$ pour le choix $p=\e$, $q=1$ et $\theta=K+2\ \,(K\ge1)$, 
il~existe une unique suite $\{\nu_k\}_{k\ge1}\in\sc C^\infty(\ob R)^{\ob N^*}$ de fonctions $2\lambda$-p\'eriodiques v\'erifiant
$$
\xi(t)=ct^{\varphi-1}+\sum_{1\le k\le K+1}\nu_k(\log_2t){t^{\varphi-1}\F(\log t)^k}
+O_K\Q({t^{\varphi-1}\log_2t\F(\log t)^{K+2}}\W)
\qquad(K\ge1,t\ge1).
$$
Alors, \'etant donn\'ees $\{\mu_k\}_{k\ge1}$ et $\{\mu_k^*\}_{k\ge1}$ les fonctions d\'etermin\'es 
par \eqref{mu1}~et~\eqref{mu2}, pour~chaque entier $k\ge2$, l'application $G_k$ d\'efinie par \eqref{IcedEarth} et la fonction $\nu_k$ v\'erifient~\eqref{rec} 
et sont des combinaisons lin\'eaires de produits dont les facteurs sont pris parmi les fonctions \eqref{oomph} 
ou~parmi les d\'eriv\'ees de $\nu_1$ d'ordre inf\'erieur \`a $k-1$. De plus, l'application $\nu_1$ satisfait \eqref{n+tn}. 
Comme la fonction $\nu_{K+1}$ est continue et p\'eriodique, nous remarquons qu'elle est \'egalement born\'ee et par cons\'equent que 
$$
\xi(t)=ct^{\varphi-1}+\sum_{1\le k\le K}\nu_k(\log_2t){t^{\varphi-1}\F(\log t)^k}
+O_K\Q({t^{\varphi-1}\F(\log t)^{K+1}}\W)
\qquad(K\ge1,t\ge1).
$$
En reportant dans \eqref{xi2}, nous obtenons le d\'eveloppement asymptotique \eqref{DGol} et nous d\'eduisons \eqref{DaC2} de l'identit\'e~\eqref{La7}. Comme $\nu_1\in\sc C(\ob R)$ v\'erifie \eqref{n+tn}~et~\eqref{Pet}, 
le r\'esultat obtenu par P\'etermann et R\'emy \CitRef{PetermannRemy}  implique alors que $\nu_1(0)\neq0$. 
Enfin, nous remarquons que~\eqref{DaC2} est v\'erifi\'e par au plus une suite $\{\nu_k\}_{k\ge1}\in\sc C^\infty(\ob R)^{\ob N^*}$ 
de fonctions $2\lambda$-p\'eriodiques. En effet, dans le cas contraire, il existerait une fonction p\'eriodique non-identiquement nulle convergeant vers $0$ en $+\infty$, ce qui est absurde.  
\hfill\qed
\bigskip

\Sect Lemmes, Lemmes et autres r\'esultats auxiliaires.

Nous avons rassembl\'e ici les lemmes sur lesquels repose la d\'emonstration du~th\'eor\`eme~\eqrefn{T3}. 
Leurs~preuves s'appuient sur des r\'esultats classiques, comme le th\'eor\`eme de d\'erivation des fonctions compos\'ees que nous rappelons ci-dessous, 
et sont pour la plupart autonomes. 
\bigskip


\theo dfc0. Soit $a\in\ob R$, soit $k\in\ob N$ et soient $f$ et $g$ 
deux fonctions d\'erivable $k$ fois respectivement en $a$ et en $f(a)$. 
Alors, 
l'application $g\circ f$ est $k$ fois d\'erivable en $a$ et 
$$
(g\circ f)^{(k)}(a)=k!\sum_{1\le s\le n}{g^{(s)}\b(f(a)\b)\F s!}
\sum_{\ss n_1+\cdots+n_s=k\atop\ss n_1\cdots n_s\neq0}
{f^{(n_1)}(a)\cdots f^{(n_s)}(a)\F n_1!\cdots n_s!}, 
$$
cette \'egalit\'e prouvant \'egalement se mettre sous la forme 
$$
(g\circ f)^{(k)}(a)=k!\sum_{n_1+\cdots+kn_k=k}g^{(n_1+\cdots+n_k)}\b(f(a)\b)
\prod_{1\le\ell\le k}{f^{(\ell)}(a)^{n_\ell}\F n_\ell!\ell^{n_\ell}}.
$$
\par
\dem. cf \CitRef{McKiernan2}. 
\hfill\qed
\bigskip


\Secti dfc, Majorations de d\'eriv\'ees successives. 


\lemm dlm1. Pour chaque nombre r\'eel $a$ et chaque entier $n\in\ob N$, nous avons
$$
\eqalignno{
{\d^n\hfill\F\d x^n}{\e^{ax}\F\exp(\e^x)}&\ll\ {1+\e^{nx}\F\exp(\e^x)}\e^{ax}
\qquad(x\in\ob R), &\eqdef{dlm1eq1}
\cr
&\ll\ \e^{-x}\qquad\quad\,
\qquad(x\ge0). & \eqdef{dlm1eq2}
}
$$
\par
\bigskip


\dem. D'apr\`es la formule de Leibniz, nous avons  
$$
{\d^n\hfill\F\d x^n}{\e^{ax}\F\exp(\e^x)}
=\sum_{0\le\ell\le n}\Q({n\atop\ell}\W)a^{n-\ell}\e^{ax}{\d^\ell\hfill\F\d x^\ell}{1\F\exp(\e^x)}
\qquad(x\in\ob R). 
$$
Comme le th\'eor\`eme de d\'erivation des foncions compos\'ees implique que 
$$
{\d^\ell\hfill\F\d x^\ell}{1\F\exp(\e^x)}=\ell!
\sum_{\ss n_1+\cdots+\ell n_\ell=\ell}{(-1)^{n_1+\cdots+n_\ell}\F\exp(\e^x)}\prod_{1\le j\le \ell}
{\e^{n_jx}\F n_j!j!^{n_j}}
\qquad(\ell\in\ob N,x\in\ob R), 
$$
nous remarquons alors d'une part que 
$$
{\d^\ell\hfill\F\d x^\ell}{1\F\exp(\e^x)}\ll{1+\e^{\ell x}\F\exp(\e^x)}\qquad(0\le \ell\le n,x\in\ob R)
$$
et d'autre part que la majoration \eqref{dlm1eq1} est satisfaite. Enfin, nous d\'eduisons~\eqref{dlm1eq2} de la~minoration 
$\exp(\e^x)\gg\e^{(a+n+1)x}\ \,(x\ge0)$. 
\hfill\qed
\bigskip


\lemm dfc1. Soient $n\in\ob N$ et $f\in\sc C^n\b([\e,\infty[,[\e,\infty[\b)$
une fonction v\'erifiant \eqref{fsim} et
$$
f(t)\olopd_n t^{\varphi-1}\qquad(t\ge\e).
\eqdef{dfc1eq1}
$$
Alors, l'application $f\circ f$ appartient \`a l'espace $\sc C^n\b([\e,+\infty[\b)$ et satisfait   
$$
f\circ f(t)\olopd_n t^{2-\varphi}
\qquad(t\ge\e).
\eqdef{dfc1eq2}
$$
\par
\bigskip


\dem. Nous remarquons que $f\circ f\in\sc C^n\b([\e, \infty[\b)$ et  
nous fixons $k\in\{0,\cdots,n\}$. 
D'apr\`es le th\'eor\`eme \eqrefn{dfc0}, de d\'erivation des fonctions compos\'ees, nous avons  
$$
(f\circ f)^{(k)}(t)=k!\sum_{n_1+\cdots+kn_k=k}{f^{(n_1+\cdots+n_k)}\big(f(t)\big)\F 1!^{n_1}\cdots k!^{n_k}}
\prod_{1\le\ell\le k}{f^{(\ell)}(t)^{n_\ell}\F n_\ell!}
\qquad(t\ge\e).
\eqdef{dfc1eq3}
$$
La~majoration \eqref{dfc1eq1} implique alors que 
$$
(f\circ f)^{(k)}(t)\ll\sum_{n_1+\cdots+kn_k=k}f(t)^{\varphi-1-n_1-\cdots-n_k}
\prod_{1\le\ell\le k}t^{(\varphi-1-\ell)n_\ell}
\qquad(t\ge\e).
$$
Comme l'application $f:[\e,\infty[\to[\e,\infty[$ satisfait \eqref{fsim}, 
nous remarquons d'une part que 
$$
f(t)\asymp t^{\varphi-1}
\qquad(t\ge \e)
\eqdef{dfc1eq4}
$$ 
et d'autre part que 
$$
(f\circ f)^{(k)}(t)\ll\sum_{n_1+\cdots+kn_k=k}t^{(\varphi-1)(\varphi-1-n_1-\cdots-n_k)}
\prod_{1\le\ell\le k}\!\!t^{(\varphi-1-\ell)n_\ell}
\qquad(t\ge\e).
$$
En particulier, nous obtenons que 
$$
(f\circ f)^{(k)}(t)\ll t^{(\varphi-1)^2-k}
\qquad(0\le k\le n, t\ge\e)
$$
et nous d\'eduisons de l'\'egalit\'e \eqref{Phi4} que la majoration \eqref{dfc1eq2} est satisfaite. 
\hfill\qed
\bigskip



\lemm dfc5. Soient $n\in\ob N$ et $f\in\sc C^{n+1}\b([\e,\infty[,[\e,\infty[\b)$ 
une fonction v\'erifiant \eqref{fsim} et 
$$
f'(t)={1+\scod_n(1)\F f\circ f(t)}\qquad(t\ge\e).\eqdef{dfc5eq1}
$$
Alors, la fonction $f$ satisfait la majoration  
$$
f(t)\olopd_{N+1}t^{\varphi-1}
\qquad(t\ge\e). 
\eqdef{dfc5eq2}
$$
\par
\bigskip




\dem. Proc\'edons par r\'ecurrence sur $k$. La majoration \eqref{dfc5eq2} pour $k=0$ d\'ecoule de~la relation \eqref{fsim}. 
Soit $j\in\{0,\cdots, n\}$ un entier pour lequel 
$$
f(t)\olopd_ j t^{\varphi-1}\ \,(t\ge\e).
$$ 
En d\'erivant $j$ fois la quantit\'e $f'(t)f\circ f(t)-1$, nous d\'eduisons alors de \eqref{dfc5eq1} que 
$$
\sum_{0\le k\le j}\Q({j\atop k}\W)f^{(k+1)}(t)(f\circ f)^{(j-k)}(t)\ll t^{-j}
\qquad(t\ge\e)
$$
et aussi que 
$$
f^{(j+1)}(t)f\circ f(t)\ll t^{-j}+\sum_{0\le k<j}
\Q|(f\circ f)^{(j-k)}(t)\W|
t^{\varphi-2-k}
\qquad(t\ge\e)
$$
Comme la fonction $f$ satisfait les hypoth\`eses du Lemme \eqrefn{dfc1} pour l'entier $j$, 
d'apr\`es~\eqref{fsim} et d'apr\`es l'hypoth\`ese de r\'ecurrence, 
nous d\'eduisons de \eqref{dfc1eq2} pour l'entier $n=j$ que 
$$
f^{(j+1)}(t)f\circ f(t)\ll t^{-j}
\qquad(t\ge\e).
$$
En remarquant que les relations \eqref{fsim} et \eqref{Phi4} impliquent que
$$
f\circ f(t)\asymp t^{2-\varphi}
\qquad(t\ge\e), 
\eqdef{dfc5eq3}
$$
nous obtenons que 
$f(t)\olopd_ {j+1} t^{\varphi-1}\ \,(t\ge\e)$ et nous en d\'eduisons \eqref{dfc5eq2}, par r\'ecurrence. 
\hfill\qed
\bigskip



\lemm dfc4. Soient $n\in\ob N^*$ et $f\in\sc C^n\b([\e,\infty[,[\e,\infty[\b)$ 
une fonction v\'erifiant \eqref{fsim} et \eqref{dfc1eq1}. 
Alors, la~fonction $f$ satisfait la majoration
$$
{1\F f\circ f(t)}\olopd_ n t^{\varphi-2}
\qquad(t\ge\e). 
\eqdef{dfc4eq1}
$$
\par



\dem. Nous proc\'edons par r\'ecurrence en remarquant que l'estimation 
$$
{1\F f\circ f(t)}\olopd_ {j-1} t^{\varphi-2}
\qquad(t\ge\e). 
\eqdef{SkyClod}
$$
est satisfaite pour l'entier $j=1$, d'apr\`es \eqref{fsim}. 
Fixons un entier $j\in[1, n]$ v\'erifiant~\eqref{SkyClod}. 
En d\'erivant $j$ fois l'identit\'e $f\circ f/f\circ f=1$ sur l'intervalle $[\e,+\infty[$, 
nous obtenons que 
$$
\sum_{0\le k\le j}\Q({j\atop k}\W)(f\circ f)^{(j-k)}(t)\Q({1\F f\circ f}\W)^{(k)}\!\!\!(t)=0
\qquad(t\ge\e)
$$
et nous d\'eduisons alors de \eqref{SkyClod} que 
$$
f\circ f(t)\Q({1\F f\circ f}\W)^{(j)}\!\!\!(t)\ll\sum_{0\le k<j}\Q|(f\circ f)^{(j-k)}(t)\W|t^{\varphi-2-k}
\qquad(t\ge\e).
$$
Appliquant le Lemme \eqrefn{dfc1} pour $n=j$ \`a la fonction $f$, 
qui satisfait \eqref{fsim}~et~\eqref{dfc1eq1}, 
nous~d\'eduisons de la majoration \eqref{dfc1eq2} que 
$$
f\circ f(t)\Q({1\F f\circ f}\W)^{(j)}\!\!\!(t)\ll t^{-j}\qquad(t\ge\e).
$$ 
Comme les estimations \eqref{fsim} et \eqref{Phi4} induisent \eqref{dfc5eq3}, 
nous obtenons que 
$$
\Q({1\F f\circ f}\W)^{(j)}\!\!\!(t)\ll t^{\varphi-2-j}\qquad(t\ge\e)
$$
et nous d\'eduisons de la relation \eqref{SkyClod} que 
$$
{1\F f\circ f(t)}\olopd_ j t^{\varphi-2}
\qquad(t\ge\e). 
$$
Par r\'ecurrence, nous concluons enfin que la majoration \eqref{dfc4eq1} est satisfaite. 
\hfill\qed
\bigskip



\lemm dfc3. Soient $n\in\ob N$ et $g\in\sc C^{n+1}\b([\e,\infty[,[\e,\infty[\b)$ une application v\'erifiant 
$$
g(t)\olopd_{n+1} t^{\varphi-1}\qquad(t\ge\e).
\eqdef{dfc3eq1}
$$
Soient $\theta>1$ et $f\in\sc C^n\b([\e,\infty[,[\e,\infty[\b)$ une fonction v\'erifiant les relations \eqref{fsim}, \eqref{dfc1eq1} et 
$$
g(t)-f(t)\olopd_ n {t^{\varphi-1}\F(\log t)^\theta}
\qquad(t\ge \e).
\eqdef{dfc3eq2}
$$
Alors, les fonctions $f$ et $g$ v\'erifient la majoration 
$$
g^{(k)}\b(g(t)\b)-f^{(k)}\b(f(t)\b)\ll{t^{2-\varphi-k(\varphi-1)}\F(\log t)^\theta}
\qquad(0\le k\le n,t\ge\e).
\eqdef{dfc3eq3}
$$
\par
\bigskip




\dem. Nous fixons un entier $k\in\{0,\cdots,n\}$ et nous remarquons que 
$$
g^{(k)}\b(g(t)\b)-f^{(k)}\b(f(t)\b)=g^{(k)}\b(f(t)\b)-f^{(k)}\b(f(t)\b)
+\int_{f(t)}^{g(t)}g^{(k+1)}(x)\d x
\qquad(t\ge\e).
$$
D'apr\`es les majorations \eqref{dfc3eq1} et \eqref{dfc3eq2}, il suit 
$$
g^{(k)}\b(g(t)\b)-f^{(k)}\b(f(t)\b)\ll{f(t)^{\varphi-1-k}\F\b(\log f(t)\b)^\theta}
+\bg|\int_{f(t)}^{g(t)}x^{\varphi-2-k}\d x\bg|
\qquad(t\ge\e).
$$
Comme les relations \eqref{fsim} et \eqref{Phi4} induisent \eqref{dfc1eq2} et 
comme \eqref{dfc3eq2} implique que 
$$
g(t)-f(t)\ll{t^{\varphi-1}\F(\log t)^\theta}
\qquad(t\ge\e), 
$$ 
nous remarquons alors que 
$$
g^{(k)}\b(g(t)\b)-f^{(k)}\b(f(t)\b)\ll{t^{(\varphi-1)(\varphi-1-k)}\F(\log t)^\theta}
\qquad(0\le k\le n,t\ge\e)
$$
et nous d\'eduisons \eqref{dfc3eq3} de l'\'egalit\'e \eqref{Phi4}.
\hfill\qed
\bigskip

\lemm dfc2. Soient $\theta>1$ et $n\in\ob N$. Soient 
$f\in\sc C^n\b([\e,\infty[,[\e,\infty[\b)$ et $g\in \sc C^{n+1}\b([\e,\infty[,[\e,\infty[\b)$ deux~applications satisfaisant les hypoth\`eses du lemme \eqrefn{dfc3}. 
Alors, elles v\'erifient 
$$
g\circ g(t)-f\circ f(t)\olopd_ n {t^{2-\varphi}\F(\log t)^\theta}
\qquad(t\ge\e).
\eqdef{dfc2eq1}
$$
\par
\bigskip




\dem. Nous fixons un entier $k\in\{0,\cdots,n\}$ et nous posons 
$$
P_{n_1,\cdots,n_k}(t):=g^{(n_1+\cdots+n_k)}\b(g(t)\b)\prod_{1\le\ell\le k}g^{(\ell)}(t)^{n_\ell}-
f^{(n_1+\cdots+n_k)}\b(f(t)\b)\prod_{1\le\ell\le k}f^{(\ell)}(t)^{n_\ell}\eqdef{dfc2eq2}
$$
pour chaque vecteur $(n_1,\cdots,n_k)\in\ob N^k$ v\'erifiant $n_1+\cdots+kn_k=k$ et 
chaque $t\ge\e$. 
L'identit\'e \eqref{dfc1eq3} \'etant 
v\'erifi\'ee par $f$~et~$g$ d'apr\`es~le th\'eor\`eme \eqrefn{dfc0}, nous remarquons que 
$$
(g\circ g)^{(k)}(t)-(f\circ f)^{(k)}(t)=k!\sum_{n_1+\cdots+kn_k=k}P_{n_1,\cdots,n_k}(t)
\prod_{1\le\ell\le k}{1\F n_\ell!(\ell!)^{n_\ell}}
\quad(t\ge\e).\!\!\!\!\!
\eqdef{dfc2eq3}
$$
Nous fixons un vecteur $(n_1,\cdots,n_k)\in\ob N^k$ v\'erifiant $n_1+\cdots+kn_k=k$ 
et nous remarquons que l'entier $s:=n_1+\cdots+n_k$ satisfait $0\le s\le k$. 
Posant $n_0:=1$ et  
$$\vcenter{
\halign{
\hfill$\ds#$&$\ds#$\hfill&\quad\qquad\hfill$\ds#$&$\ds#$\hfill\cr
\cr
a_0(t)&:=f^{(s)}\b(f(t)\b),&a_\ell(t)&:=f^{(\ell)}(t)
\cr
b_0(t)^{\strut}&:=g^{(s)}\b(g(t)\b)-f^{(s)}\b(f(t)\b),&b_\ell(t)&:=g^{(\ell)}(t)-f^{(\ell)}(t)
\cr}}\qquad\quad(1\le\ell\le k, t\ge\e),
$$
nous d\'eduisons alors de \eqref{dfc2eq2} que 
$$
P_{n_1,\cdots,n_k}(t)=\prod_{0\le \ell\le k}\b(a_\ell(t)+b_\ell(t)\b)^{n_\ell}
-\prod_{0\le \ell\le k}a_\ell(t)^{n_\ell}
\qquad(t\ge\e).
$$
Soit $E$ l'ensemble des vecteurs $(m_0,\cdots,m_k)\in\ob N^{k+1}\ssm\{0\}$ avec
$0\le m_\ell\le n_\ell\ (0\le\ell\le k)$. D\'eveloppant le produit, nous obtenons que 
$$
P_{n_1,\cdots,n_k}(t)=\sum_{(m_0,\cdots,m_k)\in E}\ \prod_{0\le \ell\le k}\bn{n_\ell}{m_\ell}
a_\ell(t)^{n_\ell-m_\ell}b_\ell(t)^{m_\ell}
\qquad(t\ge\e).
$$
Comme \eqref{dfc1eq1} et \eqref{dfc3eq1} impliquent que 
$$
a_\ell(t)\ll t^{\varphi-1-\ell} \qquad\hbox{et}\qquad
b_\ell(t)\ll{t^{\varphi-1-\ell}\F(\log t)^\theta}
\qquad(1\le\ell\le k,t\ge\e), 
$$
nous en d\'eduisons que 
$$
P_{n_1,\cdots,n_k}(t)\ll\sum_{(m_0,\cdots,m_k)\in E}\b|a_0(t)^{n_0-m_0}b_0(t)^{m_0}\b|
\prod_{1\le \ell\le k}{t^{(\varphi-1-\ell)n_\ell}\F(\log t)^{\theta m_\ell}}
\qquad(t\ge\e).
$$
Appliquant le lemme \eqrefn{dfc3} aux fonctions $f$ et $g$ 
pour $s\in\{0,\cdots, n\}$, nous observons que 
$$
b_0(t)\ll{t^{2-\varphi-s(\varphi-1)}\F(\log t)^\theta}
\qquad(t\ge\e). 
$$
De m\^eme, comme les relations \eqref{fsim} et \eqref{dfc1eq1} 
impliquent que  
$$
a_0(t)\ll t^{2-\varphi-s(\varphi-1)}
\qquad(t\ge\e),
$$ 
nous en d\'eduisons {\it a fortiori} que 
$$
P_{n_1,\cdots,n_k}(t)\ll\sum_{(m_1,\cdots,m_k)\in E}{t^{(2-\varphi-s(\varphi-1))n_0}\F(\log t)^{\theta m_0}}
\prod_{1\le \ell\le k}{t^{(\varphi-1-\ell)n_\ell}\F(\log t)^{\theta m_\ell}}
\qquad(t\ge\e).
$$
D'apr\`es les relations $n_0=1$ et $m_0+\cdots+m_k\ge1$, il suit 
$$
P_{n_1,\cdots,n_k}(t)\ll{t^{2-\varphi-s(\varphi-1)}\F(\log t)^\theta}
\prod_{1\le \ell\le k}t^{(\varphi-1-\ell)n_\ell}
\qquad(t\ge\e).
$$
Nous d\'eduisons  alors de l'\'egalit\'e $n_1+\cdots+n_k=s$ que 
$$
P_{n_1,\cdots,n_k}(t)\ll{t^{2-\varphi-k}\F(\log t)^\theta}
\qquad(n_1+\cdots+kn_k=k,t\ge\e).
$$
Reportant dans \eqref{dfc2eq3}, nous concluons 
que l'estimation \eqref{dfc2eq1} est satisfaite. 
\hfill\qed
\bigskip


\lemm dfc7. Soient $b\in\ob R$, $\theta>1$, $n\in\ob N$ et $f\in\sc C^n\b([\e,\infty[\b)$ 
une fonction v\'erifiant 
$$
f(t)\olopd_ n{1\F(\log t)^\theta}
\qquad(t\ge\e). 
\eqdef{dfc7eq1}
$$
Alors, l'application $g$ implicitement d\'efinie par $b+\int_\e^tf(x)\d x=tg(\log_2t)\ \,(t\ge\e)$ 
est de~classe~$\sc C^{n+1}$ sur l'intervalle $[0,+\infty[$ et satisfait 
$$
g^{(k)}(u)\ll\e^{(k-\theta)u}
\qquad(0\le k\le n+1,u\ge0).
\eqdef{dfc7eq2}
$$
\par
\bigskip


\dem. Nous fixons $k\in[0,n+1]$ et 
nous remarquons que la fonction $h$ d\'efinie~par  
$$
b+\int_\e^t f(x)\d x=th(t)
\qquad(t\ge\e)
\eqdef{dfc7eq3}
$$
appartient \`a l'espace $\sc C^{n+1}\b([\e,\infty[\b)$ 
et satisfait $g(u)=h\b(\exp(\e^u)\b)$ pour chaque $u\ge0$. 
En~particulier, nous~d\'eduisons du th\'eor\`eme \eqrefn{dfc0} que 
$$
g^{(k)}(u)=k!\sum_{n_1+\cdots+kn_k=k}{h^{(n_1+\cdots+n_k)}\b(\exp(\e^u)\b)\F 1!^{n_1}\cdots k!^{n_k}}
\prod_{1\le\ell\le s}{1\F n_\ell!}
\Q({\d^\ell\hfill\F\d u^\ell}\exp(\e^u)\W)^{n_\ell}
\qquad(u\ge0).
$$
Pour chaque entier $\ell\ge0$, nous remarquons que 
$$
{\d^\ell\hfill\F\d u^\ell}\exp(\e^u)=\sum_{n\ge0}{n^\ell\e^{nu}\F n!}\qquad(u\ge0)
$$
et nous d\'eduisons de la majoration $n^\ell\ll n!/(n-\ell)!\ \,(n\ge \ell)$ d'une part que 
$$
{\d^\ell\hfill\F\d u^\ell}\exp(\e^u)\ll e^{\ell u}+\sum_{n\ge\ell}{\e^{nu}\F(n-\ell)!}\ll\e^{\ell u}\exp(\e^u)
\qquad(u\ge0). 
$$
et d'autre part que 
$$
g^{(k)}(u)\ll \sum_{n_1+\cdots+kn_k=k}\Q|h^{(n_1+\cdots+n_k)}\b(\exp(\e^u)\b)\W|
\prod_{1\le\ell\le s}\e^{\ell n_\ell}\exp(n_\ell\e^u)
\qquad(u\ge0).
$$
Comme l'entier $s:=n_1+\cdots+n_k$ satisfait $1\le s\le k$
pour chaque vecteur $(n_1,\cdots,n_k)\in\ob N^k$ v\'erifiant $n_1+\cdots+kn_k=k$, il en r\'esulte que 
$$
g^{(k)}(u)\ll\e^{ku}\sum_{1\le s\le k}\B|h^{(s)}\b(\exp(\e^u)\b)\B|\exp(s\e^u)
\qquad(u\ge0).
$$
{\it A fortiori}, nous remarquons que \eqref{dfc7eq2} d\'ecoule de la majoration 
$$
h^{(s)}(t)\ll{1\F t^s(\log t)^\theta}
\qquad(t\ge\e)
\eqdef{dfc7eq4}
$$
que nous nous proposons maintenant d'\'etablir par r\'ecurrence pour  $0\le s\le n+1$. 
\bigskip


L'estimation \eqref{dfc7eq4} est satisfaite pour $s=0$ car la majoration 
\eqref{dfc7eq1} implique que 
$$
h(t)={b\F t}+{1\F t}\int_\e^tf(x)\d x\ll{1\F(\log t)^\theta}
\qquad(t\ge\e). 
$$
Supposons que \eqref{dfc7eq4} soit v\'erifi\'ee pour un entier $s\in[0,n]$. 
En d\'erivant $s+1$ fois l'identit\'e~\eqref{dfc7eq3}, nous remarquons alors que 
$$
f^{(s)}(t)=th^{(s+1)}(t)+(s+1)h^{(s)}(t)=th^{(s+1)}(t)+O\Q({1\F t^s(\log t)^\theta}\W)
\qquad(t\ge\e)
$$
et nous d\'eduisons de \eqref{dfc7eq1} que 
$$
th^{(s+1)}(t)\ll{1\F t^s(\log t)^\theta}\qquad(t\ge\e). 
$$
En particulier, l'estimation \eqref{dfc7eq4} est satisfaite pour l'entier $s+1$. Par r\'ecurrence, elle~est~donc v\'erifi\'ee pour $0\le s\le n+1$ et nous en d\'eduisons \eqref{dfc7eq2}. 
\hfill\qed
\bigskip



\lemm dfc6. Soient $n\in\ob N^*$ et $g\in\sc C^n\b([0,\infty[\b)$ une application v\'erifiant 
$$
g(u)\olop_{n-1}1
\qquad(u\ge0). 
\eqdef{dfc6eq1}
$$
ainsi que 
$$
g(u)>-c\e^u\qquad(u\ge0).\eqdef{cont}
$$  
Soit $h:\ob R\times]-c,\infty[\to\ob R$ une fonction analytique non nulle 
et soit 
$$
s:=\inf\{m+n:\partial^{(m,n)}h(0,0)\neq0\}.
$$ 
Alors, la fonction $g$ satisfait d'une part 
$$
h\Q({1\F\e^u},{g(u)\F\e^u}\W)\olop_{n-1}\e^{-su}\qquad(u\ge0) 
\eqdef{dfc6eq2a}
$$
et d'autre part 
$$
{\d^n\hfill\F\d u^n}h\Q({1\F\e^u},{g(u)\F\e^u}\W)\ll{1+\b|g^{(n)}(u)\b|\F
\e^{su}}\qquad(u\ge0). 
\eqdef{dfc6eq2b}
$$
\par
\bigskip


\dem. 
Comme $h$ est analytique en $(0,0)$ et comme $\partial^{(p,q)}h(0,0)=0\ \,(p+q<s)$, 
il existe un nombre $r>0$ et une~famille de nombre r\'eels $\{c_{p,q}\}_{p+q\ge s}$ 
tels que 
$$
h(y,z)=\sum_{p+q\ge s}c_{p,q}y^pz^q
\qquad(|y|<2r,|z|<2r). 
$$
Comme $g$ est born\'ee d'apr\`es \eqref{dfc6eq1}, nous posons 
$M:=1+\sup_{u\ge 0}|g(u)|$ et~$a:=\log(M/r)$. 
En remarquant que $|g(u)|\e^{-u}\le r$ et $\e^{-u}\le r$ pour $u\ge a$, 
nous obtenons alors que 
$$
h\Q({1\F\e^u},{g(u)\F\e^u}\W)=\sum_{p+q\ge s}c_{p,q}{g(u)^q\F\e^{(p+q)u}}
\qquad(u\ge a). 
$$
Comme l'application $u\mapsto h(\e^{-u},g(u)\e^{-u})$ est de classe $\sc C^n$ 
sur l'intervalle $[0,\infty[$ et comme 
$$
\sum_{p+q\ge s}|c_{p,q}|(1+p+2q)^nr^{p+q}<\infty,
$$ 
nous posons 
$$
\delta_{n,\ell}=\Q\{\eqalign{&1\qquad\hbox{si }n=\ell,\cr&0\qquad\hbox{si }n\neq\ell}\W.
$$ et nous d\'eduisons du th\'eor\`eme de d\'erivation terme \`a terme que les relations \eqref{dfc6eq2a}~et~\eqref{dfc6eq2b} d\'ecoulent 
de la majoration 
$$
{\d^k\hfill\F\d u^k}{g(u)^q\F\e^{(p+q)u}}\ll {1+\delta_{n,k}\b|g^{(n)}(u)\b|\F\e^{su}}(1+p+2q)^nr^{p+q}
\qquad(p+q\ge s,u\ge a)
\eqdef{dfc6eq3}
$$
que nous nous proposons maintenant d'\'etablir pour $0\le k\le n$. 
\bigskip



Pour $k\in\{0,\cdots,n\}$, nous prouvons \eqref{dfc6eq3}. 
Posant $f_m(u):=\e^{-mu}\ \,(m\in\ob N,u\ge0)$,  
il~r\'esulte en effet de la formule g\'en\'eralis\'ee de Leibniz que 
$$
{\d^k\hfill\F\d u^k}{g(u)^q\F\e^{(p+q)u}}=k!\sum_{n_0+\cdots+n_q=k}
{f_{p+q}^{(n_0)}(u)g^{(n_1)}(u)\cdots g^{(n_q)}(u)\F n_0!\cdots n_q!}
\qquad(p+q\ge s,u\ge a).
$$
Pour chaque couple $(p,q)\in\ob N^2$ satisfaisant $p+q\ge s$ et chaque vecteur 
$(n_0,\cdots,n_q)\in\ob N^{q+1}$ v\'erifiant~$n_0+\cdots+n_q=k$, nous remarquons que 
$$
\Q|f_{p+q}^{(n_0)}(u)\W|={(p+q)^{n_0}\F\e^{(p+q)u}}\le{(p+q)^{n_0}\F\e^{(p+q-s)a}}\e^{-su}
\qquad(u\ge a)
$$
et nous d\'eduisons de l'in\'egalit\'e $1+|g(u)|\le M\ \,(u\ge0)$ que  
$$
\b|g^{(n_1)}(u)\cdots g^{(n_q)}(u)\b|\le M^q
\prod_{1\le \ell\le q,n_\ell\neq0}\b|g^{(n_\ell)}(u)\b|
\qquad(u\ge a).
$$
Pour chaque couple $(p,q)\in\ob N^2$ v\'erifiant $p+q\ge s$, il suit 
$$
\Q|{\d^k\hfill\F\d u^k}{g(u)^q\F\e^{(p+q)u}}\W|\le {M^q\e^{-su}\F\e^{(p+q-s)a}}
\sum_{n_0+\cdots+n_q=k}
{k!(p+q)^{n_0}\F n_0!\cdots n_q!}\prod_{\ss1\le \ell\le q\atop\ss n_\ell\neq0}\b|g^{(n_\ell)}(u)\b|
\qquad(u\ge a).
$$
Comme $|\{\ell\in[1,q],n_\ell=n\}|\le\delta_{k,n}$ et $|\{\ell\in[1,q],n_\ell\neq0\}|\le k$, 
il r\'esulte de~\eqref{dfc6eq1} que 
$$
\prod_{1\le \ell\le q,n_\ell\neq0}g^{(n_\ell)}(u)\ll1+\delta_{k,n}\b|g^{(n)}(u)\b|
\qquad(p+q\ge s,n_0+\cdots n_q=k,u\ge a).
$$
En remarquant que la formule du multin\^ome de Newton implique que 
$$
\sum_{n_0+\cdots n_q=k}{k!(p+q)^{n_0}\F n_0!\cdots n_q!}=(1+p+2q)^k\le(1+p+2q)^n,
$$
nous obtenons alors que 
$$
{\d^k\hfill\F\d u^k}{g(u)^q\F\e^{(p+q)u}}\ll{1+\delta_{k,n}\b|g^{(n)}(u)\b|\F \e^{su}}
{M^q\F\e^{(p+q-s)a}}(1+p+2q)^n
\qquad(p+q\ge s,u\ge a).
$$
Comme $M^q\e^{(s-p-q)a}\ll r^{p+q}\ \,(p+q\ge s)$, l'estimation \eqref{dfc6eq3} est {\it a fortiori} 
satisfaite. 
\hfill\qed
\bigskip

\Secti eqloc, \'Equations approch\'ees aux diff\'erences.



\lemm L3. Soit $g\in\sc C\b([0,\infty[\b)$ une fonction satisfaisant  
$$
g(u)=o(\e^u)\qquad(u\to+\infty)
\eqdef{kmnj}
$$
telle que l'application $g^*:u\mapsto\sup_{0\le x\le u}\b|g(x)\b|$ v\'erifie 
$$
|g(u)|\le g^*\Q(u-{1\F3}\W)+O\Q({\b(1+g^*(u)\b)^2\F\e^u}+{1\F u^2}\W)\qquad(u\ge1).
\eqdef{Corcyre}
$$
Alors, l'application $g$ est born\'ee sur l'intervalle $[0,+\infty[$. 
\par
\bigskip


\dem.
Pour chaque entier $k\ge0$, nous posons $M_k:=g^*(k/3)$ et nous d\'eduisons de~l'in\'egalit\'e \eqref{Corcyre} pour $u=t/3$ que 
$$
\b|g(t/3)\b|\le M_k+O\Q({\big(1+M_{k+1}\big)^2\F\e^{k/3}}+{1\F k^2}\W)
\qquad\b(k\ge3,t\in[k,k+1]\b).
$$
La majoration pr\'ec\'edente \'etant \'egalement v\'erifi\'ee lorsque
$t\in[0,k[$, il suit
$$
M_{k+1}\le M_k+O\Q({\big(1+M_{k+1}\big)^2\F\e^{k/3}}+{1\F k^2}\W)
\qquad(k\ge3).
$$
En particulier, nous obtenons d'une part que 
$$
1+M_{k+1}+O\Q({\big(1+M_{k+1}\big)^2\F\e^{k/3}}+{1\F k^2}\W)
\le 1+
M_k
\qquad(k\ge3)
$$
et d'autre part que 
$$
1+O\Q({1+M_{k+1}\F\e^{k/3}}+{1\F k^2}\W)\le {1+M_k\F1+M_{k+1}}
\qquad(k\ge3).
$$
Comme \eqref{kmnj} implique que $M_{k+1}=o\b(\e^{k/3}\b)\ \,(k\to\infty)$, le membre de gauche de l'in\'egalit\'e pr\'ec\'edente converge vers $1$. 
A~fortiori, nous remarquons que 
$$
1\le {1+M_{k+1}\F1+M_k}\le 1+O\Q({1+M_{k+1}\F\e^{k/3}}+{1\F k^2}\W)\longrightarrow1
\qquad(k\ge3)
\eqdef{a3e3}
$$
et aussi qu'il existe un entier $K\ge3$ pour lequel  
$$
1\le {1+M_{k+1}\F1+M_k}\le\e^{1/6}\qquad(k\ge K). 
$$
Nous en d\'eduisons alors que $1+M_{k+1}\ll \e^{k/6}\ \,(k\ge3)$ et, en reportant dans \eqref{a3e3}, que 
$$
1\le {1+M_{k+1}\F1+M_k}\le 1+O\Q(\e^{-k/6}+{1\F k^2}\W)\qquad(k\ge3). \eqdef{eqp2}
$$
Comme le membre de droite  est le terme
g\'en\'eral d'un produit infini convergeant, la suite $\{M_k\}_{k=0}^\infty$ est
major\'ee. En conclusion, la fonction $g$ est born\'ee sur $[0,\infty[$.
\hfill\qed
\bigskip



\lemm L4. Soient $\alpha>0$, $\beta\in\ob R$ des constantes 
et $f\in\sc C\b([0,\infty[\b)$ une fonction telle que l'application $f^*:u\mapsto\sup_{0\le x\le u}\b|f(x)\b|$ v\'erifie  
$$
\b|f(u)\b|\le f^*(u-1/3)+O\Q({f^*(u)\F\e^{\alpha u}}+\e^{\beta u}\W)
\qquad(u\ge\lambda).  
\eqdef{tienstiens}
$$
Alors, posant $\delta:=1$ si $\beta=0$ et $\delta:=0$ sinon, la fonction $f$ satisfait 
$$
f(u)\ll 1+u^\delta\e^{\beta u}\qquad(u\ge0). \eqdef{L4eq2}
$$ 
\par
\bigskip


\dem. En posant $M_k:=f^*(k/3)$ pour $k\ge0$, nous d\'eduisons de \eqref{tienstiens} que 
$$
|f(\lambda t)|\le M_k+O\Q({M_{k+1}\F\e^{\alpha k/3}}+\e^{\beta k/3}\W)
\qquad\b(k\ge1,t\in[k/3,(k+1)/3]\b).
$$
L'identit\'e pr\'ec\'edente \'etant \'egalement v\'erifi\'ee lorsque
$t\in[0,k[$, il suit
$$
M_{k+1}\le M_k+O\Q({M_{k+1}\F\e^{\alpha k/3}}+\e^{\beta k/3}\W)
\qquad(k\ge1).
$$
En proc\'edant par r\'ecurrence sur $k$, nous obtenons alors que 
$$
M_k\le M_j+O\Bg(\sum_{j<n\le k}{M_n\F\e^{\alpha n/3}}+\sum_{j<n\le k}\e^{\beta n/3}\Bg)
\qquad(1\le j<k).
$$
La suite $\{M_k\}_{k\ge1}$ \'etant croissante, il suit 
$$
M_k\ll M_j+M_k\sum_{j<n\le k}\e^{-\alpha n/3}+\sum_{1\le n\le k}\e^{\beta n/3}
\qquad(1\le j<k).
$$
Comme $\alpha>0$, nous remarquons que $\sum_{j<n\le k}\e^{-\alpha n/3}\ll\e^{-\alpha j/3}\ \,(1\le j<k)$ 
et~aussi que 
$$
\sum_{1\le n\le k}\e^{\beta n/3}\ll 1+k^\delta\e^{\beta k/3}
\qquad(k>1). 
$$
En particulier, nous obtenons que 
$$
M_k\ll M_j+{M_k\F\e^{\alpha j/3}}+1+k^\delta\e^{\beta k/3}
\qquad(1\le j<k) 
$$
et nous d\'eduisons de l'\'egalit\'e $\lim_{j\to\infty}\e^{-\alpha j/3}=0$ 
qu'il existe un entier $j\ge1$ pour lequel
$$
M_k\ll 1+k^\delta\e^{\beta k/3}
\qquad(k>j). 
$$
{\it A fortiori}, nous remarquons d'une part que 
$$
\b|f(t/3)\b|\ll 1+t^\delta\e^{\beta t/3}\qquad(k\ge1, 0\le t\le k), 
$$
et d'autre part que l'estimation \eqref{L4eq2} est satisfaite 
\hfill\qed
\bigskip


\lemm L5. Soient $\delta>1$ et  $\Theta$ des nombres v\'erifiant \eqref{Theta1} et soit $g\in\sc C^\Theta\b([0,\infty[\b)$ 
une solution de l'\'equation approch\'ee aux diff\'erences 
$$
g(u)+g(u-\lambda )\olops_{\Theta-1}\e^{-u}\qquad(u\ge\lambda).
\eqdef{seconda}
$$
Alors, pour chaque nombre r\'eel $u$, la limite $\nu(u):=\lim_{m\to\infty}g(u+2m\lambda)$ existe et  d\'efinit une fonction $\nu\in\sc C^{K-1}(\ob R)$ v\'erifiant t \eqref{n+tn}  et 
$$
g(u)=\nu(u)+\scos_{\Theta-1}\Q(\e^{-u}\W)\qquad(u\ge0). 
\eqdef{toto}
$$
\par
\bigskip


\dem. Pour nombre $u\in\ob R$ et chaque entier $m$ v\'erifiant $-2m\lambda\le u$, nous avons
$$
\nu(u):=\lim_{n\to\infty}g(u+2n\lambda )=
g(u+2m\lambda )+\sum_{n\ge m}\B(g(u+2n\lambda+2\lambda)-g(u+2n\lambda)\B). \eqdef{serie}
$$
Comme  \eqref{seconda} induit que  $g(u+2\lambda )-g(u)\olops_{\Theta-1}\e^{-u}\ \,(u\ge0)$, 
c'est-\`a-dire que 
$$
g^{(k)}(u+2\lambda )-g^{(k)}(u)\ll \e^{-u}+u^{1_{\ob Z}(\theta)}\e^{(n+1-\theta)u}\qquad(0\le k<\Theta,u\ge0), \eqdef{majie} 
$$
et comme $\Theta+1-\theta<1$, nous observons que la limite $\nu(u)$ existe pour chaque $u\in\ob R$. 
\bigskip

Pour chaque nombre r\'eel $u$, nous substituons $u+2n\lambda$ \`a $u$ dans \eqref{seconda} pour obtenir que 
$$
g(u+2n\lambda)+g(u-\lambda+2n\lambda)\ll\e^{-\alpha n}
+n^\delta\e^{(1-\delta)2n\lambda}\qquad(n\in\ob N^*)
$$
et nous en d\'eduisons alors \eqref{n+tn} en faisant tendre $n$ vers $+\infty$. 
\bigskip

En appliquant le th\'eor\`eme de d\'erivation terme \`a terme \`a l'identit\'e \eqref{serie}, nous d\'eduisons de~la majoration \eqref{majie} 
d'une part que $\nu\in\sc C^{K-1}(\ob R)$ et d'autre part que 
$$
\nu^{(k)}(u)=g^{(k)}(u)+\sum_{n\ge0}\B(g^{(k)}(u+2n\lambda+2\lambda)-g^{(k)}(u+2n\lambda)\B)
\qquad(0\le k<\Theta,u\ge0). 
$$
\'Etant donn\'e un entier $k\in[0,\Theta[$, nous obtenons en particulier que 
$$
\nu^{(k)}(u)-g^{(k)}(u)\ll\sum_{n\ge0}\B(\e^{-u-2n\lambda}
+(u+2n\lambda)^\delta\e^{(k+1-\theta)(u+2n\lambda )}\B)
\qquad(u\ge0). 
$$
Comme la s\'erie $\sum_{n\ge0}\e^{-2n\lambda}$ converge et comme 
$(u+2n\lambda)^{\1_{\ob Z}(\theta)}\ll u^{\1_{\ob Z}(\theta)}+n^{\1_{\ob Z}(\theta)}\ \,(u\ge0,n\ge0), 
$
nous~obtenons alors que 
$$
\nu^{(k)}(u)-g^{(k)}(u)\ll\e^{-u}+\sum_{n\ge0}\b(u^{\1_{\ob Z}(\theta)}+n^{\1_{\ob Z}(\theta)}\b)\e^{(k+1-\theta)(u+2n\lambda)}
\qquad(u\ge0). 
$$
Nous d\'eduisons de la convergence des s\'eries  
$\sum_{n\ge0}\e^{2n\lambda(k+1-\theta)}$ et $\sum_{n\ge0}n\e^{2n\lambda(k+1-\theta)}$ que 
$$
\nu^{(k)}(u)-g^{(k)}(u)\ll\e^{-u}+u^{\1_{\ob Z}(\theta)}\e^{(k+1-\theta)u}\qquad(0\le k<\Theta,u\ge0)
$$   
et donc que la relation \eqref{toto} est satisfaite. 
\hfill\qed
\bigskip


\Secti D, Estimations d'un op\'erateur  int\'egral lin\'eaire.  


\lemm dl0. Soit $\{h_{m,n}\}_{(m,n)\in\ob N^2}$ la famille de fonctions holomorphes d\'efinies par  
$$
h_{m,n}(s):={\log^n(1-s/\varphi)\F n!(1-s/\varphi)^{m+1}}\qquad\b(|s|<\varphi\b)
\eqdef{hmn}
$$
et soient $(m,n,k)\in\ob N^3$ et $\epsilon_n$ 
l'application uniquement d\'etermin\'ee par 
$$
\int_0^1h_{m,n}(y){\e^x\d y\F\exp y\e^x}=
\sum_{0\le j\le k}{h_{m,n}^{(j)}(0)\F\e^{jx}}+{\epsilon_n(x)\F\e^{kx}}
\qquad(x\ge0).
\eqdef{dl0eq1}
$$
Alors, pour chaque entier $s\ge0$, la fonction $\epsilon_n$ satisfait 
$$
\epsilon_n^{(s)}(x)\ll\e^{-x}
\qquad(x\ge0). 
\eqdef{dl0eq2}
$$
\par
\bigskip




\dem. Notant $r$ l'application holomorphe implicitement d\'efinie par 
$$
h_{m,n}(s)=\sum_{0\le j\le k}h_{m,n}^{(j)}(0){s^j\F j!}+s^{k+1}r(s)\qquad\b(|s|<\varphi\b),
$$
nous d\'eduisons de \eqref{dl0eq1} que   
$$
\epsilon_n(x)=\e^{kx}\sum_{0\le j\le k}h_{m,n}^{(j)}(0)\Q(\int_0^1{y^j\F j!}{\e^x\d y\F\exp y\e^x}
-{1\F\e^{jx}}\W)+\e^{kx}\int_0^1y^{k+1}r(y){\e^x\d y\F\exp y\e^x}
\qquad(x\ge0).
$$
D'apr\`es le changement de variable $t=y\e^x$, il suit 
$$
\epsilon_n(x)=\e^{kx}\sum_{0\le j\le k}{h_{m,n}^{(j)}(0)\F\e^{jx}}\Q(\int_0^{\e^x}{t^j\F j!}\e^{-t}\d t
-1\W)+\e^{kx}\int_0^1y^{k+1}r(y){\e^x\d y\F\exp y\e^x}
\qquad(x\ge0).
$$
Pour chaque entier $j\ge0$, nous remarquons que 
$$
\int_0^{\e^x}{t^j\F j!}\e^{-t}\d t=\bg[-\sum_{0\le\ell\le j}{t^\ell\F\ell!}\e^{-t}\bg]_0^{\e^x}=
1-\sum_{0\le\ell\le j}{\e^{\ell x}\F\ell!\exp(\e^x)}\qquad(x\ge0) 
$$
et nous obtenons alors que 
$$
\epsilon_n(x)=-\e^{kx}\sum_{0\le \ell\le j\le k}{h_{m,n}^{(j)}(0)\F\e^{jx}}{\e^{\ell x}\F\ell!\exp(\e^x)}
+\int_0^1r(y){(y\e^x)^{k+1}\F\exp y\e^x}\d y
\qquad(x\ge0).
$$
Nous fixons un entier $s\ge0$, nous d\'eduisons du lemme \eqrefn{dlm1} que 
$$
{\d^s\hfill\F\d x^s}{\e^{(k+\ell-j)x}\F\exp(\e^x)}\ll\e^{-x}
\qquad(0\le\ell\le j,x\ge0)
$$
et nous remarquons que 
$$
\epsilon_n^{(s)}(x)={\d^s\F\d x^s}\int_0^1r(y){(y\e^{x})^{k+1}\F\exp y\e^x}\d y+O\Q(\e^{-x}\W)
\qquad(x\ge0).
$$
D'apr\`es le th\'eor\`eme de d\'erivation sous l'int\'egrale, il suit 
$$
\epsilon_n^{(s)}(x)=\int_0^1r(y){\partial^s\F\partial x^s}\Q({(y\e^{x})^{k+1}\F\exp y\e^x}\W)\d y+O\Q(\e^{-x}\W)
\qquad(x\ge0).
$$
En substituant $x+\log y$ \`a $x$ dans \eqref{dlm1eq1}, 
nous d\'eduisons du lemme \eqrefn{dlm1} que 
$$
{\partial^s\hfill\F\partial x^s}{(y\e^x)^{k+1}\F\exp(y\e^x)}\ll{1+(y\e^x)^s\F\exp (y\e^x)}(y\e^x)^{k+1}\qquad(x\ge0,0<y\le 1) .
$$
L'application $r$ \'etant born\'ee sur l'intervalle $[0,1]$, nous remarquons de plus que
$$
\epsilon_n^{(s)}(x)\ll\e^{-x}+\int_0^1{1+(y\e^x)^s\F\exp (y\e^x)}(y\e^x)^{k+1}\d y\qquad(x\ge0).
$$
En proc\'edant au changement de variable $t=y\e^x$, nous obtenons alors que 
$$
\epsilon_n^{(s)}(x)\ll\e^{-x}+\e^{-x}\int_0^{\e^x}{1+t^s\F\exp t}t^{k+1}\d t\ll\e^{-x}\qquad(x\ge0).
$$
En particulier, la majoration \eqref{dl0eq2} est satisfaite. 
\hfill\qed
\bigskip



\lemm dl05.  Soient $(m,\ell)\in\ob N^2$ et $g\in\sc C^\ell\b([0,\infty[\b)$. Alors,  
l'application d\'efinie par 
$$
G(x):=\varphi\e^x\int_0^xg(t){\exp(\varphi\e^t)\F\exp(\varphi\e^x)}{\d t\F\e^{mt}}
\qquad(x\ge0)
\eqdef{dl05eq1}
$$
est de classe $\sc C^\ell$ sur $[0,\infty[$ et satisfait %%%%%%%%%vrai aussi pour $C^{\ell+1}
$$
G^{(k)}(x)\ll \e^{kx}\sum_{0\le n<\max\{1,k\}}M_n(x)
\qquad(0\le k\le\ell,x\ge0), %%%%%%%%%%%%%%%  vrai aussi pour $0\le k\le \ell+1$
\eqdef{dl05eq2}
$$
la fonction $M_n$ \'etant d\'efinie pour $0\le n\le\ell$ par 
$$
M_n(x):=\sup_{0\le t\le x}\b|g^{(n)}(t)\b|
\qquad(x\ge0).
\eqdef{dl05eq3}
$$
\par
\bigskip




\dem. Il r\'esulte de l'identit\'e \eqref{dl05eq1} que la fonction $G$ appartient \`a $\sc C^\ell\b([0,\infty[\b)$ et qu'elle est solution de 
l'\'equation diff\'erentielle 
$$
G'(x)=G(x)-\varphi G(x)\e^x+\varphi g(x)\e^{(1-m)x}
\qquad(x\ge0). 
\eqdef{equaD}
$$
En majorant l'int\'egrale \eqref{dl05eq1} \`a l'aide de \eqref{dl05eq3}, nous obtenons par ailleurs que 
$$
G(x)\ll M_0(x)\e^x\int_0^x{\exp(\varphi\e^t)\F\exp(\varphi\e^x)}\d t
\qquad(x\ge0). 
$$
En particulier, il r\'esulte du changement de variable $t=x+\log(1-u\e^{-x})$ que 
$$
G(x)\ll M_0(x)\int_0^{\e^x-1}{\e^{-\varphi u}\F1-u\e^{-x}}\d u
\qquad(x\ge0).
$$
Comme $1$ est le minimum du polyn\^ome $(1-u\e^{-x})(1+u)$ lorsque $u$ varie dans $[0,\e^x-1]$, 
nous~en~d\'eduisons que
$$
G(x)\ll M_0(x)\int_0^{\e^x-1}(1+u)\e^{-\varphi u}\d u\ll M_0(x)
\qquad(x\ge0).
$$
{\it A fortiori}, la majoration \eqref{dl05eq2} est v\'erifi\'ee pour $k=0$. 
Soit~$j\in[0,\ell[$ un~entier tel~que~\eqref{dl05eq2} soit satisfaite pour $0\le k\le j$. 
En d\'erivant $j$ fois \eqref{equaD}, 
nous~observons d'une part que~$G$ est~solution de l'\'equation diff\'erentielle  
$$
G^{(j+1)}(x)=G^{(j)}(x)-\varphi\!\sum_{0\le k\le j}\!\Q({j\atop k}\W)G^{(k)}(x)\e^x
+\varphi\!\sum_{0\le k\le j}\!\Q({j\atop k}\W)g^{(k)}(x){(1-m)^{j-k}\F\e^{(m-1)x}}\!\!\!
\eqdef{equD}
$$
sur l'intervalle $[0,\infty[$ et d'autre part que 
$$
G^{(j+1)}(x)\ll\e^x\sum_{0\le k\le j}\b|G^{(k)}(x)\b|+\e^x\sum_{0\le k\le j}M_k(x)
\qquad(x\ge0). 
$$
Il r\'esulte alors de l'hypoth\`ese de r\'ecurrence que \eqref{dl05eq2} 
est satisfaite pour $k=j+1$. En~conclusion, l'estimation 
\eqref{dl05eq2} est v\'erifi\'ee. 
\hfill\qed
\bigskip




\lemm dl08. Soient $(k,\ell)\in\ob N^2$ et $f\in\sc C^{k+\ell}\b([0,\infty[\b)$. Pour chaque couple $(x,y)$ de l'ensemble $E:=[2\lambda ,\infty[\times[0,1]$, nous posons 
$$
F(x,y):=\Q\{\eqalign{
&f\b(x+\log(1-y/\varphi)\b)
\qquad\hbox{si }k=0,
\cr
&k\int_0^1(1-t)^{k-1}f^{(k)}\big(x+t\log(1-y/\varphi)\big)\d t 
\qquad\hbox{si }k\ge1.
}\W. 
\eqdef{dl08eq1}
$$
Alors, l'application $F$ est bien d\'efinie, elle est de classe $\sc C^\ell$ sur l'ensemble $E$ et elle 
satisfait 
$$
\eqalignno{
\Q|{\partial^n F\F\partial x^n}(x,y)\W|_{\strut}&\le M_{k+n}(x)
\qquad\quad\ (0\le n\le \ell,(x,y)\in E), 
&\eqdef{dl08eq2}
\cr
{\partial^{n+1}F\F\partial x^n\partial y}(x,y)&\ll M_{k+n+1}(x)
\qquad(0\le n<\ell,(x,y)\in E),  
&\eqdef{dl08eq3}
}
$$
o\`u l'on a pos\'e 
$$
M_n(x):=\sup_{0\le t\le x}\b|f^{(n)}(t)\b|
\qquad(0\le n\le k+\ell,x\ge0).\eqdef{arghfg}
$$
\par 
\bigskip





\dem. Les \'egalit\'es \eqref{Phi1} et $\lambda=\log\varphi$ impliquent d'une part que  
$$
2\lambda +\log(1-1/\varphi)=0\eqdef{Noireht}
$$
et d'autre part que
$$
0\le x+t\log(1-y/\varphi)\le x
\qquad\b(0\le t\le 1,(x,y)\in E\b). \eqdef{inter}
$$ 
Comme $f\in\sc C^{k+\ell}\b([0,\infty[\b)$, nous remarquons que l'application $F$ 
est bien d\'efinie par \eqref{dl08eq1} et nous d\'eduisons du th\'eor\`eme de d\'erivation sous l'int\'egrale 
que $F$ appartient \`a $\sc C^\ell\b(E\b)$. 
De plus, pour $0\le n\le\ell$, nous remarquons que   
$$
{\partial^n F\F\partial x^n}(x,y)=\Q\{\eqalign{
&f^{(k+n)}\b(x+\log(1-y/\varphi)\b)
\qquad\qquad\qquad\qquad\qquad\b(k=0,(x,y)\in E\b),
\cr
&k\int_0^1(1-t)^{k-1}f^{(k+n)}\big(x+t\log(1-y/\varphi)\big)\d t 
\qquad\b(k\ge1,(x,y)\in E\b).
}\W. 
\eqdef{yoi}
$$
{\it A fortiori}, l'estimation \eqref{dl08eq2} est v\'erifi\'ee d'apr\`es les identit\'es \eqref{arghfg},  \eqref{inter} et 
$$
k\int_0^1(1-t)^{k-1}\d t=1\qquad
(k\ge1).
$$ 
Pour $(x,y)\in E$ et $n\in[0,\ell[$, nous d\'erivons \eqref{yoi} par rapport \`a $y$ pour obtenir que 
$$
{\partial^{n+1} F\F\partial x^n\partial y}(x,y)=\Q\{\eqalign{
&f^{(k+n+1)}\b(x+\log(1-y/\varphi)\b)/(y-\varphi)
\qquad\qquad\qquad\qquad\qquad\ \ \hbox{ si }k=0,
\cr
&k\int_0^1t(1-t)^{k-1}f^{(k+n+1)}\big(x+t\log(1-y/\varphi)\big)\d t/(y-\varphi)
\qquad\hbox{ si }k\ge1.
}\W. 
$$
Comme $y-\varphi\le1-\varphi<0$, nous d\'eduisons alors 
\eqref{dl08eq3} 
des relations  \eqref{arghfg} et \eqref{inter}. 
\hfill\qed
\bigskip



\lemm dl09. Soit $m\in\ob N$. Sous les hypoth\`eses et notations du lemme \eqrefn{dl08}, 
la fonction $h$ implicitement~d\'efinie~par 
$$
\int_0^1h_{m,k}(y)F(x,y){\e^x\d y\F\exp y\e^x}={h(x)\F\e^{kx}}
\qquad(x\ge 2\lambda ), 
\eqdef{dl09eq1}
$$
est de classe $\sc C^\ell$ sur $[2\lambda ,\infty[$ et satisfait la majoration 
$$
\b|h^{(j)}(x)\b|\le |h_{m,k}^{(k)}(0)|M_{k+j}(x)+O\Q({\sum_{n=0}^{k+j}M_n(x)\F\e^x}\W)
\qquad(0\le j\le \ell,x\ge2\lambda ). 
\eqdef{dl09eq2} 
$$
\par
\bigskip




\dem. D'apr\`es \eqref{dl09eq1}, l'application $h$ satisfait 
$$
h(x)=\int_0^1h_{m,k}(y)F(x,y){\e^{(k+1)x}\d y\F\exp y\e^x}
\qquad(x\ge 2\lambda ). 
$$
Comme $h_{m,k}$ est holomorphe sur $\{y\in\ob C:|y|<\varphi\}$ et comme $F$ appartient \`a $\sc C^\ell(E)$, 
il r\'esulte du th\'eor\`eme de d\'erivation sous l'int\'egrale que $h$ est de classe $\sc C^\ell$ sur $[2\lambda,+\infty[$. 
De~plus, pour $0\le j\le\ell$, nous remarquons que 
$$
h^{(j)}(x)=\int_0^1h_{m,k}(y){\partial^j\F\partial x^j}\Q(F(x,y){\e^{(k+1)x}\F\exp y\e^x}\W)\d y
\qquad(x\ge 2\lambda ). 
$$
Pour $0\le a\le j$ et $b\in\ob N$, nous posons 
$$
I_{a,b}(x):=\int_0^1h_{m,k}^{(b)}(y)
{\partial^aF\F\partial x^a\hfill}(x,y){\partial^{j-a}\hfill\F\partial x^{j-a}}{\e^{(k+1-b)x}\F\exp y\e^x}\d y
\qquad(x\ge2\lambda ), 
\eqdef{dl09eq3}
$$
et nous d\'eduisons alors de la formule de Leibniz que 
$$
h^{(j)}(x)=\sum_{0\le a\le j}\Q({j\atop a}\W)I_{a,0}(x)
\qquad(x\ge 2\lambda ).
$$
En posant $J_{a,b}:=I_{a,b}-I_{a,b+1}$ nous remarquons alors que 
$$
I_{a,0}(x)=I_{a,k+1}(x)+\sum_{0\le b\le k}J_{a,b}(x)
\qquad(0\le a<j,x\ge2\lambda)
$$
et nous en d\'eduisons que 
$$
h^{(j)}(x)=I_{j,0}(x)+O\Bg(
\sum_{0\le a<j_{\strut}}\b|I_{a,k+1}(x)\b|+\sum_{\ss0\le a<j\atop\ss0\le b\le k}|J_{a,b}(x)|\Bg)
\qquad(x\ge2\lambda ).
$$
Pour d\'emontrer \eqref{dl09eq2} pour l'entier $j$, il suffit en particulier de prouver que  
$$
\b|I_{j,0}(x)\b|\le\Q|h_{m,k}^{(k)}(0)\W|M_{k+j}(x)+O\Q({M_{k+j}(x)\F\e^x}\W)
\qquad(x\ge2\lambda )
\eqdef{dl09eq4}
$$
et d'\'etablir pour chaque entier $a\in[0,j[$ les majorations  
$$
\eqalignno{
I_{a,k+1}(x)&\ll {M_{k+a}(x)\F\e^x}
\qquad\qquad\qquad\qquad\qquad\qquad\quad(x\ge2\lambda ),
&\eqdef{dl09eq5}
\cr
J_{a,b}(x)&\ll {M_{k+a}(x)^{\strut}\F\e^x}+{M_{k+a+1}(x)\F\e^x}
\qquad(0\le b\le k,x\ge2\lambda ), 
&\eqdef{dl09eq6}
}
$$
ce que nous nous proposons de faire maintenant. 
\bigskip



D'apr\`es \eqref{dl08eq2} et \eqrefn{dl09eq3}, nous avons  
$$
\b|I_{j,0}(x)\b|\le M_{k+j}(x)\int_0^1\b|h_{m,k}(y)\b|{\e^{(k+1)x}\F\exp y\e^x}\d y
\qquad(x\ge2\lambda ).
$$
Comme l'identit\'e \eqref{hmn} induit l'estimation 
$$
\b|h_{m,k}(y)\b|=\Q|h_{m,k}^{(k)}(0)\W|{y^k\F k!}+O\b(y^{k+1}\b)\qquad(0\le y\le 1),
$$
nous d\'eduisons du changement de variable $t=y\e^x$ que 
$$
\int_0^1\b|h_{m,k}(y)\b|{\e^{(k+1)x}\F\exp y\e^x}\d y=
{|h_{m,k}^{(k)}(0)|\F k!}\int_0^{\e^x}t^k\e^{-t}\d t+O\Q(
\e^{-x}\int_0^{\e^x}t^{k+1}\e^{-t}\d t\W)
\qquad(x\ge2\lambda )
$$
et nous d\'eduisons de l'identit\'e $\int_0^\infty t^n\e^{-t}\d s=n!\ \,(n\ge0)$ que 
$$
\int_0^1\b|h_{m,k}(y)\b|{\e^{(k+1)x}\F\exp y\e^x}\d y\le\b|h_{m,k}^{(k)}(0)\b|+O\b(\e^{-x}\b)
\qquad(x\ge2\lambda). 
$$ 
En particulier, l'estimation \eqref{dl09eq4} est v\'erifi\'ee. 
\bigskip





Soit $a\in[0,j[$ un entier. 
D'apr\`es \eqref{dl08eq2} et~\eqref{dl09eq3}, nous~avons 
$$
I_{a,k+1}(x)\ll M_{k+a}(x)\int_0^1
\B|h_{m,k}^{(k+1)}(y){\partial^{j-a}\hfill\F\partial x^{j-a}}{1\F\exp(y\e^x)}\B|\d y
\qquad(x\ge2\lambda ). 
$$
En substituant $x+\log y$ \`a $x$ dans \eqref{dlm1eq1}, 
nous d\'eduisons du lemme \eqrefn{dlm1} que 
$$
{\partial^{j-a}\hfill\F\partial x^{j-a}}{1\F\exp(y\e^x)}\ll{1+(y\e^x)^{j-a}\F\exp (y\e^x)}
\qquad(x\ge2\lambda ,0<y\le 1). 
$$
Comme $h_{m,k}^{(k+1)}(y)\ll1\ \,(0\le y\le 1)$, nous remarquons alors que 
$$
I_{a,k+1}(x)\ll M_{k+a}(x)\int_0^1{1+(y\e^x)^{j-a}\F\exp (y\e^x)}\d y
\qquad(x\ge0) 
$$
et nous d\'eduisons \eqref{dl09eq5} comme pr\'ec\'edemment du changement de variable $t=y\e^x$. 
\bigskip




Soit $(a,b)\in\ob [0,j[\times[0,k]$ un couple d'entiers 
et soit~$J$ l'application d\'efinie par 
$$
J(x,y):=-h_{m,k}^{(b)}(y){\partial^aF\F\partial x^a}(x,y)\qquad\b((x,y)\in E). 
\eqdef{J}
$$
En int\'egrant par partie l'int\'egrale \eqref{dl09eq3}, nous obtenons que 
$$
I_{a,b}(x)=\Q[J(x,y){\partial^{j-a}\hfill\F\partial x^{j-a}}{\e^{(k-b)x}\F\exp(y\e^x)}\W]_0^1
-\int_0^1{\partial J\F\partial y\hfill}(x,y){\partial^{j-a}\hfill\F\partial x^{j-a}}{\e^{(k-b) x}\F\exp(y\e^x)}\d y
\qquad(x\ge2\lambda ).
$$
Comme \eqref{hmn} 
induit que $h_{m,k}^{(b)}(0)=0\ \,(0\le b<k)$ et comme $(k-b)^{j-a}=0\ \,(b=k,a<j)$, 
nous~d\'eduisons de \eqref{J} que 
$$
J(x,0){\d^{j-a}\hfill\F\d x^{j-a}}\e^{(k-b)x}=0
\qquad(x\ge2\lambda )
$$
et par suite que 
$$
I_{a,b}(x)=J(x,1){\d^{j-a}\hfill\F\d x^{j-a}}{\e^{(k-b)x}\F\exp(\e^x)}
-\int_0^1{\partial J\F\partial y\hfill}(x,y){\partial^{j-a}\hfill\F\partial x^{j-a}}{\e^{(k-b)x}\F\exp(y\e^x)}\d y
\qquad(x\ge2\lambda ).
$$
Comme les relations \eqref{dl08eq2} et \eqref{J} impliquent que $J(x,1)\ll M_{k+a}(x)\ \,(x\ge2\lambda )$, il~r\'esulte alors du~lemme \eqrefn{dlm1} que 
$$
I_{a,b}(x)=O\Q({M_{k+a}(x)\F\e^x}\W)
-\int_0^1{\partial J\F\partial y\hfill}(x,y)
{\partial^{j-a}\hfill\F\partial x^{j-a}}{\e^{(k-b) x}\F\exp(y\e^x)}\d y
\qquad(x\ge2\lambda ).
$$
En d\'erivant \eqref{J} selon $y$, nous d\'eduisons des relation \eqref{dl09eq3} et 
$J_{a,b}=I_{a,b}-I_{a,b+1}$ que  
$$
J_{a,b}(x)=O\Q({M_{k+a}(x)\F\e^x}\W)+\int_0^1h_{m,k}^{(b)}(y){\partial^{a+1} F\F\partial x^a\partial y\hfill}(x,y)
{\partial^{j-a}\hfill\F\partial x^{j-a}}{\e^{(k-b) x}\F\exp(y\e^x)}\d y
\qquad(x\ge2\lambda ).
$$
Comme $h_{m,k}^{(b)}(y)\ll y^{k-b}\ \,(0\le y\le1)$ d'apr\`es \eqref{hmn}, l'identit\'e  \eqref{dl08eq3} 
implique alors que 
$$
J_{a,b}(x)\ll{M_{k+a}(x)\F\e^x}+
M_{k+a+1}(x)\int_0^1\Q|{\partial^{j-a}\hfill\F\partial x^{j-a}}{(y\e^x)^{k-b}\F\exp(y\e^x)}\W|\d y
\qquad(x\ge2\lambda ).
$$
En substituant $x+\log y$ \`a $x$ dans \eqref{dlm1eq1}, nous d\'eduisons du 
lemme \eqrefn{dlm1} que 
$$
{\partial^{j-a}\hfill\F\partial x^{j-a}}{(y\e^x)^{k-b}\F\exp(y\e^x)}
\ll{1+(y\e^x)^{j-a}\F\exp (y\e^x)}(y\e^x)^{k-b}
\qquad(x\ge0,0<y\le 1) 
$$
et nous obtenons en particulier que 
$$
J_{a,b}(x)\ll{M_{k+a}(x)\F\e^x}+
M_{k+a+1}(x)\int_0^1{1+(y\e^x)^{j-a}\F\exp (y\e^x)}(y\e^x)^{k-b}\d y
\qquad(x\ge2\lambda ).
$$
Enfin, nous en d\'eduisons \eqref{dl09eq5} 
en proc\'edant au changement de variable $t=y\e^x$. 
\hfill\qed
\bigskip




\lemm dl1. Soient $(k,\ell,m)\in\ob N^3$ et $f\in\sc C^{k+\ell}\b([0,\infty[\b)$. Alors, 
l'application $g$ d\'efinie par 
$$
\varphi\e^x\int_0^xf(t){\exp(\varphi\e^t)\F\exp(\varphi\e^x)}{\d t\F\e^{mt}}=
\sum_{0\le n\le j<k}h_{m,n}^{(j)}(0){f^{(n)}(x)\F\e^{(m+j)x}}+{g(x)\F\e^{(m+k)x}}
\qquad(x\ge0)
\eqdef{dl1eq1}
$$
est de classe $\sc C^\ell$ sur $[0,\infty[$ et satisfait  
$$
\b|g^{(j)}(x)\b|\le\sum_{0\le n\le k}|h_{m,n}^{(k)}(0)|M_{n+j}(x)
+O\Bg({\sum_{n=0}^{k+j}M_n(x)\F\e^x}\Bg)
\qquad(0\le j\le \ell,x\ge0), 
\eqdef{dl1eq2}
$$
la fonction $M_n$ \'etant d\'efinie par \eqref{dl05eq3} pour $0\le n\le k+\ell$. 
\par 
\bigskip



\dem. Soit $F$ la fonction d\'efinie par \eqref{dl08eq1} pour $(x,y)\in E:=[2\lambda ,\infty[\times[0,1]$ 
et~soient $h,\epsilon$ les applications uniquement d\'etermin\'ees 
par \eqref{dl09eq1} et par 
$$
g(x)=\sum_{0\le n<k}h_{m,n}^{(k)}(0)f^{(n)}(x)+h(x)+\epsilon(x)
\qquad(x\ge2\lambda ). 
\eqdef{shiritai}
$$   
Notant $\epsilon_0,\cdots,\epsilon_{k-1},G$ les fonctions implicitement d\'efinies 
par \eqref{dl0eq1} pour $0\le n<k$ et par \eqref{dl05eq1}, 
nous~prouvons que 
$$
\epsilon(x)=\sum_{0\le n<k}\epsilon_n(x)f^{(n)}(x)+\varphi^2{G(x-2\lambda )\F\exp(\e^x)}\e^{(m+k)x}
\qquad(x\ge2\lambda ).
\eqdef{hontoo}
$$
En effet, il r\'esulte de la d\'efinition \eqref{dl05eq1} et de la relation \eqref{Phi1} que 
$$
G(x)=\varphi\e^x\int_{x-2\lambda }^xf(t){\exp(\varphi\e^t)\F\exp(\varphi\e^x)}{\d t\F\e^{mt}}
+\varphi^2{G(x-2\lambda )\F\exp(\e^x)}
\qquad(x\ge2\lambda ).
$$ 
En proc\'edant au changement de variable $t=x+\log(1-y/\varphi)$, il r\'esulte de \eqref{Noireht} que 
$$
G(x)={1\F\e^{mx}}\int_0^1{f\big(x+\log(1-y/\varphi)\big)\F(1-y/\varphi)^{m+1}}{\e^x\d y\F\exp(y\e^x)}
+\varphi^2{G(x-2\lambda )\F\exp(\e^x)}
\qquad(x\ge2\lambda ).
\eqdef{Lucy}
$$
Comme la formule de Taylor et l'identit\'e \eqref{dl08eq1} impliquent que 
$$
f\b(x+\log(1-y/\varphi)\b)=\!\!\sum_{0\le n<k}\!\!f^{(n)}(x){\log^n(1-y/\varphi)\F n!}
+F(x,y){\log^k(1-y/\varphi)\F k!}\quad\ \b((x,y)\in E\b),
$$
il r\'esulte de \eqref{hmn} que 
$$
{f\b(x+\log(1-y/\varphi)\b)\F(1-y/\varphi)^{m+1}}=\sum_{0\le n<k}
f^{(n)}(x)h_{m,n}(y)+F(x,y)h_{m,k}(y)\qquad\b((x,y)\in E\b).  
$$
En reportant dans \eqref{Lucy}, nous d\'eduisons alors de \eqref{dl09eq1} que 
$$
G(x)=\sum_{0\le n<k}{f^{(n)}(x)\F\e^{mx}}\int_0^1h_{m,n}(y){\e^x\d y\F\exp y\e^x}
+{h(x)\F\e^{(m+k)x}}+\varphi^2{G(x-2\lambda )\F\exp(\e^x)}
\qquad(x\ge 2\lambda ).
$$
et nous d\'eduisons de \eqref{dl0eq1} pour $0\le n<k$ que  
$$
G(x)=\sum_{\ss0\le n<k\atop\ss 0\le j\le k}h_{m,n}^{(j)}(0){f^{(n)}(x)\F\e^{(m+j)x}}
+\sum_{0\le n<k}\!\!\!{f^{(n)}(x)\epsilon_n(x)\F\e^{(m+k)x}}
+{h(x)\F\e^{(m+k)x}}
+\varphi^2{G(x-2\lambda )\F\exp(\e^x)}
\quad\ (x\ge 2\lambda ).
$$
Comme $h_{m,n}^{(j)}(0)=0\ \,(0\le j<n)$, il suit 
les identit\'es \eqref{dl05eq1} et \eqref{dl1eq1} impliquent que 
$$
g(x)=\!\sum_{0\le n<k}\!h_{m,n}^{(k)}(0)f^{(n)}(x)+\!\sum_{0\le n<k}\!f^{(n)}(x)\epsilon_n(x)
+h(x)+\varphi^2{G(x-2\lambda )\F\exp(\e^x)}\e^{(m+k)x}
\quad\ (x\ge 2\lambda ).
$$
En particulier, l'estimation \eqref{hontoo} d\'ecoule de \eqref{shiritai}. 
\bigskip




\'Etablissons maintenant la majoration \eqref{dl1eq2}. 
En d\'erivant \eqref{hontoo}, nous obtenons~que
$$
g^{(j)}(x)=\sum_{0\le n<k}h_{m,n}^{(k)}(0)f^{(n+j)}(x)+h^{(j)}(x)+\epsilon^{(j)}(x)
\qquad(0\le j\le \ell,x\ge2\lambda ). 
$$
Soit $j\in[0,\ell]$ un entier. Comme le lemme \eqrefn{dl09} implique que 
$$
\b|h^{(j)}(x)\b|\le\Q|h_{m,k}^{(k)}(0)\W|M_{k+j}(x)+O\Q({\sum_{n=0}^{k+j}M_n(x)\F\e^x}\W)
\qquad(x\ge2\lambda ), 
$$
nous d\'eduisons de \eqref{dl05eq3} que 
$$
\b|g^{(j)}(x)\b|\le\sum_{0\le n\le k}\b|h_{m,n}^{(k)}(0)\b|M_{n+j}(x)
+\b|\epsilon^{(j)}(x)\b|+O\Q({\sum_{n=0}^{k+j}M_n(x)\F\e^x}\W)
\qquad(x\ge2\lambda ). 
\eqdef{dakara}
$$
En d\'erivant $j$ fois \eqref{hontoo} 
nous d\'eduisons de la formule de Leibniz que 
$$
\epsilon^{(j)}(x)=\sum_{0\le a\le j}\Q({j\atop a}\W)E_a(x)
\qquad(x\ge2\lambda ), 
\eqdef{desukara}
$$
la fonction $E_a$ \'etant d\'efinie pour $0\le a\le j$ par 
$$
E_a(x):=\sum_{0\le n<k}\epsilon_n^{(j-a)}(x)f^{(n+a)}(x)
+\varphi^2G^{(a)}(x-2\lambda ){\d^{j-a}\hfill\F\d x^{j-a}}{\e^{(m+k)x}\F\exp(\e^x)}
\qquad(x\ge2\lambda ). 
$$
Soit $a\in[0,j]$ un entier. Comme le lemme \eqrefn{dlm1} implique que 
$$
{\d^{j-a}\hfill\F\d x^{j-a}}{\e^{(m+k)x}\F\exp(\e^x)}\ll
{\e^{(m+k+j-a)x}\F\exp(\e^x)}
\qquad(x\ge0), 
$$
il r\'esulte de la minoration $\exp(\e^x)\gg\exp^{(m+k+j+1)x}\ \,(x\ge0)$ que 
$$
E_a(x)\ll\sum_{0\le n<k}\b|\epsilon_n^{(j-a)}(x)f^{(n+a)}(x)
\b|+{\b|G^{(a)}(x-2\lambda )\b|\F\e^{(a+1)x}}
\qquad(x\ge2\lambda ). 
$$
En appliquant le lemme \eqrefn{dl0} aux fonctions $\epsilon_n\ \,(0\le n<k)$, 
nous d\'eduisons de \eqref{dl0eq2} que 
$$
\epsilon_n^{(j-a)}(x)\ll\e^{-x}\qquad(0\le n<k,x\ge0)
$$ 
et nous d\'eduisons de \eqref{dl05eq3} que 
$$
E_a(x)\ll\e^{-x}\sum_{0\le n<k}M_{n+a}(x)+{\b|G^{(a)}(x-2\lambda )\b|\F\e^{(a+1)x}}
\qquad(x\ge2\lambda ). 
$$
Comme le lemme \eqrefn{dl05} implique que $G^{(a)}(x)\ll\e^{ax}\sum_{n=0}^aM_n(x)\ \,(x\ge0)$, 
il suit 
$$
E_a(x)\ll{\sum_{n=0}^{k+a}M_n(x)\F\e^x}
\qquad(0\le a\le j,x\ge2\lambda ). 
$$
En reportant dans \eqref{desukara}, nous obtenons alors que 
$$
\epsilon^{(j)}(x)\ll{\sum_{n=0}^{k+j}M_n(x)\F\e^x}\qquad(x\ge2\lambda )
$$ 
et nous d\'eduisons de \eqref{dakara} que 
$$
\b|g^{(j)}(x)\b|\le\sum_{0\le n\le k}\b|b_{m,n,k}\b|M_{n+j}(x)
+O\Q({\sum_{n=0}^{k+j}M_n(x)\F\e^x}\W)
\qquad(0\le j\le \ell,x\ge2\lambda ). 
$$
Cette estimation \'etant aussi v\'erifi\'ee pour $0\le x\le 2\lambda $ 
d'apr\`es \eqref{dl1eq1}, 
nous en d\'eduisons la majoration \eqref{dl1eq2}. 
\hfill\qed
\bigskip



\lemm dl15. Soient $\theta>1$, $\Theta$ des nombres v\'erifiant \eqref{Theta1}, soient $k\ge1$, $\ell\in\ob N$ des entiers,  
soit  $g\in\sc C^{k+\ell+1}\b([0,\infty[\b)$ une application satisfaisant
$$
g(x)\olops_{\ell+k+1}1\qquad(x\ge0)\eqdef{majg1}
$$
et soit $f\in\sc C^{k+\ell}\b([0,\infty[\b)$ une fonction v\'erifiant 
$$
f(x)=g(x)+\scos_{\ell+k}(\e^{-x})\qquad(x\ge0). \eqdef{majg2}
$$
Alors, pour chaque entier $m\ge0$, nous avons   
$$
\varphi\e^x\int_0^xf(t){\exp(\varphi\e^t)\F\exp(\varphi\e^x)}{\d t\F\e^{mt}}=
\sum_{0\le n\le j<k}h_{m,n}^{(j)}(0){f^{(n)}(x)\F\e^{(m+j)x}}+\scos_{\ell+1}\Q(\e^{-(m+k)x}\W)
\qquad(x\ge0).\eqdef{fdsq}
$$
\par
\bigskip




\dem. Nous fixons un entier $m\ge0$ et nous appliquons le lemme \eqrefn{dl1} \`a la fonction~$g$ 
pour le choix $(k^*,\ell^*,m^*)=(k,\ell+1,m)$, nous d\'eduisons des relations \eqref{dl1eq2} et \eqref{majg1} que 
la fonction~$r_1$ implicitement d\'efinie par 
$$
\varphi\e^x\int_0^xg(t){\exp(\varphi\e^t)\F\exp(\varphi\e^x)}{\d t\F\e^{mt}}=
\sum_{0\le n\le j<k}h_{m,n}^{(j)}(0){g^{(n)}(x)\F\e^{(m+j)x}}+{r_1(x)\F\e^{(m+k)x}}
\qquad(x\ge0).
$$
appartient \`a l'espace $\sc C^{\ell+1}\b([0,+\infty[\b)$ et satisfait $r_1(x)\olops_{\ell+1}1\ \,(x\ge0)$. 
{\it A fortiori}, nous~obtenons~d'une part que 
$$
{r_1(x)\F \e^{(m+k)x}}\olops_{\ell+1}\e^{-(m+k)x}\qquad(x\ge0)
$$
et d'autre part que 
$$
\varphi\e^x\int_0^xg(t){\exp(\varphi\e^t)\F\exp(\varphi\e^x)}{\d t\F\e^{mt}}=
\sum_{0\le n\le j<k}h_{m,n}^{(j)}(0){g^{(n)}(x)\F\e^{(m+j)x}}+\scos_{\ell+1}\Q(\e^{-(m+k)x}\W)
\qquad(x\ge0). \eqdef{gaga}
$$
En appliquant le lemme \eqrefn{dl1} \`a la fonction $f-g$ 
pour le choix $(k^*,\ell^*,m^*)=(k-1,\ell+1,m)$, 
nous d\'eduisons de m\^eme de \eqref{majg2} que la fonction $r_2$ implicitement d\'efinie par 
$$
\varphi\e^x\int_0^x\!\!\!\b(f(t)-g(t)\b){\exp(\varphi\e^t)\F\exp(\varphi\e^x)}{\d t\F\e^{mt}}=
\!\!\!\sum_{0\le n\le j<k-1}\!\!\!h_{m,n}^{(j)}(0){f^{(n)}(x)-g^{(n)}(x)\F\e^{(m+j)x}}+{r_2(x)\F\e^{(m+k)x}}
\quad(x\ge0)
$$
appartient \`a l'espace $\sc C^{\ell+1}\b([0,\infty[\b)$ et satisfait la majoration $r_2(x)\olops_{\ell+1}1\ \,(x\ge0)$. 
En~proc\'edant comme pr\'ec\'edemment, pour $x\ge0$, nous obtenons alors que 
$$
\varphi\e^x\!\int_0^x\!\!\b(f(t)-g(t)\b){\exp(\varphi\e^t)\F\exp(\varphi\e^x)}{\d t\F\e^{mt}}=
\!\!\!\sum_{0\le n\le j<k-1}\!\!\!h_{m,n}^{(j)}(0){f^{(n)}(x)-g^{(n)}(x)\F\e^{(m+j)x}}+\scos_{\ell+1}\Q(\e^{-(m+k)x}\W).
$$
En remarquant que la majoration \eqref{majg2} implique d'une part que
$$
f^{(n)}(x)-g^{(n)}(x)\olops_{\ell+1}\e^{-x}\qquad(0\le n\le k-1,x\ge0)
$$
et donc que 
$$
\sum_{0\le n\le k-1}h_{m,n}^{(k-1)}(0){f^{(n)}(x)-g^{(n)}(x)\F\e^{(m+k-1)x}}\olops_{\ell+1}\e^{-(m+k)x}\qquad(x\ge0), 
$$
pour $x\ge0$, nous en d\'eduisons que 
$$
\varphi\e^x\!\int_0^x\!\b(f(t)-g(t)\b){\exp(\varphi\e^t)\F\exp(\varphi\e^x)}{\d t\F\e^{mt}}=
\!\!\!\sum_{0\le n\le j<k}\!\!\!h_{m,n}^{(j)}(0){f^{(n)}(x)-g^{(n)}(x)\F\e^{(m+j)x}}+\scos_{\ell+1}\Q(\e^{-(m+k)x}\W).
$$
En ajoutant cette estimation \`a \eqref{gaga}, nous obtenons enfin la relation \eqref{fdsq}. 
\hfill\qed
\bigskip



\lemm dl2. Soient  $n\in\ob N^*$ et $g\in\sc C^n\b([0,\infty[\b)$ 
une fonction la majoration 
$$
g(u)\olop_{n-1}1\qquad(u\ge0).
\eqdef{dl2eq1}
$$ 
ainsi que la relation 
$$
\Q(c+{g(u)\F\e^u}\W)\exp\Q({\e^u\F\varphi}\W)\ge\e\qquad(u\ge0). 
\eqdef{changer}
$$
Alors les applications $G$, $M_n$ et $\phi$ d\'efinies par \eqref{dl05eq1}, \eqref{dl05eq3} et 
$$
\phi(u):=u-\lambda+\log\Q(1+{\varphi\F\e^u}\log\Q(c+{g(u)\F\e^u}\W)\W)
\qquad(u\ge0)
\eqdef{dl2eq2}
$$
v\'erifient l'estimation 
$$
{\d^n\hfill\F\d u^n}G\b(\phi(u)\b)=G^{(n)}\b(\phi(u)\b)+O\Q({1+M_n(u)\F\e^u}\W)
\qquad(u\ge0).
\eqdef{dl2eq3}
$$
\par
\bigskip



\dem. Comme l'application $g\in\sc C^n\b([0,\infty[\b)$ v\'erifie \eqref{changer}, nous observons que la fonction~$\phi$ est bien d\'efinie par l'identit\'e \eqref{dl2eq2}, qu'elle est de classe $\sc C^n$~sur~$[0,+\infty[$, 
en~tant~que~compos\'ee d'applications de classe $\sc C^n$, et qu'elle satisfait 
$$
\phi(u)\ge0\qquad(u\ge\lambda).
$$
Pour prouver le lemme \eqrefn{dl2}, nous proc\'edons de la fa\c{c}on suivante :  
pour chaque entier $n\ge1$, nous \'etablissons l'estimation  
$$
{\d^n\hfill\F\d u^n}G\b(\phi(u)\b)=G^{(n)}\b(\phi(u)\b)
+O\Q({1+M_n(u)+\b|G^{(n)}\b(\phi(u)\b)\b|\F\e^u}\W)
\qquad(u\ge0),
\eqdef{dl2eq4}
$$
en consid\'erant les deux cas selon lesquels $n=1$ ou $n\ge2$. Puis, nous prouvons que  
$$
G^{(n)}\b(\phi(u)\b)\ll 1+M_n(u)\qquad(u\ge \lambda )
\eqdef{eoui}
$$
pour en d\'eduire la relation \eqref{dl2eq3}. 
\bigskip


Supposons que $n=1$ et prouvons l'estimation \eqref{dl2eq4}.  Nous observons que 
$$
{\d^n\hfill\F\d u^n}G\b(\phi(u)\b)=\phi'(u)G'\b(\phi(u)\b)
\qquad(u\ge0).
$$
\'Etant donn\'ee $h:(x,y)\mapsto\log\b(1+\varphi^{-1}x\log(c+y)\b)-\varphi^{-1}x\log c$, 
il r\'esulte de \eqref{dl2eq2} que 
$$
\phi(u)=u-\lambda+{\log c\F \varphi}\e^{-u}+h\Q({1\F\e^u},{g(u)\F\e^u}\W)\qquad(u\ge0)\eqdef{Arch1}
$$
et nous remarquons, d'apr\`es \eqref{dl2eq1} et \eqref{changer}, que nous pouvons appliquer le lemme~\eqrefn{dfc6}
au couple d'entiers $(n^*,s^*)=(n,2)$ et aux~fonctions $g$ et $h$ pour d\'eduire de \eqref{dfc6eq2b} que 
$$
\phi'(u)=1-{\log c\F\varphi}\e^{-u}+O\Q({1+\b|g'(u)\b|\F\e^{2u}}\W)=1+O\Q({1\F\e^u}+{M_1(u)\F\e^{2u}}\W)
\qquad(u\ge0).  
\eqdef{dl2eq5}
$$
En particulier, nous obtenons que 
$$
{\d^n\hfill\F\d u^n}G\b(\phi(u)\b)=G'\b(\phi(u)\b)
+O\Q({\b|G'\b(\phi(u)\b)\b|\F\e^u}+{M_1(u)\b|G'\b(\phi(u)\b)\b|\F\e^{2u}}\W)
\qquad(u\ge0).
$$
La fonction $g$ \'etant born\'ee sur l'intervalle $[0,+\infty[$, nous d\'eduisons du lemme \eqrefn{dl05} que 
$$
G'(u)\ll\e^u\qquad(u\ge0)
$$
et nous d\'eduisons de l'identit\'e \eqref{dl2eq2} que 
$$
\phi(u)=u-\lambda+O(\e^{-u})
\qquad(u\ge0).
\eqdef{estim}
$$
{\it A fortiori}, nous remarquons d'une part que 
$$
G'\b(\phi(u)\b)\ll\e^u
\qquad(u\ge0)
$$ 
et d'autre part que 
$$
{\d^n\hfill\F\d u^n}G\b(\phi(u)\b)=G'\b(\phi(u)\b)
+O\Q({\b|G'\b(\phi(u)\b)\b|\F\e^u}+{M_1(u)\F\e^u}\W)
\qquad(u\ge0).
$$
En particulier, l'estimation \eqref{dl2eq4} est satisfaite si $n=1$. 
\bigskip


Supposons maintenant que $n\ge2$ et prouvons l'estimation \eqref{dl2eq4}. 
D'apr\`es le th\'eor\`eme de~d\'erivation des~fonctions compos\'ees, nous avons 
$$
{\d^n\hfill\F\d u^n}G\b(\phi(u)\b)=n!\!\!\sum_{k_1+\cdots+nk_n=n}\!\!
{G^{(k_1+\cdots+k_n)}\b(\phi(u)\b)\F 1!^{k_1}\cdots n!^{k_n}}
\prod_{1\le \ell\le n}{\phi^{(\ell)}(u)^{k_\ell}\F k_\ell!}
\qquad(u\ge0). 
$$
La fonction $G:[0,+\infty[\to\ob R$ \'etant d\'efinie par \eqref{dl05eq1}, %%=\eqref{defga} 
nous observons qu'elle v\'erifie~\eqref{dl1eq1} pour le triplet d'entiers $(k,\ell,m)=(0,n,0)$ et pour les fonctions $f^*=g$ et $g^*=G$. 
Comme~$g\in\sc C^n\b([0,+\infty[\b)$, nous d\'eduisons alors lemme \eqrefn{dl1} que $G\in\sc C^n\b([0,+\infty[\b)$. 
De~plus, comme les fonctions $M_0,\cdots, M_n$ sont d\'efinies par \eqref{dl05eq3} et comme $g$~v\'erifie~\eqref{dl2eq1}, 
nous~d\'eduisons de la majoration \eqref{dl1eq2} que 
$$
G(u)\olop_{n-1}1\qquad(u\ge0).\eqdef{majGd}
$$
Pour tout \'el\'ement $k=(k_1,\cdots,k_n)$ de $E:=\{k\in\ob N^n:k_1\neq n, k_n\neq1,k_1+\cdots+nk_n=n\}$, 
nous remarquons que $k_1+k_2+\cdots+k_n\le n-1$ et nous en d\'eduisons alors que 
$$
{\d^n\hfill\F\d u^n}G\b(\phi(u)\b)=\phi'(u)^nG^{(n)}\b(\phi(u)\b)+\phi^{(n)}(u)G'\b(\phi(u)\b)
+O\Bg(\sum_{k\in E}\ \prod_{\ell=1}^n\b|\phi^{(\ell)}(u)\b|^{k_\ell}\Bg)
\quad\ (u\ge0).
$$
D'apr\`es \eqref{dl2eq1} et \eqref{changer}, nous pouvons appliquer le lemme~\eqrefn{dfc6} aux~fonctions $g$ et $h$
pour le~couple d'entiers $(n^*,s)=(n,1)$. {\it A fortiori}, il r\'esulte de \eqref{dfc6eq2a} et \eqref{Arch1} d'une part que 
$$
\phi^{(\ell)}(u)=(-1)^{\ell}{\log c\F\varphi}\e^{-u}+{\d^\ell\F\d u^\ell}h\Q({1\F\e^u},{g(u)\F\e^u}\W)\ll\e^{-u}
\qquad(2\le\ell<n,u\ge0).
$$
et d'autre part que 
$$
\phi'(u)=1-{\log c\F\varphi}\e^{-u}+{\d\F\d u}h\Q({1\F\e^u},{g(u)\F\e^u}\W)=1+O\b(\e^{-u}\b)\qquad(u\ge0) 
$$ 
Comme chaque \'el\'ement $k=(k_1,\cdots,k_n)$ de $E$ satisfait $k_n=0$ et $k_2+\cdots+k_{n-1}\ge1$, 
nous en d\'eduisons que 
$$
{\d^n\hfill\F\d u^n}G\b(\phi(u)\b)=\phi'(u)^nG^{(n)}\b(\phi(u)\b)+\phi^{(n)}(u)G'\b(\phi(u)\b)
+O\Bg(\e^{-u}\sum_{k\in E}\b|\phi'(u)\b|^{k_1}\Bg)
\qquad(u\ge0)
$$
puis que 
$$
{\d^n\hfill\F\d u^n}G\b(\phi(u)\b)=G^{(n)}\b(\phi(u)\b)+\phi^{(n)}(u)G'\b(\phi(u)\b)
+O\Q({1+\b|G^{(n)}\b(\phi(u)\b)\b|\F\e^u}\W)
\qquad(u\ge0).
$$
En particulier, il r\'esulte de \eqref{majGd} que 
$$
{\d^n\hfill\F\d u^n}G\b(\phi(u)\b)=G^{(n)}\b(\phi(u)\b)
+O\Q(\b|\phi^{(n)}(u)\b|+{1+\b|G^{(n)}\b(\phi(u)\b)\b|\F\e^u}\W)
\qquad(u\ge0).
$$
De plus, en appliquant comme pr\'ec\'edemment le lemme \eqrefn{dfc6} pour $(n^*,s)=(n,1)$ 
aux fonctions $g$~et~$h$, nous d\'eduisons des relations  \eqref{dfc6eq2b} et \eqref{dl05eq3} d'une part que 
$$
\phi^{(n)}(u)=(-1)^n{\log c\F\varphi}\e^{-u}+{\d^n\F\d u^n}h\Q({1\F\e^u},{g(u)\F\e^u}\W)\ll{1+\b|g^{(n)}(u)\b|\F\e^u}\ll{1+M_n(u)\F\e^u}\qquad(u\ge0), 
$$
et d'autre part que l'estimation \eqref{dl2eq4} est satisfaite si $n\ge2$. 
\bigskip



L'estimation \eqref{dl2eq4} \'etant \'etablie, prouvons maintenant les relations \eqref{dl2eq3} et \eqref{eoui}. 
Comme les~fonctions $M_0,\cdots, M_n$ sont d\'efinies par \eqref{dl05eq3}, 
nous~appliquons le lemme \eqrefn{dl1} aux fonctions $f^*=g$ et $g^*=G$ pour les entiers $(k,\ell,m)=(0,n,0)$ 
et nous d\'eduisons des relations \eqref{dl1eq2} et \eqref{dl2eq1} et  que 
$$
G^{(n)}(x)\ll 1+M_n(x)
\qquad(x\ge0). 
\eqdef{subarashii}
$$
Comme l'estimation \eqref{estim} implique l'existence d'un nombre $a\ge \lambda $ pour lequel 
$$
\phi(u)\le u
\qquad(u\ge a)\eqdef{KelKone}
$$
et comme l'application $M_n$ est croissante, nous obtenons que 
$$
G^{(n)}\b(\phi(u)\b)\ll 1+M_n\b(\phi(u)\b)\ll 1+M_n(u)\qquad(u\ge a). 
$$
Cette majoration \'etant \'egalement  v\'erifi\'ee lorsque $\lambda \le u\le a$, nous en d\'eduisons~\eqref{eoui}. 
Enfin, en~reportant dans \eqref{dl2eq4}, nous concluons que l'estimation \eqref{dl2eq3} est satisfaite. 
\hfill\qed\null
\bigskip

\Secti B, \'Etude d'une \'equation holomorphe de plusieurs variables.

Soit $\{q_{m,n}\}_{(m,n)\in\ob N^2}$ la famille de fonctions holomorphes uniquement d\'etermin\'ee par 
$$
q_{m,n}(s):={\varphi^{-m-1}\F(1-s)^{m+1}}{\log^n(1-s)\F n!}\qquad\b((m,n)\in\ob N^2,|s|<1\b).  
\eqdef{qmn}
$$
Dans les lemmes suivants,  nous pr\'esentons quelques propri\'et\'es li\'ees \`a la fonction holomorphe $\Lambda_{K,\sc X}$ d\'efinie pour chaque entier $K\ge1$ et chaque famille $\sc X=\{\sc X_{m,n}\}_{m+n<K}$ de nombres complexes par 
$$
\Lambda_{K,\sc X}(z,s):=\e^s-c-\sum_{m+n<K}\sc X_{m,n}q_{m,n}(sz)z^{m+1}
\qquad\b(|sz|<\varphi\b). 
\eqdef{Lake1}
$$
 

\lemm  invfc0. Pour $k\ge1$, il existe un unique polyn\^ome $P_k\in\ob R[\{X_{m,n}:m+n<K\}]$, 
appartenant \`a l'anneau $\ob R\b[\{X_{m,n}:m+n<k\hbox{ et } m+1<k\}\b]$ et v\'erifiant l'identit\'e 
$$
{1\F 2\pi i}\oint\limits_{|s|=1}{1\F k!}{\partial^k\F\partial z^k}\Q(
{\partial_s\Lambda_{K,\sc X}\F\Lambda_{K,\sc X}}\W)(0,s)\e^{-\varphi s}\d s=-{\sc X_{k-1,0}\F\varphi^k}+P_k[\sc X]
\eqdef{Dimmu Borgir}
$$
pour chaque entier $K\ge k$ et pour chaque famille $\sc X=\{\sc X_{m,n}\}_{m+n<K}$ de nombres complexes, 
la~fonction $\Lambda_{K,\sc X}$ \'etant d\'efinie par \eqref{Lake1}.
\par
\bigskip

\dem.  Nous fixons $K\ge k\ge 1$, nous fixons une famille $\sc X=\{\sc X_{m,n}\}_{m+n<K}$ de~nombres complexes 
et nous d\'eduisons de \eqref{Lake1} que la fonction $\Lambda_{K,\sc X}$ 
est holomorphe au voisinage du compact $\sc K=\b\{(z,s)\in\ob C^2:z=0\hbox{ et }|s|=1\b\}$ et qu'elle satisfait 
$$
\Lambda_{K,\sc X}(0,s)=\e^s-c\qquad\b(|s|=1\b). \eqdef{Kela2}
$$ 
Comme $|\log c|\neq 1$, nous remarquons d'une part que $\e^s\neq c\ \,\b(|s|=1\b)$ et d'autre part que 
l'application $\Lambda_{K,\sc X}$ ne s'annule pas sur $\sc K$. De plus, en d\'erivant \eqref{Lake1}, nous obtenons que 
$$
\partial_s\Lambda_{K,\sc X}(z,s)=\e^s-\sum_{m+n<K}\sc X_{m,n}q_{m,n}'(sz)z^{m+2}
\qquad\b(|sz|<\varphi\b)\eqdef{Kela1}
$$
et nous en d\'eduisons d'une part que 
$$
\partial_s\Lambda_{K,\sc X}(0,s)=\e^s\qquad\b(|s|=1\b)\eqdef{Kela4}
$$
et d'autre part que l'application $\partial_s\Lambda_{K,\sc X}$ est holomorphe au voisinage du compact $\sc K$. 
{\it A~fortiori}, nous pouvons d\'eriver $k$ fois la fonction $\partial_s\Lambda_{K,\sc X}/\Lambda_{K,\sc X}$ en chaque point $(0,z)\in\sc K$ 
et la formule de Leibniz implique alors que 
$$
{\partial^k\F\partial z^k}\!\Q(
{\partial_s\Lambda_{K,\sc X}\F\Lambda_{K,\sc X}}\W)(0,s)=\!\!\!\!\sum_{k_1+k_2=k}\!{k!\F k_1!k_2!}\partial_z^{k_1}\partial_s
\Lambda_{K,\sc X}(0,s)\ \partial_z^{k_2}\Q({1\F
\Lambda_{K,\sc X}}\W)(0,s)\quad\b(|s|=1\b).\!\!\!\!\!\eqdef{Kela5}
$$
Soient $k_2\ge1$ et $s$ un nombre complexe v\'erifiant $|s|=1$. 
En appliquant le th\'eor\`eme \eqrefn{dfc0} \`a la compos\'ee de l'application $g:x\mapsto x^{-1}$ 
et de la fonction $f=\Lambda_{K,\sc X}$, nous obtenons que 
$$
\partial_z^{k_2}\Q({1\F\Lambda_{K,\sc X}}\W)(0,s)=k_2!\sum_{1\le j\le k_2}{g^{(j)}\b(\Lambda_{K,\sc X}(0,s)\b)\F j!}
\sum_{\ss n_1+\cdots+n_j=k_2\atop\ss n_1\cdots n_j\neq0}\prod_{1\le\ell\le j}{\Lambda_{K,\sc X}^{(n_\ell)}(0,s)\F n_\ell!}.
$$
Il r\'esulte alors des identit\'es \eqref{Kela2} et $g^{(j)}(x)=(-1)^j j!/x^{j+1}\ \,(j\in\ob N,x\neq0)$ que 
$$
\partial_z^{k_2}\Q({1\F\Lambda_{K,\sc X}}\W)(0,s)=k_2!\sum_{1\le j\le k_2}{(-1)^j\F(\e^s-c)^{j+1}}
\sum_{\ss n_1+\cdots+n_j=k_2\atop\ss n_1\cdots n_j\neq0}\prod_{1\le\ell\le j}{\Lambda_{K,\sc X}^{(n_\ell)}(0,s)\F n_\ell!}.\eqdef{Kela3}
$$
En d\'erivant $\ell$ fois l'identit\'e \eqref{Kela1} par rapport \`a $z$ , nous obtenons que 
$$
\partial_z^{\ell}\Lambda_{K,\sc X}(0,s)=-\sum_{m+n<K}\sc X_{m,n}\partial_z^{\ell}\B(q_{m,n}(sz)z^{m+1}\B)(0,s)
\qquad\b(\ell\ge1,|s|=1\b).
$$
et nous d\'eduisons de la formule de Leibniz que 
$$
\partial_z^{\ell}\Lambda_{K,\sc X}(0,s)=-\!\!\!\sum_{m+n<K}\!\!\sc X_{m,n}\sum_{0\le j\le\ell}{\ell!\F(\ell-j)!j!}
s^{\ell-j}q_{m,n}^{(\ell-j)}(0)\b(z^{m+1}\b)^{(j)}(0)
\quad\b(\ell\ge1,|s|=1\b).
$$
Comme la d\'efinition \eqref{qmn} de $q_{m,n}$ induit que le produit $q_{m,n}^{(\ell-j)}(0)\b(z^{m+1}\b)^{(j)}(0)$ est nul,  
sauf lorsque $m+n<\ell$ et $j=m+1$ o\`u il vaut $j!q_{m,n}^{(\ell-m-1)}(0)$, nous en d\'eduisons que   
$$
\partial_z^{\ell}\Lambda_{K,\sc X}(0,s)=-\ell!\sum_{m+n<\ell}{q_{m,n}^{(\ell-m-1)}(0)\F(\ell-m-1)!}s^{\ell-m-1}\sc X_{m,n}
\qquad\b(1\le\ell\le K,|s|=1\b).
$$
Pour $1\le j\le \le k_2\le k$, nous notons $Q_{j,k_2}$ le polyn\^ome de l'espace $\sc A[s]$ d\'efini par 
$$
Q_{j,k_2}:=\sum_{\ss n_1+\cdots+n_j=k_2\atop\ss n_1\cdots n_j\neq0}\ \prod_{1\le\ell\le j}\ 
\sum_{\ss m+n<n_\ell}{q_{m,n}^{(n_\ell-m-1)}(0)\F(n_\ell-m-1)!}s^{n_\ell-m-1}X_{m,n} 
\eqdef{Qjk}
$$
et nous d\'eduisons de l'identit\'e \eqref{Kela3} que 
$$
\partial_z^{k_2}\Q({1\F\Lambda_{K,\sc X}}\W)(0,s)=k_2!\sum_{1\le j\le k_2}{Q_{j,k_2}[\sc X](s)\F(\e^s-c)^{j+1}}\qquad\b(1\le k_2\le k,|s|=1\b).
$$
De m\^eme, pour $1\le k_1\le k$, nous notons $R_{k_1}$ le polyn\^ome de l'espace $\sc A[s]$ d\'efini par 
$$
R_{k_1}:=-\sum_{\ss m+n<k_1\atop\ss m+1<k_1}{q_{m,n}^{(k_1-m-1)}(0)\F(k_1-m-2)!}s^{k_1-m-2}\sc X_{m,n}\eqdef{Rk}
$$
et nous prouvons, en proc\'edant comme pr\'ec\'edemment, que 
$$
\partial_z^{k_1}\partial_s
\Lambda_{K,\sc X}(0,s)=k_1!R_{k_1}[\sc X](s)\qquad\b(1\le k_1\le k,|s|=1\b)
$$
En reportant ces estimations dans \eqref{Kela5}, nous d\'eduisons alors de \eqref{Kela2} et \eqref{Kela4} que 
$$
\eqalign{
{1\F k!}{\partial^k\F\partial z^k}\Q(
{\partial_s\Lambda_{K,\sc X}\F\Lambda_{K,\sc X}}\W)(0,s)=&\ \e^s\sum_{1\le j\le k}{Q_{j,k}[\sc X](s)\F(\e^s-c)^{j+1}}
+\!\!\!\sum_{\ss k_1+k_2=k\atop\ss k_1k_2\neq0}\!\!\!R_{k_1}[\sc X](s)\!\!
\sum_{1\le j\le k_2}\!\!{Q_{j,k_2}[\sc X](s)\F(\e^s-c)^{j+1}}\cr&+
{R_k[\sc X](s)\F\e^s-c}. }
$$
L'identit\'e pr\'ec\'edente \'etant satisfaite pour chaque nombre complexe $s$ v\'erifiant $|s|=1$, il~r\'esulte de la relation $\e^s\neq c\ \,\b(|s|=1\b)$ 
d'une part que l'int\'egrale \eqref{Dimmu Borgir} est bien d\'efinie et d'autre part qu'il existe 
un polyn\^ome $S\in\ob R\b[\{X_{m,n}:m+n<k\}\b]$ v\'erifiant l'identit\'e  
$$
{1/k!\F 2\pi i}\oint\limits_{|s|=1}{\partial^k\F\partial z^k}\Q(
{\partial_s\Lambda_{K,\sc X}\F\Lambda_{K,\sc X}}\W)(0,s)\e^{-\varphi s}\d s=S[\sc X]\eqdef{SX}
$$ 
pour chaque famille $\sc X=\{\sc X_{m,n}\}_{m+n<K}$ de nombres complexes. 
De plus, comme la d\'efinition~\eqref{Rk} induit que $\partial_{X_{k-1,0}}R_{k_1}=0\ \,(1\le k_1\le k)$, en d\'erivant  
l'identit\'e pr\'ec\'edente par rapport \`a l'ind\'etermin\'ee $X_{k-1,0}$, nous obtenons que 
$$
\partial_{X_{k-1,0}}{1\F k!}{\partial^k\F\partial z^k}\Q(
{\partial_s\Lambda_{K,\sc X}\F\Lambda_{K,\sc X}}\W)(0,s)=\e^s\sum_{1\le j\le k}{\partial_{X_{k-1,0}}Q_{j,k}[\sc X](s)\F(\e^s-c)^{j+1}}
\qquad\b(|s|=1\b).
$$
Comme la d\'efinition \eqref{Qjk} implique que 
$$
\partial_{X_{k-1,0}}Q_{j,k}=0\ \,(2\le j\le k)
\qquad\hbox{et}\qquad 
\partial_{X_{k-1,0}}Q_{1,k}=q_{k-1,0}(0), 
$$ 
nous obtenons que 
$$
\partial_{X_{k-1,0}}{1\F k!}{\partial^k\F\partial z^k}\Q(
{\partial_s\Lambda_{K,\sc X}\F\Lambda_{K,\sc X}}\W)(0,s)=q_{k-1,0}(0){\e^s\F(\e^s-c)^2}
\qquad\b(|s|=1\b)
$$
et nous d\'eduisons des relations $q_{k-1,0}(0)=\varphi^{-k}$ et \eqref{SX} que 
$$
\partial_{\sc X_{k-1,0}}S(\sc X)=
{1\F 2\pi i}\oint\limits_{|s|=1}\varphi^{-k}{\e^s\F(\e^s-c)^2}\e^{-\varphi s}\d s=-\varphi^{-k}. 
$$ 
{\it A fortiori}, le polyn\^ome $P_k$ implicitement d\'efini par l'identit\'e 
$$
S=-{X_{k-1,0}\F\varphi^k}+P_k
$$ 
appartient \`a l'anneau $\ob R\b[\{X_{m,n}:m+n<k\hbox{ et }m+1<k\}\b]$. 
De plus, il satisfait~\eqref{Dimmu Borgir} pour chaque famille $\sc X=\{\sc X_{m,n}\}_{m+n<K}$, d'apr\`es la relation \eqref{SX}. 
\medskip

Enfin, il existe clairement au plus un polyn\^ome de l'espace $\ob R[\{X_{m,n}:m+n<K\}]$ v\'erifiant \eqref{Dimmu Borgir} 
pour chaque famille $\sc X=\{\sc X_{m,n}\}_{m+n<K}$ de nombres complexes. 
\hfill\qed\null
\bigskip

\lemm invfc1. Soit $K\ge1$ un entier, soient $\{R_{m,n}\}_{m+n<K}$ des nombres strictement positifs, soit~$r\in]0;1[$ un nombre r\'eel v\'erifiant 
$$
r\sum_{m+n<K}R_{m,n}2^n\lambda^n\varphi^{2m+2}<{\e-c\F 2}
\eqdef{defx0}
$$
et soit $\sc K$ l'ensemble des couples $(z,\sc X)$ constitu\'es par un nombre $z$ v\'erifiant~$|z|\le r$ et par une 
famille $\sc X=\{\sc X_{m,n}\}_{m+n<K}$ de nombres complexes v\'erifiant 
$$
\b|\sc X_{m,n}\b|\le R_{m,n}\qquad(m+n<K). \eqdef{Lake2}
$$
Alors, pour chaque $(z,\sc X)\in\sc K$, la fonction $s\mapsto\Lambda_{K,\sc X}(s,z)$, 
d\'efinie par l'identit\'e \eqref{Lake1}. admet un unique z\'ero $\xi(\sc X,z)\in]-1,1[$. 
\par
\bigskip


\dem. Nous fixons $(z,\sc X)\in K$ et nous observons que l'identit\'e \eqref{Lake1} induit~que
$$
\b|\e^s-c-\Lambda_{K,\sc X}(s,z)\b|\le |z|\sum_{m+n< K}\b|\sc X_{m,n}h_{m,n}(sz)z^m\b|
\qquad\b(|s|\le 1\b). 
$$
A l'aide des in\'egalit\'es \eqref{Lake2} et $|z|\le r<1$, nous obtenons alors que 
$$
\b|\e^s-c-\Lambda_{K,\sc X}(s,z)\b|\le r\sum_{m+n< K}R_{m,n}\b|h_{m,n}(sz)\b|
\qquad\b(|s|\le 1\b). 
$$
Comme \eqref{deflambda} et \eqref{Phi1} impliquent que 
$|1-w/\varphi|\ge\varphi^{-2}$ et $\b|\log(1-w/\varphi)\b|\le 2\lambda$~pour~$|w|\le1$, 
nous~d\'eduisons de \eqref{hmn} que 
$$
\b|h_{m,n}(w)\b|\le 2^n\lambda^n\varphi^{2m+2}\qquad\b(m+n<K,|w|\le 1)
$$
et nous d\'eduisons de l'in\'egalit\'e $|z|<1$ que 
$$
\b|\e^s-c-\Lambda_{K,\sc X}(s,z)\b|\le r\sum_{m+n< K}R_{m,n}2^n\lambda^n\varphi^{2m+2}
\qquad\b(|s|\le 1\b). 
$$
D'apr\`es \eqref{defx0}, il suit alors que  
$$
\b|\e^s-c-\Lambda_{K,\sc X}(s,z)\b|\le{\e-c\F2}\qquad\b(|s|\le1\b). 
$$
Nous d\'eduisons de la majoration $|\e^s-c|\ge\e-c\ \,\b(|s|=1\b)$, d'une part que 
$$
\b|\Lambda_{K,\sc X}(s,z)\b|\ge {\e-c\F 2}\qquad\b((\sc X,z)\in\sc K,|s|=1\b)
\eqdef{yoi}
$$ 
et d'autre part que 
$$
\b|\e^s-c-\Lambda_{K,\sc X}(s,z)\b|<|\e^s-c|\qquad\b(|s|=1\b). 
$$ 
En remarquant que $s\mapsto \e^s-c$ admet un~unique~z\'ero 
dans $D:=\{s\in\ob C:|s|<1\}$,  
nous~appliquons le th\'eor\`eme de Rouch\'e \`a la fonction $s\mapsto \Lambda_{K,\sc X}(s,z)$, qui est holomorphe 
sur le~domaine $\{s\in\ob C:|s|<r^{-1}\}$ d'apr\`es les relations \eqref{qmn}, \eqref{Lake1} et $|z|\le r<1$,
et~nous en d\'eduisons qu'elle admet un unique z\'ero $\xi(\sc X,z)\in D$, de multiplicit\'e $1$. 
\bigskip

Le logarithme de la relation \eqref{hmn} \'etant pris en valeur principale, nous d\'eduisons de la~d\'efinition \eqref{Lake1} que 
$$
\Lambda_{K,\sc X}\b(\ol{\xi(\sc X,z)},z\b)=\ol{\Lambda_{K,\sc X}\b(\xi(\sc X,z),z\b)}=0
$$
Ainsi, le~nombre complexe $\ol{\xi(\sc X,z)}\in D$ est~\'egalement un z\'ero de la fonction $s\mapsto \Lambda_{K,\sc X}(s,z)$. 
Nous obtenons alors que 
$\ol{\xi(\sc X,z)}=\xi(\sc X,z)$ puis que $\xi(\sc X,z)\in]-1,1[$.
\hfill\qed\null
\bigskip



\lemm invfc2. Sous les hypoth\`eses du lemme \eqrefn{invfc1}, la fonction $(\sc X,z)\mapsto \xi(\sc X,z)$ et la fonction~$r$ implicitement d\'efinie par 
$$
\e^{-\varphi\xi(\sc X,z)}={c\F\varphi}+\sum_{0<k\le K}{1/k!\F 2\pi i}\oint\limits_{|s|=1}{\partial^k\F\partial z^k}
\Q({\partial_s\Lambda_{k,\sc X}\F\Lambda_{k,\sc X}}\W)(s,0){\d s\F\e^{\varphi s}}z^k
+r(\sc X,z)z^{K+1}\quad\b((\sc X,z)\in\sc K\b)\!\!\!\!\!\!\!\!\!
\eqdef{Lake3}
$$
sont holomorphes au voisinage du compact $\sc K$. 
\par


\dem. Sous les notations de la preuve du lemme \eqrefn{invfc1}, les in\'egalit\'es \eqref{yoi} et $\e>c$ 
induisent l'existence d'un voisinage ouvert $\sc V$ du compact $\sc K$ pour lequel  
$$
\Lambda_{K,\sc X}(s,z)\neq0
\qquad\b((\sc X,z)\in\sc V,|s|=1\b). 
\eqdef{pasdez}
$$ 
Nous rappelons que la fonction m\'eromorphe $s\mapsto s\partial_s\Lambda_{K,\sc X}(s,z)/\Lambda_{K,\sc X}(s,z)$ 
admet dans~$D$ un~unique~p\^ole~$\xi(\sc X,z)$, de~r\'esidu $\xi(\sc X,z)$, et nous d\'eduisons de la relation pr\'ec\'edente 
d'une part qu'elle est holomorphe sur le bord du disque $D$ et d'autre part que 
$$
\xi(\sc X, z)={1\F 2\pi i}\oint_{|s|=1}s{\partial_s\Lambda_{K,\sc X}(s,z)\F\Lambda_{K,\sc X}(s,z)}\d s\qquad\b((\sc X,z)\in\sc K\b).
$$ 
D'apr\`es les relations \eqref{Lake1} et \eqref{pasdez}, l'application $(\sc X,z,s)\mapsto s\partial_s\Lambda_{K,\sc X}(s,z)/\Lambda_{K,\sc X}(s,z)$ 
est holomorphe en tout point de $\sc V\times\{s\in\ob C:|s|=1\}$. Nous d\'eduisons alors du th\'eor\`eme de d\'erivation 
sous l'int\'egrale que la fonction $(\sc X,z)\mapsto\xi(\sc X,z)$ est holomorphe sur l'ouvert $\sc V$. 
\bigskip


De m\^eme, pour $(\sc X,z)\in\sc K$, la fonction m\'eromorphe $s\mapsto\e^{-\varphi s} \partial_s\Lambda_{K,\sc X}(s,z)/\Lambda_{K,\sc X}(s,z)$ admet
un unique p\^ole $\xi(\sc X,z)$, de r\'esidu $\e^{-\varphi\xi(\sc X,z)}$, dans le disque $D$ et nous d\'eduisons de la relation \eqref{pasdez} d'une part qu'elle est holomorphe sur le bord de $D$ et d'autre part que 
$$
\e^{-\varphi\xi(\sc X,z)}={1\F2\pi i}\oint_{|s|=1}{\partial_s\Lambda_{K,\sc X}(s,z)\F\Lambda_{K,\sc X}(s,z)}\e^{-\varphi s}\d s
\qquad\b((\sc X,z)\in\sc K\b). 
\eqdef{celticghast}
$$
La fonction $(\sc X,z,s)\mapsto \partial_s\Lambda_{K,\sc X}(s,z)/\Lambda_{K,\sc X}(s,z)$ \'etant holomorphe 
sur $\sc V\times\{s\in\ob C:|s|=1\}$, d'apr\`es l'in\'egalit\'e \eqref{pasdez}, 
nous~notons $R$ la fonction implicitement d\'efinie par 
$$
R(\sc X,z,s):=\int_0^1{(1-t)^K\F K!}{\partial^{K+1}\F\partial z^{K+1}}\Q({ \partial_s\Lambda_{K,\sc X}\F\Lambda_{K,\sc X}}\W)(s,tz)\d t
\qquad\b(|s|=1,(\sc X,z)\in\sc V\b)\eqdef{Lake4}
$$
et nous appliquons la formule de Taylor \`a la fonction $w\mapsto \partial_s\Lambda_{K,\sc X}(s,w)/\Lambda_{K,\sc X}(s,w)$ \`a l'ordre $K+1$ 
entre $w=0$ et $w=z$ pour obtenir que 
$$
{ \partial_s\Lambda_{K,\sc X}(s,z)\F\Lambda_{K,\sc X}(s,z)}=\sum_{0\le m\le K}{\partial^m\F\partial z^m}
\Q({ \partial_s\Lambda_{K,\sc X}\F\Lambda_{K,\sc X}}\W)(s,0){z^m\F m!}
+R_s(\sc X,z)z^{K+1}\qquad\b((\sc X,z)\in\sc K,|s|=1\b). 
$$
En reportant dans \eqref{celticghast}, pour $(\sc X,z)\in\sc K$, nous obtenons alors que  
$$
\e^{-\varphi\xi(\sc X,z)}=\sum_{0\le m\le K}{1/m!\F2\pi i}\oint\limits_{|s|=1}{\partial^m\F\partial z^m}
\Q({ \partial_s\Lambda_{K,\sc X}\F\Lambda_{K,\sc X}}\W)(s,0)
{\d s\F\e^{\varphi s}}\ z^m
+{1\F2\pi i}\oint\limits_{|s|=1}R_s(\sc X,z){\d s\F\e^{\varphi s}}\ z^{K+1}. 
$$
Comme $\log c$ est l'unique racine dans le disque $\{s\in\ob C:|s|\le 1\}$ 
de la fonction enti\`ere $s\mapsto\e^s-c$, nous d\'eduisons des relations \eqref{Kela2}, \eqref{Kela4} que 
$$
{1\F2\pi i}\oint\limits_{|s|=1}{ \partial_s\Lambda_{K,\sc X}(s,0)\F\Lambda_{K,\sc X}(s,0)}
{\d s\F\e^{\varphi s}}={1\F2\pi i}\oint\limits_{|z|=1}{\e^{(1-\varphi)s}\F\e^s-c}\d s=c^{-\varphi}. 
$$
Pour $(\sc X,z)\in\sc K$, il r\'esulte en particulier de l'\'egalit\'e \eqref{Phi2} que 
$$
\e^{-\varphi\xi(\sc X,z)}={c\F\varphi}+\sum_{1\le m\le K}{1/m!\F2\pi i}\oint\limits_{|s|=1}{\partial^m\F\partial z^m}
\Q({ \partial_s\Lambda_{K,\sc X}\F\Lambda_{K,\sc X}}\W)(s,0)
{\d s\F\e^{\varphi s}}\ z^m
+{z^{K+1}\F2\pi i}\oint\limits_{|s|=1}R(\sc X,z,s){\d s\F\e^{\varphi s}} 
$$
et nous d\'eduisons de la d\'efinition \eqref{Lake3} que 
$$
r(\sc X,z)={1\F2\pi i}\oint\limits_{|s|=1}R_s(\sc X,z){\d s\F\e^{\varphi s}}\qquad\b((\sc X,z)\in\sc K\b).  
$$
La fonction $(\sc X,z,s)\mapsto \partial_s\Lambda_{K,\sc X}(s,z)/\Lambda_{K,\sc X}(s,z)$ \'etant holomorphe 
sur $\sc V\times\{s\in\ob C:|s|=1\}$, nous~d\'eduisons de \eqref{Lake4} et du th\'eor\`eme de d\'erivation sous l'int\'egrale 
que $R$ l'est aussi. De m\^eme, nous d\'eduisons de l'identit\'e pr\'ec\'edente que $r$ est holomorphe sur $\sc V$. 
\hfill\qed\null
\bigskip



\lemm invfc3. Soient $\{q_{m,n}\}_{(m,n)\in\ob N^2}$ la famille d\'efinie par~\eqref{qmn}, $\Theta\ge K\ge1$ des entiers 
et $\sc X=\{\sc X_{m,n}\}_{m+n<K}$ une famille d'applications de l'espace $\sc C^{\Theta-K}\b([0,+\infty[\b)$ v\'erifiant 
$$
\sc X_{m,n}(x)\olop_ {\Theta-K}1\qquad(m+n<K,x\ge0). \eqdef{agrajag}
$$
Alors, il existe une fonction $\xi\in\sc C^{\Theta-K}\b([0,\infty[,]-1,1[\b)$ v\'erifiant la majoration 
$$
\xi(x)\olop_{\Theta-K}1\qquad(x\ge0), \eqdef{grip2}
$$
l'estimation 
$$
\e^{\xi(x)}=c+\sum_{m+n<K}{\sc X_{m,n}(x)\F\e^{(m+1)x}}q_{m,n}
\Q({\xi(x)\F\e^x}\W)+\sco_{\Theta-K}\Q({1\F\e^{(K+1)x}}\W)\qquad(x\ge 0)\eqdef{Windyr}
$$ 
et telle que l'application $\psi$ d\'efinie par 
$$
\psi(x)=x+\lambda+\log\Q(1-{\xi(x)\F\e^x}\W)
\qquad(x\ge0)
\eqdef{psix}
$$
v\'erifie la minoration 
$$
\psi(x)\ge0
\qquad(x\ge0). 
\eqdef{Baal}
$$
\par


\dem. Pour \'etablir le lemme \eqrefn{invfc3}, nous proc\'edons de la mani\`ere suivante : nous fixons un nombre $x_0>0$ 
et nous construisons une application $\xi\in\sc C^{\Theta-K}\b([x_0,+\infty[,]-1,1[\b)$ v\'erifiant la majoration 
$$
\e^{\xi(x)}=c+\sum_{m+n<K}{\sc X_{m,n}(x)\F\e^{(m+1)x}}q_{m,n}
\Q({\xi(x)\F\e^x}\W)\qquad(x\ge x_0). \eqdef{Windyr2}
$$ 
Puis, nous prouvons qu'elle satisfait 
$$
\xi(x)\olop_{\Theta-K}1\qquad(x\ge x_0). \eqdef{grip}
$$
Enfin, nous prolongeons la fonction $\xi$ sur l'intervalle $[0,x_0[$ pour en faire 
une application de l'ensemble $\sc C^{\Theta-K}\b([0,+\infty[,]-1,1[\b)$ et de sorte que l'application $\psi$ d\'efinie par \eqref{psix} soit positive. 
Nous concluons en remarquant que les estimations \eqref{grip2} et \eqref{Windyr} d\'ecoulent des relation \eqref{Windyr2} et \eqref{grip}, 
la~fonction $q_{m,n}$ \'etant analytique sur $]-1,1[$ 
et l'application $\sc X_{m,n}$ \'etant de classe $\sc C^{\Theta-K}$ sur l'intervalle $[0,+\infty[$ pour chaque couple d'entiers $(m,n)\in\ob N^2$ v\'erifiant $m+n<K$. 
\bigskip



Pour chaque couple d'entier positif $(m,n)$ v\'erifiant $m+n<K$, nous posons 
$$
R_{m,n}:=\sup_{x\ge0}\b|\sc X_{m,n}(x)\b|
$$ 
et nous fixons un nombre $x_0\ge2\lambda$ de sorte que la quantit\'e $r:=\e^{-x_0}$ v\'erifie~\eqref{defx0}. Commen\c{c}ons par 
construire une fonction $\xi\in\sc C^{\Theta-K}\b([x_0,+\infty[,]-1,1[\b)$ v\'erifiant~\eqref{Windyr2}. 
D'apr\`es la d\'efinition de la borne sup\'erieure $R_{m,n}$, nous avons   
$$
\b|\sc X_{m,n}(x)\b|\le R_{m,n}\qquad(m+n<K,x\ge0)  \eqdef{Man}
$$
et nous remarquons d'une part que $0<r<1$ et d'autre part que 
$$
\b|\e^{-x}\b|\le r\qquad(x\ge x_0). \eqdef{Pa}
$$
Pour $x\ge x_0$, le lemme \eqrefn{invfc1} pour le choix  $z^*=\e^{-x}$ et $\sc X^*:=\sc X(x)$ implique alors l'existence 
d'un unique z\'ero $\xi(\sc X(x),\e^{-x})$ de la fonction $s\mapsto\Lambda_{K,\sc X(x)}(s,\e^{-x})$ 
dans l'intervalle $]-1;1[$. 
A~fortiori, l'application $\xi:[x_0,+\infty[\to]-1,1[$ d\'efinie par 
$$
\xi(x):=\xi\b(\sc X(x),\e^{-x}\b)\qquad(x\ge x_0) \eqdef{ArthurDent}
$$
satisfait la relation 
$$
\Lambda_{K,\sc X(x)}\b(\xi(x),\e^{-x}\b)=0\qquad(x\ge x_0)
$$
et nous d\'eduisons de la d\'efinition \eqref{Lake1} qu'elle v\'erifie l'identit\'e \eqref{Windyr2}. 
\bigskip

\'Etablissons maintenant la majoration \eqref{grip}. Nous observons que la fonction $x\mapsto \xi(x)$ 
est la compos\'ee de la fonction $\xi_1:(\sc X,z)\mapsto\xi(\sc X,z)$ et de l'application $\xi_2:x\mapsto(\sc X(x),\e^{-x}\b)$ 
et nous~d\'eduisons du lemme \eqrefn{invfc2} que la fonction $\xi_1$ est holomorphe au voisinage du compact 
$$
\sc K:=\b\{(\sc X,z):\sc X=\{\sc X_{m,n}\}_{m+n<K} \hbox{ v\'erifie } \eqref{Lake2} \hbox{ et } |z|\le r\b\}.
$$ 
Comme \eqref{Man} et \eqref{Pa} impliquent que l'image de l'intervalle $[x_0,+\infty[$ par la~fonction~$\xi_2$  
est incluse dans le compact $\sc K$ et comme~l'estimation \eqref{agrajag} induit 
que les d\'eriv\'ees de~la fonction $\xi_2$ d'ordre inf\'erieur \`a $\Theta-K$ sont born\'ees sur l'intervalle $[x_0,+\infty[$, 
nous~en~d\'eduisons la majoration \eqref{grip}. 
\bigskip

Prolongeons $\xi$ sur $[x_0,+\infty[$ pour en faire un \'el\'ement de $\sc C^{\Theta-K}\b([0,+\infty[,]-1,1[\b)$.  
Comme~$\b|\xi(x_0)\b|<1$, nous observons qu'il existe un nombre r\'eel $\epsilon\in]0,\lambda[$ pour que
$$
\sum_{0\le k\le \Theta-K}\Q|\xi^{(k)}(x_0)\W|\ {\epsilon^k\F k!}<1, \eqdef{defxia}
$$
\'Etant donn\'ee une fonction ind\'efiniment d\'erivable $\Psi:\ob R\to[0,1]$ v\'erifiant 
$$
\Psi(x)=0\qquad(x\le x_0-\epsilon)
\qquad\hbox{et}\qquad
\Psi(x)=1\qquad(x\ge x_0),
$$ 
nous prolongeons l'application $\xi$ en une fonction de classe $\sc C^{\Theta-K}$ sur $[0,\infty[$ en posant
$$
\xi(x):=\Psi(x)\sum_{0\le k\le \Theta-K}\xi^{(k)}(x_0){(x-x_0)^k\F k!}\qquad(0\le x<x_0), \eqdef{defxib}
$$
nous remarquons que 
$$
\xi(x)=0\qquad(0\le x\le x_0-\epsilon)\eqdef{Rahhh}
$$ 
et nous d\'eduisons de \eqref{defxia} que 
$$
\b|\xi(x)\b|\le \b|\Psi(x)\b|\sum_{0\le k\le \Theta-K}\b|\xi^{(k)}(x_0)\b|{\epsilon^k\F k!}<1\qquad(x_0-\epsilon\le x<x_0).  
$$
En particulier, il r\'esulte de l'in\'egalit\'e $\xi(x)\in]-1,1[\ \,(x\ge x_0)$ que l'application $\xi$ satisfait
$$
-1<\xi(x)<1\qquad(x\ge0). 
$$
{\it A fortiori}, la fonction $\xi$ ainsi construite appartient bien \`a l'ensemble $\sc C^{\Theta-K}\b([0,\infty[,]-1,1[\b)$ 
et nous d\'eduisons les majorations \eqref{grip2} et \eqref{Windyr} des relations  \eqref{Windyr2} et \eqref{grip}.
\bigskip

Enfin, prouvons que l'application $\psi$ d\'efinie par \eqref{psix} est positive. 
Comme $x_0\ge 2\lambda$ et comme $\epsilon\le\lambda$, 
nous d\'eduisons de l'identit\'e \eqref{Rahhh} d'une part que 
$$
\xi(x)=0
\qquad(0\le x\le\lambda) 
$$ 
et d'autre part que 
$$
\psi(x)=x+\lambda\ge 0\qquad(0\le x\le \lambda). 
$$
De m\^eme, nous d\'eduisons de la minoration  $\xi(x)<1\ \,(x\ge\lambda)$ et de l'\'egalit\'e $\lambda=\log\varphi$ que 
$$
\psi(x)\ge 2\lambda+\log\Q(1-1/\varphi\W)\qquad(x\ge\lambda)
$$
et nous d\'eduisons de \eqref{Noireht} que 
$$
\psi(x)\ge0\qquad(x\ge\lambda). 
$$
En conclusion, la minoration \eqref{Baal} est v\'erifi\'ee. 
\hfill\qed\null
\bigskip



\lemm invfc5. Soient $\Theta\ge K\ge1$ deux nombres entiers, $\nu_k\in\sc C^{\Theta-k}(\ob R)\quad(1\le k\le K)$ et $g_K\in\sc C^{\Theta-K}\b([0,+\infty[\b)$ 
des fonctions telles que les hypoth\`eses du lemme \eqrefn{invfc3} soient satisfaites par les applications 
$$
\sc X_{m,n}(x)=\Q\{
\eqalign{
\nu_{m+1}^{(n)}(x+\lambda)\quad\hbox{si }m\neq K-1\cr
\nu_K(x+\lambda)+{g_K(x+\lambda)\F\varphi\e^x}\quad\hbox{si }m=K-1
}\W.\quad(m+n<K,x\ge0).\eqdef{GAHH}
$$
Soient $\{q_{m,n}\}_{(m,n)\in\ob N^2}$ la famille d\'efinie par~\eqref{qmn} et $\mu_1^*,\cdots,\mu_K^*$ les fonctions d\'efinies par~\eqref{mu2}. 
Alors, il existe une fonction $\xi$ v\'erifiant les propri\'et\'es du lemme \eqrefn{invfc3} ainsi que 
$$
\varphi\e^{-\varphi\xi(x)}=c+\varphi\sum_{1\le k\le K}{\mu_k^*(x)\F\e^{kx}}
-{g_K(x+\lambda)\F\varphi^K\e^{(K+1)x}}+\sco_{\Theta-K}\Q({1\F\e^{(K+1)x}}\W)
\qquad(x\ge0). 
\eqdef{cellela}
$$
\par


\dem. Pour $\Theta\ge K\ge 1$ et pour la famille $\sc X=\{\sc X_{m,n}\}_{m+n<K}$ d\'efinie par~\eqref{GAHH}, 
nous prouvons que la fonction $\xi$ d\'efinie dans la preuve du lemme \eqrefn{invfc3} satisfait \eqref{cellela}, 
en~proc\'edant de la fa\c{c}on suivante :  pour $x\ge x_0$, nous \'etablissons l'estimation 
$$
\e^{-\varphi\xi(x)}={c\F\varphi}+\!\sum_{1\le k\le K}
{1\F2\pi i}\oint\limits_{|s|=1}{1\F k!}{\partial^k\F\partial z^k}\!\Q({\partial_s\Lambda_{k,\sc X(x)}\F \Lambda_{K,\sc X(x)}}\W)\!(s,0)
{\d s\F\e^{\varphi s}}\e^{-kx}
+\sco_{\Theta-K}\!\!\Q({1\F\e^{(K+1)x}}\W),\eqdef{trop}
$$
nous prouvons que 
$$
\mu_k^*(x)={1/k!\F2\pi i}\oint\limits_{|s|=1}{\partial^k\F\partial z^k}\Q({\partial_s\Lambda_{K,\sc X(x)}\F \Lambda_{K,\sc X(x)}}\W)(s,0){\d s\F\e^{\varphi s}}
\qquad(1\le k<K, x\in\ob R) \eqdef{Sur1}
$$
ainsi que
$$
\mu_k^*(x)={g_K(x+\lambda)\F\varphi^{K+1}\e^x}+{1/k!\F2\pi i}\oint\limits_{|s|=1}{\partial^k\F\partial z^k}
\Q({\partial_s\Lambda_{K,\sc X(x)}\F \Lambda_{K,\sc X(x)}}\W)(s,0){\d s\F\e^{\varphi s}}
\ \quad(k=K,x\ge0)\!\!\!\!\eqdef{Sur2}
$$
et nous d\'eduisons l'estimation \eqref{cellela} de ces trois relations. 
\bigskip


\'Etablissons d'abord l'estimation \eqref{trop} sous les notations de la preuve du lemme \eqrefn{invfc3}. 
D'apr\`es le lemme \eqrefn{invfc2}, l'application implicitement d\'efinie par  \eqref{Lake3} est holomorphe au voisinage du~compact $\sc K$. 
Pour chaque $x\ge x_0$, nous remarquons que le couple $\b(\sc X(x),\e^{-x}\b)$ appartient au compact $\sc K$,  
d'apr\`es \eqref{Man} et \eqref{Pa}. En  le substituant au couple $(\sc X,z)$ dans \eqref{Lake3}, nous d\'eduisons alors 
de l'identit\'e \eqref{ArthurDent} que 
$$
\e^{-\varphi\xi(x)}={c\F\varphi}+\sum_{1\le k\le K}
{1\F2\pi i}\oint\limits_{|s|=1}{1\F k!}{\partial^k\F\partial z^k}\Q({\partial_s\Lambda_{k,\sc X(x)}\F \Lambda_{K,\sc X(x)}}\W)(s,0){\d s\F\e^{\varphi s}}\e^{-kx}
+{r\b(\sc X(x),\e^{-x}\b)\F\e^{(K+1)x}}. 
$$
Nous remarquons que la fonction $x\mapsto r\b(\sc X(x),\e^{-x}\b)$ est la compos\'ee de l'application~$r$ et de la fonction $x\mapsto(\sc X(x),\e^{-x})$, 
dont les d\'eriv\'ees d'ordre inf\'erieur \`a $\Theta-K$ sont born\'ees, d'apr\`es~\eqref{agrajag}. 
Comme l'application  $r$ est holomorphe au voisinage du compact $\sc K$ et comme 
le couple $\b(\sc X(x),\e^{-x}\b)$ appartient \`a $\sc K$ pour $x\ge x_0$, nous en d\'eduisons que  
$$
r\b(\sc X(x),\e^{-x}\b)\olop_{\Theta-K}1\qquad(x\ge x_0)
$$
puis que  
$$
{r\b(\sc X(x),\e^{-x}\b)\F\e^{(K+1)x}}\olop_{\Theta-K}\e^{-(K+1)x}\qquad(x\ge x_0). 
$$
En reportant dans l'identit\'e pr\'ec\'edente, nous obtenons alors la relation \eqref{trop}.  
\bigskip



Prouvons maintenant les identit\'es \eqref{Sur1} et \eqref{Sur2}. Nous notons $\sc Z=\{\sc Z_{m,n}\}_{m+n<K}$ la famille d'applications d\'efinie par 
$$
\sc Z_{m,n}(x):=\nu_{m+1}^{(n)}(x+\lambda)\qquad(m+n<K,x\in\ob R) \eqdef{Zmn}
$$ 
et nous d\'eduisons de \eqref{Lake1} que la d\'efinition \eqref{mu2} est \'equivalente~\`a 
$$
\mu_k^*(x)={1\F 2\pi i}\oint\limits_{|s|=1}{1\F k!		}{\partial^k\F\partial x^k}\!\Q(
{\partial_s\Lambda_{K,\sc Z(x)}\F\Lambda_{K,\sc Z(x)}}\W)\!(0,s)\e^{-\varphi s}\d s\qquad(1\le k\le K,x\in\ob R). \eqdef{mu3}
$$
Pour $1\le k\le K$, nous notons $P_k$ l'unique polyn\^ome de $\ob R[\{X_{m,n}\}_{m+n<K}]$ v\'erifiant~\eqref{Dimmu Borgir} 
pour chaque famille $\sc X=\{X_{m,n}\}_{m+n<K}$ de nombres complexes et nous d\'eduisons du~lemme \eqrefn{invfc0} d'une part que $\mu_k^*$ est bien d\'efinie par \eqref{mu2}  
et d'autre part que 
$$
\mu_k^*(x)=-{\sc Z_{k-1,0}(x)\F\varphi^k}+P_k\b(\sc Z(x)\b)\qquad(1\le k\le K,x\in\ob R).\eqdef{ArchEnemy}
$$
De plus, nous observons que l'identit\'e \eqref{Dimmu Borgir}~implique que 
$$
{1\F2\pi i}\oint\limits_{|s|=1}{1\F k!}{\partial^k\F\partial z^k}\Q({\partial_s\Lambda_{K,\sc X(x)}\F \Lambda_{K,\sc X(x)}}\W)(s,0){\d s\F\e^{\varphi s}}=
-{\sc X_{k-1,0}(x)\F\varphi^k}+P_k\b(\sc X(x)\b)
\quad\ (1\le k\le K,x\ge0). 
$$
Comme $P_k$ est un polyn\^ome de $\ob R[\{X_{m,n}\}_{m+n<K}]$ ind\'ependant de l'ind\'etermin\'ee $X_{K-1,0}$ et 
comme $\sc X_{m,n}=\sc Z_{m,n}$ pour $m+n<K$ et $m\neq K-1$, nous obtenons alors que. 
$$
P_k\b(\sc Z(x)\b)=P_k\b(\sc X(x)\b)\qquad(1\le k\le K,x\in\ob R)
$$
Pour $x\ge0$, nous en d\'eduisons que  
$$
\mu_k^*(x)={\sc X_{k-1,0}(x)\F\varphi^k}-{\sc Z_{k-1,0}(x)\F\varphi^k}+{1\F2\pi i}\oint\limits_{|s|=1}{1\F k!}{\partial^k\F\partial z^k}\Q({\partial_s\Lambda_{K,\sc X(x)}\F \Lambda_{K,\sc X(x)}}\W)(s,0){\d s\F\e^{\varphi s}}\qquad(1\le k\le K). 
$$
En remarquant que les d\'efinitions \eqref{GAHH} et \eqref{Zmn} induisent que 
$$
\sc X_{k-1,0}(x)=\Q\{\eqalign{
\sc Z_{k-1,0}(x)\qquad(1\le k<K)\cr
{g_K(x+\lambda)\F\varphi\e^x}\quad\ \qquad(k=K)}
\W.\qquad(x\ge0), 
$$
nous en d\'eduisons alors les identit\'es \eqref{Sur1} et \eqref{Sur2}. 
\bigskip


Enfin, \'etablissons l'estimation \eqref{cellela}. En reportant \eqref{Sur1} et \eqref{Sur2} dans \eqref{trop}, nous obtenons que 
$$
\e^{-\varphi\xi(x)}={c\F\varphi}+\sum_{1\le k\le K}{\mu_k^*(x)\F\e^{kx}}
-{g_K(x+\lambda)\F\varphi^{K+1}\e^{(K+1)x}}+\sco_{\Theta-K}\Q({1\F\e^{(K+1)x}}\W)
\qquad(x\ge x_0). 
$$
Comme $\nu_k\in\sc C^{\Theta-k}(\ob R)\ \,(1\le k\le K)$, nous d\'eduisons de la d\'efinition 
\eqref{Zmn} et de l'identit\'e \eqref{ArchEnemy} que les applications $\mu_k^*\ \,(1\le k\le K)$ 
appartiennent \`a l'espace $\sc C^{\Theta-K}(\ob R)$. 
Les fonctions $g_K$ et $\xi$ \'etant de classe $\sc C^{\Theta-K}$ sur $[0,+\infty[$, 
nous en d\'eduisons que l'estimation pr\'ec\'edente est aussi satisfaite sur $[0,x_0[$ et donc que la relation \eqref{cellela} est v\'erifi\'ee. 
\hfill\qed
\bigskip

\lemm invfc6. Soient $\theta>1$, soient $\Theta\ge K\ge k\ge1$ des entiers v\'erifiant l'identit\'e~\eqref{Theta1} 
et soit $h\in\sc C^{\Theta-k}\b([0,+\infty[\b)$ une application v\'erifiant 
$$
h(x)\olops_{\Theta-k}1\qquad(x\ge0). \eqdef{Magister}
$$
Soit $\{q_{m,n}\}_{(m,n)\in\ob N^2}$ la famille d\'etermin\'ee par~\eqref{qmn} et soit $\xi\in\sc C^{\Theta-K}\b([0,+\infty[,]-1,1[\b)$ 
une fonction v\'erifiant \eqref{grip2} telle que l'application $\psi$ d\'efinie par \eqref{psix} 
appartienne \`a l'ensemble $\sc C^{\Theta-K}\b([0,+\infty[,[0,+\infty[\b)$. Alors, nous avons 
$$
{h\b(\psi(x)\b)\F\e^{(k+1)\psi(x)}}=\!\!\sum_{0\le n<K-k}\!\!{h^{(n)}(x+\lambda)\F\e^{(k+1)x}}q_{k,n}\Q({\xi(x)\F\e^x}\W)+\scos_{\Theta-K}\Q(\e^{-(K+1)x}\W)
\qquad(x\ge0).\!\!\!\!\!\!\!\eqdef{Fear}
$$
\par


\dem. Pour \'etablir le lemme \eqrefn{invfc6}, nous proc\'edons de la mani\`ere suivante : \'etant~donn\'ee la fonction $L$ uniquement d\'etermin\'ee par 
$$
L(x):=\log\Q(1-{\xi(x)\F\e^x}\W)\qquad(x\ge0), \eqdef{Isle}
$$
nous montrons que l'application $H$ d\'efinie par 
$$
H(x):=(K-k)\int_0^1(1-t)^{K-k-1}h^{(K-k)}\b(x+\lambda+tL(x)\b)\d t\qquad(x\ge0) \eqdef{Ph34r}
$$
satisfait l'identit\'e 
$$
{h\b(\psi(x)\b)\F\e^{(k+1)\psi(x)}}=\!\!\!\!\sum_{0\le n<K-k}\!\!{h^{(n)}(x+\lambda)\F\e^{(k+1)x}}q_{k,n}\!\Q({\xi(x)\F\e^x}\W)
+{H(x)\F\e^{(k+1)x}}q_{k,K-k}\!\Q({\xi(x)\F\e^x}\W)\quad\ (x\ge0). \!\!\!\!\!\eqdef{IdEst}
$$
Puis, nous prouvons d'une part que 
$$
q_{k,K-k}\Q({\xi(x)\F\e^x}\W)\olop_ {\Theta-K}\e^{-(K-k)x}\qquad(x\ge0) \eqdef{Archa}
$$
et d'autre part que 
$$
H(x)\olops_{\Theta-K}1\qquad(x\ge0). \eqdef{Archb}
$$
Nous concluons en remarquant que \eqref{Fear} d\'ecoule des trois estimations pr\'ec\'edentes. 
\bigskip


Prouvons que la fonction $H$ est bien d\'efinie par \eqref{Ph34r} et qu'elle satisfait~\eqref{IdEst}. 
D'apr\`es~l'identit\'e \eqref{psix} et l'\'egalit\'e $\lambda=\log\varphi$, nous avons 
$$
{h\b(\psi(x)\b)\F\e^{(k+1)\psi(x)}}={h\b(\psi(x)\b)\F\e^{(k+1)x}}{\varphi^{-k-1}\F\Q(1-\xi(x)\e^{-x}\W)^{k+1}}\qquad(x\ge0). \eqdef{Studio}
$$
Comme $\xi\in\sc C^{\Theta-K}\b([0,+\infty[,]-1,1[\b)$, la relation \eqref{Isle} implique que l'application $L$ appartient 
\`a l'espace $\sc C^{\Theta-K}\b([0,+\infty[\b)$ et la d\'efinition \eqref{psix} induit que  
$$
\psi(x)=x+\lambda+L(x)\qquad(x\ge0).
$$ 
Comme $\psi(x)\ge0\ \,(x\ge0)$ et comme $\lambda\ge0$, nous observons que 
$$
x+\lambda+tL(x)\ge0\qquad(0\le t\le 1,x\ge0)
$$ 
et nous d\'eduisons de la relation $h\in\sc C^{K-k}\b([0,+\infty[\b)$ que $H$ est bien d\'efinie par \eqref{Ph34r}.  
En appliquant la formule de Taylor \`a la fonction $h$ entre $x+\lambda$ et $\psi(x)$ pour l'ordre $K-k$, nous obtenons de plus que 
$$
h\b(\psi(x)\b)=\sum_{0\le n<K-k}h^{(n)}(x+\lambda){L(x)^n\F n!}+H(x){L(x)^{K-k}\F(K-k)!}\qquad(x\ge0). 
$$
En reportant cette identit\'e dans \eqref{Studio}, nous d\'eduisons alors \eqref{IdEst} de \eqref{qmn} et \eqref{Isle}. 
\bigskip


\'Etablissons maintenant la majoration \eqref{Archa}. Nous d\'eduisons de \eqref{qmn} que  
l'application 
$$
p:x\mapsto {q_{k,K-k}(x)\F x^{K-k}}  
$$
est analytique sur $]-1,1[$ et nous d\'eduisons de l'in\'egalit\'e $-1<\xi(x)<1\ \,(x\ge0)$ que 
$$
q_{k,K-k}\Q({\xi(x)\F\e^x}\W)=\xi(x)^{K-k}p\Q({\xi(x)\F\e^x}\W)\e^{-(K-k)x}\qquad(x\ge0).\eqdef{Archc}
$$
Comme l'application $\xi\in\sc C^{\Theta-K}\b([0,+\infty[\b)$ v\'erifie \eqref{grip2}, nous remarquons d'une part que 
$$
{\xi(x)\F\e^x}\olop_{\Theta-K}1\qquad(x\ge0)\eqdef{taDa}
$$
et d'autre part que 
$$
p\Q({\xi(x)\F\e^x}\W)\olop_{\Theta-K}1\qquad(x\ge0). 
$$
Comme la majoration \eqref{grip2} implique \'egalement que 
$$
\xi(x)^{K-k}\olop_{\Theta-K}1\qquad(x\ge0), 
$$
nous d\'eduisons alors de l'identit\'e \eqref{Archc} que l'estimation \eqref{Archa} est satisfaite. 
\bigskip


Enfin, prouvons que $H$ satisfait \eqref{Archb} et que l'estimation \eqref{Fear} est v\'erifi\'ee. 
Nous~remarquons que l'application $q:x\mapsto \log(1-x)/x$ 
est analytique sur le segment $]-1,1[$ et nous d\'eduisons de la d\'efinition \eqref{Isle} et de l'in\'egalit\'e $-1<\xi(x)<1\ \,(x\ge0)$ que 
$$
L(x)=\xi(x)q\Q({\xi(x)\F\e^x}\W)\e^{-x}\qquad(x\ge0). 
$$
{\it A fortiori}, il r\'esulte des majorations \eqref{grip2} et \eqref{taDa} d'une part que 
$$
q\Q({\xi(x)\F\e^x}\W)\olop_{\Theta-K}1\qquad(x\ge0)
$$
et d'autre part que 
$$
L(x)\olop_{\Theta-K}\e^{-x}\qquad(x\ge0). 
$$
En particulier, nous remarquons que 
$$
x+\lambda+tL(x)=x+\lambda+\sco_{\Theta-K}(\e^{-x})\qquad(0\le t\le 1,x\ge0). 
$$
Comme la majoration \eqref{Magister} induit que
$$
h^{(K-k)}(x)\olops_{\Theta-K}1\qquad(x\ge0), 
$$
nous en d\'eduisons alors que 
$$
h^{(K-k)}(x+\lambda+tL(x))\olops_{\Theta-K}1\qquad(x\ge0).
$$
L'application $(x,t)\mapsto h^{(K-k)}(x+\lambda+tL(x))$ \'etant de classe $\sc C^{\Theta-K}$ sur $[0,+\infty[\times[0,1]$, 
nous~d\'eduisons alors de l'estimation pr\'ec\'edente et du th\'eor\`eme de d\'erivation sous l'int\'egrale et 
que la~fonction $H$, d\'efinie par \eqref{Ph34r}, satisfait \eqref{Archb}. 
En reportant les majorations \eqref{Archa} et \eqref{Archb} dans \eqref{IdEst}, nous obtenons enfin l'estimation \eqref{Fear}. 
\hfill\qed
\bigskip


\lemm invfc7. Soit $\theta>1$, soient $\Theta\ge K>k\ge0$ des nombres entiers v\'erifiant~\eqref{Theta1} 
et soit $g\in\sc C^{\Theta-k-1}\b([0,+\infty[\b)$ une application v\'erifiant 
$$
g(x)\olops_{\Theta-k-1}\e^{-x}\qquad(x\ge0). \eqdef{Magister2}
$$
Soit $\{q_{m,n}\}_{(m,n)\in\ob N^2}$ la famille d\'etermin\'ee par~\eqref{qmn} et soit $\xi\in\sc C^{\Theta-K}\b([0,+\infty[,]-1,1[\b)$ 
une fonction v\'erifiant \eqref{grip2} telle que l'application $\psi$ d\'efinie par \eqref{psix} 
appartienne \`a l'ensemble $\sc C^{\Theta-K}\b([0,+\infty[,[0,+\infty[\b)$. Alors, nous avons 
$$
{g\b(\psi(x)\b)\F\e^{(k+1)\psi(x)}}=\sum_{0\le n<K-k}{g^{(n)}(x+\lambda)\F\e^{(k+1)x}}q_{k,n}\Q({\xi(x)\F\e^x}\W)+\scos_{\Theta-K}\Q(\e^{-(K+1)x}\W)
\qquad(x\ge0).\!\!\!\!\!\!\!\eqdef{Fear2}
$$
\par

\dem. Nous prouvons que \eqref{Fear2} d\'ecoule du lemme \eqrefn{invfc6} 
et de la majoration 
$$
{g^{(\ell)}(x+\lambda)\F\e^{(k+1)x}}\!\!\!\!\!\!\sum_{\ell\le n<K-k-1}\!\!\!\Q({\ss n\atop\ss\ell}\W)q_{k+1,n}\Q({\xi(x)\F\e^x}\W)\olops_{\Theta-K}\e^{-(K+1)x}
\quad(0\le\ell<K-k,x\ge0),\!\!\!\!\!\!\!\!\!\!\eqdef{wages}
$$
puis nous \'etablissons effectivement la majoration \eqref{wages}. 
\bigskip

Supposons la majoration \eqref{wages} v\'erifi\'e et d\'eduisons l'estimation \eqref{Fear2} du lemme~\eqrefn{invfc6}. 
Nous~remarquons que la fonction $h:u\mapsto \e^ug(u)$ appartient \`a l'espace $\sc C^{\Theta-k-1}\b([0,+\infty[\b)$ 
et nous~d\'eduisons de \eqref{Magister2} qu'elle satisfait
$$
h(x)\olops_{\Theta-k-1}1\qquad(x\ge0).
$$
Nous pouvons alors lui appliquer le lemme \eqrefn{invfc6} pour les entiers $\Theta$, $K$ et $k^*=k+1$ pour obtenir que 
$$
{h\b(\psi(x)\b)\F\e^{(k+2)\psi(x)}}=\sum_{0\le n<K-k-1}{h^{(n)}(x+\lambda)\F\e^{(k+2)x}}q_{k+1,n}\Q({\xi(x)\F\e^x}\W)+\scos_{\Theta-K}\Q(\e^{-(K+1)x}\W)
\qquad(x\ge0).
$$
Comme $\{(n,\ell):0\le n<K-k-1,0\le\ell\le n\}=\{(n,\ell):0\le\ell<K-k,\ell\le n<K-k-1\}$, 
nous d\'eduisons alors de l'\'egalit\'e $\lambda=\log\varphi$ et de l'identit\'e $h(u)=\e^ug(u)\ \,(u\ge0)$ que 
$$
h^{(n)}(x+\lambda)=\varphi\e^x\sum_{0\le\ell\le n}\Q({\ss n\atop\ss\ell}\W)g^{(\ell)}(x+\lambda)\qquad(0\le n<K-k,x\ge0). 
$$
et aussi que,  pour $x\ge0$, la fonction $g$ satisfait  
$$
{g\b(\psi(x)\b)\F\e^{(k+1)\psi(x)}}=\varphi\!\sum_{0\le\ell<K-k}\!{g^{(\ell)}(x+\lambda)\F\e^{(k+1)x}}\!\!\sum_{\ell\le n<K-k-1}\!\!\Q({\ss n\atop\ss\ell}\W)q_{k+1,n}\!\Q({\xi(x)\F\e^x}\W)+\scos_{\Theta-K}\Q(\e^{-(K+1)x}\W).
$$
{\it A fortiori}, pour $x\ge0$, il r\'esulte de la majoration \eqref{wages} que
$$
{g\b(\psi(x)\b)\F\e^{(k+1)\psi(x)}}=\varphi\!\sum_{0\le\ell<K-k}\!{g^{(\ell)}(x+\lambda)\F\e^{(k+1)x}}\sum_{n\ge\ell}\Q({\ss n\atop\ss\ell}\W)q_{k+1,n}\!\Q({\xi(x)\F\e^x}\W)+\scos_{\Theta-K}\Q(\e^{-(K+1)x}\W).
$$
En remarquant que $-1<\xi(x)\e^{-x}<1\ \,(x\ge0)$ et que l'identit\'e \eqref{qmn} implique que 
$$
\varphi\sum_{n\ge\ell}\Q({\ss n\atop\ss\ell}\W)q_{k+1,n}(z)=q_{k,\ell}(z)\qquad(\ell\in\ob N,|z|<1), 
$$
nous concluons alors que l'estimation \eqref{Fear2} est satisfaite. 
\bigskip


\'Etablissons maintenant \eqref{Fear2}. D'apr\`es \eqref{qmn}, la s\'erie de fonctions holomorphes 
$$
Q:z\mapsto \sum_{n\ge K-k-1}\Q({\ss n\atop\ss\ell}\W)q_{k,n}(z)z^{k+1-K}
$$
converge uniform\'ement sur tout compact du disque $\{z\in\ob C:|z|<1\}$. {\it A fortiori}, la fonction $Q$ est 
analytique sur le segment $]-1,1[$ et satisfait  
$$
\sum_{n\ge K-k-1}\Q({\ss n\atop\ss\ell}\W)q_{k,n}\Q({\xi(x)\F\e^x}\W)=\xi(x)^{K-k-1}Q\Q({\xi(x)\F\e^x}\W)\e^{-(K-k+1)x}\qquad(x\ge0). 
$$
Comme $\xi\in\sc C^{\Theta-K}\b([0,+\infty[\b)$ satisfait \eqref{grip2} et v\'erifie {\it a fortiori} l'estimation \eqref{taDa}, 
nous~observons alors d'une part que 
$$
Q\Q({\xi(x)\F\e^x}\W)\olop_{\Theta-K}1\qquad(x\ge0)
$$ 
et d'autre part que 
$$
\sum_{n\ge K-k-1}\Q({\ss n\atop\ss\ell}\W)q_{k,n}\Q({\xi(x)\F\e^x}\W)\olop_{\Theta-K}\e^{-(K-k+1)x}\qquad(x\ge0). 
$$
Pour $0\le\ell<K-k$, il r\'esulte alors de la majoration \eqref{Magister2} que 
$$
g^{(\ell)}(x+\lambda)\olops_{\Theta-K}\e^{-x}\qquad(x\ge0)
$$
et nous en d\'eduisons que l'estimation \eqref{Fear2} est satisfaite. 
\hfill\qed
\bigskip



\Sect Theo, Preuve du th\'eor\`eme \eqrefn{T3}. 


\Secti Plan, Plan de la d\'emonstration.


La preuve du th\'eor\`eme \eqrefn{T3} comporte trois grandes \'etapes ventil\'ees dans trois~paragraphes. 
Au paragraphe \CitSec{DL0}, nous prouvons que les fonctions $f$ satisfaisant les hypoth\`eses 
du~th\'eor\`eme~\eqrefn{T3} v\'erifient l'estimation~\eqref{fsim}. Ce r\'esultat est du \`a P\'etermann \CitRef{Petermann1} et 
la preuve que nous proposons, par soucis d'autonomie,  s'inspire largement de sa d\'emonstration. 
\bigskip


Au paragraphe \CitSec{Reductio}, nous prouvons le th\'eor\`eme \eqrefn{T3} d\'ecoule du th\'eor\`eme plus~faible suivant, 
que nous d\'emontrons au paragraphe \CitSec{Theo5}. 

\theo T3faible. Soient $p\ge\e$, $\theta>1$ des nombres r\'eels et $\Theta$ l'entier d\'efini par \eqref{Theta1}. Alors, 
pour chaque fonction $f\in\sc C^{\Theta+1}\b([\e,\infty[,[\e,\infty[\b)$ v\'erifiant 
$$
f'(t)={1+\scod_\Theta\b(1/(\log t)^\theta\b)\F f\circ f(t)}\qquad(t\ge\e), 
\eqdef{Sabbat}
$$
il existe une~unique~suite $\nu_k\in\sc C^{\Theta-k}(\ob R)\ \,(1\le k\le\Theta)$
de~fonctions $2\lambda$-p\'eriodiques  v\'erifiant 
$$
f(t)=ct^{\varphi-1}+\sum_{1\le k\le\Theta}
\nu_k(\log_2t){t^{\varphi-1}\F(\log t)^k}
+O\Q({t^{\varphi-1}\F(\log t)^\theta}+\1_{\ob Z}(\theta) {t^{\varphi-1}\log_2t\F(\log t)^\theta}\W)
\qquad(t\ge p).
$$
L'application $\nu_1$ satisfait \eqref{n+tn} et, pour~$2\le k\le \Theta$, l'application $G_k$ d\'efinie par \eqref{IcedEarth} et la fonction $\nu_k$ v\'erifient l'identit\'e~\eqref{rec} 
et sont des combinaisons lin\'eaires de produits dont les facteurs sont pris parmi les fonctions \eqref{oomph} 
ou~parmi les d\'eriv\'ees $\nu_1,\cdots,\nu_1^{(k-1)}$. 
\par
\bigskip


Pour ce faire, nous proc\'edons de la fa\c{c}on suivante : nous supposons le th\'eor\`eme \eqrefn{T3faible}~\'etabli, nous fixons des constantes $p,q,\theta$ et une application $f$ satisfaisant les hypoth\`eses 
du th\'eo\-r\`e\-me \eqrefn{T3} puis  nous construisons une fonction $\phi\in\sc C^{\Theta+1}\b([\e,\infty[,[\e,\infty[\b)$ v\'erifiant \eqref{Sabbat} et 
$$
f(t)=\phi(t)+O\Q({t^{\varphi-1}\F(\log t)^\theta}\W)\qquad(t\ge p), \eqdef{Satan}
$$
c'est-\`a-dire satisfaisant les hypoth\`eses du th\'eor\`eme \eqrefn{T3faible} 
et de m\^eme comportement asymptotique que~la fonction $f$ modulo un reste $\ll t^{\varphi-1}/(\log t)^\theta$. 
Il r\'esulte alors du th\'eor\`eme~\eqrefn{T3faible} appliqu\'e \`a la~~fonction $\phi$ qu'il existe 
une unique suite $\nu_k\in\sc C^{\Theta-k}(\ob R)\ \,(1\le k\le\Theta)$ de~fonctions $2\lambda$-p\'eriodiques v\'erifiant 
$$
\phi(t)=ct^{\varphi-1}+\sum_{1\le k\le\Theta}
\nu_k(\log_2t){t^{\varphi-1}\F(\log t)^k}
+O\Q({t^{\varphi-1}(\log_2t)^{\1_{\ob Z}(\theta)}\F(\log t)^\theta}\W)
\qquad(t\ge p). 
$$
En reportant dans l'estimation \eqref{Satan}, nous obtenons la relation \eqref{res}, qui est satisfaite par au plus une suite $\nu_k\in\sc C^{\Theta-k}(\ob R)\ \,(1\le k\le\Theta)$ de~fonctions $2\lambda$-p\'eriodiques, 
et nous en d\'eduisons que  la conclusion 
du th\'eor\`eme \eqrefn{T3} est vraie pour la fonction $f$ . Par suite, le~th\'eor\`eme \eqrefn{T3} d\'ecoule du th\'eor\`eme \eqrefn{T3faible}.
\bigskip


Enfin, au paragraphe \CitSec{Theo5}, nous d\'emontrons le th\'eor\`eme \eqrefn{T3faible} : \'etant donn\'es des nombres 
$p\ge\e$, $\theta>1$ et $\Theta$ v\'erifiant~\eqref{Theta1} et une~fonction $f\in\sc C^{\Theta+1}\b([\e,\infty[,[\e,\infty[\b)$ v\'erifiant~\eqref{Sabbat},  nous prouvons en trois \'etapes, ventil\'ees 
dans les paragraphes \CitSec{DL1}, \CitSec{DL2} et \CitSec{DLn}, 
que~les~conclusions du th\'eor\`eme \eqrefn{T3faible} sont satisfaites pour l'application $f$. 
\bigskip


\Secti DL0, Estimation de la fonction $f$ \`a l'ordre $0$. 


Soient $(p,q,\theta)$ des nombres et $f$ une application v\'erifiant les hypoth\`eses du th\'eor\`eme~\eqrefn{T3}.
Dans~les paragraphes suivants, nous prouvons que la fonction $f$ satisfait l'estimation~\eqref{fsim} en proc\'edant de la fa\c{c}on suivante : 
Nous \'etablissons dans un premier temps que 
$$
\lim_{t\to\infty}f(t)=+\infty.
\eqdef{deathf}
$$ 
Dans un second temps, nous prouvons que l'application~$h$ implicitement d\'efinie par 
$$
f(t)=ct^{\varphi-1}\exp h(t)
\qquad\b(t\ge\max\{p,q\}\b) 
\eqdef{T1eq6}
$$
satisfait, pour un certain nombre r\'eel $b$, l'\'equation fonctionnelle approch\'ee
$$
\b(1+\varphi th'(t)\b)\exp\B(\varphi h(t)+h\b(f(t)\b)\B)=1+O\Q({1\F(\log t)^\theta}\W)
\qquad(t\ge b).
\eqdef{T1eq7}
$$
Enfin, nous diff\'erencions deux cas, selon que l'application $h$ est monotone ou pas, et nous prouvons dans chacun deux que 
$$
\lim_{t\to\infty} h(t)=0, 
\eqdef{limf}
$$ 
ce qui suffit pour \'etablir l'estimation \eqref{fsim}, d'apr\`es l'identit\'e \eqref{T1eq6}. 
\bigskip


Pour commencer, prouvons que la fonction $f$ satisfait \eqref{deathf}. 
Comme $f\circ f(t)>0\ \,(t\ge p)$, 
la relation~\eqref{T1eq1} implique l'existence
d'un nombre r\'eel $a\ge p$ pour lequel
$$
f'(t)\ge {1/2\F f\circ f(t)}>0
\qquad(t\ge a). 
\eqdef{T1eq4}
$$
Nous en d\'eduisons que $f$ est croissante sur $[a,+\infty[$ 
et nous posons $L:=\lim_{t\to\infty}f(t)$. 
En~int\'egrant la relation $2f'(t)f\circ f(t)\ge1$ sur l'intervalle $[a,+\infty[$, 
nous obtenons que
$$
2\int_{f(a)}^L f(t)\d t=+\infty.
$$
Comme $f$ est continue sur $[f(a),+\infty[$, nous en d\'eduisons les relations $L=+\infty$ et  \eqref{deathf}.
\bigskip





D'apr\`es \eqref{deathf} et dd'apr\`es la croissante de la fonction $f$ sur l'intervalle $[a,+\infty[$, 
il existe un nombre $b\ge \max\{p,q\}$ v\'erifiant  
$$
f(t)\ge\max\{p,q\}\qquad(t\ge b).\eqdef{jala}
$$  
L'in\'egalit\'e  $p>0$ et la positivit\'e de $f$ sur $\b[q,+\infty\b[$ impliquent alors que $h$ est bien d\'efinie par l'identit\'e~\eqref{T1eq6}. 
Prouvons de plus qu'elle satisfait l'\'equation fonctionnelle approch\'ee~\eqref{T1eq7}. 
Comme $f\in\sc C^1\b([p,\infty[\b)$, l'identit\'e \eqref{T1eq6} implique que $h\in\sc C^1\b([b,\infty[\b)$ et aussi~que 
$$
f'(t)=ct^{\varphi-2}\b(\varphi-1+th'(t)\b)\exp h(t)
\qquad(t\ge b).
$$
{\it A fortiori}, il r\'esulte des \'egalit\'es \eqref{Phi1}, \eqref{Phi2} et \eqref{Phi4} que 
$$
f'(t)=c^{-\varphi}t^{-(\varphi-1)^2}\b(1+\varphi th'(t)\b)\exp h(t)
\qquad(t\ge b).
$$
Comme la fonction $f$ satisfait \eqref{jala}, l'identit\'e \eqref{T1eq6} implique de m\^eme que  
$$
f\b(f(t)\b)=c^\varphi t^{(\varphi-1)^2}\exp\B((\varphi-1)h(t)+h\b(f(t)\b)\B)
\qquad(t\ge b). 
$$
En reportant les deux estimations pr\'ec\'edentes dans \eqref{T1eq1}, 
nous en d\'eduisons alors \eqref{T1eq7}. 
\bigskip




Lorsque la fonction $h$ est monotone au voisinage de $+\infty$, prouvons qu'elle v\'erifie la relation~\eqref{limf}. 
Soit $\ell:=\lim_{t\to\infty}h(t)$. Si la fonction $h$ est croissante \`a l'infini, 
nous~observons que $h'(t)\ge0\ \,(t\to\infty)$ 
et nous d\'eduisons de \eqref{T1eq7} que   
$$
\exp\B(\varphi h(t)+h\b(f(t)\b)\B)\le1+O\Q({1\F(\log t)^\theta}\W)
\qquad(t\to\infty).
$$
Comme $f$ satisfait \eqref{deathf}, nous en d\'eduisons que la limite $\ell$ est n\'ecessairement finie.  
Si~$h$~est~d\'ecroissante \`a l'infini, nous montrons de m\^eme que 
$$
\exp\B(\varphi h(t)+h\b(f(t)\b)\B)\ge1+O\Q({1\F(\log t)^\theta}\W)
\qquad(t\to\infty)
$$
et que la limite $\ell$ est~finie. Dans les deux cas, i.e. lorsque $h$ est monotone \`a l'infini, 
nous~observons que la limite $\ell$ est finie et nous d\'eduisons alors de \eqref{T1eq7} d'une part que 
$$
h'(t)={\e^{-(\varphi+1)\ell}-1+o(1)\F\varphi t}
\qquad(t\to\infty)
$$
et d'autre part que $\e^{-(\varphi+1)\ell}=1$. En effet, 
comme la fonction $h$ converge \`a l'infini, sa~d\'eriv\'ee ne peut \^etre asymptotiquement \'equivalente  
\`a une fonction du type $t\mapsto \alpha/t$ pour un nombre $\alpha\neq0$. 
{\it A fortiori}, nous obtenons que $\ell=0$ et nous d\'eduisons de \eqref{T1eq6} que la fonction $f$ satisfait \eqref{deathf}. 
\bigskip






De m\^eme, \'etablissons~\eqref{limf} lorsque $h$ n'est pas monotone au voisinage de l'infini.   
Comme~$f$ satisfait l'\'equation diff\'erentielle approch\'ee~\eqref{T1eq1} et l'estimation \eqref{deathf}, 
nous~ob\-te\-nons d'une part que $\lim_{t\to+\infty}f'(t)=0$ et d'autre part que 
$$
f(t)=o(t)\qquad(t\to\infty).
$$
Comme l'hypoth\`ese sur $h$ induit que l'ensemble $H:=\{t\ge b:h'(t)=0\}$ contient des nombres  
arbitrairement~grands, nous en d\'eduisons qu'il existe un \'el\'ement $t_0\in H$ pour lequel $t_0\ge a$ et  
$$
f(t)\le t
\qquad(t\ge t_0).
\eqdef{T1eq8}
$$
Pour chaque $t\ge t_0$, nous posons $\alpha_t:=\sup\{x\in H:x\le t\}$, 
$\beta_t:=\inf\{x\in H:x\ge t\}$ 
et nous observons que la fonction $h$ est monotone sur $\b[\alpha_t,\beta_t\b]$. 
{\it A fortiori}, nous observons~que 
$$
\b|h(t)\b|\le \max\B\{\b|h(\alpha_t)\b|,\b|h(\beta_t)\b|\B\}
\qquad(t\ge t_0). 
$$
Comme $h'(x)=0\ \,(x\in H)$, l'\'equation fonctionnelle approch\'ee \eqref{T1eq7} induit d'une part que 
$$
\varphi h(x)+h\b(f(x)\b)\ll {1\F(\log x)^\theta}
\qquad(x\in H) 
$$ 
et d'autre part que
$$
\varphi \b|h(t)\b|\le \max\bg\{\B|h\b(f(\alpha_t)\b)\B|,\B|h\b(f(\beta_t)\b)\B|\bg\}+O\Q({1\F(\log \alpha_t)^\theta}\W)
\qquad(t\ge t_0). 
\eqdef{T1eq9}
$$
Comme l'ensemble $H$ n'est pas major\'e et comme la fonction $f$ satisfait \eqref{deathf}, 
nous~d\'e\-fi\-nis\-sons bien par r\'ecurrence une suite strictement croissante $\{t_n\}_{n=0}^\infty$ en posant 
$$
t_{n+1}:=\inf\b\{t\in H:f(t)\ge t_n \hbox{ et } t\ge 2t_n\b\}
\qquad(n\ge0). 
\eqdef{T1eq10}
$$
Nous fixons $n\in\ob N$, $t\in[t_{n+1},t_{n+2}]$ et nous observons que $t_{n+1}\le\alpha_t\le\beta_t\le t_{n+2}$. 
Comme~la fonction $f$ est croissante sur l'intervalle $[t_0,\infty[$ d'apr\`es l'in\'egalit\'e $t_0\ge a$, nous~en~d\'eduisons que 
$$
f(t_n)\le f(\alpha_t)\le f(\beta_t)\le f(t_{n+1}). 
$$
Les relations \eqref{T1eq8} et \eqref{T1eq10} impliquent alors que  
$$
t_{n-1}\le f(\alpha_t)\le f(\beta_t)\le t_{n+1}. 
$$
Posant $M_n:=\sup_{t_n\le t\le t_{n+1}}\b|h(t)\b|$ pour chaque entier $n\ge0$, 
nous en  d\'eduisons que
$$
\max\B\{\b|h(\alpha_t)\b|,\b|h(\beta_t)\b|\B\}
\le \max\{M_n,M_{n+1}\}
\qquad(n\ge0,t_{n+1}\le t\le t_{n+2}). 
$$
Reportant dans \eqref{T1eq9}, l'in\'egalit\'e 
$\alpha_t\ge t_{n+1}\ \,(t\ge t_{n+1})$ induit que 
$$
\varphi|h(t)|\le \max\{M_n,M_{n+1}\}+O\Q({1\F(\log t_n)^\theta}\W)
\qquad(n\ge0, t_{n+1}\le t\le t_{n+2}).
$$
Comme \eqref{T1eq10} implique que $t_{n+1}\ge 2t_n$ 
et donc que $t_n\ge t_02^n\ \,(n\ge0)$, il suit 
$$
\varphi M_{n+1}\le\max\{M_n, M_{n+1}\}+O\Q({1\F(n+1)^\theta}\W)
\qquad(n\ge0). 
$$
Lorsque $M_{n+1}=\max\{M_n, M_{n+1}\}$ nous remarquons que 
$$
(\varphi-1)M_{n+1}\ll{1\F(n+1)^\theta}\qquad(n\ge0) 
$$
et nous en d\'eduisons que 
$$
\varphi M_{n+1}\le M_n+O\Q({1\F(n+1)^\theta}\W)
\qquad(n\ge0). 
$$
L'in\'egalit\'e pr\'ec\'edente \'etant aussi v\'erifi\'ee si $M_n=\max\{M_n, M_{n+1}\}$, 
par r\'ecurrence sur~$n$, nous montrons que  
$$
M_n\ll \varphi^{-n}+\sum_{0\le k<n}{\varphi^{k-n}\F (k+1)^\theta}
\qquad(n\ge0).  
$$
D'apr\`es le th\'eor\`eme de convergence domin\'ee, 
le membre de droite tends vers $0$ lorsque~$n$ tends vers $+\infty$. 
{\it A fortiori}, nous obtenons que  $\lim_{n\to\infty} M_n=0$ et par suite que l'estimation~\eqref{limf} est satisfaite. 
\hfill\qed
\bigskip


\Secti Reductio, R\'eduction du probl\`eme. 

Dans les paragraphes suivants, nous prouvons que le th\'eor\`eme \eqrefn{T3} d\'ecoule du th\'eor\`eme~\eqrefn{T3faible} via la m\'ethode expos\'ee au paragraphe \CitSec{Plan}, c'est-\`a-dire 
en construisant une fonction $\phi$ appartenant \`a l'ensemble $\sc C^{\Theta+1}\b([\e,\infty[,[\e,\infty[\b)$ et v\'erifiant les~estimations \eqref{Sabbat}~et~~\eqref{Satan} : 
pour~chaque $n\ge0$, nous construisons une application~$f_n\in\sc C^{n+1}\b([\e,\infty[,[\e,\infty[\b)$~v\'erifiant 
$$
f_n'(t)={1+\scod_n\b(1/(\log t)^\theta\b)\F f_n\b(f_n(t)\b)}\qquad(t\ge\e)\leqno{(\sc P_n)}
$$
telle que 
$$
f(t)=f_n(t)+O\Q({t^{\varphi-1}\F(\log t)^\theta}\W)
\qquad(t\ge p), 
\leqno{(\sc Q_n)}
$$
en proc\'edant par r\'ecurrence, puis nous posons $\phi=f_\Theta$. 
\bigskip


Soient $p,q,\theta$ des constantes et $f$ une fonction v\'erifiant les hypoth\`eses du th\'eor\`eme~\eqrefn{T3}.  
Commen\c{c}ons par construire une fonction $f_0\in\sc C^1\b([\e,\infty[,[\e,\infty[\b)$ satisfaisant $(\sc P_0)$~et~$(\sc Q_0)$. 
Comme $f$ v\'erifie \eqref{fsim}, il existe un nombre $x_0>a$ pour lequel $f(x_0)>a$ et $f\circ f(x_0)>a$. Nous~posons $x_1:=\min\{x_0,f(x_0)\}$ 
et nous d\'eduisons des in\'egalit\'es $a\ge\e$ et \eqref{T1eq4} que 
$$
x_1>a\ge\e,\qquad f(x_1)>\e \qquad\hbox{et}\qquad f'(x_1)>0.\eqdef{inegal}
$$  
Alors, nous remarquons d'une part que les nombres $\alpha$ et $\beta$ d\'efinis par 
$$
\alpha:={x_1f'(x_1)\F f(x_1)-\e}\qquad\hbox{et}\qquad\beta:={f(x_1)-\e\F x_1^\alpha}
\eqdef{T1eq13}
$$
sont strictement positifs et d'autre part que l'application $f_0$ l'application d\'efinie par
$$
f_0(x):=\Q\{\eqalign{
&\e+\beta x^\alpha\qquad(\e\le x<x_1)
\cr
&f(x)\vbox to4mm{}\qquad(x\ge x_1)
\cr
}\W.
\eqdef{T1eq12}
$$
appartient \`a $\sc C^1\b([\e,x_1[,[\e,\infty[\b)$.  La fonction $f$ 
\'etant croissante et de classe $\sc C^1$ sur~$[a,\infty[$, nous d\'eduisons \'egalement de 
\eqref{inegal} que l'application $f_0$ appartient \`a $\sc C^1\b([x_1,\infty[,[\e,\infty[\b)$. 
Comme les nombres $\alpha$ et $\beta$ satisfont le syst\`eme
$$
\Q\{\eqalign{
&\e+\beta x_1^\alpha=f(x_1)_{\strut}
\cr
&\alpha\beta x_1^{\alpha-1}=f'(x_1)
}
\W.\Longleftrightarrow\Q\{\eqalign{
&\beta x_1^\alpha=f(x_1)_{\strut}-\e
\cr
&\alpha (f(x_1)-\e)=x_1f'(x_1)
}\W.\Longleftrightarrow\eqref{T1eq13},
$$
le th\'eor\`eme de raccordement $\sc C^1$ appliqu\'e en $x_1$ implique que $f_0\in\sc C^1\b([\e,\infty[,[\e,\infty[\b)$. 
Les~fonctions $f$ et $f_0$ \'etant continues sur $[p,+\infty[$, nous d\'eduisons alors de \eqref{T1eq12} que   
$$
f(x)=f_0(x)
\qquad(x\ge x_1)
$$
et donc que la relation $(\sc Q_0)$ est satisfaite. 
Comme la fonction $f$ est croissante sur~$[x_0,+\infty[$ 
et comme $x_1=\min\{x_0,f(x_0)\}$, nous d\'eduisons de \eqref{T1eq12} que $f(x)=f_0(x)\ge x_1\ \,(x\ge x_0)$
et nous d\'eduisons de l'estimation \eqref{T1eq1} que   
$$
f_0'(x)=f'(x)={1+O\b(1/(\log x)^\theta\b)\F f\b(f(x)\b)}={1+O\b(1/(\log x)^\theta\b)\F f_0\b(f_0(x)\b)}\qquad(x\ge x_0). 
$$
Cette estimation \'etant \'egalement v\'erifi\'ee sur l'intervalle $[\e,x_0[$ par la fonction $f_0$, 
nous~concluons que la relation $(\sc P_0)$ est satisfaite. 
\bigskip



Soit $\{f_n\}_{n=1}^\infty$ la famille d'applications d\'efinie par 
$$
f_{n+1}(t):=\e+\int_\e^t{\d x\F
f_n\b(f_n(x)\b)}\qquad(n\ge0,t\ge\e).
\eqdef{T1eq14}
$$
Pour chaque $n\ge0$, nous montrons par r\'ecurrence que la fonction $f_n:[\e,+\infty[\to[\e,+\infty[$ est bien d\'efinie 
et de classe $\sc C^{n+1}$. En effet, ces propri\'et\'es sont v\'erifi\'ees pour $n=0$ et l'identit\'e~\eqref{T1eq14} induit qu'elles sont v\'erifi\'es pour l'entier $n+1$ si elles le sont pour $n\ge0$. 
\bigskip


Soit $\{\epsilon_n\}_{n=0}^\infty$ la suite de fonctions d\'efinie par
$$
f_n'(t)={1+\epsilon_n(t)\F f_n\b(f_n(t)\b)}\qquad(n\ge0,t\ge\e).\eqdef{T1eq15}
$$
Pour $n\ge0$, nous d\'eduisons de la relation $f_n\in\sc C^{n+1}\b([\e,\infty[,[\e,\infty[\b)$ 
que l'application~$\epsilon_n$ appartient \`a l'espace $\sc C^n\b([\e,\infty[\b)$ et nous remarquons que 
la majoration $(\sc P_n)$ s'\'ecrit aussi 
$$
\epsilon_n(t)\ll_n{1\F(\log t)^\theta}\qquad(t\ge\e).
\leqno{(\sc P_n)}
$$
Nous rappelons que $f_n\in\sc C^{n+1}\b([\e,+\infty[,[\e,+\infty[\b)\ \,(n\ge0)$, que $(\sc P_0)$ et $(\sc Q_0)$ sont satisfaites et 
nous~\'etablissons maintenant $(\sc P_n)$ et $(\sc Q_n)$ par r\'ecurrence sur l'entier $n\ge1$. 
\bigskip




\'Etant donn\'e un entier $n\ge0$ pour lequel $(\sc P_n)$ et $(\sc Q_n)$ sont satisfaites, prouvons $(\sc Q_{n+1})$. 
En~d\'erivant la d\'efinition \eqref{T1eq14}, nous obtenons 
$$
f_{n+1}'(t)={1\F f_n\b(f_n(t)\b)}
\qquad(t\ge\e).
\eqdef{T1eq16}
$$
Nous d\'eduisons alors de \eqref{T1eq15} que 
$$
f_n'(t)-f_{n+1}'(t)=\epsilon_n(t)f_{n+1}'(t)
\qquad(t\ge\e).
\eqdef{T1eq17}
$$
Nous remarquons que les estimations $(\sc Q_n)$ et \eqref{fsim} impliquent que
$$
f_n(t)\sim ct^{\varphi-1}
\qquad(t\to\infty). 
\eqdef{T1eq18}
$$
{\it A fortiori}, il r\'esulte des relations \eqref{Phi4} et \eqref{T1eq16} que
$$
f_{n+1}'(t)\ll t^{\varphi-2}
\qquad(t\ge\e)
$$
et nous d\'eduisons de \eqref{T1eq17} que
$$
f_n'(t)-f_{n+1}'(t)\ll\epsilon_n(t)t^{\varphi-2}
\qquad(t\ge\e).
$$
D'apr\`es la majoration $(\sc P_n)$, il suit
$$
f_n'(t)-f_{n+1}'(t)\ll {t^{\varphi-2}\F(\log t)^\theta}
\qquad(t\ge\e).
$$
En int\'egrant cette estimation sur l'intervalle $[\e,t]$, nous obtenons alors que
$$
f_n(t)-f_{n+1}(t)\ll{t^{\varphi-1}\F(\log t)^\theta}
\qquad(t\ge\e). 
\eqdef{T1eq19}
$$
Reportant dans $(\sc Q_n)$, nous concluons que l'estimation $(\sc Q_{n+1})$
est satisfaite.
\bigskip



\'Etant donn\'e un entier $n\ge0$ pour lequel 
$(\sc P_n)$ et $(\sc Q_n)$ sont satisfaites, prouvons $(\sc P_{n+1})$. 
D'apr\`es \eqref{T1eq15} et \eqref{T1eq16}, nous avons 
$$
\epsilon_{n+1}(t)=f_{n+1}'(t)\B(f_{n+1}\b(f_{n+1}(t)\b)-f_n\b(f_n(t)\b)\B)
\qquad(t\ge\e). 
$$
Nous fixons un entier $k\in\{0,\cdots,n+1\}$ et nous d\'eduisons de la formule de
Leibniz que
$$
\epsilon_{n+1}^{(k)}(t)=\sum_{0\le\ell\le k}\Q({k\atop\ell}\W)f_{n+1}^{(k-\ell+1)}(t)
\B(f_{n+1}\circ f_{n+1}-f_n\circ f_n\B)^{(\ell)}(t)
\qquad(t\ge\e).
\eqdef{T1eq20}
$$
D'apr\`es \eqref{T1eq18}~et~$(\sc P_n)$, nous pouvons appliquer 
le lemme~\eqrefn{dfc5} \`a $f_n\in\sc C^{n+1}\b([\e,\infty[,[\e,\infty[\b)$ 
pour~l'entier~$n$ et nous obtenons que 
$$
f_n(t)\olopd_{n+1} t^{\varphi-1}\qquad(t\ge\e). \eqdef{Yess}
$$ 
En particulier, la fonction $f_n$ satisfait les hypoth\`eses du lemme \eqrefn{dfc4} 
pour l'entier $n+1$ et nous~d\'eduisons de \eqref{T1eq16} que
$$
f_{n+1}'(t)\olopd_{n+1} t^{\varphi-2}\qquad(t\ge\e). 
\eqdef{T1eq21}
$$  
Reportant dans \eqref{T1eq20}, il suit 
$$
\epsilon_{n+1}^{(k)}(t)\ll\sum_{0\le\ell\le k}
\bg|(f_{n+1}\circ f_{n+1})^{(\ell)}(t)-(f_n\circ f_n)^{(\ell)}(t)\bg|t^{\varphi-2+\ell-k}
\qquad(t\ge\e). 
\eqdef{T1eq22}
$$
Pour \'etablir la relation $(\sc P_{n+1})$, c'est-\`a-dire pour d\'emontrer que 
$$
\epsilon_{n+1}^{(k)}(t)\ll{1\F t^k(\log t)^\theta}
\qquad(0\le k\le n+1,t\ge\e), 
$$
nous remarquons alors qu'il suffit d'\'etablir la majoration 
 $$
f_{n+1}\circ f_{n+1}(t)-f_n\circ f_n(t)\olopd_{n+1}{t^{2-\varphi}\F(\log t)^\theta}
\qquad(t\ge\e), 
$$
ce que nous faisons maintenant, en prouvant que les fonctions $f_n$ et $f_{n+1}$ satisfont les~hypoth\`eses du lemme \eqrefn{dfc2} pour l'entier $n+1$. 
Nous fixons un entier $j\in[1, n+1]$ et nous d\'erivons $j-1$ fois l'identit\'e \eqref{T1eq17} pour obtenir que  
$$
f_n^{(j)}(t)-f_{n+1}^{(j)}(t)=
\sum_{0\le\ell<j}\Q({j-1\atop\ell}\W)\epsilon_n^{(\ell)}(t)f_{n+1}^{(j-\ell)}(t)
\qquad(t\ge\e). 
$$
Comme les relations \eqref{T1eq21} et $(\sc P_n)$ impliquent que   
$$
f_n^{(j)}(t)-f_{n+1}^{(j)}(t)\ll\sum_{0\le\ell<j}\b|\epsilon_n^{(\ell)}(t)\b|t^{\varphi-1+\ell-j}
\ll{t^{\varphi-1-j}\F(\log t)^\theta}
\qquad(t\ge\e), 
$$
nous d\'eduisons alors de \eqref{T1eq19} que 
$$
f_n(t)-f_{n+1}(t)\olopd_{n+1} {t^{\varphi-1}\F(\log t)^\theta}\qquad(t\ge\e).
$$
De m\^eme, comme les relations \eqref{T1eq18} et \eqref{T1eq19} impliquent que $f_{n+1}(t)\ll t^{\varphi-1}\ \,(t\ge\e)$, 
nous d\'eduisons de \eqref{T1eq21} que 
$$
f_{n+1}(t)\olopd_{n+2} t^{\varphi-1}\qquad(t\ge\e). 
$$ 
En particulier, la fonction $f_{n+1}$ satisfait \eqref{dfc3eq1} et \eqref{dfc3eq2} pour l'entier $n'=n+1$. 
De~m\^eme, il r\'esulte de \eqref{T1eq18} et de \eqref{Yess} que $f_n$ satisfait  \eqref{fsim}~et~\eqref{dfc1eq1} 
pour l'entier $n'=n+1$.  
{\it A~fortiori}, les hypoth\`eses du lemme \eqrefn{dfc2} pour l'entier $n+1$ sont bien v\'erifi\'ees par $f_n$ et $f_{n+1}$, 
de sorte que la relation $(\sc P_{n+1})$ est satisfaite. 
\medskip

Nous avons \'etabli les relations $(\sc P_0)$ et $(\sc Q_0)$ et prouv\'e l'implication suivante
$$
(\sc P_n)\hbox{ et }(\sc Q_n)\qquad \Longrightarrow\qquad (Q_{n+1})\hbox{ et }(\sc P_{n+1})\qquad\qquad(n\ge0). 
$$
Par r\'ecurrence, nous concluons que $(\sc P_n)$ et $(\sc Q_n)$ sont v\'erifi\'ees pour chaque entier $n\ge0$ 
et nous remarquons que la fonction $\phi=f_{\Theta}$ poss\`ede alors toutes les qualit\'es requises.  
\hfill\qed
\bigskip

\Secti Theo5, Preuve du th\'eor\`eme \eqrefn{T3faible}.

\Sectio Plan2, Plan de la d\'emonstration. 

Soient  $p\ge\e$, $\theta>1$ et $\Theta$ des nombres v\'erifiant \eqref{Theta1} 
et soit $f\in\sc C^{\Theta+1}\b([\e,\infty[,[\e,\infty[\b)$ 
une fonction satisfaisant \eqref{Sabbat}, i.e. v\'erifiant la majoration
$$
f'(x)f\circ f(x)-1\olopd_\Theta {1\F(\log x)^\theta}\qquad(x\ge\e).\eqdef{burning}
$$
Dans les paragraphes \CitSec{DL1}, \CitSec{DL2} et \CitSec{DLn}, nous d\'emontrons en trois \'etapes que le~th\'eo\-r\`e\-me~\eqrefn{T3faible} s'applique \`a $f$, i.e.  
qu'il~existe une~unique~suite $\nu_k\in\sc C^{\Theta-k}(\ob R)\ \,(1\le k\le\Theta)$ de~fonctions $2\lambda$-p\'eriodiques v\'erifiant l'estimation~\eqref{res}, que~la fonction $\nu_1$ v\'erifie \eqref{n+tn} et 
que l'application $G_k$ d\'efinie par \eqref{IcedEarth} et la fonction $\nu_k$ v\'erifient l'identit\'e~\eqref{rec} 
et sont des combinaisons lin\'eaires de produits dont les facteurs sont pris parmi les fonctions~\eqref{oomph} 
ou~parmi les d\'eriv\'ees $\nu_1,\cdots,\nu_1^{(k-1)}$. 
\bigskip

Pour \'etudier le  comportement \`a l'infini de $f$, nous proc\'edons au changement de fonction
$$
f(t)=ct^{\varphi-1}+g(\log_2t){t^{\varphi-1}\F\log t}
\qquad(t\ge\e)
\eqdef{defg}
$$
et nous estimons le comportement asymptotique de la fonction $g$ ainsi d\'efinie pour des pr\'ecisions de plus en plus fines 
sur l'\'echelle exponentielle. 
\medskip

Au paragraphe \CitSec{DL1}, nous prouvons que l'application $g$, implicitement d\'efinie par \eqref{defg}, 
appartient \`a l'espace $\sc C^{[\theta]+1}\b([0,\infty[\b)$ et qu'elle est born\'ee sur $[0,\infty[$. {\it A fortiori}, 
il r\'esulte de l'identit\'e \eqref{defg} que la fonction $f$ satisfait le d\'eveloppement limit\'e \`a l'ordre 1 
$$
f(t)=ct^{\varphi-1}+O\Q({t^{\varphi-1}\F\log t}\W)
\qquad(t\ge\e). \eqdef{DL1f}
$$
\medskip

%Nous remarquons que l'estimation \eqref{vardi} \'etablie par Vardi \CitRef{Vardi} est une cons\'equence de \eqref{DL1f}. 
%En~effet, nous rappelons que la fonction croissante $\xi\in\sc C^\infty\b(]0,\infty[,]0,\infty[\b)$ d\'efinie par \eqref{defxi} 
%satisfait les estimations \eqref{xi2}, \eqref{xi1} et {\it a fortiori} l'\'equation diff\'erentielle approch\'ee \eqref{nimp}. 
%En particulier, la fonction $\xi$ satisfait les hypoth\`eses du th\'eor\`eme \eqrefn{T3} 
%pour les nombres $p=3$, $q=\xi(3)$~et~$\theta=3$.  D'apr\`es l'\'etude men\'ee dans la section \CitSec{Reductio}, 
%il existe alors une fonction $K$ satisfaisant les hypoth\`eses du th\'eor\`eme \eqrefn{T3faible} telle que 
%$$
%\xi(t)=K(t)+O\Q({t^{\varphi-1}\F(\log t)^3}\W)
%\qquad(t\ge 3). \eqdef{foncK}
%$$
%Comme une telle application satisfait \eqref{DL1f}, nous d\'eduisons alors de \eqref{xi2} que 
%$$
%u_n=\xi(n)+O(1)=K(n)+O\Q({n^{\varphi-1}\F(\log n)^3}\W)=cn^{\varphi-1}+O\Q({n^{\varphi-1}\F\log n}\W)\qquad(n\ge3).
%$$
%En particulier, la suite $\{r_n\}_{n=1}^\infty$ implicitement d\'efinie par \eqref{La7} satisfait l'estimation \eqref{vardi}. 
%\bigskip

Au paragraphe \CitSec{DL2}, nous prouvons que la fonction $g$ satisfait la majoration 
$$
g(u)\olops_ \Theta 1\qquad(u\ge0)
\eqdef{first}
$$
et nous construisons une application $\nu_1\in\sc C^{\Theta-1}\b(\ob R\b)$ v\'erifiant \eqref{n+tn} et
$$
g(u)=\nu_1(u)+\scos_{\Theta-1}\Q(\e^{-u}\W)\qquad(u\ge0).\eqdef{audre}
$$
Si $\theta>2$, nous remarquons que $\Theta\ge2$ et nous d\'eduisons de \eqref{defg} et de \eqref{var3a} 
que  la fonction $f$ satisfait alors le d\'eveloppement limit\'e \`a l'ordre 2
$$
f(t)=ct^{\varphi-1}+\nu_1(\log_2t){t^{\varphi-1}\F\log t}
+O\Q({t^{\varphi-1}\F(\log t)^2}\W)
\qquad(t\ge\e). \eqdef{ahahahah}
$$
\medskip

%Nous observons que la conjecture $(C')$ \'enonc\'ee par P\'etermann, R\'emy et Vardi \CitRef{PetermannRemyVardi} 
%est une cons\'equence imm\'ediate de l'estimation \eqref{ahahahah}. En effet, une solution $f$ du syst\`eme \eqref{conjsys} 
%satisfait les hypoth\`eses du  th\'eor\`eme \eqrefn{T3} pour les nombres $p'=p+\e$, $q=p$ et $\theta=3$. D'apr\`es l'\'etude men\'ee 
%dans la section \CitSec{Reductio}, il existe alors une fonction $\phi$ satisfaisant les hypoth\`eses du th\'eor\`eme \eqrefn{T3faible} et 
%admettant le m\^eme comportement asymptotique que~la fonction $f$ modulo un reste $\ll t^{\varphi-1}/(\log t)^3$. Comme~une~telle application satisfait \eqref{ahahahah}, 
%nous obtenons alors que 
%$$
%f(t)=\phi(t)+O\Q({t^{\varphi-1}\F(\log t)^3}\W)=ct^{\varphi-1}+\nu_1(\log_2t){t^{\varphi-1}\F\log t}+O\Q({t^{\varphi-1}\F(\log t)^2}\W)\qquad(t\ge p').
%$$
%En particulier, la fonction $f$ satisfait la conjecture $(C')$. 
%\medskip
%
%De m\^eme, nous remarquons que la conjecture $(C)$ de Vardi \CitRef{Vardi} d\'ecoule de l'estimation~\eqref{ahahahah}. 
%En~effet, comme l'application $K$ pr\'ec\'edemment d\'efinie satisfait les hypoth\`eses du th\'eor\`eme \eqrefn{T3faible}, 
%elle~v\'erifie n\'ec\'essairment \eqref{ahahahah} et nous d\'eduisons de \eqref{xi2} et \eqref{foncK} que 
%$$
%u_n=\xi(n)+O(1)=K(n)+O\Q({n^{\varphi-1}\F(\log n)^3}\W)=cn^{\varphi-1}+\nu_1(\log_2n){n^{\varphi-1}\F\log n}
%+O\Q({n^{\varphi-1}\F(\log n)^2}\W)\qquad(n\ge3).
%$$
%En particulier, la suite $\{r_n\}_{n=1}^\infty$ implicitement d\'efinie par \eqref{La7} satisfait l'estimation \eqref{vardi}. 
%\bigskip


Enfin, au paragraphe \CitSec{DLn},  nous construisons par r\'ecurrence sur l'entier $k\in\{2,\cdots, \Theta\}$ 
une application $2\lambda$-p\'eriodique $\nu_k\in\sc C^{\Theta-k}(\ob R)$,  qui est une combinaisons lin\'eaires de produits dont les facteurs sont pris parmi les fonctions~\eqref{oomph} 
ou~parmi $\nu_1,\cdots,\nu_1^{(k-1)}$, v\'erifiant les relations \eqref{rec} et
$$
g(u)=\sum_{0\le m<k}{\nu_{m+1}(u)\F\e^{mu}}+\scos_{\Theta-k}\Q(\e^{-ku}\W)\qquad(u\ge0).  
\eqdef{KOL}
$$
Comme la fonction $\nu_1\in\sc C^{\Theta-1}(\ob R)$ construite au paragraphe \CitSec{DL2} satisfait~\eqref{n+tn} 
et comme~\eqref{res} d\'ecoule des relations \eqref{defg} et \eqref{KOL}, cela~suffit pour \'etablir le th\'eor\`eme~\eqrefn{T3faible}.
En~effet, prouvons par l'absurde qu'il existe au plus une suite $\nu_k\in\sc C^{\Theta-k}(\ob R)\ \,(1\le k\le\Theta)$ d'applications $2\lambda$-p\'eriodiques 
v\'erifiant~\eqref{res}. \'Etant donn\'ee $\eta_k\in\sc C^{\Theta-k}(\ob R)\ \,(1\le k\le\Theta)$ une autre suite poss\'edant les propri\'et\'es requises, nous posons 
$$
m=\min\{k\in[1,\Theta]:\nu_k\neq\eta_k\}
$$ 
et nous d\'eduisons de la relation \eqref{res} v\'erifi\'ee  
par les suites $\{\nu_k\}_{1\le k\le\Theta}$ et $\{\eta_k\}_{1\le k\le\Theta}$ que  
$$
0=\sum_{m\le k\le\Theta}
\b(\nu_k(\log_2t)-\eta_k(\log_2t)\b){t^{\varphi-1}\F(\log t)^k}
+O\Q({t^{\varphi-1}(\log_2t)^{\1_{\ob Z}(\theta)}\F(\log t)^\theta}\W)
\qquad(t\ge p).
$$
Comme les applications $\nu_k-\eta_k\ \,(1\le k\le\Theta)$ sont continues et $2\lambda$-p\'eriodiques sur l'intervalle $\ob R$, nous remarquons d'une part qu'elles sont born\'ees et d'autre part que 
$$
\nu_m(\log_2t)-\eta_m(\log_2t)\ll {\log_2t\F\log t}\qquad(t\ge p).
$$
Pour chaque nombre r\'eel $u$, nous substituons le nombre $\exp\e^{u+2\lambda n}$ \`a $t$ et  
nous d\'eduisons de la $2\lambda$-p\'eriodicit\'e de la fonction $\nu_m-\eta_m$ que 
$$
\nu_m(u)-\eta_m(u)=\lim_{n\to\infty}\b(\nu_\ell(u+2\lambda n)-\eta_\ell(u+2\lambda n)\b)=0\qquad(u\in\ob R).
$$ 
{\it A fortiori}, nous obtenons que $\nu_\ell=\eta_\ell$, ce qui contredit la d\'efinition du minimum~$m$. 
\bigskip

\Sectio DL1, Estimation de la fonction $f$ \`a l'ordre $1$.

Dans ce paragraphe, nous prouvons que la fonction $g$ implicitement d\'efinie par l'identit\'e~\eqref{defg} est born\'ee. 
Pour ce faire, nous notons $E$ et $\phi$ les fonctions uniquement d\'etermin\'ees par 
$$
\eqalignno{
{c\e^\varphi\F\varphi}+\int_\e^{f(t)}\!\!f(x)\d x=t+tE(\log_2t)
&\qquad(t\ge\e), 
&
\eqdef{defE}
\cr
\log_2 f(t)=\phi(\log_2t)^{\strut}\qquad\qquad
&\qquad(t\ge\e). 
&
\eqdef{defphi}
}
$$
et nous prouvons que l'application $g$ satisfait l'\'equation int\'egrale
$$
{c\F\varphi}\Q(c+{g(u)\F\e^u}\W)^\varphi+{1\F\exp(\e^u)}
\int_0^{\phi(u)}\!\!g(x)\exp(\varphi\e^x)\d x=1+E(u)
\qquad(u\ge0).
\eqdef{eqfonc}
$$
Puis, nous \'etablissons les majorations 
$$
\eqalignno{
E(u)_{\strut}\ll\e^{-\theta u}\qquad
\qquad
&(u\ge0),
&
\eqdef{Kort6}
\cr
\phi(u)=u-\lambda+O(\e^{-u})
\qquad&(u\ge0)
&
\eqdef{Kort5}
}
$$
et nous d\'emontrons que la fonction $G$ implicitement d\'efinie par l'identit\'e
$$
G(u):=\varphi\e^u\int_0^ug(x){\exp(\varphi\e^x)\F\exp(\varphi\e^u)}\d x\qquad(u\ge0)
\eqdef{defga}
$$
est une solution de l'\'equation fonctionnelle approch\'ee  
$$
g(u)+G\b(\phi(u)\b)\ll{\b|g(u)\b|+g(u)^2\F\e^u}+\e^{(1-\theta)u}
\qquad(u\ge0). 
\eqdef{equaqua}
$$
Enfin, nous en d\'eduisons que la fonction $g$ est born\'ee sur l'intervalle $[0,+\infty[$. 
\bigskip




\'Etant donn\'ees $E$ et $\phi$ les fonctions uniquement d\'etermin\'ees  par \eqref{defE} et \eqref{defphi}, 
prouvons~que $g$ est une solution de l'\'equation \eqref{eqfonc}. 
En reportant \eqref{defg} dans l'identit\'e~\eqref{defE}, nous obtenons que 
$$
{cf(t)^\varphi\F\varphi}+\int_\e^{f(t)}g(\log_2 x){x^{\varphi-1}\F\log x}\d x=t+tE(\log_2t)
\qquad(t\ge\e).
$$
Il r\'esulte alors du changement de variable $x=\exp(\e^y)$ que  
$$
{cf(t)^\varphi\F\varphi}+\int_0^{\log_2f(t)}g(y)\exp(\varphi\e^y)\d y=t+tE(\log_2t)
\qquad(t\ge\e).
$$
En divisant les deux membres de l'\'egalit\'e par $t$, nous d\'eduisons alors de \eqref{defphi} que  
$$
{cf(t)^\varphi\F \varphi t}+{1\F t}\int_0^{\phi(\log_2t)}g(y)\exp(\varphi\e^y)\d y=1+E(\log_2t)
\qquad(t\ge\e).
$$
et de l'identit\'e \eqref{defg} que 
$$
{c\F\varphi}\Q(c+{g(\log_2t)\F\log t}\W)^\varphi
+{1\F t}\int_0^{\phi(\log_2t)}g(x)\exp(\varphi\e^x)\d x=1+E(\log_2t)
\qquad(t\ge\e).
$$
Enfin, en posant $u=\log_2t$, nous concluons que l'application $g$ satisfait \eqref{eqfonc}. 
\bigskip


Prouvons maintenant  que les majorations \eqref{Kort6} et \eqref{Kort5} sont v\'erifi\'ees par $E$ et $\phi$.  
Comme $f\in\sc C^\Theta\b([\e,+\infty[,[\e,+\infty[\b)$, il r\'esulte de la d\'efinition \eqref{defphi}~que
$$
\phi\in\sc C^\Theta\b([0,\infty[,[0,+\infty[\b). \eqdef{powa}
$$ 
De plus, d'apr\`es l'identit\'e \eqref{defE}, nous avons 
$$
tE(\log_2t)={c\e^\varphi\F\varphi}+\int_\e^{f(\e)}f(x)\d x+\int_{f(\e)}^{f(t)}f(x)\d x-t
\qquad(t\ge\e). 
$$
En remarquant que $t=\e+\int_e^t\d x\ \,(t\ge\e)$ et que 
$$
\int_{f(\e)}^{f(t)}f(x)\d x=\int_\e^tf'(x)f\circ f(x)\d x\qquad(t\ge\e),
$$
nous obtenons alors que 
$$
tE(\log_2t)={c\e^\varphi\F\varphi}+\int_\e^{f(\e)}f(x)\d x-\e+\int_\e^t\B(f'(x)f\circ f(x)-1\B)\d x
\qquad(t\ge\e)\eqdef{Emper}
$$ 
et nous d\'eduisons de \eqref{burning} que 
$$
E(\log_2t)\ll{1\F(\log t)^\theta}\qquad(t\ge\e)
$$ 
puis que l'estimation \eqref{Kort6} est satisfaite, d'apr\`es le changement de variable~$t=\exp\e^u$. 
Par ailleurs, l'identit\'e \eqref{defg} implique que  
$$
\log f(t)=(\varphi-1)\log t+\log\Q(c+{g(\log_2t)\F\log t}\W)
\qquad(t\ge\e). 
$$
D'apr\`es \eqref{Phi1}, il suit 
$$
\log_2f(t)=\log_2 t-\lambda+\log\Q(1+{\varphi\F \log t}\log\Q(c+{g(\log_2t)\F\log t}\W)\W)
\qquad(t\ge\e).
$$
En posant $u=\log_2t$, il r\'esulte alors de la d\'efinition \eqref{defphi} que 
$$
\phi(u)=u-\lambda+\log\Q(1+{\varphi\F\e^u}\log\Q(c+{g(u)\F\e^u}\W)\W)
\qquad(u\ge0). 
\eqdef{KoKor}
$$
Comme $f$ satisfait les hypoth\`eses du th\'eor\`eme \eqrefn{T1} pour $p=q=\e$, 
elle~v\'erifie~\eqref{fsim}. A~fortiori, nous en d\'eduisons que $g(\log_2t)=o(\log t)\ \,(t\to\infty)$ et donc que  
$$
g(u)=o(\e^u)
\qquad(u\to\infty). 
\eqdef{Kor4}
$$
Enfin, en reportant dans \eqref{KoKor}, nous d\'eduisons \eqref{Kort5} de la relation $\phi\in\sc C\b([0,+\infty[\b)$. 
\bigskip


Comme $f\in\sc C^{\Theta+1}\b([\e,+\infty[)$, nous d\'eduisons de \eqref{defg} d'une part que $g\in\sc C^{\Theta+1}\b([0,\infty[\b)$ 
et d'autre part que l'application $G$ d\'efinie par \eqref{defga} appartient \`a l'espace  $\sc C^{\Theta+2}\b([0,\infty[\b)$. 
Prouvons que les applications  $g$ et $G$ satisfont \'egalement l'\'equation fonctionnelle~\eqref{equaqua}. 
Nous remarquons que l'identit\'e \eqref{KoKor} implique d'une part que 
$$
\varphi\e^{\phi(u)}=\e^u+\varphi\log \Q(c+{g(u)\F\e^u}\W)
\qquad(u\ge0)
\eqdef{fghj}
$$
et d'autre part que 
$$
\exp\B(\varphi\e^{\phi(u)}\B)=\Q(c+{g(u)\F\e^u}\W)^\varphi\exp(\e^u)
\qquad(u\ge0).  
$$
En reportant dans \eqref{eqfonc}, il suit 
$$
\Q(c+{g(u)\F\e^u}\W)^\varphi
\Bg({c\F\varphi}+\int_0^{\phi(u)}g(x){\exp(\varphi\e^x)\F\exp\b(\varphi\e^{\phi(u)}\b)}\d x\Bg)=1+E(u)
\qquad(u\ge0).
$$
Nous d\'eduisons alors de \eqref{defga} et \eqref{fghj} d'une part que 
$$
\Q(c+{g(u)\F\e^u}\W)^\varphi\Bg({c\F\varphi}+{G\b(\phi(u)\b)\F \e^u+\varphi\log\b(c+g(u)\e^{-u}\b)}\Bg)
=1+E(u)
\qquad(u\ge0). 
$$ 
et d'autre part que 
$$
G\b(\phi(u)\b)=\Bg({1+E(u)\F\b(c+g(u)\e^{-u}\b)^\varphi}-{c\F \varphi}\Bg)
\Q(\e^u+\varphi\log\Q(c+{g(u)\F\e^u}\W)\W)
\qquad(u\ge0). 
\eqdef{Larela}
$$
Comme l'\'egalit\'e \eqref{Phi4} implique que  
$$
{1\F(c+u)^\varphi}={c\F\varphi}-u+O\b(u^2\b)
\qquad(u\to0), 
\eqdef{relinte}
$$
il r\'esulte alors des estimations \eqref{Kort5} et \eqref{Kor4} que 
$$
G\b(\phi(u)\b)=\Q(-{g(u)\F\e^u}+O\Q({g(u)^2\F\e^{2u}}+\e^{-\theta u}\W)\W)
\b(\e^u+O(1)\b)
\qquad(u\to\infty). 
$$
En particulier, nous obtenons 
$$
g(u)+G\b(\phi(u)\b)\ll{\b|g(u)\b|+g(u)^2\F\e^u}+\e^{(1-\theta)u}
\qquad(u\to\infty).
$$
Comme les fonctions $g$ et $G\circ\phi$ sont continues sur $[0,+\infty[$, 
nous en d\'eduisons \eqref{equaqua}. 
\bigskip


Enfin, prouvons que $g$ est born\'ee sur l'intervalle $[0,+\infty[$. 
Nous notons $\{h_{m,n}\}_{(m+n)\in\ob N^2}$ la famille de fonctions d\'efinies par \eqref{hmn} et nous posons 
$$
M_n(u):=\sup_{0\le x\le u}\b|g^{(n)}(x)\b|
\qquad(0\le n\le \Theta+1,u\ge0). 
\eqdef{defMn}
$$
La fonction $G:[0,+\infty[\to\ob R$ \'etant d\'efinie par \eqref{defga} et continue, 
nous remarquons qu'elle satisfait \eqref{dl1eq1} pour le triplet d'entiers $(k,\ell,m)=(0,0,0)$ 
et pour  $(f^*,g^*)=(g,G)$. 
En~appliquant le lemme \eqrefn{dl1}, nous d\'eduisons alors de la relation  \eqref{dl1eq2} que 
$$
\b|G(x)\b|\le \b|h_{0,0}(0)\b|M_0(x)+O\Q({M_0(x)\F\e^x}\W)
\qquad(x\ge0). 
$$
Comme \eqref{hmn} implique que $h_{0,0}(0)=1$ et comme les relations \eqref{defphi} et $f(t)\ge\e\ \,(t\ge\e)$ 
impliquent que $\phi(u)\ge0\ \,(u\ge0)$, nous remarquons que 
$$
\B|G\b(\phi(u)\b)\B|\le M_0\b(\phi(u)\b)+O\Q({M_0\b(\phi(u)\b)\F\e^{\phi(u)}}\W)\qquad(u\ge0).
$$
La fonction positive $M_0$ \'etant croissante, nous d\'eduisons de l'in\'egalit\'e $1/3<\lambda<1$ et de~l'identit\'e \eqref{Kort5} 
qu'il existe un~nombre $\alpha\ge1$ pour lequel 
$$
u-1\le \phi(u)\le u-1/3\ \,(u\ge \alpha)\eqdef{Slaughter}
$$ 
et donc que 
$$
\b|G\b(\phi(u)\b)\b|\le M_0(u-1/3)
+O\Q({1+M_0(u)\F\e^u}\W)
\qquad(u\ge1).
$$
Reportant cette in\'egalit\'e dans \eqref{equaqua}, il r\'esulte de la d\'efinition \eqref{defMn} que   
$$
\b|g(u)\b|\le M_0(u-1/3)
+O\Q({\b(1+M_0(u)\b)^2\F\e^u}+\e^{(1-\theta)u}\W)
\qquad(u\ge1).
$$
Enfin, comme l'application continue $g:[0,+\infty[\to\ob R$ satisfait \eqref{kmnj} et \eqref{Corcyre}, 
nous~d\'e\-dui\-sons du~lemme \eqrefn{L3} qu'elle est born\'ee sur l'intervalle $[0,\infty[$. 
\hfill\qed
\bigskip


\Sectio DL2, Estimation de la fonction $f$ \`a l'ordre $2$.


Dans ce paragraphe, nous prouvons la majoration \eqref{first} par r\'ecurrence puis nous construisons une fonction $\nu_1\in\sc C^{\Theta-1}\b(\ob R\b)$ 
v\'erifiant  \eqref{n+tn} et~\eqref{audre}.  Pour~\'etablir~\eqref{first}, nous proc\'edons de la fa\c{c}on suivante : 
la~fonction~$g$ \'etant born\'ee sur l'intervalle $[0,+\infty[$,  elle satisfait $g(u)\olop_ 01\ \,(u\ge0)$. Alors, nous fixons un entier $n\in[1,\Theta]$ v\'erifiant 
$$
g(u)\olop_ {n-1}1\qquad(u\ge0), 
\eqdef{hehehehe}%=dl2eq1
$$
nous posons $\alpha:=\min\{1,\theta-1\}$ et nous montrons que la fonction $g$ satisfait la majoration 
$$
g^{(n)}(u)+G^{(n)}\b(\phi(u)\b)\ll{M_n(u)\F\e^{\alpha u}}+\e^{-u}+\e^{(n+1-\theta)u}
\qquad(u\ge \lambda), 
\eqdef{atrouver}
$$
les fonctions $G$ et $M_n$ \'etant d\'efinies par \eqref{defga} et \eqref{defMn}, puis nous en d\'eduisons que 
$$
g^{(n)}(u)\ll \Q\{\eqalign{&1\qquad\qquad\qquad\qquad\quad\hbox{si } n<\Theta\cr
&1+u^{\1_{\ob Z}(\theta)}\e^{(\Theta+1-\theta)u}\qquad\hbox{si }n=\Theta}\W.\qquad(u\ge0). \eqdef{lalalo}
$$ 
Nous d\'eduisons alors de l'hypoth\`ese \eqref{hehehehe} que  
$$
\Q\{\eqalign{&g(u)\olop_ n1 \qquad\hbox{si }n<\Theta\cr&g(u)\olops_ n1\qquad\hbox{si }n=\Theta}\W.\qquad(u\ge0). \eqdef{huhuhu}
$$ 
Par r\'ecurrence, nous concluons alors que la majoration \eqref{first} est satisfaite. 
\bigskip


\'Etant donn\'e un entier $n\in[1,\Theta]$ v\'erifiant  \eqref{hehehehe}, prouvons que la fonction $g$ v\'erifie~\eqref{atrouver}. 
Comme $f\in\sc C([\e,+\infty[,[\e,+\infty[$, nous d\'eduisons de \eqref{Phi1} et de \eqref{defg}  que 
$$
f(t)=\Q(c+{g(\log_2t)\F \log t}\W)t^{1/\varphi}\ge\e\qquad(t\ge\e).
$$
{\it A fortiori}, la relation \eqref{changer} r\'esulte du changement de variable $t=\exp(\e^u)$. 
Comme $g$ est de classe $\sc C^n$ sur $[0,+\infty[$ et satisfait \eqref{hehehehe}, 
nous pouvons lui appliquer le lemme~\eqrefn{dl2}. 
Nous~notons $W$ la fonction d\'efinie par  
$$
W(u):=g(u)+G\b(\phi(u)\b)\qquad(u\ge0)\eqdef{Deathpsi}
$$
et nous d\'eduisons alors de \eqref{dl2eq3} que 
$$
W^{(n)}(u)=g^{(n)}(u)+G^{(n)}\b(\phi(u)\b)+O\Q({1+M_n(u)\F\e^u}\W)\qquad(u\ge0).
\eqdef{besoin}
$$
Par ailleurs, en notant $G_1$ et $G_2$ les fonctions analytiques d\'efinies par 
$$
\eqalign{
G_1(y,z)&:=z+\Q((c+z)^{-\varphi}-c/\varphi\W)\b(1+\varphi y\log(c+z)\b)\qquad
(y\in\ob R, z>-c)_{\strut},
\cr
G_2(y,z)&:=(c+z)^{-\varphi}\big(1+\varphi y\log(c+z)\big)\qquad\qquad\qquad\quad\ \,(y\in\ob R,z>-c),
}
$$
nous d\'eduisons de \eqref{Larela} et \eqref{Deathpsi} que 
$$
W(u)=\e^uG_1\Q({1\F\e^u},{g(u)\F\e^u}\W)+\e^uE(u)G_2\Q({1\F\e^u},{g(u)\F\e^u}\W)
\qquad(u\ge0). 
$$
Pour $u\ge0$, la formule de Leibniz g\'en\'eralis\'ee induit alors que 
$$
W^{(n)}(u)=\sum_{k+\ell=n}{n!\e^u\F k!\ell!}{\d^\ell\hfill\F\d u^\ell}G_1\!\Q({1\F\e^u},{g(u)\F\e^u}\W)
+\!\!\sum_{j+k+\ell=n}\!{n!\e^u\F j!k!\ell!}E^{(k)}(u){\d^\ell\hfill\F\d u^\ell}G_2\!\Q({1\F\e^u},{g(u)\F\e^u}\W).
$$
Nous d\'eduisons des relations \eqref{changer} et de \eqref{hehehehe} 
que la fonction $g\in\sc C^n\b([0,+\infty[\b)$ v\'erifie les in\'equations \eqref{dfc6eq1}~et~\eqref{cont}. 
Comme \eqref{relinte} implique que $\partial^{(m,n)}G_1(0,0)=0\ \,(m+n\le 1)$, 
nous pouvons alors appliquer le~lemme \eqrefn{dfc6} aux fonctions $G_1$ et $g$ pour l'entier $s=2$  
et nous en d\'eduisons d'une part que $G_1(\e^{-u},g(u)\e^{-u})\olop_ {n-1}\e^{-2u}\ \,(u\ge0)$ 
et d'autre part que  
$$
{\d^n\hfill\F\d u^n}G_1\Q({1\F\e^u},{g(u)\F\e^u}\W)\ll {1+M_n(u)\F\e^{2u}}\qquad(u\ge0). 
$$
{\it A fortiori}, nous obtenons que 
$$
W^{(n)}(u)\ll{1+M_n(u)\F\e^u}
+\sum_{j+k+\ell=n}\Q|\e^uE^{(k)}(u){\d^\ell\hfill\F\d u^\ell}G_2\!\Q({1\F\e^u},{g(u)\F\e^u}\W)\W|
\qquad(u\ge0).
$$
En posant $b:=c\e^\varphi/\varphi-\e+\int_\e^{f(\e)}f(x)\d x$, 
nous d\'eduisons de \eqref{Emper} que $E$ satisfait  
$$
b+\int_\e^t\b(f'(x)f\circ f(x)-1\b)\d x=tE(\log_2t)
\qquad(t\ge\e). 
$$
D'apr\`es le lemme \eqrefn{dfc7} appliqu\'e pour l'entier $\Theta$ \`a la fonction $f^*:=(f\circ f)f'-1$, qui~v\'erifie~\eqref{burning},
la fonction $E$ est alors de classe $\sc C^{\Theta+1}$ sur $[0,+\infty[$ et satisfait 
$$
E^{(k)}(u)\ll\e^{(k-\theta)u}
\qquad(0\le k\le\Theta+1,u\ge0).\eqdef{mrter}
$$
En particulier, nous en d\'eduisons que 
$$
W^{(n)}(u)\ll{1+M_n(u)\F\e^u}
+\sum_{k+\ell\le n}\Q|{\d^\ell\hfill\F\d u^\ell}G_2\!\Q({1\F\e^u},{g(u)\F\e^u}\W)\W|\e^{(k+1-\theta)u}
\qquad(u\ge0).
$$
En appliquant cette fois-ci le lemme \eqrefn{dfc6} aux fonctions $G_2$ et $g$ pour l'entier $s=0$, nous en d\'eduisons d'une part que 
$G_2(\e^{-u},g(u)\e^{-u})\olop_{n-1}1\ \,(u\ge0)$ et d'autre part que 
$$
{\d^n\hfill\F\d u^n}G_2\Q({1\F\e^u},{g(u)\F\e^u}\W)
\ll 1+M_n(u)
\qquad(u\ge0)  
$$
En particulier, nous obtenons alors que 
$$
W^{(n)}(u)\ll{1+M_n(u)\F\e^u}
+M_n(u)\e^{(1-\theta)u}+\e^{(n+1-\theta)u}
\qquad(u\ge0)
$$
et nous d\'eduisons la majoration \eqref{atrouver} des relations \eqref{besoin} et $\alpha:=\min\{1,\theta-1\}$. 
\bigskip


\'Etant donn\'e un entier $n\!\in\![1,\Theta]$ v\'erifiant  \eqref{hehehehe}, prouvons maintenant \eqref{lalalo}~et~\eqref{huhuhu}. 
La~fonction $G:[0,+\infty[\to\ob R$ \'etant d\'efinie par \eqref{defga}, 
nous rappelons qu'elle satisfait~\eqref{dl1eq1} pour le triplet d'entiers $(k,\ell,m)=(0,n,0)$ 
et pour $f^*=g\in\sc C^n\b([0,+\infty[\b)$ et $g^*=G$. Nous~pouvons alors appliquer le lemme \eqrefn{dl1} 
et nous d\'eduisons de \eqref{dl05eq3} et de \eqref{hehehehe} que 
$$
\b|G^{(n)}(x)\b|\le M_n(x)+O\Q({1+M_n(x)\F\e^x}\W)
\qquad(x\ge0). 
$$ 
En particulier, il r\'esulte de \eqref{Slaughter} et de la croissance de la fonction $M_n$ que 
$$
\b|G^{(n)}(\phi(u))\b|\le M_n(u-1/3)+O\Q({1+M_n(u)\F\e^u}\W)
\qquad(u\ge\alpha). 
$$
Cette estimation \'etant satisfaite pour $1\le u\le \alpha$, 
nous posons $\beta:=\max\{n+1-\theta,-1\}$ et nous d\'eduisons de \eqref{atrouver} que  
$$
\b|g^{(n)}(u)\b|\le M_n(u-1/3)+O\Q({M_n(u)\F\e^{\alpha u}}+\e^{\beta u}\W)
\qquad(u\ge1).
$$
En particulier,  la fonction continue $f^*=g^{(n)}$ satisfait la relation \eqref{tienstiens} pour $M^*=M_n$. 
Comme~$\alpha=\min\{1,\theta-1\}>0$ et comme le nombre $\beta=\max\{n+1-\theta,-1\}$ satisfait 
$$
\Q\{\eqalign{
&\beta<0\qquad\qquad\qquad\qquad\!\!(n<\Theta),
\cr
&\beta=\Theta+1-\theta=0\qquad(n=\Theta,\theta\in\ob N),
\cr
&\beta=\Theta+1-\theta>0\qquad(n=\Theta,\theta\notin\ob N),
}\W.
$$
nous d\'eduisons alors du lemme \eqrefn{L4} que la majoration \eqref{lalalo} est satisfaite. 
Comme~$g$ v\'erifie~\eqref{hehehehe} et \eqref{lalalo}, 
il r\'esulte alors des d\'efinitions \eqref{var3a} et \eqref{var3b} qu'elle satisfait \eqref{huhuhu}. 
\bigskip


En proc\'edant par r\'ecurrence sur $n$, nous d\'eduisons \eqref{first} des r\'esultats pr\'ec\'edents. Pour~finir, 
construisons une application $\nu_1\in\sc C^{\Theta-1}\b(\ob R\b)$ satisfaisant \eqref{n+tn} et~\eqref{audre}. 
Comme~la majoration \eqref{first} implique que l'estimation \eqref{hehehehe} est v\'erifi\'ee pour $1\le n\le \Theta$, 
nous~remarquons que l'estimation \eqref{atrouver} est satisfaite pour $1\le n\le\Theta$. 
L'application~$g$ \'etant born\'ee, il~r\'esulte~de \eqref{equaqua} d'une part que 
$$
g(u)+G\b(\phi(u)\b)\ll\e^{-u}+\e^{(1-\theta)u}
\qquad(u\ge0)
$$
et d'autre part que la relation \eqref{atrouver} est v\'erifi\'ee pour chaque entier $n\in\{0,\cdots, \Theta\}$. Comme~$G\in\sc C^\Theta\b([0,+\infty[\b)$, 
nous en d\'eduisons que 
$$
G^{(n)}\b(\phi(u)\b)=G^{(n)}(u-\lambda)+\int_{u-\lambda}^{\phi(u)}G^{(n+1)}(x)\d x
\qquad(0\le n<\Theta,u\ge\lambda). \eqdef{Axle}
$$
Comme $g$ est de classe $\sc C^\Theta$ sur $[0,+\infty[$ et comme la~fonction $G$ est d\'efinie par \eqref{defga}, 
nous~pouvons appliquer le lemme \eqrefn{dl1} pour le triplet d'entiers $(k,\ell,m)=(0,\Theta,0)$ aux~fonctions $f^*=g$ et $g^*=G$
et nous d\'eduisons alors de la majoration \eqref{dl1eq2} que 
$$
G^{(n)}(x)\ll\sum_{0\le j\le n}M_j(x)\qquad(0\le n\le\Theta ,x\ge0). 
$$ 
{\it A fortiori}, il r\'esulte des relations \eqref{dl05eq3} et \eqref{first} que 
$$
G(x)\olops_\Theta1\qquad(x\ge0). 
$$
Nous d\'eduisons alors de l'estimation \eqref{Kort5} que 
$$
\int_{u-\lambda}^{\phi(u)}G^{(n+1)}(x)\d x
\ll\Q\{
\eqalign{&\e^{-u}\qquad\qquad\ \ \!(0\le n<\Theta-1,u\ge\lambda),\cr &u^{\1_{\ob Z}(\theta)}\e^{(\Theta-\theta)u}\quad\qquad(n=\Theta-1,u\ge\lambda).}\W.
$$
En reportant dans \eqref{Axle}, nous d\'eduisons alors de \eqref{first} et de \eqref{atrouver} que 
$$
g^{(n)}(u)+G^{(n)}(u-\lambda)\ll\e^{-\alpha u}+\e^{-u}+u^{\1_{\ob Z}(\theta)}\e^{(n+1-\theta)u}
\qquad(0\le n<\Theta, u\ge \lambda). 
$$
Lorsque le minimum $\alpha:=\min\{1,\theta-1\}$ n'est pas \'egal \`a $1$, nous~remarquons que $1<\theta<2$ 
et nous~d\'eduisons de la d\'efinition \eqref{Theta1} que $u^{\1_{\ob Z}(\theta)}\e^{(n+1-\theta)}=\e^{-\alpha u}\ \,(0\le n<\Theta,u\ge0)$. 
En~particulier, nous~obtenons que 
$$
g^{(n)}(u)+G^{(n)}(u-\lambda )\ll \e^{-u}+u^{\1_{\ob Z}(\theta)}\e^{(n+1-\theta)u}
\qquad(0\le n<\Theta,u\ge \lambda ).\eqdef{cradle}
$$
Comme $g$ est de classe $\sc C^\Theta$ sur $[0,+\infty[$ et comme la~fonction $G$ est d\'efinie par \eqref{defga}, 
nous pouvons appliquer le lemme \eqrefn{dl1} pour le triplet d'entiers $(k,\ell,m)=(1,\Theta-1,0)$ 
\`a la~fonction $f^*=g$ et \`a l'application $g^*$ implicitement d\'efinie par 
$$
G(x)=h_{0,0}(0)g(x)+{g^*(x)\F\e^x}\qquad(x\ge0). 
$$
Comme $h_{0,0}(0)=1$ d'apr\`es \eqref{hmn}, nous d\'eduisons alors des relations \eqref{dl1eq2} et \eqref{first} que 
$$
G(x)=g(x)+\scos_{\Theta-1}\Q(\e^{-x}\W)\qquad(x\ge0) \eqdef{LordBelial}
$$
En particulier, nous d\'eduisons de \eqref{cradle} que 
$$
g(u)+g(u-\lambda)\olops_ {\Theta-1}\e^{-u}\qquad(u\ge\lambda). 
$$
{\it A fortiori}, nous pouvons appliquer le lemme \eqrefn{L5} pour les nombres $\theta$  et $\Theta$ \`a la fonction $g$. 
Alors, nous~en d\'eduisons 
que la limite  
$$
\nu_1(u):=\lim_{n\to\infty}g(u+2n\lambda)\eqdef{defnu11}
$$
existe pour chaque nombre r\'eel $u\in\ob R$ et qu'elle d\'efinit une fonction $\nu_1\in\sc C^{\Theta-1}(\ob R)$ v\'erifiant les relations \eqref{n+tn} 
et~\eqref{audre}. 
\hfill\qed
\bigskip


\Sectio DLn, Estimation de la fonction $f$ \`a l'ordre $n$. 


Dans les paragraphes suivants, pour $2\le k\le\Theta$, nous construisons 
une application $2\lambda$-p\'eriodique $\nu_k\in\sc C^{\Theta-k}(\ob R)$,  qui est une combinaisons lin\'eaires de produits dont les facteurs sont pris parmi les fonctions~\eqref{oomph} 
ou~parmi $\nu_1,\cdots,\nu_1^{(k-1)}$, v\'erifiant \eqref{rec} et
$$
g(u)=\sum_{0\le m<k}{\nu_{m+1}(u)\F\e^{mu}}+\scos_{\Theta-k}\Q(\e^{-ku}\W)\qquad(u\ge0), 
\eqdef{KOL}
$$
ce qui suffit pour \'etablir le th\'eor\`eme \eqrefn{T3faible} d'apr\`es l'argument avanc\'e au paragraphe \CitSec{Plan2}. 
\bigskip


Pour construire ces applications, nous proc\'edons par r\'ecurrence de la fa\c{c}on suivante : 
l'estimation~\eqref{KOL} \'etant \'etablie au paragraphe \CitSec{DL2} pour les entiers $k=0$~et~$k=1$, 
d'apr\`es les relations~\eqref{first}~et~\eqref{audre}, nous~fixons un~entier $K\in[2,\Theta]$ 
et  des fonctions $2\lambda$-p\'eriodiques $\nu_k\in\sc C^{\Theta-k}(\ob R)\ \,(1\le k<K)$ v\'erifiant~la relation \eqref{KOL} pour $0\le k<K$. 
Puis, nous notons $\sc Y:=\{\sc Y_{m,n}\}_{m+n<K,m+1<K}$ la famille de fonctions d\'efinie par 
$$
\sc Y_{m,n}(u):=\nu_{m+1}^{(n)}(u+\lambda)\qquad(m+n<K,m+1<K,u\in\ob R)  
\eqdef{Ymn}
$$
et nous nous munissons des polyn\^omes $P_k\in\ob R[\{X_{m,n}:m+n<k\hbox{ et }m+1<k\}\b]\ \,(k\ge1)$ 
uniquement  d\'etermin\'es par l'identit\'e \eqref{Dimmu Borgir}.  Alors, nous posons  
$$
G_K^*(u)=P_K\b[\sc Y(u)\b]-{1\F\varphi}\sum_{\ss m+n<K\atop\ss m+1<K}c_{k,m,n}\,\sc Y_{m,n}(u)\qquad(u\in\ob R), 
\eqdef{Gketoile}
$$ 
et nous notons $\nu_K$ l'applications $2\lambda$-p\'eriodique, de classe $\sc C^{\Theta-K}$ sur $\ob R$, d\'efinie par 
$$
\nu_K(u)={\varphi^KG_K^*(u)-\varphi G_K^*(u+\lambda)\F\varphi^{K-1}-\varphi^{1-K}}
\qquad(u\in\ob R). \eqdef{nuK}
$$
\'Etant~donn\'ees les applications $\mu_1,\cdots, \mu_K, \mu_1^*,\cdots,\mu_K^*, G_1,\cdots,G_k$ respectivement d\'efinies par \eqref{mu1}, \eqref{mu2} et \eqref{IcedEarth} pour $1\le k\le K$, 
nous \'etablissons \eqref{rec} pour $2\le k\le K$, nous~en d\'eduisons que $G_k$ et $\nu_k$ sont des combinaisons lin\'eaires de produits dont les facteurs sont pris parmi les fonctions \eqref{oomph} 
ou parmi $\nu_1,\cdots,\nu_1^{k-1}$ pour $2\le k\le K$ et nous prouvons que la fonction $\nu_K$ satisfait la relation \eqref{KOL} pour $k=K$. 
Par r\'ecurrence, la famille $\nu_2,\cdots, \nu_\Theta$ ainsi construite poss\`ede alors les propri\'et\'es requises. 
\bigskip


Fixons un entier $K\in[2,,\Theta\}$, des fonctions $2\lambda$-p\'eriodiques $\nu_k\in\sc C^{\Theta-k}(\ob R)\ \,(1\le k<K)$   
v\'erifiant l'estimation \eqref{KOL} pour $0\le k<K$ et prouvons que l'identit\'e \eqref{nuK} d\'efinit bien une application $2\lambda$-p\'eriodique $\nu_K\in\sc C^{\Theta-K}(\ob R)$. 
D'apr\`es \eqref{Ymn}, nous avons que 
$$
\sc Y_{m,n}\in\sc C^{\Theta-K}(\ob R)\qquad(m+n<K,m+1<K). 
$$
Comme le lemme \eqrefn{invfc0} induit que \eqref{Dimmu Borgir} d\'etermine bien une unique suite de polyn\^ome 
$$
P_k\in\ob R\b[\{X_{m,n}:m+n<k\hbox{ et }m+1<k\}\b]\qquad(k\ge1), \eqdef{Pkespace}
$$
nous remarquons que l'application $G_K^*$ est bien d\'efinie par \eqref{Gketoile}, qu'elle est $2\lambda$-p\'eriodique et de~classe $\sc C^{\Theta-K}$ sur l'intervalle $\ob R$. 
{\it A fortiori}, la fonction $\nu_K$ est bien d\'efinie par \eqref{nuK}. De plus, elle  est $2\lambda$-p\'eriodique et  appartient \`a l'espace $\sc C^{\Theta-K}\b(\ob R\b)$. 
\bigskip


Soient $\mu_1,\cdots, \mu_K$ les applications d\'efinies par \eqref{mu1}, $\mu_1^*,\cdots,\mu_K^*$ les fonctions d\'efinies par~l'identit\'e \eqref{mu2} 
et $G_1,\cdots,G_k$ les applications d\'etermin\'ees par \eqref{IcedEarth} pour $1\le k\le K$. 
Prouvons maintenant que~l'identit\'e \eqref{rec} est satisfaite par $\nu_K$ pour l'entier $k=K$.
Pour~$1\le k\le K$, les relations \eqref{mu1} et $c_{k,k-1,0}=1$ impliquent que 
$$
\mu_k(u)={\nu_k(u)\F\varphi}+{1\F\varphi}\sum_{\ss m+n<k\atop\ss m+1<k}c_{k,m,n}\,\nu_{m+1}^{(n)}(u)\qquad(k\ge1,u\in\ob R)
$$
et nous d\'eduisons de l'identit\'e \eqref{IcedEarth} que 
$$
G_k(u)=\mu_k^*(u)+{\nu_k(u+\lambda)\F\varphi^k}-{1\F\varphi}\sum_{\ss m+n<k\atop\ss m+1<k}c_{k,m,n}\,\nu_{m+1}^{(n)}(u)
\qquad(u\in\ob R).   
$$ 
Soient $\{q_{m,n}\}_{(m,n)\in\ob N^2}$ la famille de fonctions holomorphes d\'efinie par \eqref{qmn} 
et $\Lambda_{K,\sc X}$ la fonction holomorphe d\'etermin\'ee par \eqref{Lake1} 
pour chaque famille $\sc X=\{\sc X_{m,n}\}_{m+n<K}$ de nombres complexes. 
Nous notons  $\sc X(x):=\{X_{m,n}(x)\}_{m+n<K}$ la famille de nombres complexes d\'efinie par 
$$
\sc X_{m,n}(x):=\nu_{m+1}^{(n)}(x+\lambda)\qquad(m+n<K,x\in\ob R). \eqdef{KingD}
$$  
et nous observons que l'identit\'e \eqref{mu2} peut se mettre sous la forme 
$$
\mu_k^*(x)={1\F 2\pi i}\oint\limits_{|s|=1}{1\F k!}{\partial^k\F\partial x^k}\Q(
{\partial_s\Lambda_{K,\sc X(x)}\F \Lambda_{K,\sc X(x)}}\W)(0,s)\e^{-\varphi s}\d s\qquad(1\le k\le K,x\in\ob R). 
\eqdef{mu3}
$$
Pour $1\le k\le K$ et $u\in\ob R$, nous obtenons en particulier que  
$$
G_k(u)={1\F 2\pi i}\oint\limits_{|s|=1}{1\F k!}{\partial^k\F\partial x^k}\Q(
{\partial_s\Lambda_{K,\sc X(u)}\F \Lambda_{K,\sc X(u)}}\W)(0,s){\d s\F \e^{\varphi s}}
+{\nu_k(u+\lambda)\F\varphi^k}-{1\F\varphi}\sum_{\ss m+n<k\atop\ss m+1<k}c_{k,m,n}\,\nu_{m+1}^{(n)}(u). 
$$ 
et nous appliquons le lemme \eqrefn{invfc0} \`a l'entier $k$ pour d\'eduire de \eqref{Dimmu Borgir} que 
$$
G_k(u)=P_k\b[\sc X(u)\b]-{1\F\varphi}\sum_{\ss m+n<k\atop\ss m+1<k}c_{k,m,n}\,\nu_{m+1}^{(n)}(u)
\qquad(1\le k\le K,u\in\ob R). \eqdef{SeewhatIsee}
$$ 
Comme $\sc X_{m,n}=\sc Y_{m,n}$ pour $m+n<K$ et $m+1<K$, nous d\'eduisons de \eqref{Pkespace} que   
$$
P\b[\sc X(u)\b]=P\b[\sc Y(u)\b]\qquad(u\in\ob R)
$$ 
et nous d\'eduisons de \eqref{Gketoile} que 
$$
G_K(u)=G_K^*(u)\qquad(u\in\ob R).
$$
En reportant dans \eqref{nuK}, nous concluons que $\nu_K$ satisfait \eqref{rec} pour l'entier $k=K$. 
\bigskip

Pour prouver que $\nu_1,\cdots,\nu_K$ v\'erifient \eqref{KOL} pour $0\le k\le K$ et \eqref{rec}~pour~\hbox{$2\le k\le K$}, 
il~nous~reste~\`a d\'emontrer \eqref{KOL} pour l'entier $k=K$ et l'identit\'e \eqref{rec} pour $2\le k<K$. 
Pour cela,  il suffit d'\'etablir que les fonctions $g_k \in\sc C^{\Theta-k}\b([0,+\infty[\b)\ \,(0\le k\le K)$ d\'efinies~par 
$$
g(u)=\sum_{0\le m<k}{\nu_{m+1}(u)\F\e^{mu}}+{g_k(u)\F\e^{ku}}
\qquad(0\le k\le K,u\ge0), 
\eqdef{DeathOL}
$$
satisfont les estimations 
$$
\eqalignno{
G(u)&=\varphi\sum_{0\le k<K}{\mu_{k+1}(u)\F\e^{ku}}+{g_K(u)\F\e^{Ku}}+\scos_{\Theta-K}\Q(\e^{-Ku}\W)
\qquad\qquad(u\ge0), 
&\eqdef{Carcass}
\cr 
&=\varphi\sum_{0\le k<K}{\mu_{k+1}^*(u)\F\e^{ku}}-{g_K(u+\lambda)\F\varphi^K\e^{Ku}}+\scos_{\Theta-K}\Q(\e^{-Ku}\W)
\qquad(u\ge0). 
&\eqdef{Carcariass}
\cr 
}
$$
En effet, nous prouvons que les estimations \eqref{Carcass} et \eqref{Carcariass} impliquent d'une part que 
$$
\mu_k(u)=\mu_k^*(u)\qquad(1\le k<K, u\in\ob R)
\eqdef{Blonde}
$$
et d'autre part que la fonction $g_K$ satisfait l'\'equation aux diff\'erences approch\'ee
$$
g_K(u)+\varphi^{-K}g_K(u+\lambda)\olops_{\Theta-K}1\qquad(u\ge0).
\eqdef{Brune}
$$
Puis, nous d\'emontrons que la relations \eqref{Blonde} induit l'identit\'e \eqref{rec} pour $2\le k<K$ et 
que~\eqref{Brune} induit l'estimation  \eqref{KOL} pour l'entier $k=K$. 
Enfin, nous \'etablissons les~relations~\eqref{Carcass} et \eqref{Carcariass}, 
ce~qui~~conclut la preuve du  th\'eor\`eme \eqrefn{T3faible}. 
\bigskip


Prouvons que les estimations \eqref{Carcass} et \eqref{Carcariass} induisent l'identit\'e \eqref{Blonde}. 
Nous supposons que les~relations \eqref{Carcass} et \eqref{Carcariass} sont satisfaites et nous remarquons alors que 
$$
\sum_{0\le k<K}{\mu_{k+1}(u)-\mu_{k+1}^*(u)\F\e^{ku}}+{g_K(u)+\varphi^{-K}g_K(u+\lambda)\F\e^{Ku}}\olops_{\Theta-K}\e^{-Ku}
\qquad(u\ge0). \eqdef{Lakedaemon}
$$
La relation \eqref{KOL} \'etant v\'erifi\'ee pour $0\le k<K$, d'apr\`es l'hypoth\`ese de r\'ecurrence, 
nous~d\'eduisons de la d\'efinition \eqref{DeathOL} d'une part que 
$$
{g_k(u)\F\e^{ku}}\olops_ {\Theta-k}\e^{-ku}\qquad(0\le k<K,u\ge0)
$$
et d'autre part que 
$$
g_k(u)\olops_ {\Theta-k}1\qquad(0\le k<K,u\ge0). 
$$
Comme $\nu_K\in\sc C^{\Theta-K}(\ob R)$ est $2\lambda$-p\'eriodique et comme \eqref{DeathOL} implique que 
$$
g_K(u)=-\nu_K(u)\e^u+g_{K-1}(u)\e^u\qquad(u\ge0), 
$$ 
il en r\'esulte alors que 
$$
g_K(u)\olop_ {\Theta-K}\e^u\qquad(u\ge0). \eqdef{Rahhh}
$$
En reportant dans \eqref{Lakedaemon}, nous d\'eduisons alors de l'in\'egalit\'e $2\le K\le\Theta$ que 
$$
\sum_{0\le k<K}{\mu_{k+1}(u)-\mu_{k+1}^*(u)\F\e^{ku}}\ll \e^{-(K-1)u}\qquad(u\ge0). 
$$
Les applications $\mu_{k+1}$ et $\mu_{k+1}^*$ \'etant $2\lambda$-p\'eriodiques pour $0\le k<K$, nous remarquons qu'il en est de m\^eme 
pour la fonction $u\mapsto\mu_{k+1}(u)-\mu_{k+1}^*(u)$ et nous d\'eduisons de la majoration pr\'ec\'edente 
que l'identit\'e \eqref{Blonde} est n\'ecessairement satisfaite. En effet, une~fonction p\'eriodique de limite nulle en $+\infty$ est identiquement nulle. 
\bigskip


Prouvons maintenant que \eqref{Carcass} et \eqref{Carcariass} induisent l'identit\'e \eqref{Brune}. 
Nous~supposons que les~estimations \eqref{Carcass} et \eqref{Carcariass} sont satisfaites et nous en d\'eduisons \eqref{Blonde} et \eqref{Lakedaemon}. En les combinant, nous observons alors que  
$$
{\mu_K(u)-\mu_K^*(u)\F\e^{(K-1)u}}+{g_K(u)+\varphi^{-K}g_K(u+\lambda)\F\e^{Ku}}\olops_ {\Theta-K}\e^{-Ku}\qquad(u\ge0).  \eqdef{Loudblast}
$$
Nous rappelons que l'application $G_K$ est $2\lambda$-p\'eriodique, que \eqref{rec} est satisfaite pour l'entier $k=K$ 
et nous en d\'eduisons d'une part que 
$$
\nu_K(u+\lambda)={\varphi^ KG_k(u+\lambda)-\varphi G_K(u)\F\varphi^{K-1}-\varphi^{1-K}}
\qquad(u\in\ob R)
$$
et d'autre part que $\nu_K$ est une solution de l'\'equation fonctionnelle \eqref{metal} pour $k=K$. 
Nous remarquons alors que la~d\'efinition \eqref{IcedEarth} de la fonction $G_K$ induit l'\'egalit\'e $\mu_K=\mu_K^*$, 
que~nous~reportons dans \eqref{Loudblast} pour obtenir que 
$$
{g_K(u)+\varphi^{-K}g_K(u+\lambda)\F\e^{Ku}}\olops_ {\Theta-K}\e^{-Ku}\qquad(u\ge0) 
$$
et par suite que l'estimation \eqref{Brune} est satisfaite. 
\bigskip



Prouvons maintenant que l'identit\'e \eqref{Blonde} induit l'estimation \eqref{rec} pour $2\le k<K$. Nous~supposons que l'identit\'e \eqref{Blonde} est v\'erifi\'ee, 
nous la reportons dans la d\'efinition \eqref{IcedEarth} et nous en d\'eduisons que la fonction~$\nu_k$ est une solution 
de l'\'equation fonctionnelle \eqref{metal} pour $2\le k<K$. L'application $\nu_k$ \'etant $2\lambda$-p\'eriodique, nous 
substituons alors $u+\lambda$ \`a $u$ dans \eqref{metal} pour obtenir que 
$$
{\nu_k(u+\lambda)\F\varphi}+{\nu_k(u)\F\varphi^k}=G_k(u+\lambda)\qquad(2\le k<K,u\in\ob R)
$$
et nous d\'eduisons du syst\`eme form\'e par \eqref{metal} et par la relation pr\'ec\'edente que \eqref{rec} est~satisfaite 
pour $2\le k<K$.
\bigskip


Enfin, montrons que la relation \eqref{Brune} induit  l'estimation  \eqref{KOL} pour l'entier $k=K$. Nous~supposons que 
la fonction $g_K$ est une solution de l'\'equation aux diff\'erences~\eqref{Brune}. Pour $L\ge0$ et $u\ge0$,  
nous d\'eduisons du principe des sommes t\'elescopiques que 
$$
g_K(u)=(-1)^L\ {g_K(u+L\lambda)\F\varphi^{LK}}+
\sum_{0\le\ell<L}(-1)^\ell\ {g_K(u+\ell\lambda)+\varphi^{-K}g_K(u+\ell\lambda+\lambda)\F\varphi^{\ell K}}.
$$
Nous rappelons que la majoration \eqref{Rahhh}, qui d\'ecoule de \eqref{KOL} pour $0\le k<K$, 
est~v\'erifi\'ee et nous d\'eduisons de \eqref{Brune} et de \eqref{Rahhh} d'une part que 
$$
g_K(u)\olop_ {\Theta-K-1} {\e^{u+L\lambda}\F\varphi^{LK}}+
\sum_{0\le\ell<L}\varphi^{-\ell K}
\qquad(u\ge0, L\ge0)\eqdef{At}
$$
et d'autre part que 
$$
g_K^{(\Theta-K)}(u)\ll {\e^{u+L\lambda}\F\varphi^{LK}}+
\sum_{0\le\ell<L}{1+(u+\ell\lambda)^{\1_{\ob Z}(\theta)}\e^{(\Theta+1-\theta)(u+\ell\lambda)}
\F\varphi^{\ell K}}
\qquad(u\ge0, L\ge0). 
$$
Le nombre $\1_{\ob Z}(\theta)$ \'etant \'egal \`a $0$ ou \`a $1$, nous observons que 
$$
(u+\ell\lambda)^{\1_{\ob Z}(\theta)}\le u^{\1_{\ob Z}(\theta)}+\lambda\ell\qquad(u\ge0,\ell\ge0).
$$
et donc que 
$$
g_K^{(\Theta-K)}(u)\ll {\e^{u+L\lambda}\F\varphi^{LK}}+
\!\!\sum_{0\le\ell<L}\!\!{1+(u^{\1_{\ob Z}(\theta)}+\ell\lambda)\e^{(\Theta+1-\theta)(u+\ell\lambda)}
\F\varphi^{\ell K}}
\quad\ (u\ge0, L\ge0).\!\!\!\!\eqdef{The}
$$
Comme $\lambda=\log\varphi$ et comme $0\le\Theta+1-\theta\le 1<K$, nous remarquons que les s\'eries 
$$
\sum_{\ell\ge0}\varphi^{-LK}, \qquad 
\sum_{\ell\ge0}{e^{(\Theta+1-\theta)\ell\lambda}\F \varphi^{\ell K}}, \quad \hbox{et}\quad
\sum_{\ell\ge0}{\ell\e^{(\Theta+1-\theta)\ell\lambda}\F\varphi^{\ell K}}
$$ 
sont convergentes et nous faisons tendre le param\`etre $L$ vers $+\infty$ dans les majorations uniformes \eqref{At}~et~\eqref{The} pour obtenir d'une part que 
$$
g_K(u)\olop_  {\Theta-K-1}1\qquad(u\ge0)
$$
et d'autre part que 
$$
g_K^{(\Theta-K)}(u)\ll \Q(u^{\1_{\ob Z}(\theta)}+1\W)\e^{(\Theta+1-\theta)u}\qquad(u\ge0). 
$$
Comme ces deux majorations impliquent que 
$$
g_K(u)\olops_{\Theta-K}1\qquad(u\ge0), 
$$
nous en d\'eduisons que 
$$
{g_K(u)\F\varphi^{Ku}}\olops_{\Theta-K}\e^{-Ku}\qquad(u\ge0) 
$$
et nous concluons que \eqref{KOL} est satisfaite pour l'entier $k=K$, d'apr\`es l'identit\'e \eqref{DeathOL}. 
\bigskip


\'Etablissons maintenant  l'estimation~\eqref{Carcass}, en proc\'edant de la fa\c{c}ons suivante : 
nous~montrons que l'on d\'efinit une application $\rho$ de l'espace $\sc C^{\Theta-K}\b([0,+\infty[\b)$ en posant  
$$
G(u)=\varphi\sum_{0\le k<K}{\mu_{k+1}(u)\F\e^{ku}}+{g_K(u)\F\e^{Ku}}+{\rho(u)\F\e^{Ku}}\qquad(u\ge0). 
\eqdef{Carcass2}
$$
Puis, \'etant donn\'ee la famille $\{h_{m,n}\}_{(m,n)\in\ob N^2}$ de fonctions holomorphes d\'efinies~par~\eqref{hmn}, 
nous~notons $\rho_1,\cdots,\rho_{K-1}$ les applications uniquement d\'etermin\'ees par l'identit\'e 
$$
\varphi\e^u\int_0^u{\nu_{m+1}(u)\F\e^{mu}}{\exp(\varphi\e^x)\F\exp(\varphi\e^u)}\d x
=\sum_{0\le n\le j<K-m}
h_{m,n}^{(j)}(0){\nu_{m+1}^{(n)}(x)\F\e^{(m+j)x}}+{\rho_{m+1}(u)\F\e^{Ku}}
\qquad(u\ge0)
\eqdef{Chuck}
$$
pour chaque $m\in\{0,\cdots,K-2\}$, nous~notons~$\rho_K$ l'application implicitement d\'efinie par  
$$
\varphi\e^u\int_0^u{g_{K-1}(u)\F\e^{(K-1)u}}{\exp(\varphi\e^x)\F\exp(\varphi\e^u)}\d x
=h_{K-1,0}(0){g_{K-1}(u)\F\e^{(K-1)u}}+{\rho_K(u)\F\e^{Ku}}\qquad(u\ge0)
\eqdef{Schuldiner}
$$
et nous prouvons que 
$$
\rho(u)=\sum_{1\le m\le K}\rho_m(u)\qquad(u\ge0).\eqdef{CosmicSea}
$$
Enfin, pour $1\le m\le K$, nous \'etablissons que 
$$
\rho_m(u)\olops_{\Theta-K}1\qquad(u\ge0) \eqdef{Major}
$$
et nous d\'eduisons de l'identit\'e \eqref{Carcass2} que l'estimation \eqref{Carcass} est satisfaite. 
\bigskip


Pour commencer, prouvons que \eqref{Carcass2} d\'efinit bien 
une application $\rho\in\sc C^{\Theta-K}\b([0,+\infty[\b)$. 
Comme $\nu_k\in\sc C^{\Theta-k}(\ob R)\ \,(1\le k\le K)$,  
il r\'esulte de \eqref{mu1} que les fonctions $\mu_1,\cdots,\mu_K$ appartiennent \`a l'espace $\sc C^{\Theta-K}(\ob R)$ 
et nous~d\'eduisons de  \eqref{DeathOL} que la fonction $g_K$ est  de~classe~$\sc C^{\Theta-K}$ sur~$[0,+\infty[$. 
L'application $G$ appartenant \`a l'espace $\sc C^{\Theta+1}\b([0,+\infty[\b)$,  
nous~concluons alors que $\rho$ est bien d\'efinie par \eqref{Carcass2} 
et que $\rho\in\sc C^{\Theta-K}\b([0,+\infty[\b)$.
\bigskip


\'Etant donn\'ees la famille $\{h_{m,n}\}_{(m,n)\in\ob N^2}$ de fonctions holomorphes d\'etermin\'ee~par~\eqref{hmn} et les~applications 
$\rho_1,\cdots,\rho_K$ implicitement d\'efinies par \eqref{Chuck} et \eqref{Schuldiner}, prouvons \eqref{CosmicSea}. 
En~reportant l'identit\'e \eqref{DeathOL} pour l'entier $k=K-1$ dans \eqref{defga}, nous obtenons que 
$$
G(u)=\!\!\!\sum_{0\le m<K-1}\!\!\!\varphi\e^u\int_0^u{\nu_{m+1}(u)\F\e^{mu}}{\exp(\varphi\e^x)\F\exp(\varphi\e^u)}\d x
+\varphi\e^u\int_0^u{g_{K-1}(u)\F\e^{(K-1)u}}{\exp(\varphi\e^x)\F\exp(\varphi\e^u)}\d x
\qquad(u\ge0).
$$
Pour $u\ge0$, nous d\'eduisons alors de \eqref{Chuck} et de \eqref{Schuldiner} que  
$$
G(u)=\sum_{0\le m<K-1}\ \sum_{0\le n\le j<K-m}h_{m,n}^{(j)}(0){\nu_{m+1}^{(n)}(u)\F\e^{(m+j)u}}
+h_{K-1,0}(0){g_{K-1}(u)\F\e^{(K-1)u}}+\sum_{0\le m<K}{\rho_{m+1}(u)\F\e^{Ku}}.
$$
En remarquant que \eqref{DeathOL} implique l'identit\'e  
$$
{g_{K-1}(u)\F\e^{(K-1)u}}={\nu_K(u)\F\e^{(K-1)u}}+{g_K(u)\F\e^{Ku}}\qquad(u\ge0), \eqdef{aaaa}
$$
nous d\'eduisons de l'\'egalit\'e $h_{K-1,0}(0)=1$ que 
$$
G(u)=\sum_{0\le m<K}\ \sum_{0\le n\le j<K-m}h_{m,n}^{(j)}(0){\nu_{m+1}^{(n)}(u)\F\e^{(m+j)u}}
+{g_K(u)\F\e^{Ku}}+\sum_{0\le m<K}{\rho_{m+1}(u)\F\e^{Ku}}
\qquad(u\ge0).
$$
En intervertissant l'ordre des sommations, nous observons alors que 
$$
G(u)=\sum_{0\le s<K}\e^{-su}\sum_{m+n\le s}h_{m,n}^{(s-m)}(0)\,\nu_{m+1}^{(n)}(u)
+{g_K(u)\F\e^{Ku}}+\sum_{0\le m< K}{\rho_{m+1}(u)\F\e^{Ku}}
\qquad(u\ge0).
$$
Comme l'identit\'e \eqref{hmn} implique que $h_{m,n}^{(k)}(0)=0\ \,(m\in\ob N,0\le k<n)$ et aussi que 
$$
\eqalign{
{(1-s/\varphi)^v\F1-s/\varphi-u}&=\sum_{(m,n)\in\ob N^2}h_{m,n}(s)u^mv^n\cr
&=\sum_{m+n<k}h_{m,n}^{(k-m-1)}(0){s^{k-m-1}u^mv^n\F (k-m-1)!}
\qquad\b(|u|+|s|<1,v\in\ob C\b),}
$$
nous d\'eduisons alors de l'identit\'e \eqref{cmn} que
$$
c_{k,m,n}=h_{m,n}^{(k-m-1)}(0)\qquad(m+n<k)
$$ 
et nous remarquons que 
$$
G(u)=\sum_{0\le s<K}\e^{-su}\sum_{m+n\le s}c_{s+1,m,n}\,\nu_{m+1}^{(n)}(u)
+{g_K(u)\F\e^{Ku}}+\sum_{0\le m<K}{\rho_{m+1}(u)\F\e^{Ku}}
\qquad(u\ge0).
$$
Il r\'esulte alors de la d\'efinition \eqref{mu1} des fonctions $\mu_1,\cdots,\mu_K$ que 
$$
G(u)=\varphi\sum_{0\le s<K}{\mu_{s+1}(u)\F\e^{s u}}+{g_K(u)\F\e^{Ku}}
+\sum_{0\le m<K}{\rho_{m+1}(u)\F\e^{Ku}}
\qquad(u\ge0)
$$
et nous d\'eduisons de \eqref{Schuldiner} que l'application $\rho$ satisfait~l'identit\'e \eqref{CosmicSea}.  
\bigskip


Enfin,  prouvons que $\rho_m$ satisfait la majoration \eqref{Major} pour $1\le m\le K$ et d\'eduisons-en l'estimation \eqref{Carcass}.   
Comme $g\in\sc C^\Theta\b([0,+\infty[\b)$ et comme $\nu_k\in\sc C\b([0,+\infty[\b)\ \,(0\le k\le K)$, l'identit\'e \eqref{DeathOL} pour  $k=K-1$ implique 
que $g_{K-1}$ est de classe $\sc C^{K-\Theta+1}$ sur~$[0,\infty[$, 
En~lui~appliquant le lemme \eqrefn{dl1} pour le triplet d'entiers $(k^*,\ell^*,m^*):=(1,\Theta-K, K-1)$, 
nous~obtenons alors d'une part que l'application~$\rho_K$, implicitement d\'efinie par \eqref{Schuldiner},  
appartient \`a l'espace $\sc C^{\Theta-K}\b([0,\infty[\b)$ et d'autre part que
$$
\rho_K^{(j)}(u)\ll\!\sum_{0\le n\le 1}\ \sup_{0\le x\le u}\b|g_{K-1}^{(n+j)}(x)\b|+\e^{-u}\!\!\!\sum_{0\le n\le j+1}\ \sup_{0\le x\le u}\b|g_{K-1}^{(n)}(x)\b|
\quad(0\le j\le\Theta-K,u\ge0). 
$$
L'estimation \eqref{KOL} \'etant satisfaite pour l'entier $k=K-1$, d'apr\`es l'hypoth\`ese de~r\'ecurrence, nous remarquons que 
l'identit\'e \eqref{DeathOL} implique d'une part que 
$$
{g_{K-1}(u)\F\e^{(K-1)u}}\olops_{\Theta+1-K}\e^{-(K-1)u}\qquad(u\ge0)
$$
et d'autre part que  
$$
g_{K-1}(u)\olops_{\Theta+1-K}1\qquad(u\ge0). \eqdef{bbbb}
$$
En particulier, les d\'eriv\'ees d'ordre inf\'erieur \`a $\Theta-K$ de $g_{K-1}$ sont born\'ees sur $[0,+\infty[$ 
et  la d\'eriv\'ee d'ordre $\Theta+1-K$ de $g_{K-1}$ est $\ll 1+u^{\1_{\ob Z}(\theta)}\e^{(\Theta+1-\theta)u}$ sur ce m\^eme intervalle. 
Comme $\Theta+1-\theta<1$, nous d\'eduisons alors de la majoration  de la d\'eriv\'ee $\rho_K^{(j)}$ que 
$$
\rho_K(u)\olops_{\Theta-K}1\qquad(u\ge0). \eqdef{eac}
$$
\'Etant donn\'e  $m\in\{0,\cdots,K-2\}$, la relation \eqref{DeathOL} pour les entiers $k=m$ et $k=m+1$ implique que 
$$
\nu_{m+1}(u)=g_m(u)-{g_{m+1}(u)\F\e^u}\qquad(u\ge0). \eqdef{ahahha}
$$
L'estimation \eqref{KOL} \'etant satisfaite pour l'entier $k=m+1$, il r\'esulte de \eqref{DeathOL} que 
$$
{g_{m+1}(u)\F\e^{(m+1)u}}\olops_{\Theta-m-1}\e^{-(m+1)u}\qquad(u\ge0). 
$$
{\it A fortiori}, nous en d\'eduisons d'une part que 
$$
g_{m+1}(u)\olops_{\Theta-m-1}1\qquad(u\ge0) \eqdef{bhbh}
$$
et d'autre part que 
$$
\nu_{m+1}(u)=g_m(u)+\scos_{\Theta-m-1}\Q(\e^{-u}\W)
\qquad(u\ge0). 
$$
Comme $g$ appartient \`a l'espace $\sc C^\Theta\b([0,+\infty[\b)$ et comme $\nu_k\in\sc C^{\Theta-k}(\ob R)\ \,(1\le k\le\Theta)$, 
nous~remarquons que la fonction $g_m$ est de classe $\sc C^{\Theta-m}$ sur l'intervalle $[0,+\infty[$ et nous~d\'eduisons, comme pr\'ec\'edemment, 
de l'estimation \eqref{KOL} pour l'entier $k=m$ que 
$$
g_m(u)\olops_{\Theta-m}1\qquad(u\ge0). \eqdef{Matal}
$$
{\it A fortiori}, nous pouvons appliquer le lemme \eqrefn{dl15} au couple $(f^*,g^*)=(\nu_{m+1}, g_m)$ et  
au~\hbox{$5$-uplet} d'entiers  $(m^*, k^*,\ell^*,\delta^*, \theta^*)=(\nu_{m+1}, g_m, m, K-m, \Theta-K-1,\1_{\ob Z}(\theta),\theta-m)$ 
pour~obtenir d'une part que~la fonction $\rho_{m+1}$, implicitement d\'efinie par \eqref{Chuck}, 
est de~classe~$\sc C^{\Theta-K}$ sur $[0,\infty[$ et d'autre part qu'elle satisfait 
$$
\rho_{m+1}(u)\olops_{\Theta-K}1\qquad(u\ge0).
$$
Cette relation \'etant v\'erifi\'ee pour $0\le m\le K-2$, nous d\'eduisons de \eqref{eac} 
que la~majoration~\eqref{Major} est satisfaite pour $1\le m\le K$. {\it A fortiori}, 
il r\'esulte de \eqref{CosmicSea} que 
$$
\rho(u)\olops_{\Theta-K}1\qquad(u\ge0)
$$ 
et par cons\'equent que 
$$
{\rho(u)\F\e^{Ku}}\olops_{\Theta-K}\e^{-Ku}\qquad(u\ge0).
$$
Enfin, nous obtenons \eqref{Carcass} en reportant cette majoration dans l'identit\'e \eqref{DeathOL}.   
\bigskip


Enfin, il nous reste \`a \'etablir l'estimation \eqref{Carcariass} pour achever la preuve du th\'eor\`eme~\eqrefn{T3faible}. 
Pour~cela, nous proc\'edons de la fa\c{c}on suivante : \'etant donn\'ees les familles de fonctions $\{q_{m,n}\}_{(m,n)\in\ob N^2}$ et~$\{\sc X_{m,n}\}_{m+n<K}$ 
d\'efinies par 
\eqref{qmn} et par  
$$
\sc X_{m,n}(x):=\Q\{\eqalign{&\nu_{m+1}^{(n)}(x+\lambda)\qquad\hbox{si }m\neq K-1\cr
&g_{K-1}(x+\lambda)\qquad\hbox{sinon}
}\W.\qquad(m+n<K,x\ge0), \eqdef{cccc}
$$
nous construisons une application $\xi:[0,+\infty[\to]-1,1[$, 
dont les d\'eriv\'ees d'ordre inf\'erieur \`a $\Theta-K$ sont continues et born\'ees sur l'intervalle $[0,+\infty[$,  
v\'erifiant les estimations 
$$
\eqalignno{
\e^{\xi(x)}=c+\sum_{m+n<K}{\sc X_{m,n}(x)\F\e^{(m+1)x}}q_{m,n}\Q({\xi(x)\F\e^x}\W)+\sco_{\Theta-K}\Q(\e^{-(K+1)x}\W)&
\qquad(x\ge 0),&\eqdef{POSIX}
\cr
\qquad\qquad\varphi\e^{-\varphi\xi(x)}=c+\varphi\sum_{0\le k<K}{\mu_{k+1}^*(x)\F\e^{(k+1)x}}
-{g_K(x+\lambda)\F\varphi^K\e^{Kx}}+\sco_{\Theta-K}\Q(\e^{-(K+1)x}\W)&
\qquad(x\ge0)&
\eqdef{cellela}}
$$
et telle que l'application $\psi$ d\'efinie par 
$$
\psi(x)=x+\lambda+\log\Q(1-{\xi(x)\F\e^x}\W)
\qquad(x\ge0)
\eqdef{psix}
$$
v\'erifie la minoration 
$$
\psi(x)\ge0
\qquad(x\ge0). 
\eqdef{Baal}
$$
Puis, nous prouvons que 
$$
c+{g\b(\psi(x)\b)\F\e^{\psi(x)}}=\e^{\xi(x)}+\scos_{\Theta-K}\Q(\e^{-(K+1)x}\W)\qquad(x\ge0)\eqdef{durdur}
$$
et nous en d\'eduisons que 
$$
G(u)=G\b(\phi\circ \psi(u)\b)+\scos_{\Theta-K}\b(\e^{-(K+2)u}\b)\qquad(u\ge0). \eqdef{durdur0}
$$
Enfin, nous \'etablissons la relation 
$$
G\b(\phi\circ \psi(u)\b)=\varphi\sum_{0\le k<K}{\mu_{k+1}^*(u)\F\e^{ku}}-{g_K(u+\lambda)\F\varphi^K\e^{Ku}}+\scos_{\Theta-K}\Q(\e^{-Ku}\W)
\qquad(u\ge0), \eqdef{fini0}
$$
que nous reportons dans la relation pr\'ec\'edente, pour obtenir l'estimation \eqref{Carcariass}. 
\bigskip


Prouvons l'existence de deux applications $\xi:[0,+\infty[\to]-1,1[$ et $\psi:[0,+\infty[\to[0,+\infty[$, de classe $\sc C^{\Theta-K}$ sur l'intervalle $[0,+\infty[$, 
v\'erifiant les relations \eqref{POSIX}, \eqref{cellela}, \eqref{psix} et 
$$
\xi(u)\olop_{\Theta-K}1\qquad(u\ge0). \eqdef{hiv}
$$
Nous rappelons que $\nu_k\in\sc C^{\Theta-k}(\ob R)\ \,(1\le k\le K)$ et que les fonctions $g_{K-1}$ et $g_K$ 
sont respectivement de classe $\sc C^{\Theta-K+1}$ et de classe $\sc C^{\Theta-K}$ sur l'intervalle $[0,+\infty[$. 
{\it A fortiori}, il~r\'esulte de la $2\lambda$-p\'eriodicit\'e des applications $\nu_1,\cdots,\nu_K$ que 
$$
\nu_k(x)\olop_{\Theta-k}1\qquad(1\le k\le K,x\in\ob R).
$$
Nous d\'eduisons alors de \eqref{bbbb} que les applications de la famille $\sc X:=\{\sc X_{m,n}\}_{m+n<K}$ 
d\'efinie par \eqref{cccc} 
appartiennent \`a $\sc C^{\Theta-K}\b([0,+\infty[\b)$ et satisfont la majoration~\eqref{agrajag}. 
Comme~la~relation \eqref{aaaa} et l'\'egalit\'e $\lambda=\log\varphi$ impliquent que 
$$
\sc X_{K-1,0}(x)=g_{K-1}(x+\lambda)=\nu_K(x+\lambda)+{g_K(x+\lambda)\F\varphi\e^x}\qquad(x\ge0), 
$$
nous observons que les hypoth\`eses du lemme \eqrefn{invfc5} sont v\'erifi\'ees et nous en d\'eduisons 
qu'il existe une application $\xi\in\sc C^{\Theta-K}\b([0,+\infty[,]-1,1[\b)$ satisfaisant \eqref{POSIX}, \eqref{cellela}, \eqref{hiv}~et 
telle~~que~la~fonction $\psi$ d\'efinie par l'identit\'e \eqref{psix} v\'erifie l'in\'egalit\'e \eqref{Baal}. Comme~$\xi$ est de classe $\sc C^{\Theta-K}$ et \`a valeurs dans $]-1,1[$ sur l'intervalle $[0,+\infty[$ , nous d\'eduisons enfin de \eqref{psix} et de \eqref{Baal} que la~fonction~$\psi$ appartient 
\`a l'ensemble $\sc C^{\Theta-K}\b([0,+\infty[,[0,+\infty[\b)$. 
\bigskip



\'Etant donn\'ees les fonctions $\xi$ et $\psi$ d\'efinies au paragraphe pr\'ec\'edent, prouvons~\eqref{durdur}. 
D'apr\`es  l'identit\'e \eqref{DeathOL} pour l'entier $k=K-1$, nous avons 
$$
c+{g(u)\F \e^u}=c+\sum_{0\le m<K-1}{\nu_{m+1}(u)\F\e^{(m+1)u}}+{g_{K-1}(u)\F\e^{Ku}}
\qquad(u\ge0).
$$
Comme $\psi$ satisfait \eqref{Baal}, nous pouvons substituer $\psi(x)$ au nombre $u$ pour obtenir que 
$$
c+{g\b(\psi(x)\b)\F \e^{\psi(x)}}=c+\sum_{0\le m<K-1}{\nu_{m+1}\b(\psi(x)\b)\F\e^{(m+1)\psi(x)}}+{g_{K-1}\b(\psi(x)\b)\F\e^{K\psi(x)}}
\qquad(x\ge0).
$$
L'identit\'e $\nu_{m+1}(u)=g_m(u)+(\nu_{m+1}-g_m)(u)\ \,(0\le m<K-1,u\ge0)$ implique alors que 
$$
c+{g\b(\psi(x)\b)\F \e^{\psi(x)}}=c+\sum_{0\le m<K}{g_m\b(\psi(x)\b)\F\e^{(m+1)\psi(x)}}
+\sum_{0\le m<K-1}{(\nu_{m+1}-g_m)\b(\psi(x)\b)\F\e^{(m+1)\psi(x)}}\qquad(x\ge0). \eqdef{boye}
$$
Pour $0\le m<K$, nous rappelons que la fonction $g_m$ appartient \`a l'espace $\sc C^{\Theta-m}\b([0,+\infty[\b)$ et qu'elle satisfait la majoration \eqref{Matal}. En appliquant le lemme \eqrefn{invfc6} pour $h=g_m$ et $k^*=m$, 
nous~d\'eduisons alors de \eqref{Fear} que 
$$
{g_m\b(\psi(x)\b)\F\e^{(m+1)\psi(x)}}=\!\!\!\!\sum_{0\le n<K-m}\!\!\!\!{g_m^{(n)}(x+\lambda)\F\e^{(m+1)x}}q_{m,n}\Q({\xi(x)\F\e^x}\W)+\scos_{\Theta-K}\Q(\e^{-(K+1)x}\W)
\qquad(x\ge0).\!\!\!\!\eqdef{ouaye}
$$
Pour $0\le m<K-1$, nous remarquons que $u\mapsto\nu_{m+1}(u)-g_m(u)$ est de classe $\sc C^{\Theta-m-1}$ sur l'intervalle $[0,+\infty[$ 
et nous d\'eduisons de \eqref{ahahha} et de \eqref{bhbh} que 
$$
\nu_{m+1}(u)-g_m(u)=-{g_{m+1}(u)\F\e^u}\olops_{\Theta-m-1}\e^{-u}\qquad(u\ge0).
$$
En appliquant le lemme \eqrefn{invfc7} aux quantit\'es $g^*=\nu_{m+1}-g_m$ et $k^*=m$, pour $x\ge0$, 
nous~d\'eduisons alors de \eqref{Fear2} que 
$$
{(\nu_{m+1}-g_m)\b(\psi(x)\b)\F\e^{(m+1)\psi(x)}}=\!\!\!\sum_{0\le n<K-m}\!\!\!\!{(\nu_{m+1}-g_m)^{(n)}(x+\lambda)\F\e^{(m+1)x}}q_{m,n}\!\Q({\xi(x)\F\e^x}\W)+\scos_{\Theta-K}\Q(\e^{-(K+1)x}\W).
$$
En sommant cette estimation avec \eqref{ouaye} nous d\'eduisons de \eqref{boye}  l'estimation 
$$
\eqalign{
c+{g\b(\psi(x)\b)\F \e^{\psi(x)}}=\ &c+\sum_{0\le m<K-1}\ \sum_{0\le n<K-m}{\nu_{m+1}^{(n)}(x+\lambda)\F\e^{(m+1)x}}q_{m,n}\Q({\xi(x)\F\e^x}\W)
\cr&+
{g_{K-1}(x+\lambda)\F\e^{Kx}}q_{K-1,0}\Q({\xi(x)\F\e^x}\W)+\scos_{\Theta-K}\Q(\e^{-(K+1)x}\W)\qquad(x\ge0), 
}
$$
que nous mettons, \`a l'aide des quantit\'es d\'efinies par \eqref{cccc}, sous la forme plus compacte  
$$
c+{g\b(\psi(x)\b)\F \e^{\psi(x)}}=c+\sum_{m+n<K}{\sc X_{m,n}(x)\F\e^{(m+1)x}}q_{m,n}\Q({\xi(x)\F\e^x}\W)+\scos_{\Theta-K}\Q(\e^{-(K+1)x}\W)\qquad(x\ge0). 
$$
Enfin, nous d\'eduisons de \eqref{POSIX} que l'estimation \eqref{durdur} est satisfaite. 
\bigskip



\'Etant donn\'ees les fonctions $\xi$ et $\psi$ d\'efinies pr\'ec\'edemment, prouvons maintenant~\eqref{durdur0}. 
Nous rappelons que l'application $G$ est de classe $\sc C^{\Theta+1}$ sur $[0,+\infty[$. 
Comme~\eqref{powa} implique que  $\phi(x)\ge0\ \,(x\ge0)$, nous d\'eduisons de l'in\'egalit\'e \eqref{Baal} que 
$$
G(u)=G\b(\phi\circ\psi(u)\b)+\int_{\phi\circ\psi(u)}^uG'(t)\d t\qquad(u\ge0). 
$$ 
Notant $\epsilon$ l'application implicitement d\'efinie par 
$$
\phi\circ\psi(u)=u+\epsilon(u)\qquad(u\ge0), \eqdef{epsi1}
$$
il r\'esulte alors du changement de variable affine $t=u+x\epsilon(u)$ que 
$$
G(u)=G\b(\phi\circ\psi(u)\b)-\epsilon(u)\int_0^1G'\b(u+x\epsilon(u)\b)\d x\qquad(u\ge0). \eqdef{durdur4}
$$ 
En substituant $\psi(x)$ \`a $u$ dans \eqref{KoKor}, nous obtenons que 
$$
\phi\circ\psi(x)=\psi(x)-\lambda-x+\log\Q(1+{\varphi\F\e^{\psi(x)}}\log\Q(c+{g(\psi(x))\F\e^{\psi(x)}}\W)\W)
\qquad(x\ge0). 
$$
{\it A fortiori}, l'identit\'e \eqref{psix} et l'\'egalit\'e $\lambda=\log\varphi$ impliquent que 
$$
\phi\circ\psi(x)=x+\log\Q(1-{\xi(x)\F\e^x}\W)+\log\Q(1+{\e^{-x}\F1-\xi(x)\e^{-x}}\log\Q(c+{g(\psi(x))\F\e^{\psi(x)}}\W)\W)
\qquad(x\ge0).
$$
En reportant dans \eqref{epsi1}, nous obtenons alors que 
$$
\epsilon(x)=\log\Q(1-{\xi(x)\F\e^x}+\e^{-x}\log\Q(c+{g(\psi(x))\F\e^{\psi(x)}}\W)\W)
\qquad(x\ge0). \eqdef{epsi2}
$$
Comme l'application $\xi$ satisfait la majoration \eqref{hiv}, nous remarquons que 
$$
\e^{-\xi(x)}\olop_{\Theta-K}1\qquad(x\ge0)
$$
et nous d\'eduisons de l'estimation \eqref{durdur} que 
$$
c+{g\b(\psi(x)\b)\F\e^{\psi(x)}}=\e^{\xi(x)}\Q(1+\scos_{\Theta-K}\Q(\e^{-(K+1)x}\W)\W)\qquad(x\ge0). \eqdef{wearedoomed}
$$
Le membre de droite est strictement positif, puisque les relations~\eqref{defg} et $f(t)\ge\e\ \,(t\ge\e)$ impliquent que 
$$
c+{g\b(\psi(x)\b)\F\e^{\psi(x)}}=f\Q(\exp\e^{\psi(x)}\W)\exp\e^{(1-\varphi)\psi(x)}>0\qquad(x\ge0), 
$$
de sorte que l'identit\'e \eqref{epsi2} peut \'egalement s'\'ecrire 
$$
\epsilon(x)=\log\Q(1+\e^{-x}\log\Q(1+\scos_{\Theta-K}\Q(\e^{-(K+1)x}\W)\W)\W)
\qquad(x\ge0). 
$$
La fonction $\sc L:z\mapsto\log(1+z)/z$ \'etant holomorphe sur $\ob C\ssm]-\infty,-1]$, nous observons que 
$$
\eqalign{
\log\Q(1+\scos_{\Theta-K}\Q(\e^{-(K+1)x}\W)\W)&=\scos_{\Theta-K}\Q(\e^{-(K+1)x}\W)\times \sc L\Q(\scos_{\Theta-K}\Q(\e^{-(K+1)x}\W)\W)\cr
&\olops_{\Theta-K}\e^{-(K+1)x}\qquad(x\ge0)}. 
$$
En proc\'edant de la m\^eme mani\`ere, nous obtenons alors d'une part que 
$$
\epsilon(x)=\log\Q(1+\scos_{\Theta-K}\Q(\e^{-(K+2)x}\W)\W)
\qquad(x\ge0)
$$
et d'autre part que 
$$
\epsilon(x)\olops_{\Theta-K}\e^{-(K+2)x}
\qquad(x\ge0). \eqdef{durdur3}
$$
Comme $\Theta<\theta$ et comme $K\ge2$, nous d\'eduisons des estimation \eqref{first} et \eqref{LordBelial} 
que 
$$
G(x)\olop_{\Theta-1}1\qquad(x\ge0) 
$$
et par cons\'equent que
$$
G'(x)\olop_{\Theta-K}1\qquad(x\ge0). 
$$
En remarquant que 
$$
u+x\epsilon(u)=u+\scos_{\Theta-K}\Q(\e^{-(K+2)x}\W)\qquad(0\le x\le 1,u\ge0), 
$$
nous obtenons alors d'une part que 
$$
G'\b(u+x\epsilon(u)\b)\olop_{\Theta-K}1
\qquad(0\le x\le 1,u\ge0)
$$
et d'autre part que 
$$
\int_0^1G'\b(u+x\epsilon(u)\b)\d x\olop_{\Theta-K}1\qquad(u\ge0).
$$
En reportant dans \eqref{durdur4}, nous d\'eduisons alors l'estimation \eqref{durdur0} de la majoration \eqref{durdur3}. 
\bigskip



Enfin, \'etant donn\'ees les fonctions $\xi$ et $\psi$ d\'efinies pr\'ec\'edemment, \'etablissons~\eqref{fini0}. 
En~substituant  $\psi(x)$ \`a $u$ dans \eqref{Larela}, nous obtenons que   
$$
G\b(\phi\circ\psi(x)\b)=\Bg({1+E(\psi(x))\F\b(c+g(\psi(x))\e^{-\psi(x)}\b)^\varphi}-{c\F \varphi}\Bg)
\Q(\e^{\psi(x)}+\varphi\log\Q(c+{g(\psi(x))\F\e^{\psi(x)}}\W)\W)\qquad(x\ge0). 
$$
Pour $x\ge0$, il r\'esulte alors de l'estimation \eqref{wearedoomed} que le nombre $G\b(\phi\circ\psi(x)\b)$ vaut 
$$
\Bg({\e^{-\varphi\xi(x)}+E(\psi(x))\e^{-\varphi\xi(x)}\F\b(1+\scos_{\Theta-K}(\e^{-(K+1)x})\b)^\varphi}-{c\F \varphi}\Bg)
\bg(\e^{\psi(x)}+\varphi\xi(x)+\log\B(1+\scos_{\Theta-K}\b(\e^{-(K+1)x}\b)\B)\bg). 
$$
En remarquant que 
$$
\log\B(1+\scos_{\Theta-K}\b(\e^{-(K+1)x}\b)\B)\olops_{\Theta-K}\e^{-(K+1)x}\qquad(x\ge0), 
$$
pour $x\ge0$, nous obtenons alors que 
$$
G\b(\phi\circ\psi(x)\b)=\Bg({\e^{-\varphi\xi(x)}+E(\psi(x))\e^{-\varphi\xi(x)}\F\b(1+\scos_{\Theta-K}(\e^{-(K+1)x})\b)^\varphi}-{c\F \varphi}\Bg)
\Q(\e^{\psi(x)}+\varphi\xi(x)+\scos_{\Theta-K}\Q(\e^{-(K+1)x}\W)\W). 
$$
Pour $x\ge0$, l'identit\'e \eqref{psix} et l'\'egalit\'e $\lambda=\log\varphi$ impliquent alors d'une part que 
$$
\e^{\psi(x)}=\varphi\e^x-\varphi\xi(x)
$$
et d'autre part que
$$
G\b(\phi\circ\psi(x)\b)=\varphi\e^x\Bg({\e^{-\varphi\xi(x)}+E(\psi(x))\e^{-\varphi\xi(x)}\F\b(1+\scos_{\Theta-K}(\e^{-(K+1)x})\b)^\varphi}-{c\F \varphi}\Bg)
\Q(1+\scos_{\Theta-K}\Q(\e^{-(K+2)x}\W)\W). \eqdef{termination}
$$
Comme l'application $\xi\in\sc C^{\Theta-K}\b([0,+\infty[,]-1,1[\b)$ satisfait la majoration \eqref{grip2}, nous~observons que 
$$
\e^{-\varphi\xi(x)}\olop_{\Theta-K}1\qquad(x\ge0). \eqdef{Mackie}
$$
Par ailleurs, nous remarquons que 
$$
{\xi(x)\F\e^x}\olop_{\Theta-K}\e^{-x}\qquad(x\ge0)
$$
et nous d\'eduisons des relation \eqref{psix} et $-1<\xi(x)<1\ \,(x\ge0)$ que 
$$
\psi(x)=x+\sco_{\Theta-K}\Q(\e^{-x}\W)\qquad(x\ge0). \eqdef{ze}
$$
Pour chaque entier $k\in\{0,\cdots,\Theta-K\}$, nous appliquons le th\'eor\`eme \eqrefn{dfc0} pour d\'eriver $k$-fois la compos\'ee des applications 
$E\in\sc C^{\Theta+1}\b([0,+\infty[\b)$ et $\psi\in\sc C^{\Theta-K}\b([0,+\infty[,[0,+\infty[\b)$ pour~obtenir~que 
$$
{\d^k\F\d x^k}E\b(\psi(x)\b)=k!\sum_{n_1+\cdots+kn_k=k}E^{(n_1+\cdots+n_k)}\b(\psi(x)\b)
\prod_{1\le j\le k}{\psi^{(j)}(a)^{n_j}\F n_j!j^{n_j}}\qquad(x\ge0). 
$$
Comme la majoration \eqref{mrter}  et l'estimation \eqref{ze} impliquent que 
$$
{\d^k\F\d x^k}E\b(\psi(x)\b)\ll\sum_{n_1+\cdots+kn_k=k}\e^{(n_1+\cdots+n_k-\theta)\psi(x)}\ll \e^{(k-\theta)x}\qquad(x\ge0) 
$$
et comme $k-\theta\le-K-1\ \,(0\le k<\Theta-K)$, nous en d\'eduisons alors que 
$$
E\b(\psi(x)\b)\olops_{\Theta-K}\e^{-(K+1)x}\qquad(x\ge0). 
$$
En particulier, pour $x\ge0$, il r\'esulte de \eqref{Mackie} et de \eqref{termination} d'une part que 
$$
E\b(\psi(x)\b)\e^{-\varphi\xi(x)}\olops_{\Theta-K}\e^{-(K+1)x}
$$
et d'autre part que 
$$
G\b(\phi\circ\psi(x)\b)=\varphi\e^x\Bg({\e^{-\varphi\xi(x)}+\scos_{\Theta-K}\Q(\e^{-(K+1)x}\W)\F\b(1+\scos_{\Theta-K}(\e^{-(K+1)x})\b)^\varphi}-{c\F \varphi}\Bg)
\Q(1+\scos_{\Theta-K}\Q(\e^{-(K+2)x}\W)\W).  
$$
En remarquant que 
$$
\B(1+\scos_{\Theta-K}\B(\e^{-(K+1)x}\B)\B)^{-\varphi}=1+\scos_{\Theta-K}\Q(\e^{-(K+1)x}\W)\qquad(x\ge0), 
$$
pour $x\ge0$, nous d\'eduisons alors de la majoration \eqref{Mackie} que 
$$
G\b(\phi\circ\psi(x)\b)=\varphi\e^x\bg(\e^{-\varphi\xi(x)}-{c\F \varphi}+\scos_{\Theta-K}\Q(\e^{-(K+1)x}\W)\bg)
\Q(1+\scos_{\Theta-K}\Q(\e^{-(K+2)x}\W)\W). 
$$
Comme l'application $\xi\in\sc C^{\Theta-K}\b([0,+\infty[\b)$ satisfait \eqref{grip2}, nous remarquons que 
$$
\e^{-\varphi\xi(x)}\olop_{\Theta-K}1\qquad(x\ge0)
$$
et nous en d\'eduisons que  
$$
G\b(\phi\circ\psi(x)\b)=\e^x\Q(\varphi\e^{-\varphi\xi(x)}-c+\scos_{\Theta-K}\Q(\e^{-(K+1)x}\W)\W)\qquad(x\ge0). 
$$
En particulier, il suit 
$$
G\b(\phi\circ\psi(x)\b)=\e^x\Q(\varphi\e^{-\varphi\xi(x)}-c\W)+\scos_{\Theta-K}\Q(\e^{-Kx}\W)\qquad(x\ge0) 
$$
et nous d\'eduisons l'estimation \eqref{fini0} de la relation \eqref{cellela}. 
\bigskip


Enfin, pour $2\le k\le K$, montrons que $G_k$ et $\nu_K$ sont des combinaisons lin\'eaires de produits dont les facteurs sont pris parmi \eqref{oomph} ou parmi les fonctions $\nu_1,\cdots,\nu_1^{(k-1)}$. 
Pour $k\in\{2,\cdots,K\}$, nous d\'eduisons de \eqref{Pkespace} et de \eqref{KingD} que l'application $P_k\b[\sc X(u)\b]$  est une combinaison lin\'eaire de produits dont les facteurs sont pris parmi les fonctions \eqref{oomph}. 
Les relations \eqref{SeewhatIsee} et \eqref{rec} impliquent alors qu'il en est de m\^eme pour $G_k$ et $\nu_k$. 
En proc\'edant par r\'ecurrence, nous obtenons alors que ces deux fonctions sont \'egalement des combinaisons lin\'eaires de produits dont les facteurs sont pris parmi $\nu_1,\cdots,\nu_1^{(k-1)}$. 
\hfill\qed
\bigskip


\Sect Annexe, Annexe.

\Secti Point fixe, Existence, unicit\'e et conditions initiales. 

Dans ce paragraphe,  nous \'etudions certaines variantes de l'\'equation diff\'erentielle \eqref{ol2}. Nous cherchons plus particuli\`erement \`a d\'eterminer l'existence et  l'unicit\'e de leurs solutions 
et \`a caract\'eriser celles-ci par une condition initiale. 
\bigskip




L'introduction de cet article sugg\`ere que l'int\'er\^et port\'e \`a l'\'equation diff\'erentielle~\eqref{ol2} 
est une cons\'equence de l'\'etude de la suite de Golomb, 
ce qui n'est pas compl\`etement vrai. En~1957, c'est-\`a-dire onze ann\'ees avant que Golomb ne propose l'\'etude de la suite~$\{u_n\}_{n=1}^\infty$,  
McKiernan \CitRef{McKiernan} prouve ainsi, pour chaque nombre complexe p v\'erifiant $|p|\ge\varphi$, 
qu'il existe une unique fonction holomorphe $f$ d\'efinie au voisinage de $p$ et v\'erifiant 
$$
\Q\{\eqalign{
f'(t)&={1\F f\circ f(t)},
\cr
f(p)&=p.
}
\W.\eqdef{Kov}
$$

En~1998, %% A verifier
P\'etermann \CitRef{Petermann2} montre que la fonction $t\mapsto ct^{\varphi-1}$ est
l'unique solution $f\in\sc C^1\b(\ob R_+^*,\ob R_+^*\b)$ de l'\'equation
diff\'erentielle \eqref{ol2} satisfaisant la condition initiale 
$$
\lim_{t\to 0^+}f(t)=0.
$$

Un an plus tard, P\'etermann, R\'emy et Vardi \CitRef{PetermannRemyVardi} prouvent que le syst\`eme 
$$
\Q\{\eqalign{
f'(t)&={1\F f\circ f(t)}\qquad(t\ge p),
\cr
f(p)&=p
}
\W.\eqdef{conjsys}
$$
poss\`ede, pour chaque nombre r\'eel $p>0$, une unique solution $f\in\sc C^1\big([p,\infty[,[p,\infty[\big)$. 
R\'eciproquement, pour chaque solution maximale et positive $f$
de l'\'equation~\eqref{ol2}, ils~montrent que $f$ admet au moins un point fixe $p>0$ et que $f\in\sc
C^1\big([p,\infty[,[p,\infty[\big)$.
Lorsque $f$ et $g$ sont deux solutions maximales et positives de~\eqref{ol2},
P\'etermann, R\'emy et Vardi prouvent \'egalement qu'il existe des racines
arbitrairement grandes de l'\'equation
$$
f(x)=g(x).
$$


Nous remarquons que les \'equations diff\'erentielles \eqref{ol2}, \eqref{nimp}, \eqref{eqP} et \eqref{T1eq1} \'evoqu\'ees dans l'introduction 
sont toutes des cas particuliers, 
pour certains nombres r\'eels $p>0$ et pour certaines applications $h\in\sc C\b([p,\infty[\b)$ ``proches'' de $1$, 
de~l'\'equation diff\'erentielle 
$$
f'(t)={h(t)\F f\circ f(t)}\qquad(t\ge p). \eqdef{eqOL}
$$
Il est alors naturel de s'interroger d'une part sur les conditions que doivent v\'erifier le nombre~$p>0$ 
et la fonction $h\in\sc C\b([p,+\infty[\b)$ pour que l'\'equation diff\'erentielle pr\'ec\'edente admette des solutions 
et~d'autre part sur les conditions initiales assurant existence et unicit\'e pour ces solutions de \eqref{eqOL}.  
La propri\'et\'e suivante, que nous prouvons plus~loin, r\'epond~partiellement \`a ces deux probl\'ematiques en proposant pour les \'equations dif\-f\'e\-ren\-tiel\-les \eqref{eqOL} 
associ\'ees \`a un nombre $p>0$ et \`a une fonction $\in\sc C\b([p,\infty[\b)$ v\'erifiant l'in\'egalit\'e
$$
0\le h(t)<t\qquad(t\ge p) \eqdef{condd}
$$
une condition initiale assurant existence et unicit\'e des solutions : la donn\'ee d'un nombre $q\in]0,p]$ 
et d'une fonction $g\in\sc C\b([q,p],\ob R_+^*\b)$ v\'erifiant $g(p)=q$. 

\prop TTr. Soient $p>0$ un nombre r\'eel et $h\in\sc
C\b([p,\infty[\b)$ une fonction v\'erifiant~\eqref{condd}. 
Pour chaque nombre r\'eel $q\in]0,p]$ et chaque fonction $g\in\sc C\b([q,p],\ob R_+^*\b)$ v\'erifiant $g(p)=q$, 
il existe une unique solution $f\in\sc C\b([q,\infty[\b)\cap \sc C^1\b([p,\infty[,[q,\infty[\b)$ du syst\`eme 
$$
\Q\{
\eqalign{
f(t)=&g(t)\qquad\qquad(q\le t\le p),
\cr
f'(t)=&{h(t)\F f\circ f(t)}\qquad(t\ge p).
\cr
}
\W. 
\eqdef{manowar}
$$
\par

Nous fixons un nombre $p>0$  et une fonction $h\in\sc C\b([p,\infty[\b)$ v\'erifiant la condition~\eqref{condd} et nous observons 
qu'il~existe une unique solution $f\in\sc C^1\b([p,+\infty[,[p,+\infty[\b)$ de~\eqref{eqOL} admettant le point~fixe $p$,  
d'apr\`es la propri\'et\'e \eqrefn{TTr} appliqu\'ee \`a $q=p$ et \`a ~$g:p\mapsto p$. 
Au~vu~de~cette~remarque, il para\^\i t raisonnable de caract\'eriser 
les solutions de l'\'equation diff\'erentielle~\eqref{eqOL} par~la donn\'ee d'un point fixe. 
Malheureusement, cette condition initiale pr\'esente deux d\'efauts : 
\medskip

D'abord, il est possible qu'une solution $f$ 
de l'\'equation~\eqref{eqOL} soit sans point~fixe. 
Pour~l'\'equation diff\'erentielle~\eqref{eqOL} associ\'ee au nombre~$p=3$ et \`a la fonction $h:t\mapsto t-2$, 
qui~v\'erifient~la~condition~\eqref{condd}, c'est~ainsi le cas pour la solution~$f:t\mapsto t-1$. 
\medskip

Ensuite, plusieurs~solutions~distinctes de  \eqref{eqOL} peuvent admettre le m\^eme~point~fixe. 
Lorsque $h(p)>0$, c'est toujours le cas. Pour $0<q<p$ et pour $a\ge1$, nous posons 
$$
h_a(t):=\Q\{\eqalign{
&h(p){t^a\F p^a}\qquad(0<t<p),\cr
&h(t)\qquad\quad\qquad(t\ge p),
}\W.
$$
nous observons que la fonction $h_a$ est continue sur l'intervalle $]0,+\infty[$ et nous d\'eduisons de~\eqref{condd} qu'elle v\'erifie la condition 
$$
0\le h_a(t)<t\qquad(t\ge q).
$$ 
En appliquant la propri\'et\'e \eqrefn{TTr} \`a $p^*=q^*=q$ et aux applications $h^*=h_\alpha$ et $g:q\mapsto q$, nous~en~d\'eduisons alors que l'\'equation diff\'erentielle
$$
f'(t)={h(t)\F f\circ f(t)}\qquad(t\ge q)
$$ 
admet une unique solution $f_{q,a}\!\in\!\sc C^1\b([q,+\infty[,[q,+\infty[\b)$ v\'erifiant l'\'egalit\'e $f_{q,a}(q)=q$. Pour~$0<q<p$, nous remarquons 
que les fonctions $f_{q,a}\ \,(a\ge1)$ sont des solutions de l'\'equation diff\'erentielle \eqref{eqOL} admettant $q$ pour point fixe 
et nous d\'eduisons de l'identit\'e 
$$
(f_{q,a}\circ f_{q,a})f'_{q,a}=h_a\neq h_b=(f_{q,b}\circ f_{q,b})f'_{q,b}\qquad(a>b\ge 1)
$$ 
qu'elles sont distinctes deux \`a deux. En conclusion, chaque nombre $q\in]0,p[$ est un point fixe pour une infinit\'e de solutions de l'\'equation diff\'erentielle \eqref{eqOL}. 
\bigskip

Comme la donn\'ee d'un point fixe ne permet pas de caract\'eriser toutes les solutions de l'\'equation diff\'erentielles~\eqref{eqOL}, 
recherchons une autre condition initiale~plus~ad\'equate. 
Dans la mesure o\`u elle assure l'existence et l'unicit\'e des solutions de l'\'equation~\eqref{eqOL}, 
la~condition initiale \'enonc\'ee dans la propri\'et\'e \eqrefn{TTr} semble \^etre une bonne candidate. De~plus,  
montrons qu'une solution strictement positive $f$ de l'\'equation diff\'erentielle \eqref{eqOL}, 
i.e.~une fonction continue $f$ \`a valeurs strictement positives, d\'erivable sur $[p,+\infty[$ et  v\'erifiant~\eqref{eqOL},  
est compl\`etement d\'etermin\'ee sur un voisinage de $+\infty$ par une condition initiale semblable, 
c'est-\`a-dire par la donn\'ee du nombre $q=f(p)$ et de la restriction $g$ de $f$ \`a $[q,p]$ si 
$$
\sup_{t\ge p}f(t)\le p \eqdef{brr1}
$$
et par la donn\'ee de deux nombres $p'>q'>0$ v\'erifiant $q'=f(p')$ et de la restriction $g$ de l'application $f$ \`a $[q',p']$ si
$$
\sup_{t\ge p}f(t)>p.\eqdef{brr2}
$$ 

\'Etant donn\'ee une solution strictement positive $f$ de l'\'equation diff\'erentielle \eqref{eqOL} v\'erifiant \eqref{brr1}, 
les relations \eqref{eqOL} et \eqref{condd} impliquent d'une part que l'application $f$ est croissante sur $[p,+\infty[$ et d'autre part que 
$$
0<f(p)\le f(t)\le p\qquad(t\ge p)
$$
En posant $q=f(p)$ et en notant $g$ de la fonction $f$ \`a l'intervalle $[q,p]$, nous d\'eduisons alors de~\eqref{eqOL} que l'application $f$ 
est une solution du probl\`eme diff\'erentiel autonome 
$$
\Q\{\eqalign{f'(t)&={1\F g\b(f(t)\b)}\qquad(t\ge p)\cr f(p)&=q}\W.
$$
et donc que $f$ est enti\`erement d\'etermin\'ee sur $[p,+\infty[$ par la donn\'ee de la fonction $g$ et de l'\'egalit\'e $f(p)=q$. 
\bigskip

\'Etant donn\'ee une solution strictement positive $f$ de l'\'equation \eqref{eqOL} v\'erifiant~\eqref{brr1}, 
nous~d\'eduisons de \eqref{eqOL} et de \eqref{condd} d'une part que 
$$
f'(t)\ge0\qquad(t\ge p)
$$
et d'autre part que la fonction $f$ est croissante sur l'intervalle $[p,+\infty[$. Montrons par l'absurde qu'il existe 
un nombre $p'\ge p$ v\'erifiant $f(p')<p'$. Dans le cas contraire, nous aurions en effet  
$$
f(t)\ge t\qquad(t\ge p) \eqdef{Savage}
$$ 
et les relations \eqref{eqOL} et \eqref{condd} impliqueraient alors que 
$$
f'(t)<{t\F f\circ f(t)}\le {t\F f(t)}\le 1\qquad(t\ge p), 
$$
ce qui contredit l'in\'egalit\'e \eqref{Savage}. {\it A fortiori}, il existe un nombre $p'\ge p$ v\'erifiant $f(p')<p'$. 
La quantit\'e $f(x)$ \'etant d\'efinie, continue et strictement positive pour $f(p')\le x\le p'$ si l'\'equation \eqref{eqOL} a un sens, 
nous~d\'eduisons de l'in\'egalit\'e \eqref{brr2} que la restriction $g$ de la fonction~$f$ au segment $[q',p']$ est d\'efinie, continue et strictement positive. 
Par ailleurs, nous remarquons 
que la solution $f$ satisfait le syst\`eme \eqref{manowar} pour $(p^*,q^*)=(p',q')$ et par cons\'equent qu'elle est compl\`etement d\'etermin\'ee sur l'intervalle $[p',+\infty[$ par la donn\'ee des nombres $p'>q'=f'(p')>0$ et de sa restriction $g$ au segment  $[q',p']$. 
\bigskip


Pour finir, \'etablissons la propri\'et\'e \eqrefn{TTr}. 
Nous pouvons sans perte de
g\'en\'eralit\'e nous ramener au cas o\`u la fonction~$g$
cro\^\i t sur le segment~$[q,p]$. Lorsque $q<p$, il existe en effet 
un plus grand \'el\'ement $p'\in]p,\infty]$ tel que
$$
\int_q^pg(u)\d u\ge\int_p^{p'}h(u)\d u.
$$
L'application $g$ \'etant strictement positive sur $[q,p]$, quel que soit $t\in[p,p']$, 
il existe {\it a~fortiori} un~unique nombre r\'eel
$g(t)\in[q,p]$ v\'erifiant
$$
\int_q^{g(t)}g(u)\d u=\int_p^th(u)\d u.
$$
Ce prolongement de la fonction  $g$ sur $[p,p']$ est croissant, contin\^ument d\'erivable et satisfait 
$$
g'(t)={h(t)\F g\circ g(t)}\qquad(p\le t\le p').\eqdef{equoi}
$$
Par ailleurs, il constitue l'unique prolongement possible de la fonction $f$ 
\`a $[p,p']$ en une solution 
de l'\'equation diff\'erentielle \eqref{equoi}. Une~fonction
$f\in\sc C\b([q,\infty[\b)\cap \sc C^1\big([p,\infty[,[q,\infty[\big)$ est~a~fortiori solution de \eqref{manowar} si, et~seulement si, elle satisfait $f(t)=g(t)\ \,(q\le t<p)$ et 
$$
\Q\{
\eqalign{
f(t)=&g(t)\qquad\qquad(p\le t\le p'),
\cr\cr
f'(t)=&{h(t)\F f\circ f(t)}\qquad(t\ge p').
}
\W.
$$
Lorsque $p'=+\infty$, la propri\'et\'e \eqrefn{TTr} est d\'emontr\'ee et lorsque $p'\in]p,\infty[$, nous observons  
d'une part que $g(p')=p$ et d'autre part que nous nous sommes bien ramen\'es \`a l'\'etude du~probl\`eme \eqref{manowar} 
dans le cas o\`u la fonction $g$ est croissante sur l'intervalle $[q,p]$. 
\medskip

Nous prouvons dor\'enavant la propri\'et\'e \eqrefn{TTr} avec l'hypoth\`ese suppl\'ementaire 
que $g$ cro\^\i t sur le segment $[q,p]$.
Soit $\sc E$ l'ensemble
des~fonctions croissantes $f\in\sc F\big([q,\infty[\big)$ v\'erifiant
$$
f(t)=g(t)\qquad(q\le t\le p)
$$
et soit $\sc T:\sc E\to\sc E$ l'unique op\'erateur d\'etermin\'e par
$$
\sc T f(t)=q+\int_p^t{h(u)\F f\circ f(u)}\d u\qquad(f\in\sc E,t\ge p).
\eqdef{defT}
$$
Nous notons $f_0$ l'unique application de $\sc E$
v\'erifiant $f_0(t)=q\ \,(t\ge p)$ et nous posons 
$$
f_n:=\sc T^nf_0\qquad(n\ge1). \eqdef{deffn}
$$
Nous allons montrer que nous d\'efinissons bien sur l'intervalle $[q,\infty[$ une application $f$ 
poss\'edant les qualit\'es requises en posant 
$$
f(t):=\lim_{n\to\infty}f_{2n}(t)\qquad(t\ge q).\eqdef{deff2}
$$
\medskip

Pour commencer, prouvons que l'on d\'efinit deux fonctions $f\le \phi$ de $\sc E$ v\'erifiant $f=\sc T\phi$ et $\phi=\sc Tf$ en posant \eqref{deff2} et 
$$
\phi(t):=\lim_{n\to\infty}f_{2n+1}(t)\qquad(t\ge q).
$$
Pour chaque couple $(h_0,h_1)\in\sc E^2$ v\'erifiant $h_0\le h_1$, nous obtenons l'in\'egalit\'e 
$$
h_0\big(h_0(t)\big)\le h_1\big(h_0(t)\big)\le h_1\big(h_1(t)\big)
\qquad(t\ge p)
$$
que nous reportons dans l'identit\'e \eqref{defT} pour obtenir que $\sc Th_1\le\sc Th_0$. 
Comme le plus petit \'el\'ement de l'ensemble~$\sc E$ est $f_0$, nous remarquons que $f_0\le f_1$ et 
que $f_0\le f_2$. De~l'observation pr\'ec\'edente et de la~d\'efinition \eqref{deffn}, nous d\'eduisons 
alors que $f_2\le f_1$ puis que $f_0\le f_2\le f_1$.
En proc\'edant de m\^eme, nous prouvons que  
la suite $\{f_{2n}\}_{n\ge0}$ est~croissante, que la suite $\{f_{2n+1}\}_{n\ge0}$
est d\'ecroissante et qu'elles v\'erifient 
$$
f_{2n}\le f_{2n+1}\qquad(n\ge0). \eqdef{Froog}
$$
En particulier, les fonctions $f$ et $\varphi$ sont bien d\'efinies sur l'intervalle $[q,\infty[$ 
et elles appartiennent \`a l'ensemble $\sc E$, qui est ferm\'e pour la topologie de la convergence simple. 
Par passage \`a la limite dans l'in\'egalit\'e \eqref{Froog} et dans les identit\'es 
$$
f_{2n}=\sc Tf_{2n-1}\qquad\hbox{et}\qquad
f_{2n+1}=\sc Tf_{2n}\qquad(n\ge1),
$$ 
nous obtenons enfin que $f\le\phi$, que $f=\sc T\phi$ et que $\phi=\sc Tf$.  
\bigskip

Nous remarquons qu'une application $\psi$ appartient \`a $\sc C\b([q,\infty[\b)\cap\sc C^1\b([p,\infty[,[q,\infty[\b)$ et satisfait 
le syst\`eme \eqref{manowar} si, et seulement si, elle v\'erifie $\psi\in\sc E$ et $\psi=\sc T \psi$. 
De plus, nous~d\'emontrons qu'une telle application satisfait n\'ecessairement l'in\'egalit\'e 
$$
f\le \psi\le \phi.\eqdef{Sluug}
$$
{\it A~fortiori}, il suffit d'\'etablir l'\'egalit\'e $f=\phi$ pour \'etablir que 
la fonction $f$ d\'efinie par~\eqref{deff2} est une solution du syst\`eme \eqref{manowar}
appartenant \`a l'ensemble $\sc C\b([q,\infty[\b)\cap\sc C^1\b([p,\infty[,[q,\infty[\b)$ et que c'est la seule. 
Comme la fonction $f_0$ est le plus petit \'el\'ement de l'ensemble~$\sc E$, nous~remarquons en effet que $f_0\le\psi$ et  
nous d\'eduisons de l'\'egalit\'e $\psi=\sc T\psi$ et de~l'observation du paragraphe pr\'ec\'edent que 
$$
f_{2n}\le\psi\le f_{2n+1}\qquad(n\ge0). 
$$
En passant \`a la limite, nous obtenons alors l'in\'egalit\'e \eqref{Sluug}. 
\bigskip

Nous proc\'edons par l'absurde pour \'etablir que $f=\phi$. Nous supposons donc que $f\neq \phi$ 
et nous consid\'erons le nombre
$$
a:=\sup\big\{x\ge q:\forall t\in[q,x],f(t)=\phi(t)\big\}, \eqdef{defaaa}
$$
qui est fini et sup\'erieur \`a $p$ car les fonctions $f$ et $\phi$ appartiennent \`a l'ensemble $\sc E$. 
Nous~prouvons d'une part que $a$ est un point fixe de la fonction $f$ 
%$a=f(a)$ 
et d'autre part qu'il existe un nombre r\'eel $\epsilon>0$ pour lequel
$$
f(t)=\phi(t)\qquad(q\le t\le a+\epsilon), \eqdef{contradiction}
$$
ce qui est en contradiction avec la d\'efinition du nombre $a$. Les identit\'es $f=\sc T \phi$ et $\phi=\sc Tf$ 
induisent que les fonctions $f$ et $\phi$ sont de classe $\sc C^1$ 
sur $[p,\infty[$. De plus, en les d\'erivant, nous obtenons que 
$$
{f'(t)\F f\circ f(t)}={h(t)\F f\circ f(t)\phi\circ\phi(t)}={\phi'(t)\F\phi\circ\phi(t)}\qquad(t\ge p). \eqdef{egalimp}
$$
En remarquant que $f(t)=\phi(t)$ et que $f'(t)=\phi'(t)\neq 0$ pour $p\le t\le a$, il suit 
$$
f(y)=\phi(y)\qquad\Big(y\in f\big([p,a]\big)\Big).
$$
Comme $f(p)=q$ et comme l'application $f$ est continue et cro\^\i t sur l'intervalle $[q,\infty[$, 
nous~obtenons alors que $f(y)=\phi(y)\ \,\b(q\le y\le f(a)\b)$ et nous d\'eduisons que $q\le f(a)\le a$ de la d\'efinition \eqref{defaaa}.  
Par suite, la borne sup\'erieure  
$$
b:=\sup\{x\ge a:\forall t\in[a,x],q\le f(t)\le a\} \eqdef{defbbb}
$$
existe et nous d\'eduisons de l'in\'egalit\'e $q\le f(t)\le a\ \,(a\le t\le b)$ et de \eqref{defaaa} que 
$$
f\circ f(t)=\phi\big(f(t)\big)\qquad(a\le t\le b).
$$
Les identit\'es \eqref{egalimp} et $f(a)=\phi(a)$ impliquent alors que les fonctions $f$ et $\phi$
sont deux solutions du~m\^eme probl\`eme de Cauchy
$$
\Q\{
\eqalign{
&{y'(t)\F \phi\big(y(t)\big)}={\phi'(t)\F \phi\circ\phi(t)}\qquad(a\le t\le b),
\cr
&y(a)=\phi(a), 
}\W.
$$
et par cons\'equent qu'elles co\"\i ncident sur le segment $[a,b]$. Enfin, nous d\'eduisons que $b=a$ 
de la~d\'efinition~\eqref{defaaa} puis, comme $f$ est croissante, continue et satisfait $f(a)\le a$, 
nous~d\'eduisons que $f(a)=a$ de la d\'efinition  \eqref{defbbb}. Ainsi, $a$ est un point fixe de $f$. 
\bigskip


D\'erivant l'identit\'e $\phi-f=\sc Tf-\sc T\phi$, nous montrons que
$$
\phi'(t)-f'(t)=h(t){\phi\circ\phi(t)-f\circ f(t)\F f\circ f(t)\phi\circ\phi(t)}
\qquad(t\ge a).
$$
Comme l'in\'egalit\'e $f\le \phi$ induit que $f\circ f(t)\le \phi\circ\phi(t)\ \,(t\ge p)$ et comme 
les fonctions $h$, $f$ et $\phi$ sont continues et positives, l'application $\phi-f$ cro\^\i t sur l'intervalle
$[a,\infty[$ et satisfait
$$
\phi'(t)-f'(t)\ll \phi\circ\phi(t)-f\circ f(t)\qquad(a\le t\le  a+1).
$$
Comme $\phi$ et $f$ sont croissantes, de classe $\sc C^1$ sur $[a,+\infty[$ 
et v\'erifient $\phi(a)=f(a)=a$, nous remarquons que 
$$
\phi\circ\phi(t)-f\circ f(t)=\phi\b(\phi(t)\b)-f\b(\phi(t)\b)+\int_{f(t)}^{\phi(t)}f'(u)\d u\qquad(a\le t\le a+1).
$$
La fonction $f'$ \'etant born\'ee sur le segment $[a,a+1]$, il suit 
$$
\phi'(t)-f'(t)\ll (\phi-f)\circ \phi(t)+\phi(t)-f(t)\qquad(a\le t\le a+1).\eqdef{king}
$$
En d\'erivant la relation $\phi=\sc Tf$ au point $a$,  il r\'esulte de la relation $f(a)=a$ que 
$$
\phi'(a)={h(a)\F f\circ f(a)}={h(a)\F a}
$$
et nous remarquons que l'hypoth\`ese \eqref{condd} induit l'in\'egalit\'e $\phi'(a)<1$. 
Comme $\phi(a)=a$, il~existe {\it a fortiori} un nombre r\'eel $\epsilon\in]0,1[$ pour lequel
$$
a\le \phi(t)\le t\qquad(a\le t\le a+\epsilon).
$$
L'application $\Delta=\phi-f$ \'etant croissante sur le segment $[a,a+\epsilon]$, l'estimation 
\eqref{king} induit l'existence d'une constante $\kappa>0$ pour laquelle
$$
\Delta'(t)\le \kappa\Delta(t)\qquad(a\le t\le  a+\epsilon)
$$
et nous observons alors que la fonction $t\mapsto\Delta(t)\e^{-\kappa t}$ est d\'ecroissante sur $[a,a+\epsilon]$  
d'apr\`es l'in\'egalit\'e 
$$
{\d\F\d t}\b(\Delta(t)\e^{-\kappa t}\b)=\Delta'(t)\e^{-\kappa t}-\Delta(t)\e^{-\kappa t}\le 0\qquad(a\le t\le a+\epsilon). 
$$
Enfin, comme les relations $f(a)=\phi(a)$ et $f\le \phi$ impliquent d'une part que $\Delta(a)=0$ et d'autre part que 
$\Delta(t)\ge0\ \,(t\ge q)$, nous concluons que l'application $\Delta$ est identiquement nulle 
sur le segment $[a,a+\epsilon]$, 
c'est \`a dire que l'identit\'e \eqref{contradiction} est satisfaite. 
\hfill\qed
\bigskip



\Secti Algo, M\'ethode de calcul. 

Dans les paragraphes suivants, nous expliquons comment calculer num\'eriquement $u_n$ pour de grandes valeurs de l'entier $n$ 
et nous pr\'esentons les sources d'un programme, \'ecrit pour le logiciel {\it Mathematica 5}, permettant 
de tracer le graphe de la fonction $\aleph$ sur~l'intervalle $[6\lambda, 13\lambda]$.
\bigskip


Nous observons que les suites $\{u_n\}_{n=1}^\infty$ et $\{v_m\}_{m=1}^\infty$, d\'efinies par \eqref{defF} et par \eqref{defvm},  
peuvent \^etre consid\'er\'ees comme deux fonctions $u$ et $v$ 
du groupe $\b(\sc F(\ob N^*,\ob N^*),\circ\b)$~si~l'on~pose 
$$
u(n):=u_n\qquad(n\ge1)\qquad\hbox{ et }\qquad v(m):=v_m\qquad(m\ge1). 
$$
Pour chaque \'el\'ement $f$ du groupe $\b(\sc F(\ob N^*,\ob N^*),\circ\b)$, nous posons
$$
f^{[0]}(n):=t,\qquad f^{[k]}(n):=\underbrace{f\circ f\circ\cdots\circ f}_{k\hbox{ \sevenrm fois}}(n)\qquad(k\ge1,n\in\ob N^*)
$$
et nous d\'eduisons de la relation \eqref{fond} que 
$$
u\B(v^{[k+1]}(n)\B)=v^{[k]}(n)\qquad(k\ge0,n\ge1). 
$$
En particulier, le nombre $v^{[k]}(n)$ est l'\'el\'ement de rang $v^{[k+1]}(n)$ de la suite de Golomb. 
Par ailleurs, comme \eqref{DFSGol} implique l'existence de deux nombres 
$b>a>0$ tels que 
$$
a n^\varphi\le v(n)\le b n^\varphi\qquad(n\ge1), 
$$
en proc\'edant par r\'ecurrence sur l'entier $k$, nous d\'eduisons de l'identit\'e \eqref{Phi1} que 
$$
a^{-\varphi}(a^\varphi n)^{\varphi^k}\le v^{[k]}(n)\le b^{-\varphi}(b^\varphi n)^{\varphi^k}\qquad(k\ge0,n\ge1). 
$$
\'Etant donn\'e un entier $n_0>a^{-\varphi}$, nous en d\'eduisons que la suite $\b\{v^{[k]}(n)\b\}_{k=0}^\infty$ poss\`ede une croissance tr\`es rapide car 
doublement exponentielle pour chaque entier $n\ge n_0$ fix\'e. 
Pour~calculer num\'eriquement $u_n$ pour de grandes valeurs de l'entier $n$, 
il suffit alors de calculer les premiers termes de la suite $\b\{v^{[k]}(n)\b\}_{k=0}^\infty$ 
pour des valeurs fix\'ees de $n\ge n_0$. 
\bigskip


Pour chaque $k\ge0$, nous prouvons l'existence d'un polyn\^ome $P_k\in\ob Q[X,X_0,\cdots,X_{k-1}]$ 
v\'erifiant l'identit\'e  
$$
v^{[k+1]}(n)=1+\sum_{2\le m\le n}P_k\Q(u_m,v^{[0]}(m),\cdots,v^{[k-1]}(m)\W)\qquad(n\ge1).  \eqdef{Hern}
$$
En particulier, nous remarquons qu'il suffit de d\'eterminer les nombres $u_1, \cdots, u_N$ et les~polyn\^omes $P_0,\cdots, P_K$ 
pour d\'eduire de l'identit\'e pr\'ec\'edente les nombres $$
v^{[k]}(n)\ \,(0\le k\le K+1,1\le n\le N).
$$ 
Le programme propos\'e adopte cette strat\'egie et permet de calculer num\'eriquement les nombres $
v^{[k]}(n)\qquad(0\le k\le 9,1\le n\le 7000)$
en 13 heures, \`a l'aide un ordinateur monoposte muni de 512 Mo de m\'emoire vive et d'un microprocesseur cadenc\'e \`a 2,2 GHz. 
Le rang $n$ le plus \'elev\'e pour lequel nous avons calcul\'e $u_n$ est ainsi
$$
n=v^{[9]}(7000)\approx 4\times10^{276}. 
$$
Par ailleurs, nous observons que le calcul des nombres $u_1,\cdots, u_N$ est tr\`es rapide via \eqref{defF}. 
\bigskip



\'Etant donn\'es la suite  $\{\sc B_n\}_{n=0}^\infty$ 
des polyn\^omes de Bernoulli uniquement d\'etermin\'ee par 
$$
z{\e^{zx}\F\e^z-1}=\sum_{n=0}^\infty\sc B_n(x){z^n\F n!}\qquad(z\in\ob C,x\in\ob R), \eqdef{Bernou}
$$
l'anneau $\sc A:=\ob R\b[\{X_k:k\ge0\}\b]$ et l'unique endomorphisme $\sc T$ du $\sc A$-module $\sc A[X]$ v\'erifiant 
$$
\sc T\b(X^k\b)={\sc B_{k+1}(X)-\sc B_{k+1}(0)\F k+1}\qquad(k\ge0), \eqdef{ope}
$$
nous prouvons maintenant que l'identit\'e \eqref{Hern} est satisfaite, pour chaque entier $k\ge0$, 
par la suite de polyn\^omes $P_k\in\sc R[X,X_0,\cdots,X_{k-1}]\ \,(k\in\ob N)$ d\'efinie par r\'ecurrence par  
$$
\Q\{\eqalign{
P_0&:=X, 
\cr 
P_1&:=XX_0,
\cr
P_{k+2}&:=\sc T\B(P_{k+1}\b(X_0,X_1-P_0,\cdots, X_{k+1}-P_k\b)\B)\qquad(k\ge0).
}\W.\eqdef{defpoly} 
$$
Comme $v(1)=v_1=1$, nous remarquons que $v^{[k+1]}(1)=1\ \,(k\ge0)$ et nous d\'eduisons du principe 
des sommes t\'elescopiques que 
$$
v^{[k+1]}(n)=1+\sum_{2\le m\le n}\Q(v^{[k+1]}(m)-v^{[k+1]}(m-1)\W)\qquad(k\ge0,n\ge1).  
$$
Comme la relation \eqref{defvm} induit que $v(m)=v(m-1)+u_m\ \,(m\ge2)$, nous remarquons que 
$$
v^{[k+1]}(n)=1+\sum_{2\le m\le n}\Q(v^{[k+1]}(m)-v^{[k]}\b(v(m)-u_m)\W)\qquad(k\ge0,n\ge1).  
$$
Ainsi, pour prouver que l'identit\'e \eqref{Hern} est satisfaite par $P_k$, il suffit d'\'etablir que 
$$
v^{[k+1]}(m)-v^{[k]}\b(v(m)-\ell\b)=P_k\b(\ell,v^{[0]}(m),\cdots,v^{[k-1]}(m)\b)\quad(m\ge2,0\le \ell\le u_m).\!\!\!\!\!\eqdef{LO}
$$
Comme $v^{[0]}=\hbox{Id}_{\ob N^*}$ et comme 
$P_0(\ell)=\ell\ \,(\ell\in\ob R)$, 
l'identit\'e \eqref{LO} est v\'erifi\'ee pour $k=0$. 
Par ailleurs, nous d\'eduisons de la d\'efinition \eqref{defvm} que 
$$
v^{[2]}(m)-v\b(v(m)-\ell\b)=\sum_{v(m)-\ell<n\le v(m)}u_n\qquad(m\ge2,0\le\ell\le u_m). 
$$
Comme $v(m-1)\le v(m)-\ell\le v(m)$ lorsque $0\le \ell\le u_m$, la relation \eqref{fond} implique que 
$$
v^{[2]}(m)-v\b(v(m)-\ell\b)=\sum_{v(m)-\ell<n\le v(m)}m=\ell m\qquad(m\ge2,0\le\ell\le u_m). 
$$
L'identit\'e \eqref{LO} est par cons\'equent 
bien satisfaite pour $k=1$, d'apr\`es l'\'egalit\'e $P_1=XX_0$. \'Etant donn\'e 
un entier $K\ge1$ pour lequel la relation \eqref{LO} soit v\'erifi\'ee pour $0\le k\le K$, nous \'etablissons maintenant \eqref{LO} pour $k=K+1$. 
Nous fixons $n\ge2$ et $\ell\in\{0,\cdots,u_m\}$ et nous d\'eduisons du principe des sommes t\'elescopiques que 
$$
v^{[K+2]}(m)-v^{[K+1]}\b(v(m)-\ell\b)=\sum_{0\le a<\ell}\B(v^{[K+1]}\b(v(m)-a\b)-v^{[K+1]}\b(v(m)-a-1\b)\B). 
$$
Comme la d\'efinition \eqref{defvm} implique que $v\b(v(m)-a-1\b)=v\b(v(m)-a)-u\b(v(m)-a\b)$ pour $0\le a<\ell$, nous notons $\Delta_{m,\ell}$ le membre de gauche de l'identit\'e pr\'ec\'edente et nous~remarquons que 
$$
\Delta_{m,\ell}=\sum_{0\le a<\ell}\bg(v^{[K+1]}\b(v(m)-a\b)-v^{[K]}\B(v\b(v(m)-a\b)-u\b(v(m)-a\b)\B)\bg)
$$
et nous d\'eduisons de \eqref{LO} pour les entiers $k=K$, $m^*=v(m)-a$ et $\ell=u(m^*)$ que 
$$
\Delta_{m,\ell}=\sum_{0\le a<\ell}P_K\B(u\b(v(m)-a)\b),v(m)-a, \cdots, v^{[K-1]}\b(v(m)-a\b)\B). 
$$
Comme $v(m-1)<v(m)-a\le v(m)$ pour $0\le a<\ell$, la relation \eqref{fond} induit que 
$$
\Delta_{m,\ell}=\sum_{0\le a<\ell}P_K\B(m,v(m)-a, \cdots, v^{[K-1]}\b(v(m)-a\b)\B)
$$
et nous d\'eduisons de l'\'egalit\'e $P_0=X$ et de l'identit\'e \eqref{LO} pour $0\le k<K$ que 
$$
\Delta_{m,\ell}=\!\!\sum_{0\le a<\ell}\!\!P_K\!\B(\!m,v(m)-P_0(a), \cdots, v^{[K]}(m)-P_{K-1}
\b(a,m,\cdots,v^{[K-2]}(m)\b)\!\B)\!
$$
et aussi que 
$$
\Delta_{m,\ell}=\sum_{0\le a<\ell}P_K\b(X_0,X_1-P_0,\cdots, X_K-P_{K-1}\b)
\Q[a, v^{[0]}(m), \cdots, v^{[K]}(m)\W]. \eqdef{Nocturnus}
$$
Comme la d\'efinition \eqref{Bernou} implique que 
$$
\sum_{n=0}^\infty(\sc B_n(x+1)-\sc B_n(x)\b){z^n\F n!}=z\e^{zx}=\sum_{n=1}^\infty x^{n-1}{z^n\F(n-1)!}\qquad(z\in\ob C,x\in\ob R),
$$
nous remarquons d'une part que 
$$
{\sc B_n(x+1)-\sc B_n(x)\F n}=x^{n-1}\qquad(n\ge1,x\in\ob R)
$$
et d'autre part que 
$$
{\sc B_{n+1}(\ell)-\sc B_{n+1}(0)\F n+1}=\sum_{0\le a<\ell}{\sc B_{n+1}(a+1)-\sc B_{n+1}(a)\F n}=\sum_{0\le a<\ell}a^n\qquad(n\ge0,\ell\ge0).
$$
En particulier, nous d\'eduisons de la relation \eqref{ope} que 
$$
\sc T(X^n)[\ell]=\sum_{0\le a<\ell}a^n\qquad(n\ge0,\ell\ge0).
$$ 
{\it A fortiori}, pour $K\ge0$ et $P\in\ob R[X,X_0,\cdots,X_K]$, l'op\'erateur $\sc T$ satisfait la relation 
$$
\sc T(P)(\ell,x_0,\cdots,x_k)=\sum_{0\le a<\ell}P(a,x_0,\cdots,x_K)\qquad\b(\ell\ge0,(x_0,\cdots,x_K)\in\ob R^{K+1}\b) 
$$
et nous d\'eduisons de l'identit\'e \eqref{Nocturnus} que 
$$
\Delta_{m,\ell}=\sc T\B(P_K\b(X_0,X_1-P_0,\cdots, X_K-P_{K-1}\b)\B)
\Q[\ell, v^{[0]}(m), \cdots, v^{[K]}(m)\W]. 
$$
Pour $m\ge2$ et $0\le\ell\le u_m$, la relation de r\'ecurrence \eqref{defpoly} implique alors que 
$$
v^{[K+2]}(m)-v^{[K+1]}\b(v(m)-\ell\b)=\Delta_{m,\ell}=P_{K+1}\b(\ell,v^{[0]}(m), \cdots, v^{[K]}(m)\b). 
$$
En particulier, l'identit\'e \eqref{LO} est satisfaite pour l'entier $k=K+1$. \hfill\qed\null
\bigskip


\Secti Source, Programme. 

\noindent
{\bf Initialisation}
\medskip

\noindent
ClearAll["Global`*"];
\medskip

\noindent
{\bf 
D\'efinition de l'op\'erateur T
}
\medskip

\noindent
T[n\_]:=T[n]=(Bernoulli[n+1,x]-Bernoulli[n+1,0])/(n+1);
\medskip

\noindent
{\bf 
D\'efinition r\'ecurrente des polyn\^omes P[k]
}
\medskip


\noindent
P[0]=x;
P[1]=x X[0]; P[k\_/;2$\le$k$\le$7]:=
P[k]=

\noindent
Expand[ReplaceAll[Expand[x$^2$ 
ReplaceAll[P[k-1],$\{$x$\rightarrow$X[0],X[n\_]$\rightarrow$X[n+1]-P[n]$\}$]],

\noindent
x$^{n\_}\rightarrow$T[n-2]]];
\medskip

\noindent
{\bf 
D\'efinition r\'ecurrente de la suite de Golomb U[n]
}
\medskip

\noindent
U[1]=1;

\noindent
U[n\_/;n$>1$]:=IIfn-1==Sum[U[k],$\{$k,1,U[n-1]$\}$],U[n]=U[n-1]+1,U[n]=U[n-1]];
\medskip

\noindent
{\bf 
D\'efinition r\'ecurrente des nombres V[k,n]
}
\medskip

\noindent
V[k\_/;k==,n\_]:=V[0,n]=n;

\noindent
V[k\_/;k$>$0,n\_/;n==1]:=VVk,n]=1;

\noindent
V[k\_/;k==1,n\_/;n$>$1]:=VVk,n]=VVk,n-1]+U[n]; 

\noindent
V[k\_/;2$\le$k$\le9$,n\_/;n$>$1]:=VVk, n]=V(k,n-1]+RReplaceAllExpand[x$^2$ ReplaceAll[P[k - 2],

\noindent
Flatten[$\{$x$\rightarrow$n,Table[X[h]$\rightarrow$V[h+1,n]-q[h,n],$\{$h, 0, K-1$\}$]$\}$,1]]], x$^{d\_}\rightarrow$T[d-2,U[n]]];
\medskip

\noindent
{\bf
Choix des param\`etres K et L. 
{\it Valeurs conseill\'ees :  $0\le K\le 9$ et $2\le L\le10000$. }
}
\medskip

\noindent
K=4;
L=100;
\medskip

\noindent
{\bf 
D\'etermination et temps de calcul des quantit\'es P[k], U[n] et V[k+1,n]. 
}
\medskip

\noindent
Print["Calcul des P[k] pour 0$\le$k$\le$",Max[0,Min[7,K-2]]," effectu\'e en  ", 
Timing[Table[P[k],

\noindent
$\{$k,0,Min[7,K-2]$\}$]][[1]]];

\noindent
Print["Calcul des U[n] pour 1$\le$n$\le$", L, " effectu\'e en ",
Timing[Table[U[n],$\{$n,2,L$\}$]][[1]]);

\noindent
Do[

\noindent
Print["Calcul des V[", a,",n] pour 1$\le$n$\le$,L, effectu\'e en ",Timing[Table[V[a,n],$\{$n,0,L$\}$]]

\noindent
[[1]]],$\{$a,0,K+1$\}$];
\medskip
\noindent
{\bf 
Calcul des valeurs de la fonction $\aleph$. 
}
\medskip

\noindent
$\aleph$ = Sort[N[Flatten[Table[$\{$Log[Log[V[k,n]]],(V[k-1,n] V[k,n]$^{\theta}$-c)Log[V[k,n]]$\}$,$\{$k,2,K+1$\}$,

\noindent
$\{$n,3,L$\}$],1]]];
\medskip

\noindent
{\bf
Graphe de la fonction $\aleph$ sur l'intervalle [6$\lambda$,13$\lambda$]
}
\medskip

\noindent
ListPlot[$\aleph$,PlotRange$\rightarrow\{\{$5$\lambda$,13$\lambda\}$,$\{$
-0.0075,0.0055$\}\}$,Axes$\rightarrow$True,AxesOrigin$\rightarrow\{$5$\lambda$,0$\}$,

\noindent
PlotStyle$\rightarrow$PointSize[0.002],Ticks$\rightarrow$
$\{$Table[$\{$k $\lambda$, k "$\lambda$"$\}$,$\{$k,6,13$\}$],

\noindent
Table[$\{$k 0.0025,k 0.0025$\}$,
$\{$k,-3,2$\}$]$\}$];
\bigskip
\hfill\eject
\Secti Biblio, Bibliographie.

{\eightpts
\DefRef Fine ; N. J. Fine, Solution to Problem 5407 ; Amer. Math. Monthly ; 74 ; (1967), 740--743.

\DefRef Golomb ; S. W. Golomb, Problem 5407 ; Amer. Math. Monthly ; 73 ; (1966), 674.

\DefRef Knuth ; R. L. Graham, D. E. Knuth and O. Patashnik, ``Concrete Mathematics'' ; Addison-Wesley, Reading, MA ; { } ; $\!\!$(1988), 567.

\DefRef McKiernan2 ; M. A. McKiernan, On the nth derivative of composite functions ; \it Amer. Math. Monthly ; 63 ; (1956).

\DefRef McKiernan ; M. A. McKiernan, The functional differential equation $Df=1/f\circ f$ ; Proc. Amer. Math. Soc. ; 8 ; (1957), 230-233.

\DefRef Petermann1 ; Y.-F. S. P\'etermann, On Golomb's self describing sequence ; J. Number Theory ; 53 ; (1995), 13--24.

\DefRef Petermann2 ; Y.-F. S. P\'etermann, On Golomb's self describing sequence II ; Arch. Math. ; 67 ; (1996), 473--477.

\DefRef Petermann3 ; Y.-F. S. P\'etermann, Problem 10753 ; Amer. Math. Monthly ; ? ; (199?), ?--?.

\DefRef PetermannRemy ; Y.-F. S. P\'etermann et J. L. R\'emy, Golomb's selfdescribing sequence 
and functional differential equations ; Illinois J. Math. ; 42 ; (1998), 420--440.

\DefRef PetermannRemyVardi ; Y.-F. S. P\'etermann, J. L. R\'emy et I. Vardi, On a functional-differential equation related 
to Golomb's self-described sequence ; Journal de Th\'eorie des nombres de Bordeaux ; 11 ; (1999), 211--230.

\DefRef Remy1 ; J. L. R\'emy, Sur la suite autod\'ecrite de Golomb ; J. Number. Theory ; 66 ; (1997), 1--28.

\DefRef Vardi ; I. Vardi, The error term in Golomb's sequence ; J. Number. Theory ; 40 ; (1992), 1--11.
\par
}
\bye









%%% Local Variables:
%%% mode: plain-tex
%%% TeX-master: t
%%% End: