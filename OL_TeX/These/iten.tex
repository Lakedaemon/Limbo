% MISE EN PAGE



%\magnification 1100

%
% Definitions pour les notes en bas de page
% [Format : \note{texte de la note} avec numerotation automatique]
%
\let\footnotea=\footnote
\catcode`@=11
\message{output routines,}
\newinsert\footins
\def\footnote#1{\let\@sf\empty % parameter #2 (the text) is read later
  \ifhmode\edef\@sf{\spacefactor\the\spacefactor}\/\fi
\hbox{$^{(#1)}$}\@sf\vfootnote{#1.}}
\def\vfootnote#1{\insert\footins\bgroup
  \interlinepenalty\interfootnotelinepenalty
  \splittopskip\ht\strutbox % top baseline for broken footnotes
  \splitmaxdepth\dp\strutbox \floatingpenalty\@MM
  \leftskip\z@skip \rightskip\z@skip \spaceskip\z@skip \xspaceskip\z@skip
  \textindent{#1}\footstrut\futurelet\next\fo@t}
\def\fo@t{\ifcat\bgroup\noexpand\next \let\next\f@@t
  \else\let\next\f@t\fi \next}
\def\f@@t{\bgroup\aftergroup\@foot\let\next}
\def\f@t#1{#1\@foot}
\def\@foot{\strut\egroup}
\def\footstrut{\vbox to\splittopskip{}}
\skip\footins=\bigskipamount % space added when footnote is present
\count\footins=800 % footnote magnification factor (1 to 1)
\dimen\footins=8in % maximum footnotes per page

%
%\anote=\footnote de plain (2 arguments)
%
\def\anote#1#2{\footnotea{\hbox{$^{#1}$}}{\eightpoint#2}}
\catcode`@=12 % at signs are no longer letters

%reference de note
 \def\defrefnote#1{\definexref{#1}{{\the\footnotenumber}}{refnotes}}
 \def\refnote#1{\ref{#1}}

%
% Fin notes
%


%\def\eqlgno#1{$$\leqalignno{#1}$$}


 % choix numerotation gauche ou droite: defaut=gauche, \numadroite{oui} -> a droite
 \def\testd{oui}
 \def\choixlat{\ifx\numadroite\testd \let\leqno=\eqno\let\leqalignno=\eqalignno
          \else\let\eqno=\leqno\let\eqalignno=\leqalignno \fi}
  \choixlat
  \input eplaingt


\interfootnoteskip=0pt
\let\note=\numberedfootnote
\everyfootnote={\eightpoint\leftskip=5truemm\rightskip5truemm}

\hsize150truemm\vsize 240truemm\hoffset=5truemm\voffset=-3truemm
\def\dimstand{\hsize 150truemm\vsize 240truemm\hoffset=5truemm\voffset=0truemm}
\def\dimart{\hsize126truemm\vsize186truemm\hoffset16truemm\voffset=24truemm}
\pretolerance=500\tolerance=1000\brokenpenalty=5000
\parindent3mm

\countdef\temps=170
\temps=\time
\countdef\nminutes=171{\nminutes = \time}
\countdef\nheure=172
\def\heure{\begingroup                     % heure a la francaise
   \temps = \time \divide\temps by 60
   \nheure = \temps                      % l'heure, de 0 \`a 23
   \nminutes = \time
   \multiply\temps by 60
   \advance\nminutes by -\temps            % Les minutes, de 0 a 59.
   \ifnum\nminutes<10 \toks1 = {0}%
   \else\toks1 = {}%
   \fi
   \number\nheure h\the\toks1 \number\nminutes
\endgroup}%


\newcount\chstart
\chstart=\pageno
 \headline={\ifnum\pageno=\chstart {\hfill} \else {\hss \tenrm --\ \folio\ --\hss}\fi}
\footline={\hfill}
\normalbaselines
\frenchspacing
  \def\dater{\vglue-10mm\rightline{(\the\day/\the\month/\the\year)}}
\def\dateheure{\vglue-10mm\rightline{(\the\day/\the\month/\the\year,\ \heure)}}

\newif\ifpagetitre \pagetitretrue
\newtoks\hautpagetitre \hautpagetitre={\hfill}
\newtoks\baspagetitre \baspagetitre={\hfill}
\newtoks\auteurcourant \auteurcourant={\hfill}
\newtoks\titrecourant \titrecourant={\hfill}
\newtoks\hautpagegauche
\newtoks\hautpagedroite
\newtoks\hautpagemilieu
\hautpagemilieu={\tenrm\hfil -- \folio\ -- \hfil}
\hautpagegauche={\ifx\midfolio\oui\the\hautpagemilieu\else\tenrm\folio\hfil\the\auteurcourant\hfil\fi}
\hautpagedroite={\ifx\midfolio\oui\the\hautpagemilieu\else\hfil\the\titrecourant\hfil\tenrm\folio\fi}
\newtoks\baspagegauche \baspagegauche={\hfil}
\newtoks\baspagedroite \baspagedroite={\hfil}
\headline={\ifpagetitre\the\hautpagetitre
\else\ifodd\pageno\the\hautpagedroite\else\the\hautpagegauche\fi\fi }
\footline={\ifpagetitre\the\baspagetitre
\else\ifodd\pageno\the\baspagedroite
\else\the\baspagegauche\fi\fi \global\pagetitrefalse}

\def\nopagenumber{\headline={\hfill}\footline={\hfill}}
%\def\numpgsuiv{ \chstart=\pageno\pageno=\chstart}

\def\pageblanche{\vfill\eject\pagetitretrue
\null\vfill\eject
\pagetitretrue
}


\def\hautspages#1#2{\auteurcourant={\ninepcap#1}\titrecourant={\nineit#2}}

\ifnum\chstart=\pageno \pagetitretrue\fi
%\ifnum\pageno=\chstart \pagetitretrue \fi

%pour eviter les rectangles noirs:
%\overfullrule=Opt
% pour que les @ soient des lettres
\def\signeat{\catcode`@=11}


%% pour aller a la ligne dans un \proclaim:
\def\PAR{\par}

%% Note en marge gauche
\def\leftnote#1{\vadjust{\setbox1=\vtop{\hsize 20mm\parindent=0pt\eightpoint
\baselineskip=9pt\rightskip=4mm plus 4mm\vskip-4.7mm#1}\hbox{\kern-2cm\smash{\box1}}}}
\def\OK{\leftnote{\bf OK ?}}

%% Pour couper automatiquement un titre trop long

\def\raggedcenter{\leftskip=20pt plus 10em  % reglages initial 4em
     \rightskip=\leftskip
      \parfillskip=0pt
       \spaceskip=.3333em \xspaceskip=.5em
        \pretolerance=9999 \tolerance=9999
         \hyphenpenalty=9999 \exhyphenpenalty=9999 }

\def\titrecentre#1{{\parindent0mm\raggedcenter\tit#1\par}}


%%%%%% D\'ebut num\'erotation automatique des paragraphes et/ou des enonces


\def\oui{oui}

 \def\fontetitreun{\ifx\paradouze\oui\douzepts\gpdouze\twelvebf\else
\quatorzepts\gpquatorze\fourteenbf\fi}
 \def\fontetitredeux{\ifx\paradouze\oui\onzepts\textfont1=\eleveni\scriptfont1=\ninei\elevenit\else
                      \douzepts\twelveit\fi}

 \def\fontetitredeuxb{\ifx\paradouze\oui\onzepts\eleventi\gponze\textfont1=\elevenib\scriptfont1=\nineib
                       \else\douzepts\twelveti\scriptfont1=\twelveib\scriptfont1=\tenib\gpdouze\fi}

\def\fontetitretrois{\textfont0=\elevenrm\scriptfont0=\eightrm\textfont1=\eleveni
                    \scriptfont1=\eighti\scriptscriptfont1=\sixi\elevenit}

\newcount\titreun\titreun=0
\newcount\titredeux\titredeux=0
\newcount\titretrois\titretrois=0
\newcount\enonce\enonce=0

\def\incr#1{\global\advance#1 by 1 {\the #1}}
\def\avance#1{\global\advance#1 by 1}
\def\init#1{\global#1=0}

\long\def\Indentation#1#2{\setbox10=\hbox{\fontetitreun#1}
                          \ifdim\wd10 < 4mm
                       \setbox10=\hbox to 4mm{\box10\hfill}
                     \else\ifdim\wd10 < 6mm
                       \setbox10=\hbox to 6mm{\box10\hfill}
                          \else\ifdim\wd10 < 8mm
                       \setbox10=\hbox to 8mm{\box10\hfill}
                     \else\ifdim\wd10 < 12mm
                       \setbox10=\hbox to 12mm{\box10\hfill}
                     \fi\fi\fi\fi
                     \dimen10=\hsize
                     \advance \dimen10 by -\wd10
                     \noindent \box10 %\ignorespaces
                     \hbox{\vtop{\hsize=\dimen10\raggedright\noindent\fontetitreun#2}}}

\long\def\paraun#1{\removelastskip\par\bigskip\bigskip\goodbreak\vskip0pt plus.01\vsize\penalty-100
              \vskip0pt plus-.01\vsize
                    \init{\titredeux}\ifnum\optionparag=1{\init\eqnumber\init\enonce}\else{}\fi
                \goodbreak{\fontetitreun
                      \Indentation{\incr{\titreun}.\ }{\fontetitreun #1\par}}\nobreak\medskip}

 %
 % titres de paragraphes centres
 %
 \long\def\paraunc#1{\removelastskip\bigskip\goodbreak\vskip0pt plus.01\vsize\penalty-100
              \vskip0pt plus-.01\vsize
                    \init{\titredeux}
               \ifnum\optionparag=1{\init{\eqnumber}\init\enonce}\else{}\fi
                \goodbreak
                      {\parindent0mm\raggedcenter\fontetitreun\incr{\titreun}.\
                   \fontetitreun #1\par}\nobreak\medskip}

\long\def\paradeux#1{\init{\titretrois}\vskip0pt plus.01\vsize\penalty-10
              \vskip0pt plus-.01\vsize{
               \Indentation{\fontetitredeux\the\titreun.\incr{\titredeux}.\ }
                            {\fontetitredeux#1}}\nobreak\smallskip}

\long\def\paradeuxb#1{\init{\titretrois}\vskip0pt plus.001\vsize\penalty-10
              \vskip0pt plus-.01\vsize{
                \Indentation
{\fontetitredeuxb\the\titreun.\incr{\titredeux}.}{ \fontetitredeuxb#1}}\nobreak\smallskip
\par}


\long\def\paratrois#1{\ifdim\lastskip<\smallskipamount
              \removelastskip\smallskip\fi
               \vskip0pt plus.01\vsize\penalty-10
                \vskip0pt
plus-.01\vsize{\Indentation{\fontetitretrois\the\titreun.\the\titredeux.\incr{\titretrois}.\ }
{\hskip0mm\fontetitretrois#1}\smallskip\nobreak\par}}

% Pour memoriser le no de section d'un titre de paragraphe
\def\drefun#1{\definexref{�#1}{{\the\titreun}}{}}
\def\drefdeux#1{\definexref{�#1}{{\the\titreun}.{\the\titredeux}}{}}
\def\dreftrois#1{\definexref{�#1}{{\the\titreun}.{\the\titredeux}.{\the\titretrois}}{}}
% Dans le cas de \drefun, coder \paraun{Titre}\drefun{label} dans le cas de \drefdeux, on peut coder
%                                \paradeux{Sous-titre\drefdeux{label}}
% pour appeler la reference "label", taper \refn{�label}


%% num\'erotation des \'enonc\'es
%%Exemples
%%\propt{leb}{ (Lebesgue)}{ L'espace $L^1$ est complet.}
%%\par
%%Ca marche, comme le dit le \ref{sleb}.
%%\Propt{leb2}{L'espace $L^1$ est vraiment complet.}\par
%%Ce \ref{leb2} est moins bien que les Th\'eor\`emes \ref{sleb} et \ref{sleb2}.


\long\def\propdeux#1#2#3#4{\avance{\enonce}{\definexref{#2}{#1~{\the\titreun.\the\enonce}}{}}
{\definexref{s#2}{{\the\titreun.\the\enonce}}{}}
 \medbreak
   \noindent{\bf#1\ {\bf\the\titreun.\the\enonce{#3}.}\enspace}{\sl#4\par}%
   \ifdim\lastskip<\medskipamount \removelastskip\penalty55\medskip\fi
}

\long\def\propun#1#2#3#4{\avance{\enonce}{\definexref{#2}{#1~{\the\enonce}}{}}
{\definexref{s#2}{{\the\enonce}}{}}
 \medbreak
   \noindent{\bf#1\ {\bf\the\enonce{#3}.}\enspace}{\sl#4\par}%
   \ifdim\lastskip<\medskipamount \removelastskip\penalty55\medskip\fi
}

\long\def\prop#1#2#3#4{\ifnum\optionparag=1
                        \propdeux{#1}{#2}{#3}{#4} \else\propun{#1}{#2}{#3}{#4}\fi}


\long\def\propt#1#2#3{\ifx\tpf\oui \prop{Th\'eo\-r\`eme}{#1}{#2}{#3}\par
                     \else\prop{Theorem}{#1}{#2}{#3}\par\fi}
\long\def\Propt#1#2{\propt{#1}{}{#2}}
\long\def\propl#1#2#3{\ifx\tpf\oui\prop{Lem\-me}{#1}{#2}{#3}\par
                       \else\prop{Lemma}{#1}{#2}{#3}\par\fi}
\long\def\Propl#1#2{\propl{#1}{}{#2}}
\long\def\propc#1#2#3{\ifx\tpf\oui\prop{Corol\-laire}{#1}{#2}{#3}\par
                       \else\prop{Corollary}{#1}{#2}{#3}\par\fi}
\long\def\Propc#1#2{\propc{#1}{}{#2}}
\long\def\propp#1#2#3{\prop{Pro\-po\-si\-tion}{#1}{#2}{#3}\par}
\long\def\Propp#1#2{\propp{#1}{}{#2}}
\long\def\propd#1#2#3{\ifx\tpf\oui\prop{D\'efi\-nition}{#1}{#2}{#3}\par
                     \else\prop{Definition}{#1}{#2}{#3}\par\fi}
\long\def\Propd#1#2{\propd{#1}{}{#2}}

% Pour une numerotation des enonces et des formules avec un seul nombre
% coder \optionparag=2

% Choisir numerotation manuelle ou automatique des paragraphes:
% utiliser \section au lieu de \paraun et choisir \optionparag=2
% Cela induit aussi une num\'erotation des formules avec un seul nombre

\newcount\optionparag\optionparag=1

\long\def\section#1#2{\ifnum\optionparag=1 \paraun{#2}
                      \else\goodbreak{\fontetitreun
                      \Indentation{#1.\ }{#2}}\nobreak\smallskip\fi}


\long\def\sectionl{\section}

\def\numauto{\optionparag=1}  % numerotation automatique (par defaut)
\def\nummanu{\optionparag=2}  % numerotation manuelle
\def\eqconstruct#1{\ifnum\optionparag=1{\the\titreun\hbox{$\cdot$}#1}\else{#1}\fi}



%%%%%%% Fin num\'erotation automatique des paragraphes

%% num\'erotation bibliographie

\def\numref{oui}  % numeroter la biblio par defaut avec bibtem

\newcount\mesref\mesref=0
\def\defbib#1{\ifx\numref\oui\global\advance\mesref by 1 \definexref{#1}{{\the
               \mesref}}{}\else\definexref{#1}{#1}{}\fi}
\def\bibtem#1{\defbib{#1}\item{\citer{#1}}}
\def\citer#1{[\ref{#1}]}



% FONTES & FAMILLES

\font\seventeenmsa=msam10 at 17pt    % symboles d'AMSTEX
\font\fourteenmsa=msam10 at 14pt
\font\twelvemsa=msam10 at 12pt
\font\tenmsa=msam10
\font\ninemsa=msam10 at 9pt
\font\eightmsa=msam10 at 8pt
\font\sevenmsa=msam7
\font\sixmsa=msam10 at 6pt
\font\fivemsa=msam5
\newfam\msafam\textfont\msafam=\tenmsa\scriptfont\msafam=\sevenmsa\scriptscriptfont\msafam=\fivemsa

\font\seventeenbb=msbm10 at 17pt     % Lettres evidees pour titres
\font\fourteenbb=msbm10 at 14pt
\font\twelvebb=msbm10 at 12pt
\font\tenbb=msbm10                   %Lettres evidees
\font\ninebb=msbm10 at 9pt
\font\eightbb=msbm10 at 8pt
\font\sevenbb=msbm7
\font\sixbb=msbm10 at 6pt
\font\fivebb=msbm5
\newfam\bbfam\textfont\bbfam=\tenbb\scriptfont\bbfam=\sevenbb\scriptscriptfont\bbfam=\fivebb
\def\bb{\fam\bbfam\tenbb}%


\font\seventeenscaln=eusm10 at 17pt   % Lettres de ronde
\font\twelvescaln=eusm10 at 12pt
\font\tenscaln=eusm10
\font\ninescaln=eusm10 scaled 900
\font\eightscaln=eusm10 scaled 800
\font\sevenscaln=eusm10 scaled 700
\font\sixscaln=eusm10 scaled 600
\font\fivescaln=eusm5
\newfam\scalnfam\textfont\scalnfam=\tenscaln\scriptfont\scalnfam=\sevenscaln
\def\scaln{\fam\scalnfam\tenscaln}%
\def\scal{\scaln}

\font\tenscalb=eusb10                % Lettres de ronde grasses
\font\ninescalb=eusb10 scaled 900
\font\eightscalb=eusb10 scaled 800
\font\sevenscalb=eusb10 scaled 700
\font\sixscalb=eusb10 scaled 600
\font\fivescalb=eusb5
\newfam\scalbfam\textfont\scalbfam=\tenscalb\scriptfont\scalbfam=\sevenscalb
\def\scalb{\fam\scalbfam\tenscalb}%

%
% Romain
%
\font\fourteenrm=cmr12 scaled 1200
\font\elevenrm=cmr10 at 11pt
\font\twelverm=cmr12
\font\ninerm=cmr9
\font\eightrm=cmr8
\font\sevenrm=cmr7
\font\sixrm=cmr6



\font\seventeenpcap=cmcsc10 at 17pt
\font\tenpcap=cmcsc10                        % Petites capitales
\font\ninepcap=cmcsc9
\font\eightpcap=cmcsc8
\font\sevenpcap=cmcsc10 scaled 700
\font\sixpcap=cmcsc10 scaled 600
\newfam\pcapfam\textfont\pcapfam=\tenpcap\scriptfont\pcapfam=\sevenpcap
\def\pcap{\fam\pcapfam\tenpcap}

\font\gras=cmbx10 scaled 1100             % Gras romain (boldface)
\font\TIT=cmbx12 scaled 1600              % Titres
\font\seventeenrm=cmbx12 scaled 1400
\font\chiffre=cmbx10 scaled 1600
\font\titre=cmbx12 scaled 1200
\font\pti=cmbx12 scaled 1200
\font\fourteenbf=cmbx10 scaled 1400
\font\thirteenbf=cmbx10 at 13pt
\font\twelvebf=cmbx12
\font\elevenbf=cmbx10 at 11pt
\font\ninebf=cmbx9
\font\eightbf=cmbx8
\font\sixbf=cmbx6

\font\tengot=eufm10                           % Lettres gothiques
\font\ninegot=eufm10 at 9pt
\font\eightgot=eufm10 at 8truept
\font\sevengot=eufm7
\font\sixgot=eufm10 at 6 truept
\font\fivegot=eufm5
\newfam\gotfam
\textfont\gotfam=\tengot\scriptfont\gotfam=\sevengot
\def\got{\fam\gotfam\tengot}%


%% Pour les titres (168pt)

\def\tit{%
\textfont0=\seventeenrm\scriptfont0=\tenrm\def\rm{\fam0\seventeenrm}%
\textfont1=\seventeenib\scriptfont1=\twelveib%
\textfont2=\seventeensy\scriptfont2=\twelvesy\scriptscriptfont2=\ninesy
\textfont3=\seventeenex
\textfont\itfam=\seventeenti
\def\it{\fam\itfam\seventeenti}%
\textfont\bbfam=\seventeenbb \scriptfont\bbfam=\twelvebb
\def\bb{\fam\bbfam\seventeenbb}%
\textfont\msafam=\seventeenmsa\scriptfont\msafam=\twelvemsa
\textfont\scalnfam=\seventeenscaln
\def\pcap{\fam\pcapfam\seventeenpcap}
\normalbaselineskip=25pt\normalbaselines\rm}


% italiques grasses pour titres
\font\seventeenti=cmbxti10 scaled 1680
\font\fourteenti=cmbxti10 at 14pt
\font\thirteenti=cmbxti10 at 13pt
\font\twelveti=cmbxti10 scaled 1200
\font\eleventi=cmbxti10 at 11pt
\font\tenti=cmbxti10
\font\nineti=cmbxti10 at 9pt

%
% italiques
%
\font\twelveit=cmti12
\font\elevenit=cmti10 scaled 1100
\font\nineit=cmti9
\font\eightit=cmti8
\font\sevenit=cmti7
\font\sixit=cmti10 scaled 600
\font\fiveit=cmti10 scaled 500

%
% italique mathematique gras pour titres
%
\font\seventeenib=cmmib10 scaled 1680
\font\fourteenib=cmmib10 scaled 1400
\font\twelveib=cmmib10 scaled 1200
\font\elevenib=cmmib10 scaled 1100
\font\tenib=cmmib10
\font\nineib=cmmib10 scaled 900


%
% italique mathematique
%
\font\twelvei=cmmi10 scaled 1200
\font\eleveni=cmmi10 scaled 1100
\font\ninei=cmmi9
\font\eighti=cmmi8
\font\seveni=cmmi7
\font\sixi=cmmi6

\font\ninesl=cmsl9                  % slanted
\font\eightsl=cmsl8
\font\sevensl=cmsl10 at 7pt
\font\sixsl=cmsl10 at 6pt
\font\fivesl=cmsl10 at 5pt

\font\ninett=cmtt9                    % typewriter
\font\eighttt=cmtt8
\font\seventt=cmtt10 scaled 700
\font\sixtt=cmtt10 scaled 550
\font\fivett=cmtt10 scaled 500

\font\seventeensy=cmsy10 scaled 1680    % symboles pour titres
\font\fourteensy=cmsy10 scaled 1400
\font\twelvesy=cmsy10 scaled 1176
\font\ninesy=cmsy9                      % symboles
\font\eightsy=cmsy8
\font\sixsy=cmsy6
\font\seventeenex=cmex10 at 17pt
\font\fourteenex=cmex10 at 14pt
\font\twelveex=cmex10 at 12pt
\font\nineex=cmex10 at 9pt
\font\eightex=cmex10 at 8pt
\font\sevenex=cmex10 at 7pt
\font\sixex=cmex10 at 6pt
\font\fiveex=cmex10 at 5pt

 %% Lettres grecques plombees              %% Lettres grecques plombees

\font\fourteengp=cmmi10 at 14pt
\font\thirteengp=cmmib10 at 13pt
\font\twelvegp=cmmib10 at 12pt
\font\elevengp=cmmib10 at 11pt
\font\tengp=cmmib10
\font\ninegp=cmmib10 at 9pt
\font\eightgp=cmmib8
\font\sevengp=cmmib7
\font\sixgp=cmmib6
\font\fivegp=cmmib5

 \def\gphuit{\textfont0=\eightbf\scriptfont0=\sixbf\scriptscriptfont0=\fivebf
         \textfont1=\eightgp\scriptfont1=\sixgp\scriptscriptfont1=\fivegp}
 \def\gp{\textfont0=\tenbf\scriptfont0=\sevenbf\scriptscriptfont0=\fivebf
         \textfont1=\tengp\scriptfont1=\sevengp\scriptscriptfont1=\fivegp}
\def\gponze{\textfont0=\elevenbf\scriptfont0=\eightbf\scriptscriptfont0=\sixbf
         \textfont1=\elevengp\scriptfont1=\eightgp\scriptscriptfont1=\sixgp}
\def\gpdouze{\textfont0=\twelvebf\scriptfont0=\tenbf\scriptscriptfont0=\ninebf
         \textfont1=\twelvegp\scriptfont1=\tengp\scriptscriptfont1=\ninegp}
\def\gptreize{\textfont0=\thirteenbf\scriptfont0=\elevenbf\scriptscriptfont0=\tenbf
         \textfont1=\thirteengp\scriptfont1=\elevengp\scriptscriptfont1=\tengp}
 \def\gpquatorze{\textfont0=\fourteenbf\scriptfont0=\twelvebf\scriptscriptfont0=\elevenbf
         \textfont1=\fourteengp\scriptfont1=\twelvegp\scriptscriptfont1=\elevengp}


% CARACTERES SPECIAUX

\expandafter\chardef\csname pre amssym.def at\endcsname=\the\catcode`\@
\catcode`\@=11
\def\undefine#1{\let#1\undefined}
\def\newsymbol#1#2#3#4#5{\let\next@\relax
 \ifnum#2=\@ne\let\next@\msafam@\else
 \ifnum#2=\tw@\let\next@\bbfam@\fi\fi
 \mathchardef#1="#3\next@#4#5}
\def\mathhexbox@#1#2#3{\relax
 \ifmmode\mathpalette{}{\m@th\mathchar"#1#2#3}%
 \else\leavevmode\hbox{$\m@th\mathchar"#1#2#3$}\fi}
\def\hexnumber@#1{\ifcase#1 0\or 1\or 2\or 3\or 4\or 5\or 6\or 7\or 8\or
 9\or A\or B\or C\or D\or E\or F\fi}

\def\setboxz@h{\setbox\z@\hbox}
\def\wdz@{\wd\z@}
\def\boxz@{\box\z@}

\edef\msafam@{\hexnumber@\msafam}
\mathchardef\dabar@"0\msafam@39

\edef\bbfam@{\hexnumber@\bbfam}
\def\widehat#1{\setboxz@h{$\m@th#1$}%
 \ifdim\wdz@>\tw@ em\mathaccent"0\bbfam@5B{#1}%
 \else\mathaccent"0362{#1}\fi}
\def\widetilde#1{\setboxz@h{$\m@th#1$}%
 \ifdim\wdz@>\tw@ em\mathaccent"0\bbfam@5D{#1}%
 \else\mathaccent"0365{#1}\fi}
\newsymbol\leqq 1335          % superieur ou egal(=)
\newsymbol\leqslant 1336
\newsymbol\lessgtr 1337       % superieur ou inferieur
\newsymbol\backprime 1038     % apostrophe de gauche a droite
\newsymbol\risingdotseq 133A  % egal entre points (bas puis haut)
\newsymbol\fallingdotseq 133B % egal entre points (haut puis bas)
\newsymbol\succcurlyeq 133C   % superieur ou egal tordu
\newsymbol\geqq 133D          % inferieur ou egal(=)
\newsymbol\geqslant 133E
\newsymbol\nmid 232D
\newsymbol\nexists 2040
\newsymbol\smallsetminus 2272
\newsymbol\varnothing 203F

\catcode`\@=\active

% typographie francaise ou anglaise

\catcode`\@=11

\newcount\typofr\typofr=1

\catcode`\;=\active
\def;{\ifnum\typofr=1\relax\ifhmode\ifdim\lastskip>\z@\unskip\fi
   \kern.2em\fi\string;\else\string;\fi}

\catcode`\:=\active
\def:{\ifnum\typofr=1\relax\ifhmode\ifdim\lastskip>\z@\unskip\fi
\penalty\@M\ \fi\string:\else\string:\fi}

\catcode`\!=\active
\def!{\ifnum\typofr=1\relax\ifhmode\ifdim\lastskip>\z@\unskip\fi
   \kern.2em\fi\string!\else\string!\fi}

\catcode`\?=\active
\def?{\ifnum\typofr=1\relax\ifhmode\ifdim\lastskip>\z@\unskip\fi
   \kern.2em\fi\string?\else\string?\fi}


\def\francais{\typofr=1\def\tpf{oui}}
\def\anglais{\typofr=2\def\tpf{non}}
\def\oui{oui}
\francais

\catcode`\@=12
%\catcode`\@=\active



% FIN CARACTERES SPECIAUX


% MACROS DIVERSES

\def\og{\leavevmode\raise.24ex\hbox{$\scriptscriptstyle\langle\!\langle\>$}}    % guillemets ouvrants
\def\fg{\leavevmode\raise.24ex\hbox{$\scriptscriptstyle\>\rangle\!\rangle$}}    % guillemets fermants
\def\chv#1{\left\langle#1\right\rangle}
\def\d{\,{\rm d}}
\def\dt{\d t}
\def\du{\d u}
\def\dx{\d x}
\def\dy{\d y}
\def\z{{\bb Z}}
\def\r{{\bb R}}
\def\CC{{\bb C}}
\def\N{{\bb N}}
\def\Q{{\bb Q}}
\def\F{{\bb F}}
\def\PP{{\bb P}}
\def\TT{{\bb T}}
\def\UU{{\bb U}}

\def\A{{\cal A}}
\def\B{{\cal B}}
\def\C{{\scaln C}}
\def\D{{\cal D}}
\def\E{{\cal E}}
\def\G{{\cal G}}
\def\HH{{\cal H}}
\def\I{{\cal I}}
\def\J{{\cal J}}
\def\K{{\cal K}}
\let\LL=\L
\def\L{{\cal L}}
\def\M{{\cal M}}
\def\n{{\cal N}}
\def\O{{\cal O}}
\def\P{{\scaln P}}
\def\QQ{{\scaln Q}}
\def\R{{\cal R}}
\def\s{{\cal S}}
\def\T{{\cal T}}
\def\V{{\cal V}}
\def\W{{\cal W}}
\def\X{{\cal X}}
\def\Y{{\cal Y}}
\def\Z{{\cal Z}}
\def\f{{\cal F}}
\def\frac#1#2{{#1\over #2}}
\def\di#1#2{\sct#1\atop{\sct#2}}
\def\tri#1#2#3{{\sct#1\atop\sct#2}\atop\sct#3}
\def\page#1{\rm p.\thinspace#1}
\def\chapitre#1{\rm chap.\thinspace#1}
\def\qedbox{$\rlap{$\sqcap$}\sqcup$}           % carr\'e blanc fin de d\'emo
\def\qed{\nobreak\hfill\penalty250 \hbox{}\nobreak\hfill\qedbox\par\medskip}
\def\Ssi{si, et seulement si, }
\def\CN{condition n\'ecessaire}
\def\cn{\CN}
\def\CS{condition suffisante}
\def\CNS{condition n\'ecessaire et suffisante}
\def\cns{\CNS}
\def\HR{hypoth\`ese de Riemann}
\def\sD{s\'erie de Dirichlet}
\def\sDS{s\'eries de Dirichlet}
\def\tnp{th\'eo\-r\`eme des nombres premiers}
\def\TNP{\tnp}
\def\tan{th\'eorie analytique des nombres}
\def\TL{transform\'ee de Laplace}
\def\TF{transform\'ee de Fourier}
\def\TFs{transform\'ees de Fourier}
\def\plaf#1{\left\lceil#1\right\rceil}
\def\planch#1{\left\lfloor#1\right\rfloor}
\def\�{\S\thinspace}

%SYMBOLES MATHS

\def\e{{\rm e}}
\def\mod{\mathop{\rm mod}\nolimits}
\def\md#1#2{\equiv#1\,({\rm mod\,}#2)}
\def\no#1{\Vert#1\Vert}
\def\NO#1{\left\Vert#1\right\Vert}
\def\ddp#1#2{\dsp{\partial#1\over\partial #2}}
\def\ent#1{\left[#1\right]}
\def\tdv{\rightarrow}
\def\ssi{\Leftrightarrow}
\def\tvi{\tdv +\infty}
%\def\v{\varepsilon}
\def\ep{\varepsilon}
\def\epsilon{\varepsilon}
\def\om{\omega}
\def\Om{\Omega}
%\def\phi{\varphi}           %ol
\def\theta{\vartheta}
\def\rho{\varrho}
\def\dm{{\textstyle{1\over 2}}}
\def\txt{\textstyle}
\def\dsp{\displaystyle}
\def\sct{\scriptstyle}
\def\nid{\par\noindent{\it D\'emonstration. }}
\def\pf{\noi{\it Proof. }}
\def\noi{\noindent}
\def\rem{\noi{\it Remarque.}\ }
\def\re{{\Re e\,}}
\def\im{{\Im m\,}}
\def\ov{\overline}
\def\un{\underline}
\def\sset{\smallsetminus}
\def\setminus{\sset}
\def\emptyset{{\varnothing}}
\def\pl{\partial}
\def\pp{{\rm pp}}
\def\det{\mathop{\hbox{\rm d\'et}}\nolimits}
\def\meas{\mathop{\rm meas}\nolimits}
\def\dim{\mathop{\rm dim}\nolimits}
\def\1{{\bf 1}}
\def\|{\Vert}
\let\vieuxle=\le
\let\vieuxge=\ge
\def\le{\leqslant}\def\leq{\leqslant}
\def\ge{\geqslant}\def\geq{\geqslant}
\def\wh{\widehat}
\def\cf{{cf.}}
\def\ie{{i.e.\ }}
\def\eg{{e.g.}}

%% operateurs

\def\card{\mathop{\rm card}\nolimits}
\def\sh{\mathop{\rm sh}\nolimits}
\def\th{\mathop{\rm th}\nolimits}
\def\tg{\mathop{\rm tg}\nolimits}
\def\arctg{\mathop{\rm Arctg}\nolimits}
\def\arcsin{\mathop{\rm Arcsin}\nolimits}
\def\arccos{\mathop{\rm Arccos}\nolimits}
\def\ch{\mathop{\rm ch}\nolimits}
\def\li{\mathop{\rm li}\nolimits}
\def\sgn{\mathop{\rm sgn}\nolimits}
\def\supp{\mathop{\rm supp}\nolimits}
\def\grad{\mathop{\rm grad}\nolimits}
\def\log{\mathop{\rm log}\nolimits}
\def\ft#1#2{{\txt{#1\over #2}}}
\def\fs#1#2{\scriptstyle{#1\over #2}}
%% Grande asterisque operateur
\font\gm=cmbx10 scaled 2000
\def\bst{\smash{\atop\hbox{\gm *}}}
\def\gast#1#2{\mathop{\bst}_{\phantom A \atop{\sct #1}}#2}
%% Grande asterisque sur la ligne
\def\glast#1#2{\mathop{\bst}_{\atop\sct #1}#2}



%fleches

\def\gradv{\mathop{\overrightarrow{\rm grad}}\nolimits}
\def\vec{\overrightarrow}


%% Lettres plombees 'pauvres'
\def\pmb#1{\setbox0=\hbox{#1}%
\kern-.025em\copy0\kern-\wd0\kern.05em\copy0\kern-\wd0\kern-.025em\raise .0433em\box0 }
\def\pd{\pmb{$\delta$}}


% NOTES EN BAS DE PAGE

% Pour que les accents se placent correctement en mode math en corps 8 et 6
\skewchar\eighti='177 \skewchar\sixi='177
\skewchar\eightsy='60 \skewchar\sixsy='60

\def\eightpoint{%
\textfont0=\eightrm\scriptfont0=\sixrm\scriptscriptfont0=\fiverm
\def\rm{\fam0\eightrm}%
\textfont1=\eighti\scriptfont1=\sixi
\scriptscriptfont1=\fivei\def\oldstyle{\fam1\seveni}%
\textfont2=\eightsy\scriptfont2=\sixsy\scriptscriptfont2=\fivesy
\textfont3=\eightex\scriptfont3=\sixex
\textfont\itfam=\eightit
\def\it{\fam\itfam\eightit}%
\textfont\slfam=\eightsl
\def\sl{\fam\slfam\eightsl}%
\textfont\bbfam=\eightbb \scriptfont\bbfam=\sixbb\scriptscriptfont\bbfam=\fivebb
\def\bb{\fam\bbfam\eightbb}%
\textfont\msafam=\eightmsa\scriptfont\msafam=\sixmsa
\textfont\scalnfam=\eightscaln
\def\scaln{\fam\scalnfam\eightscaln}
\textfont\ttfam=\eighttt
\def\tt{\fam\ttfam\eighttt}%
\textfont\bffam=\eightbf\scriptfont\bffam=\sixbf\scriptscriptfont\bffam=\fivebf
\def\bf{\fam\bffam\eightbf}%
\textfont\pcapfam=\eightpcap
\def\pcap{\fam\pcapfam\eightpcap}
\abovedisplayskip=2pt plus2pt minus 2pt
\belowdisplayskip=2pt plus1pt minus 2pt
\abovedisplayshortskip= 1pt plus 2pt minus 1pt
\belowdisplayshortskip= 1pt plus 2pt minus 1pt
\smallskipamount=2pt plus 1pt minus 2pt
\medskipamount=3pt plus 2pt minus 2pt
\bigskipamount=7pt plus 3pt minus 3pt
\setbox\strutbox=\hbox{\vrule height 5pt depth 2pt width 0pt}%
\normalbaselineskip=9pt\normalbaselines\rm}


%\def\note#1#2{\footnote{#1}{\noi\eightpoint #2\vskip-10pt\par}}
%\def\note#1#2{\hbox{$^{(#1)}$}\footnote{}{\eightpoint$#1.$#2\vskip-10pt\par}}
\def\pclm#1#2{\noi\bf#1\sl\ #2}

\def\({\left(}
\def\){\right)}

\def\footnoterule{\kern -2pt\hrule width 7truecm\kern 2.4pt}

%reference croisee a des numeros de notes
\def\xnotedef#1{\definexref{#1}{\noexpand\number\footnotenumber}{Note}}%
\def\noteref{\ref}

%Fin des definitions pour les notes en bas de page

% PARAGRAPHES EN NEUF POINTS

\def\ninepoint{%
\textfont0=\ninerm\scriptfont0=\sixrm\scriptscriptfont0=\fiverm
\def\rm{\fam0\ninerm}%
\textfont1=\ninei\scriptfont1=\sixi
\scriptscriptfont1=\fivei\def\oldstyle{\fam1\ninei}%
\textfont2=\ninesy\scriptfont2=\sixsy\scriptscriptfont2=\fivesy
\textfont3=\nineex\scriptfont3=\sixex
\textfont\itfam=\nineit
\def\it{\fam\itfam\nineit}%
\textfont\slfam=\ninesl
\def\sl{\fam\slfam\ninesl}%
\textfont\bbfam=\ninebb\scriptfont\bbfam=\sixbb\scriptscriptfont\bbfam=\fivebb
\def\bb{\fam\bbfam\ninebb}%
\textfont\msafam=\ninemsa\scriptfont\msafam=\sixmsa\scriptscriptfont\msafam=\fivemsa
\textfont\scalnfam=\ninescaln
\def\scaln{\fam\scalnfam\ninescaln}
\textfont\ttfam=\ninett
\def\tt{\fam\ttfam\ninett}%
\textfont\bffam=\ninebf\scriptfont\bffam=\sixbf\scriptscriptfont\bffam=\fivebf
\def\bf{\fam\bffam\ninebf}%
\abovedisplayskip=3pt plus2pt minus 2pt
\belowdisplayskip=3pt plus1pt minus 2pt
\abovedisplayshortskip= 2pt plus 2pt minus 1pt
\belowdisplayshortskip= 2pt plus 2pt minus 1pt
\smallskipamount=2pt plus 1pt minus 2pt
\medskipamount=3pt plus 2pt minus 2pt
\bigskipamount=7pt plus 3pt minus 3pt
\setbox\strutbox=\hbox{\vrule height 5pt depth 2pt width 0pt}%
\normalbaselineskip=10.5pt plus.3pt minus.3pt\normalbaselines\rm}

\def\neufpts#1{{\ninepoint#1}\par}

\def\sevenpoint{%
\textfont0=\sevenrm\scriptfont0=\sixrm\scriptscriptfont0=\fiverm
\def\rm{\fam0\sevenrm}%
\textfont1=\seveni\scriptfont1=\sixi
\scriptscriptfont1=\fivei\def\oldstyle{\fam1\seveni}%
\textfont2=\sevensy\scriptfont2=\sixsy\scriptscriptfont2=\fivesy
\textfont3=\sevenex\scriptfont3=\fiveex
\textfont\itfam=\sevenit
\def\it{\fam\itfam\sevenit}%
\textfont\slfam=\sevensl
\def\sl{\fam\slfam\sevensl}%
\textfont\bbfam=\sevenbb \scriptfont\bbfam=\sixbb\scriptscriptfont\bbfam=\fivebb
\def\bb{\fam\bbfam\sevenbb}%
\textfont\msafam=\sevenmsa\scriptfont\msafam=\sixmsa
\textfont\scalnfam=\sevenscaln
\def\scaln{\fam\scalnfam\sevenscaln}
\textfont\bffam=\sevenbf\scriptfont\bffam=\sixbf\scriptscriptfont\bffam=\fivebf
\def\bf{\fam\bffam\sevenbf}%
\textfont\ttfam=\seventt
\abovedisplayskip=2pt plus2pt minus 2pt
\belowdisplayskip=2pt plus1pt minus 2pt
\abovedisplayshortskip= 1pt plus 2pt minus 1pt
\belowdisplayshortskip= 1pt plus 2pt minus 1pt
\smallskipamount=2pt plus 1pt minus 2pt
\medskipamount=3pt plus 2pt minus 2pt
\bigskipamount=7pt plus 3pt minus 3pt
\setbox\strutbox=\hbox{\vrule height 5pt depth 2pt width 0pt}%
\normalbaselineskip=9pt\normalbaselines\rm}

 \def\onzepts{%
 \textfont0=\elevenrm\scriptfont0=\ninerm
 \textfont1=\elevenib\scriptfont1=\ninei}

\def\douzepts{%
\textfont0=\twelverm\scriptfont0=\tenrm\def\rm{\fam0\twelverm}%
\textfont1=\twelveib\scriptfont1=\teni%
\textfont2=\twelvesy\scriptfont2=\tensy\scriptscriptfont2=\eightsy
\textfont3=\twelveex
\textfont\itfam=\twelveti
\def\it{\fam\itfam\twelveti}%
\textfont\bffam=\twelvebf\scriptfont\bffam=\tenbf\scriptscriptfont\bffam=\eightbf
\def\bf{\fam\bffam\twelvebf}%
\textfont\bbfam=\twelvebb \scriptfont\bbfam=\tenbb
\def\bb{\fam\bbfam\twelvebb}%
\textfont\msafam=\twelvemsa\scriptfont\msafam=\tenmsa
\textfont\scalnfam=\twelvescaln
\normalbaselineskip=15pt\normalbaselines\rm}

\def\quatorzepts{%
\textfont0=\fourteenrm\scriptfont0=\twelverm\def\rm{\fam0\fourteenrm}%
\textfont1=\fourteenib\scriptfont1=\twelveib%
\textfont2=\fourteensy\scriptfont2=\twelvesy\scriptscriptfont2=\tensy
\textfont3=\fourteenex
\textfont\itfam=\fourteenti
\def\it{\fam\itfam\fourteenti}%
\textfont\bffam=\fourteenbf\scriptfont\bffam=\twelvebf\scriptscriptfont\bffam=\tenbf
\def\bf{\fam\bffam\fourteenbf}%
\textfont\bbfam=\fourteenbb \scriptfont\bbfam=\twelvebb
\def\bb{\fam\bbfam\fourteenbb}%
\textfont\msafam=\fourteenmsa\scriptfont\msafam=\twelvemsa
\textfont\scalnfam=\twelvescaln
\normalbaselineskip=18pt\normalbaselines\rm}


% Bibliographies, journaux

\def\AA{{\it Acta Arith.}}
\def\BSMF{{\it Bull. Soc. Math. France}}
\def\Crelle{{\it J. reine angew. Math.}}
\def\JLMS{{\it J. London Math. Soc.}}
\def\JNT{{\it J. number theory}}
\def\PAMS{{\it Proc. Amer. Math. Soc.}}
\def\PLMS{{\it Proc. London Math. Soc.}}
\def\TAMS{{\it Trans. Amer. Math. Soc.}}


%Insertion de figures et illustrations
\def\picture #1 by #2 (#3){\leavevmode\vbox to #2{
   \hrule width #1 height 0pt depth 0pt
    \vfill
     \special{picture #3}}}