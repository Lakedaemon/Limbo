\catcode`@=11\relax
\input LD@Header.tex
\input qhsort.tex
\catcode`@=11\relax
\input LD@Library.tex
\input LD@Typesetting.tex
\input LD@Akuma@Japanese.tex
\catcode`@=11\relax

\lccode"2019="0027\relax% to get hyphen with words with apostrophes

%%
%% Fontes utilisées
%%

\def\FontCSName #1 #2\EndFontCSName{%
	\EA\gdef\CS #1\EC##1{%
		\unless\ifcsname #1##1\endcsname 
			\EA\font\CS #1##1\EC="#2" at ##1pt%
		\fi
		\CS #1##1\EC
	}%
}%
\def\FontCSDef #1.{%
	\for\OLFont:=#1\do{%
		\EA\FontCSName\OLFont\EndFontCSName
	}%
}%

\FontCSDef Arial Arial,ArialB Arial Bold,ArialIT Arial Italic,ArialBIT Arial Bold Italic,Kaisho HGSeikaishotaiPRO,Mincho MS PMincho.%%% Automated but might bring Errors

\def\FontDef #1.{%
	\def\FontName ##1=##2@##3\EndFontName{%
		\EA\global\EA\font\CS ##1\EC="##2" at ##3pt%
	}%
	\for\OLFont:=#1\do{%
		\EA\FontName \OLFont\EndFontName
	}%
}%

\FontDef Romaji=Arial@5,Grade=Arial@4.






%\font\Fontindex="HGSeikaishotaiPRO" at 10pt


\font\StrokeFont="Arial:color=FF0F0Fl" at 4pt

%%%%%%%%%%%%%%%%%%%%%%%%%%%%%%%%%%%%%%%%%%%%%%%%%%%%%%%%%%%%%%%%%%%%%%%%%%%%%%%%%%%%%%%%%%%%%%%%%%%%%%%%%%%%%%%%%%%%%%
%%%%%%%%                                                                                                            Fontes
%%%%%%%%%%%%%%%%%%%%%%%%%%%%%%%%%%%%%%%%%%%%%%%%%%%%%%%%%%%%%%%%%%%%%%%%%%%%%%%%%%%%%%%%%%%%%%%%%%%%%%%%%%%%%%%%%%%%%%
% -> Dingbats
\font\LD@Dingbats@Japan="japanapush" at 12pt%      r=rice bowl,   t=bentou,    v=temple,     w=zen gate,     z=japan
\font\LD@Dingbats@X="Geisha" at 12pt
\font\LD@Dingbats@MartialArts="McCoy Dingbat Karate" at 12pt


\font\LD@Font@KanaSubtitle="HGSeikaishotaiPRO" at 10pt
\font\LD@Font@KanaTitle="HGSeikaishotaiPRO" at 24pt
\font\LD@Font@IndexNumber="Arial" at 6pt
\font\LD@Font@Nihongo="HGSeikaishotaiPRO" at 28pt
\font\LD@Font@BushuArray="HGSeikaishotaiPRO" at 10pt
\font\LD@Font@BushuMeaningsArray="Arial" at 6pt
\font\LD@Font@BigKanji="HGSeikaishotaiPRO" at 66pt
\font\LD@Font@Mincho="MS PMincho" at 18pt
\font\LD@Font@Word="HGSeikaishotaiPRO" at 10pt
\font\LD@Font@Kana="HGSeikaishotaiPRO" at 10pt
\font\LD@Font@WordKana="HGSeikaishotaiPRO" at 10pt
\font\LD@Font@Bushu="HGSeikaishotaiPRO" at 10pt
\font\LD@Font@Meanings="Arial" at 8pt
\font\LD@Font@BushuMeanings="Arial" at 6pt
\font\LD@Font@Numbers="Arial" at 4pt
\font\LD@Font@Index="Arial" at 5pt
\font\LD@Font@IndexItalic="Arial Italic" at 5pt
\font\LD@Font@TinyKaisho="HGSeikaishotaiPRO" at 6pt
\font\LD@Font@TinyArial="Arial" at 4pt
\font\LD@Font@One="Arial" at 8pt
\font\LD@Font@Two="Arial Italic" at 8pt
\newcount\LD@Font@Number\global\LD@Font@Number=0\relax
\def\LD@Font@Set{% 
	\global\advance\LD@Font@Number by 1\relax
	\ifcase\LD@Font@Number
	\or
		\global\let\LD@Font@Sens\LD@Font@One
	\else
		\global\let\LD@Font@Sens\LD@Font@Two
		\global\LD@Font@Number=0\relax
	\fi
}%
\def\LD@Font@Reset{%
	\global\LD@Font@Number=0\relax
	\LD@Font@Set
}%
%%%%%%%%%%%%%%%%%%%%%%%%%%%%%%%%%%%%%%%%%%%%%%%%%%%%%%%%%%%%%%%%%%%%%%%%%%%%%%%%%%%%%%%%%%%%%%%%%%%%%%%%%%%%%%%%%%%%%%
%%%%%%%%                                                                                                            Stats 
%%%%%%%%%%%%%%%%%%%%%%%%%%%%%%%%%%%%%%%%%%%%%%%%%%%%%%%%%%%%%%%%%%%%%%%%%%%%%%%%%%%%%%%%%%%%%%%%%%%%%%%%%%%%%%%%%%%%%%
\def\JLPTKanji#1{%
\begingroup
	\LD@Counter@A=#1\relax
	\advance\LD@Counter@A by1\relax
	\ifcase\LD@Counter@A
		{\Kaisho6片仮名}\or
		{\Kaisho6平仮名}\or
		JLPT 4\or
		JLPT 3\or
		JLPT 2\or
		JLPT 1\or
		NF\or
		Autre%
	\fi
\endgroup
}%
\def\GradeKanji#1{%
\begingroup
	\LD@Counter@A=#1\relax
	\advance\LD@Counter@A by1\relax
	\ifcase\LD@Counter@A
	 	{\Kaisho6片仮名}\or
		{\Kaisho6平仮名}\or
		CP\or
		CE1\or
		CE2\or
		CM1\or
		CM2\or
		6ème\or
		College\or
		Nom\or
		Autre\or
	\fi
\endgroup
}%
\def\JLPTTango#1{%
\begingroup
	\LD@Counter@A=#1\relax
	\ifcase\LD@Counter@A
	 	\or
		JLPT 4\or
		JLPT 3\or
		JLPT 2\or
		JLPT 1\or
		NF\or
		Autre\or
	\fi
\endgroup
}%
\def\LD@Stats{%
%\immediate\write16{Displaying Stats...}%
\begingroup
	\def\JLPTSwap##1{%
		\LD@Counter@A=##1\relax
		\global\advance\LD@Counter@A by1\relax
		\ifcase\LD@Counter@A
			\def##1{-1}\or
			\def##1{0}\or
			\def##1{4}\or
			\def##1{3}\or
			\def##1{2}\or
			\def##1{1}\or
		\fi
	}%
	\PAR
	\centerline{\ArialB{20}Statistiques}
	\medskip
	{\Arial 8Tableau representant pour chaque niveau de vocabulaire, le nombre de mots {\Kaisho8(単語)} que l'on peut écrire en utilisant les caractères {\Kaisho8(漢字)} d'un niveau inférieur à un seuil fixé.}
	\medskip
	\centerline{%
		\tikzpicture
			\foreach \x in{1,2,...,6}
				\foreach \y in{-1,0,...,6}
					\node[rectangle,draw,text width=1cm,text height=10pt,text depth=2pt,inner sep=0pt,text centered] at (\x*1cm,-\y*12pt) {%
					\let\LDX=\x\JLPTSwap\LDX\LD@Stat@Display{\LDX|\y}};
               		\foreach \x in{1,2,...,6}
				\node[rectangle, text width=1cm,text height=10pt,text depth=2pt,inner sep=0pt, text centered] (a\x)at (\x*1cm,24pt) {\Arial6\JLPTTango{\x}};
			\draw [snake=brace] (a1.north west) -- (a6.north east) node[midway](G){};
			\node [above of=G, node distance=8pt]  {\Kaisho{10}単語};
			\foreach \y in{-1,0,...,6}
			\node[rectangle, text width=0.8cm,text height=10pt,text depth=2pt,inner sep=0pt] at (0cm,-\y*12pt) (b\y) {\noindent\hfill\Arial6\JLPTKanji{\y}\hskip4pt};
			\draw [snake=brace] (b6.south west) -- (b-1.north west) node[midway](H){};
			\node (J) at (H.west) {};\node [left of=J,node distance=4pt](K){};
			\node[above of=K,node distance=5pt] {\Kaisho{10}漢};
			\node[below of=K,node distance=5pt] {\Kaisho{10}字};
		\endtikzpicture
	}%
	\centerline{%
		\tikzpicture
			\foreach \x in{1,2,...,6}
				\foreach \y in{-1,0}
					\node[rectangle,draw, text width=1cm,text height=10pt,text depth=2pt,inner sep=0pt, text centered] at (\x*1cm,-\y*12pt) {%
					\let\LDX=\x\JLPTSwap\LDX\LD@Stat@Display{\LDX|\y}};
			\foreach \x in{1,2,...,6}
				\foreach \y in{1,2,...,9}
					\node[rectangle,draw, text width=1cm,text height=10pt,text depth=2pt,inner sep=0pt, text centered] at (\x*1cm,-\y*12pt) {%
					\let\LDX=\x\JLPTSwap\LDX\LD@Stat@Display{\LDX|G\y}};
               		\foreach \x in{1,2,...,6}
				\node[rectangle, text width=1cm,text height=10pt,text depth=2pt,inner sep=0pt, text centered] (a\x)at (\x*1cm,24pt) {\Arial6\JLPTTango{\x}};
			\draw [snake=brace] (a1.north west) -- (a6.north east) node[midway](G){};
			\node [above of=G, node distance=8pt]  {\Kaisho{10}単語};
			\foreach \y in{-1,0,...,9}
			\node[rectangle, text width=0.9cm,text height=10pt,text depth=2pt,inner sep=0pt] at (0cm,-\y*12pt) (b\y) {\noindent\hfill\Arial6\GradeKanji{\y}\hskip4pt};
			\draw [snake=brace] (b9.south west) -- (b-1.north west) node[midway](H){};
			\node (J) at (H.west) {};\node [left of=J,node distance=4pt](K){};
			\node[above of=K,node distance=5pt] {\Kaisho{10}漢};
			\node[below of=K,node distance=5pt] {\Kaisho{10}字};
		\endtikzpicture
	}%
	\endgroup
}%
%%%%%%%%%%%%%%%%%%%%%%%%%%%%%%%%%%%%%%%%%%%%%%%%%%%%%
%												%
%					Language choice					%
%												%
%%%%%%%%%%%%%%%%%%%%%%%%%%%%%%%%%%%%%%%%%%%%%%%%%%%%%

\def\LD@Akuma@Language{fre}%

%%%%%%%%%%%%%%%%%%%%%%%%%%%%%%%%%%%%%%%%%%%%%%%%%%%%%
%												%
%					Loading Kanjidic					%
%												%
%%%%%%%%%%%%%%%%%%%%%%%%%%%%%%%%%%%%%%%%%%%%%%%%%%%%%

\LD@Kanjidic@Load

%%%%%%%%%%%%%%%%%%%%%%%%%%%%%%%%%%%%%%%%%%%%%%%%%%%%%
%												%
%					Kanjidic Preprocessing				%
%												%
%%%%%%%%%%%%%%%%%%%%%%%%%%%%%%%%%%%%%%%%%%%%%%%%%%%%%

\LD@Yomigana@Kanji@Preprocessing
\LD@Yomigana@Preprocessing@Default

%%%%%%%%%%%%%%%%%%%%%%%%%%%%%%%%%%%%%%%%%%%%%%%%%%%%%
%												%
%					Loading JMdict					%
%												%
%%%%%%%%%%%%%%%%%%%%%%%%%%%%%%%%%%%%%%%%%%%%%%%%%%%%%

\LD@JMdict@Load@Kanji

%%%%%%%%%%%%%%%%%%%%%%%%%%%%%%%%%%%%%%%%%%%%%%%%%%%%%
%												%
%					Filter the multiple Tango Entries			%
%												%
%%%%%%%%%%%%%%%%%%%%%%%%%%%%%%%%%%%%%%%%%%%%%%%%%%%%%
\LD@Time@Benchmark{Filtering the entries...}%

\global\LD@One=0\relax
\global\LD@Two=0\relax
\gdef\LD@Entries@Filter{}%
\loop
	\global\advance\LD@One by1\relax
	\LD@Data@Set{Kanji}{E\the\LD@One}\LD@Kanji
	\LD@Data@Set{Furigana}{E\the\LD@One}\LD@Furigana
	\LD@Data@Set{Sense}{E\the\LD@One}\LD@Sense
	\LD@Data@Set{Entry}{E\the\LD@One}\LD@Entry
	\global\LD@Two=\LD@One\relax
	{\loop
		\def\LD@Loop@Switch{N}%
		\ifnum\LD@Two<\LD@Entry@Count\relax
			\global\advance\LD@Two by1\relax
			\LD@Data@Set{Kanji}{E\the\LD@Two}\LD@Kanji@Temp
			\LD@Data@Set{Furigana}{E\the\LD@Two}\LD@Furigana@Temp
			\LD@Data@Set{Sense}{E\the\LD@Two}\LD@Sense@Temp
			\LD@Data@Set{Entry}{E\the\LD@Two}\LD@Entry@Temp
			\ifx\LD@Entry\LD@Entry@Temp
				\ifx\LD@Kanji\LD@Kanji@Temp
					\ifx\LD@Sense\LD@Sense@Temp
						\LD@String@IfSubString\LD@Entries@Filter{,\the\LD@Two,}{}{% 
							\LD@String@IfSubString\LD@Entries@Filter{,the\LD@One,}{%
								\xdef\LD@Entries@Filter{\LD@Entries@Filter\the\LD@Two,}%
							}{%
								\xdef\LD@Entries@Filter{\LD@Entries@Filter |,\the\LD@One,\the\LD@Two,}%
							}%
						}%
					\fi
				\fi
				\def\LD@Loop@Switch{Y}%
			\fi
		\fi
	\if Y\LD@Loop@Switch
	\repeat}%
\ifnum\LD@One<\LD@Entry@Count\relax
\repeat
\xdef\LD@Entries@Filter{\LD@Entries@Filter |}%

\LD@Time@Benchmark{Filtering the entries II...}%
\global\LD@One=0\relax
\global\LD@Two=0\relax
\gdef\LD@Entries@Filter@Kana{}%
\loop
	\global\advance\LD@One by1\relax
	\LD@Data@Set{Kana}{E\the\LD@One}\LD@Kana
	\LD@Data@Set{Furigana}{E\the\LD@One}\LD@Furigana
	\LD@Data@Set{Sense}{E\the\LD@One}\LD@Sense
	\LD@Data@Set{Entry}{E\the\LD@One}\LD@Entry
	\LD@Data@Set{Level}{E\the\LD@One}\LD@Level
	\global\LD@Two=\LD@One\relax
	{\loop
		\def\LD@Loop@Switch{N}%
		\ifnum\LD@Two<\LD@Entry@Count\relax
			\global\advance\LD@Two by1\relax
			\LD@Data@Set{Kana}{E\the\LD@Two}\LD@Kana@Temp
			\LD@Data@Set{Furigana}{E\the\LD@Two}\LD@Furigana@Temp
			\LD@Data@Set{Sense}{E\the\LD@Two}\LD@Sense@Temp
			\LD@Data@Set{Entry}{E\the\LD@Two}\LD@Entry@Temp
			\LD@Data@Set{Level}{E\the\LD@Two}\LD@Level@Temp
			\ifx\LD@Entry\LD@Entry@Temp
				\ifx\LD@Kana\LD@Kana@Temp
				\ifx\LD@Level\LD@Level@Temp
					\ifx\LD@Sense\LD@Sense@Temp
						\def\LD@Entries@Filter@OK{N}%
						\LD@String@IfSubString\LD@Entries@Filter{,\the\LD@One,}{%
							\LD@String@IfSubString\LD@Entries@Filter{|,\the\LD@One,}{%
								\LD@String@IfSubString\LD@Entries@Filter{|,\the\LD@Two,}{%
									\def\LD@Entries@Filter@OK{Y}%
								}{}%
							}{}%
						}{%
							\LD@String@IfSubString\LD@Entries@Filter{,\the\LD@Two,}{}{%
								\def\LD@Entries@Filter@OK{Y}%
							}%
						}%
						\if Y\LD@Entries@Filter@OK
							\LD@String@IfSubString\LD@Entries@Filter@Kana{,\the\LD@Two,}{}{% 
								\LD@String@IfSubString\LD@Entries@Filter@Kana{,\the\LD@One,}{%
									\xdef\LD@Entries@Filter@Kana{\LD@Entries@Filter@Kana\the\LD@Two,}%
								}{%
									\xdef\LD@Entries@Filter@Kana{\LD@Entries@Filter@Kana |,\the\LD@One,\the\LD@Two,}%
								}%
							}%
						\fi
					\fi\fi
				\fi
				\def\LD@Loop@Switch{Y}%
			\fi
		\fi
	\if Y\LD@Loop@Switch
	\repeat}%
\ifnum\LD@One<\LD@Entry@Count\relax
\repeat
\xdef\LD@Entries@Filter@Kana{\LD@Entries@Filter@Kana |}%

%%%%%%%%%%%%%%%%%%%%%%%%%%%%%%%%%%%%%%%%%%%%%%%%%%%%%
%												%
%					Compute the Tango Data				%
%												%
%%%%%%%%%%%%%%%%%%%%%%%%%%%%%%%%%%%%%%%%%%%%%%%%%%%%%
\LD@Time@Benchmark{Computing the data...}%
\def\LD@Glue{{\LD@Font@Word\hskip1em plus 0.5em minus 0.5em}}%
\global\LD@One=0\relax
\global\LD@Two=0\relax
\loop
	\global\advance\LD@One by1\relax
	\LD@Data@Set{Kanji}{E\the\LD@One}\LD@Kanji
	\LD@Data@Set{Furigana}{E\the\LD@One}\LD@Furigana
	\unless\ifx\LD@Furigana\LD@Empty
		\LD@Tango@Info@Get\LD@Kanji
		\LD@Data@Set{KanjiInfo}{E\the\LD@One}\LD@Kanji@Info
		\LD@Akuma@Option@Process\LD@Kanji@Info
		\ifx\LD@JLPT\LD@Empty
			\ifx\LD@nf\LD@Empty
				\LD@Stat@Update{6|\TangoMaxKanjiLevel}%
				\LD@Stat@Update{6|G\TangoMaxKanjiGrade}%
			\else
				\LD@Stat@Update{5|\TangoMaxKanjiLevel}%
				\LD@Stat@Update{5|G\TangoMaxKanjiGrade}%
			\fi
			\def\LD@JLPT{{}}%
		\else
			\LD@Stat@Update{\LD@JLPT|\TangoMaxKanjiLevel}%
			\LD@Stat@Update{\LD@JLPT|G\TangoMaxKanjiGrade}%
		\fi
		\LD@Three=\TangoMaxKanjiLevel\relax
		\ifnum\LD@Three<6\relax
			\def\LD@Proceed@OK{Y}%
			\LD@String@IfSubString\LD@Entries@Filter{|,\the\LD@One,}{}{%
				\LD@String@IfSubString\LD@Entries@Filter{,\the\LD@One,}{%
					\def\LD@Proceed@OK{N}%
				}{}%
			}%
			\LD@String@IfSubString\LD@Entries@Filter@Kana{|,\the\LD@One,}{}{%
				\LD@String@IfSubString\LD@Entries@Filter@Kana{,\the\LD@One,}{%
					\def\LD@Proceed@OK{N}%
				}{}%
			}%
			\if Y\LD@Proceed@OK
				% We don't want Kanji without frequence, there shouldn't be kana at this point
				\global\advance\LD@Two by1\relax
				\LD@Data@Set{Furigana}{E\the\LD@One}\LD@Furigana
				\LD@Tango@Format\LD@Kanji\LD@Furigana\LD@JLPT\LD@nf\LD@Tango@Kanji\LD@Tango@Kana
				\LD@Meaning@Format
				\LD@String@IfSubString\LD@Entries@Filter{|,\the\LD@One,}{%
					\EA\LD@String@Split\EA\LD@Entries@Filter\EA{\EA|\EA,\the\LD@One,}\LD@Trash\LD@Temp
					\LD@String@Split\LD@Temp{,|}\LD@Temp\LD@Trash
					\gdef\LD@Entries@Others{}%
					\LD@Loop@For\LD@Token=\LD@Temp\WithSeparator ,\Do{%
						\LD@Data@Set{Furigana}{E\LD@Token}\LD@Furigana@Temp
						{\LD@Tango@Format\LD@Kanji\LD@Furigana@Temp\LD@JLPT\LD@nf\LD@Tango@Kanji@Temp\LD@Tango@Kana@Temp
						\xdef\LD@Entries@Others{\LD@Entries@Others/\LD@Tango@Kana@Temp}%
						}%
					}%
				}{%
					\gdef\LD@Entries@Others{}%
				}%	
				\LD@String@IfSubString\LD@Entries@Filter@Kana{|,\the\LD@One,}{%
					\EA\LD@String@Split\EA\LD@Entries@Filter@Kana\EA{\EA|\EA,\the\LD@One,}\LD@Trash\LD@Temp
					\LD@String@Split\LD@Temp{,|}\LD@Temp\LD@Trash
					\gdef\LD@Entries@Kanji@Others{}%
					\LD@Loop@For\LD@Token=\LD@Temp\WithSeparator ,\Do{%
						\LD@Data@Set{Furigana}{E\LD@Token}\LD@Furigana@Temp
						\LD@Data@Set{Kanji}{E\LD@Token}\LD@Kanji@Temp
						\LD@Data@Set{KanjiInfo}{E\LD@Token}\LD@Kanji@Info
						\LD@Akuma@Option@Process\LD@Kanji@Info
						{\LD@Tango@Format\LD@Kanji@Temp\LD@Furigana@Temp\LD@JLPT\LD@nf\LD@Tango@Kanji@Temp\LD@Tango@Kana@Temp
						\xdef\LD@Entries@Kanji@Others{\LD@Entries@Kanji@Others/\LD@Tango@Kanji@Temp}%
						}%
					}%
				}{%
					\gdef\LD@Entries@Kanji@Others{}%	
				}%		
				\EA\xdef\CS Text\the\LD@Two\EC{%
					{\LD@Font@Word\LD@Tango@Kanji\LD@Entries@Kanji@Others}\LD@Glue
					{\LD@Font@Kana\JapaneseLBrace\LD@Tango@Kana\LD@Entries@Others\JapaneseRBrace}\LD@Glue{\LD@Meaning}%
				}%
				\setbox\LD@Box@One=\hbox{\CS Text\the\LD@Two\EC}%
				\LD@Dim@One=\wd\LD@Box@One\relax
				\divide\LD@Dim@One by10000\relax
				\advance\LD@Dim@One by\TangoMaxKanjiRank pt%
				\advance\LD@Dim@One by -1pt%
				\EA\xdef\CS\the\LD@Two\EC{\the\LD@Dim@One}%
			\fi
		\fi
	\fi	
\ifnum\LD@One<\LD@Entry@Count\relax
\repeat
\global\n=\LD@Two\relax

%%%%%%%%%%%%%%%%%%%%%%%%%%%%%%%%%%%%%%%%%%%%%%%%%%%%%
%												%
%					Sort the Tango Entries				%
%												%
%%%%%%%%%%%%%%%%%%%%%%%%%%%%%%%%%%%%%%%%%%%%%%%%%%%%%

\LD@Time@Benchmark{Sorting the Entries...}
\def\cmpdim#1#2{%#1, #2 are def-s
%Result: \status= 0, 1, 2, if
%        \val{#1} =, >, < \val{#2}
	\ifdim#1>#2
		\global\status1\relax
	\else
		\ifdim#1<#2
			\global\status2\relax
		\else
			\global\status0\relax
		\fi
	\fi
}%
\def\xchtango#1#2{%#1, #2 counter variables
	\EA\let\EA\auxone\CS Text\the#1\EC
	\EA\let\EA\auxtwo\CS Text\the#2\EC
	\EA\global\EA\let\CS Text\the#2\EC\auxone
	\EA\global\EA\let\CS Text\the#1\EC\auxtwo
	\EA\let\EA\auxone\CS\the#1\EC
	\EA\let\EA\auxtwo\CS\the#2\EC
	\EA\global\EA\let\CS\the#2\EC\auxone
	\EA\global\EA\let\CS\the#1\EC\auxtwo
}%
\let\cmp\cmpdim
\let\xch\xchtango
\heapsort


%%%%%%%%%%%%%%%%%%%%%%%%%%%%%%%%%%%%%%%%%%%%%%%%%%%%%
%												%
%					Reading Index						%
%												%
%%%%%%%%%%%%%%%%%%%%%%%%%%%%%%%%%%%%%%%%%%%%%%%%%%%%%

\newcount\LD@Lectures\global\LD@Lectures=0\relax

%-> if necessary, creates the pointers
\def\UpdateReadingsIndex#1#2#3#4{% #1->Kanji, #2-> reading -> #3 displayed reading #4-> sorting reading
	\unless\ifcsname RI#2\endcsname
		\global\advance\LD@Lectures by1\relax
		\EA\xdef\CS RI#2\EC{\the\LD@Lectures}%
		\EA\xdef\CS RIL\the\LD@Lectures\EC{#4}%
		\EA\xdef\CS RIC\the\LD@Lectures\EC{%
			\noindent
			\rlap{%
				\LD@Font@Index#3%
			}%
		}% the first word of the content is the lecture
	\fi
	\LD@Data@Set{Frequency}#1\LD@Frequency
	\EA\xdef\CS RIC\CS RI#2\EC\EC{%
		\CS RIC\CS RI#2\EC\EC
		\noindent\hfill\LD@Font@Word#1\LD@Font@IndexNumber\hskip2.5em
		\llap{%
			\LD@Frequency
		}%
		\PAR
	}%
}%

%%%%%%%%%%%%%%%%%%%%%%%%%%%%%%%%%%%%%%%%%%%%%%%%%%%%%
%												%
%					\LD@Kanji@Info@Make				%
%												%
%%%%%%%%%%%%%%%%%%%%%%%%%%%%%%%%%%%%%%%%%%%%%%%%%%%%%


\def\LD@Kanji@Info@Make{%
	\edef\LD@KanjiInfo{\LD@Font@Meanings}%
	\LD@Data@Set{On}\LD@Kanji\On% On Readings
	\ifx\relax\On
	\else
		\for\LD@Lecture:=\On\do{%
			\let\LD@DisplayedLecture\LD@Lecture
			{\LD@Romanize\LD@DisplayedLecture}%
			{\let\LD@SortingLecture\LD@DisplayedLecture
			\LD@LowerCase\LD@SortingLecture}%
			\edef\LD@SortingLecture{\LD@SortingLecture-}%
			\LD@Split\LD@SortingLecture-\LD@SortingLecture\LD@Temp
			\ifx\LD@SortingLecture\LD@Empty
				\edef\LD@SortingLecture{\LD@Temp-}%
				\LD@Split\LD@SortingLecture-\LD@SortingLecture\LD@Trash
			\fi
			\UpdateReadingsIndex\LD@Kanji\LD@Lecture\LD@DisplayedLecture\LD@SortingLecture
			\edef\LD@KanjiInfo{\LD@KanjiInfo\space\LD@DisplayedLecture}%
		}%
	\fi
	\LD@Data@Set{Kun}\LD@Kanji\Kun% Kun Readings
	\ifx\relax\Kun
	\else
		\for\LD@Lecture:=\Kun\do{%
			\edef\LD@Temp{\LD@Lecture.}%
			\LD@Split\LD@Temp.\LD@Head\LD@Tail
			\ifx\LD@Tail\LD@Empty
				\let\LD@DisplayedLecture\LD@Lecture
				{\LD@Romanize\LD@DisplayedLecture}%
				\edef\LD@SortingLecture{\LD@DisplayedLecture-}%first remove the last - if there is one
				\LD@Split\LD@SortingLecture-\LD@SortingLecture\LD@Temp
				\ifx\LD@SortingLecture\LD@Empty
					\edef\LD@SortingLecture{\LD@Temp-}%
					\LD@Split\LD@SortingLecture-\LD@SortingLecture\LD@Trash
				\fi
				\edef\LD@KanjiInfo{\LD@KanjiInfo\space\LD@DisplayedLecture}%
			\else
				\LD@Split\LD@Tail.\LD@Tail\LD@Trash
				{\LD@Romanize\LD@Head}%
				{\LD@Romanize\LD@Tail}%
				\edef\LD@DisplayedLecture{\LD@Head(\LD@Tail)}%
				\edef\LD@Head{\LD@Head-}%
				\LD@Split\LD@Head-\LD@Head\LD@Temp
				\unless\ifx\LD@Temp\LD@Empty
					\LD@Split\LD@Temp-\LD@Head\LD@Trash
				\fi
				\edef\LD@SortingLecture{\LD@Head\LD@Tail}%
				\edef\LD@KanjiInfo{\LD@KanjiInfo\space\LD@Head(\LD@Tail)}%%%%% takes care of "waru.i" Kun readings
			\fi
			\UpdateReadingsIndex\LD@Kanji\LD@Lecture\LD@DisplayedLecture\LD@SortingLecture
		}%
	\fi
	\LD@Data@Set{Meanings}\LD@Kanji\Meanings% Meanings
	\prems=0\relax
	\for\LD@Lecture:=\Meanings\do{%
		\advance\prems by1\relax
		\ifnum\prems=1\relax
			\edef\LD@KanjiInfo{\LD@KanjiInfo\space\LD@Lecture}%
		\else
			\edef\LD@KanjiInfo{\LD@KanjiInfo,\space\LD@Lecture}%
		\fi
	}%
	\LD@Data@Set{Nanori}\LD@Kanji\Nanori% Nanori Readings
	\ifx\relax\Nanori
	\else
		\for\LD@Lecture:=\Nanori\do{%
			\let\LD@DisplayedLecture\LD@Lecture
			{\LD@Romanize\LD@DisplayedLecture}
			\let\LD@SortingLecture\LD@DisplayedLecture
			\UpdateReadingsIndex\LD@Kanji\LD@Lecture\LD@DisplayedLecture\LD@DisplayedLecture
			\edef\LD@KanjiInfo{\LD@KanjiInfo\space\LD@DisplayedLecture}%
		}%
		\ifx\empty\Nanori
		\else
			\edef\LD@KanjiInfo{\LD@KanjiInfo\LD@Font@Meanings.}%
		\fi
	\fi
}%

%%%%%%%%%%%%%%%%%%%%%%%%%%%%%%%%%%%%%%%%%%%%%%%%%%%%%
%												%
%					\LD@Tango@Retrieve				%
%												%
%%%%%%%%%%%%%%%%%%%%%%%%%%%%%%%%%%%%%%%%%%%%%%%%%%%%%

\newcount\LD@Pointer\global\LD@Pointer=1\relax % Fix this
\def\LD@Tango@Retrieve{{%
	\xdef\LD@StringOne{}%
	\def\LD@LoopOne{Y}%
	\loop
		\EA\LD@Dim@One\EA=\the\LD@Kanji@Counter pt\relax
		\EA\ifdim\EA\LD@Dim@One\EA>\CS\the\LD@Pointer\EC\relax
			\xdef\LD@StringOne{%
				\LD@StringOne\break\CS Text\the\LD@Pointer\EC\hfil
			}%
			\global\advance\LD@Pointer by 1\relax
			\unless\ifnum\LD@Pointer<\n\relax
				\def\LD@LoopOne{N}%
			\fi
		\else
			\def\LD@LoopOne{N}%
		\fi
	\if Y\LD@LoopOne
	\repeat
}}%


%%%%%%%%%%%%%%%%%%%%%%%%%%%%%%%%%%%%%%%%%%%%%%%%%%%%%
%												%
%					\LD@Entry@Display					%
%												%
%%%%%%%%%%%%%%%%%%%%%%%%%%%%%%%%%%%%%%%%%%%%%%%%%%%%%


\newdimen\Tangoheight
\newdimen\Tangoheightd
\newdimen\Tangowidth
\newdimen\Tangowidthd
\newdimen\OLoffset
\newdimen\OLvoffset
\newdimen\LD@Dimen@WidthOne
\newdimen\LD@Dimen@HSize
\def\LD@Entry@Display{%
	% Make the main box
	\setbox\LD@Box@One=\hbox{%
		\vrule 
		\vbox{%
			\hsize10mm\noindent\hfill
			\hbox{%
				\LD@Font@Word\the\LD@Kanji@Counter\strut
			}%
			\hfill\null\PAR
			\noindent\hrule\noindent\hfill
			\hbox{%
				\LD@Font@IndexNumber G\LD@Grade\space JLPT\LD@Level\strut
			}%
			\hfill\null\PAR
			\noindent\hrule\noindent\hfill
			\hbox{%
				\LD@Font@Kana\CS Bushu\LD@Bushu\EC\strut
			}%
			\hfill\null\PAR
			\noindent\hrule\noindent\hfill\LD@Font@Mincho\LD@Kanji\raise-3pt\vbox to 20pt{}\hfill\null
		}%
		\vrule
		{\LD@Font@BigKanji\LD@Kanji}%\PutB(-1.22,0.85){\StrokeOrder{#6}}%
	}%
	\Tangowidth=\hsize\advance\Tangowidth by-\wd\LD@Box@One\advance\Tangowidth by -10.8pt%
	\Tangoheight=\ht\LD@Box@One\relax
	\Tangoheightd=\Tangoheight\relax
	\advance\Tangoheightd by 4pt%
	\OLoffset=\wd\LD@Box@One\relax
	\Tangowidthd=\hsize\advance\Tangowidthd by-10.8pt%
	\wd40=\Tangowidth\relax
	\setbox40=\vtop{%
		\parshape 1 \OLoffset \Tangowidth\hfuzz0pt\parskip0pt\baselineskip=9pt plus 1pt\noindent
		\LD@KanjiInfo
		\vtop to 5pt{}\hfil
		\LD@StringOne\vfill
	}%
	\setbox39=\vsplit40 to\Tangoheightd
	\goodbreak
	\ifvoid40%
		\LD@Dimen@HSize=\hsize\advance\LD@Dimen@HSize by -0.8pt%
		\LD@Dimen@WidthOne=\hsize\advance\LD@Dimen@WidthOne by-\wd\LD@Box@One\advance\LD@Dimen@WidthOne by-0.4pt%
		\setbox38=\vbox{%
			\unvcopy39\vfill
		}%
		\OLvoffset=\ht\LD@Box@One
		\advance\OLvoffset by -\ht38%
		\noindent
		\setbox41=\vbox{%%%%% hsize ?
		\moveright\LD@Dimen@WidthOne
		\box\LD@Box@One
		\vskip-\Tangoheight
		\advance\OLoffset by 5pt%
		\advance\OLvoffset by 4pt%
		\noindent\parshape 1 5pt \Tangowidth
		\baselineskip=9pt plus 2pt minus 0.5pt%
		\LD@KanjiInfo
		\hfil
		\LD@StringOne\vfil
		\vskip\OLvoffset plus 5pt minus 5 pt\nobreak
		}%
		\advance\LD@Dimen@HSize by-0pt%
		\wd41=\LD@Dimen@HSize
		\box41%
		\hrule
		\smash{%
			\hskip10.4pt\hskip\Tangowidth\vrule width 0.4pt height \Tangoheight depth 0pt\hskip 10mm\vrule width 0.4pt height \Tangoheight depth 0pt%
		}%
	\else	
		\LD@Dimen@WidthOne=\Tangowidth\advance\LD@Dimen@WidthOne by10.4pt%
		\advance\OLoffset by 5pt%
	\moveright\LD@Dimen@WidthOne\vbox{\noindent
			\copy\LD@Box@One\hrule  width \wd\LD@Box@One height 0pt depth 0.4pt%
		}%
		\nobreak\vskip-\Tangoheight\nobreak
		\parshape 7 5pt \Tangowidth 5pt \Tangowidth  5pt \Tangowidth 5pt \Tangowidth  5pt \Tangowidth 5pt \Tangowidth 5pt \Tangowidthd
		\baselineskip=9pt plus 2pt minus 0.5pt\noindent
		\LD@KanjiInfo
		\hfil
		\LD@StringOne
		\nobreak
		\vskip 5pt plus 4pt minus 4pt\nobreak\hrule
	\fi
}%


%%%%%%%%%%%%%%%%%%%%%%%%%%%%%%%%%%%%%%%%%%%%%%%%%%%%%
%												%
%				Stroke Count and Bushu Indexes				%
%												%
%%%%%%%%%%%%%%%%%%%%%%%%%%%%%%%%%%%%%%%%%%%%%%%%%%%%%



\def\LD@Index@Update{% 
	%% index par nb de traits -> info is stored in Traits+Stroke Count+Bushu(Kanji) control sequence = \CS Traits\Stroke\CS Bushu\Bushu\EC\EC
	\LD@Data@Set{Stroke Count}\LD@Kanji\LD@StrokeCount
	\edef\LD@Temp{\LD@StrokeCount ,}% in case there are more than one correct stroke counts
	\LD@String@Split\LD@Temp ,\LD@StrokeCount\LD@Trash
	\ifcsname Traits\LD@StrokeCount\CS Bushu\LD@Bushu\EC\endcsname
		\EA\xdef\CS Traits\LD@StrokeCount\CS Bushu\LD@Bushu\EC\EC{\CS Traits\LD@StrokeCount\CS Bushu\LD@Bushu\EC\EC\noindent\quad\hfill{\LD@Font@Word\LD@Kanji}\hfill\qquad\llap{\LD@Font@IndexNumber\the\LD@Kanji@Counter}\PAR}%
	\else	
		\EA\xdef\CS Traits\LD@StrokeCount\CS Bushu\LD@Bushu\EC\EC{\noindent\quad\hfill{\LD@Font@Word\LD@Kanji}\hfill\qquad\llap{\LD@Font@IndexNumber\the\LD@Kanji@Counter}\PAR}%
	\fi
%% index par clef -> info is stored in Clef+Bushu(Kanji)+Stroke Count control sequence = \CS Clef\CS Bushu\Bushu\EC\Stroke\EC
	\ifcsname Clef\CS Bushu\LD@Bushu\EC\LD@StrokeCount\endcsname
		\EA\xdef\CS Clef\CS Bushu\LD@Bushu\EC\LD@StrokeCount\EC{\CS Clef\CS Bushu\LD@Bushu\EC\LD@StrokeCount\EC\noindent\quad\hfill{\LD@Font@Word\LD@Kanji}\hfill\qquad\llap{\LD@Font@IndexNumber\the\LD@Kanji@Counter}\PAR}%
	\else
		\EA\xdef\CS Clef\CS Bushu\LD@Bushu\EC\LD@StrokeCount\EC{\noindent\quad\hfill{\LD@Font@Word\LD@Kanji}\hfill\qquad\llap{\LD@Font@IndexNumber\the\LD@Kanji@Counter}\PAR}%
	\fi
}%



%%%%%%%%%%%%%%%%%%%%%%%%%%%%%%%%%%%%%%%%%%%%%%%%%%%%%
%												%
%					Title Page						%
%												%
%%%%%%%%%%%%%%%%%%%%%%%%%%%%%%%%%%%%%%%%%%%%%%%%%%%%%

\def\LD@Akuma@Title@en{%
	\null\vfill
	\centerline{%
		$\mathop{\hbox{\LD@Font@Nihongo 湖悪魔}}\limits^{\hbox{\LD@Font@One Lakedaemon's}}_{\hbox{\LD@Font@KanaSubtitle こあくま}}$%
	}%
	\bigskip
	\bigskip
	\centerline{%
		\hbox{\LD@Font@KanaTitle の}%
	}%
	\bigskip\bigskip
	\centerline{%
		$\LD@Font@Nihongo\mathop{\hbox{日本語}}\limits^{\hbox{\LD@Font@One Japanese}}\hbox{\LD@Font@Nihongo-}%
		\mathop{\hbox{\LD@Font@Nihongo 英語}}\limits^{\hbox{\LD@Font@One English}}\hbox{\LD@Font@Nihongo -}\mathop{\hbox{\LD@Font@Nihongo 辞典}}\limits^{\hbox{\LD@Font@One dictionnary}}$%
	}%
	\vfill\null
	\eject
}%
\def\LD@Akuma@Title@fr{%
	\null\vfill
	\centerline{%
		$\mathop{\hbox{\LD@Font@Nihongo 湖悪魔}}\limits^{\hbox{\LD@Font@One Lakedaemon}}_{\hbox{\LD@Font@KanaSubtitle こあくま}}$%
	}%
	\bigskip
	\bigskip
	\bigskip
	\centerline{%
		\hbox{\LD@Font@KanaTitle の}%
	}%
	\bigskip
	\bigskip
	\bigskip
	\centerline{%
		$\mathop{\hbox{\LD@Font@Nihongo 漢字の辞典}}\limits^{\hbox{\LD@Font@One Manuel de Kanji}}$%
	}%
	\vfill\null
	\eject
}%
\CS LD@Akuma@Title@\LD@Akuma@Language\EC


%%%%%%%%%%%%%%%%%%%%%%%%%%%%%%%%%%%%%%%%%%%%%%%%%%%%%
%												%
%						Bushu 						%
%												%
%%%%%%%%%%%%%%%%%%%%%%%%%%%%%%%%%%%%%%%%%%%%%%%%%%%%%

\immediate\write16{Composing the Document...}%

\headline={}%
\AFiveLayout
\def\LInsert{\vrule\hskip-0.4pt\hfil}%
\def\RInsert{\hfil\vrule\hskip-0.4pt}%
\def\CInsert{\vrule}%
\def\BInsert{\vskip-0.4pt\hrule}%
\def\TInsert{\hrule}%
\singlecolumn
\XeTeXlinebreaklocale "fr"
\hfuzz0pt%
\removelastskip\vskip-15pt

\centerline{\LD@Font@Nihongo  部首} 
\gutter=0.6pc
\fivecolumns
{\baselineskip=10pt plus 1pt
%\newwrite\FilteredFile
%\immediate\openout\FilteredFile=NewBushu.utf\relax%
\newcount\BushuCount\global\BushuCount=0\relax
\LD@ParseFile{Bushu.utf}{#1 [#2,#3]#4}{%
	\global\advance\BushuCount by 1\relax
	\EA\xdef\CS Bushu\the\BushuCount\EC{#1}%
	{\LD@Font@BushuArray #1 \LD@Font@BushuMeaningsArray #3}\PAR
	%\immediate\write\FilteredFile{#2 []#3}%
}{}%
%\closeout\FilteredFile
}%
\eject
\onecolumn





%%%%%%%%%%%%%%%%%%%%%%%%%%%%%%%%%%%%%%%%%%%%%%%%%%%%%%%%%%%%%%%%%%%%%%%%%%%%%%%%%%%%%%%%%%%%%%%%%%%%%%%
%%%%%%%%                                                                                              Stroke Count (for later)
%%%%%%%%%%%%%%%%%%%%%%%%%%%%%%%%%%%%%%%%%%%%%%%%%%%%%%%%%%%%%%%%%%%%%%%%%%%%%%%%%%%%%%%%%%%%%%%%%%%%%%%





\newcount\StrokeCount
\newdimen\Unit
\Unit=66pt
%\font\LD@Font@NihongoBig="HGSeikaishotaiPRO" at \Unit
\def\StrokeOrder#1{\global\StrokeCount=0\relax\edef\Tail{#1;}\looptrue
\loop\global\advance\StrokeCount by 1\relax%
\LD@Split\Tail{}\Head\Tail
\if(\Head\LD@Split\Tail)\Head\Tail\expandafter\PutB\expandafter(\Head){\StrokeFont\the\StrokeCount}\fi\if;\Tail\loopfalse\fi\ifloop\repeat}%


%%%%%%%%%%%%%%%%%%%%%%%%%%%%%%%%%%%%%%%%%%%%%%%%%%%%%
%												%
%						Kanji Manual					%
%												%
%%%%%%%%%%%%%%%%%%%%%%%%%%%%%%%%%%%%%%%%%%%%%%%%%%%%%





\def\space{ }% fix this

\newcount\prems % usefull ?
\newcount\LD@Kanji@Counter % fix with common counter
\global\LD@Kanji@Counter=0\relax

\loop
	\global\advance\LD@Kanji@Counter by1\relax
	\edef\LD@Kanji{\CS KanjiN\the\LD@Kanji@Counter\EC}%
	\LD@Data@Set{Grade}\LD@Kanji\LD@Grade
	\ifx\LD@Grade\relax
		\def\LD@Grade{}%
	\fi
	\LD@Data@Set{Level}\LD@Kanji\LD@Level
	\ifx\LD@Level\relax
		\def\LD@Level{6}%
	\else
		\ifx\LD@Level\LD@Empty
			\def\LD@Level{6}%
		\fi
	\fi
	\LD@Data@Set{Bushu}\LD@Kanji\LD@Bushu
	\LD@Kanji@Info@Make
	\LD@Tango@Retrieve
	\LD@Entry@Display
	\LD@Index@Update
\ifnum\LD@Kanji@Counter<2501\relax
\repeat


\def\TInsert{\hrule}%
\def\BInsert{\hrule}%
\singlecolumn


%%%%%%%%%%%%%%%%%%%%%%%%%%%%%%%%%%%%%%%%%%%%%%%%%%%%%
%												%
%						 Stroke Count Index				%
%												%
%%%%%%%%%%%%%%%%%%%%%%%%%%%%%%%%%%%%%%%%%%%%%%%%%%%%%


\goodbreak
\centerline{\Arial{20}Index par nombre de traits}
\LD@Font@IndexNumber
\medskip\gutter=2pc
\sixcolumns


\newif\ifPrint
\for\Traits:=1,2,3,4,5,6,7,8,9,10,11,12,13,14,15,16,17,18,19,20,21,22,23,24\do{%
	\Printfalse
	\global\LD@One=0\relax
	\loop
		\global\advance\LD@One by1\relax
		\EA\ifx\EA\relax\CS Traits\Traits\CS Bushu\the\LD@One\EC\EC
		\else
			\Printtrue
		\fi
	\ifnum\LD@One<214\repeat
	\ifPrint
		\removelastskip\goodbreak
		\LD@One=\Traits\relax
		\ifnum\LD@One=1\relax
			\centerline{\LD@Font@IndexNumber\Traits~Trait}%
		\else
			\centerline{\LD@Font@IndexNumber\Traits~Traits}%
		\fi
		\vskip1pt\hrule\vskip1pt
	\fi
	\global\LD@One=0\relax
	\loop
		\global\advance\LD@One by1\relax
		\EA\ifx\EA\relax\CS Traits\Traits\CS Bushu\the\LD@One\EC\EC
		\else
			\noindent
			\penalty-200\noindent\rlap{\LD@Font@Word\CS Bushu\the\LD@One\EC}%
		\fi
		\CS Traits\Traits\CS Bushu\the\LD@One\EC\EC
	\ifnum\LD@One<214\repeat
}%
\vfill\null 
\singlecolumn


%%%%%%%%%%%%%%%%%%%%%%%%%%%%%%%%%%%%%%%%%%%%%%%%%%%%%
%												%
%						 Bushu Index					%
%												%
%%%%%%%%%%%%%%%%%%%%%%%%%%%%%%%%%%%%%%%%%%%%%%%%%%%%%


\goodbreak
\centerline{\Arial{20} Index par clef}
\gutter=2pc
\sixcolumns  

\newif\ifPrint
\global\LD@One=0\relax
\loop
	\global\advance\LD@One by1\relax
	\Printfalse
	\for\Traits:=1,2,3,4,5,6,7,8,9,10,11,12,13,14,15,16,17,18,19,20,21,22,23,24\do{%
		\EA\ifx\EA\relax\CS Clef\CS Bushu\the\LD@One\EC\Traits\EC
		\else
			\Printtrue
		\fi
	}%
	\ifPrint
		\removelastskip\goodbreak
		\centerline{\LD@Font@IndexNumber Clef \LD@Font@Word\CS Bushu\the\LD@One\EC}\vskip1pt\hrule\vskip1pt%
	\fi
	\for\Traits:=1,2,3,4,5,6,7,8,9,10,11,12,13,14,15,16,17,18,19,20,21,22,23,24\do{%
		\EA\ifx\EA\relax\CS Clef\CS Bushu\the\LD@One\EC\Traits\EC
		\else
			\penalty-200\noindent\rlap{\LD@Font@IndexNumber\Traits}%
		\fi
		\CS Clef\CS Bushu\the\LD@One\EC\Traits\EC
	}%
\ifnum\LD@One<214\relax\repeat
\vfill\null
\singlecolumn

%%%%%%%%%%%%%%%%%%%%%%%%%%%%%%%%%%%%%%%%%%%%%%%%%%%%%
%												%
%						 ReadingIndex					%
%												%
%%%%%%%%%%%%%%%%%%%%%%%%%%%%%%%%%%%%%%%%%%%%%%%%%%%%%

\immediate\write16{Sorting the readings...}%

% First set and sort the lectures

\def\xcholri#1#2{%#1, #2 counter variables
	\EA\let\EA\auxone\CS\the#1\EC
	\EA\let\EA\auxtwo\CS\the#2\EC
 	\EA\global\EA\let\CS\the#2\EC\auxone
	\EA\global\EA\let\CS\the#1\EC\auxtwo
	\EA\let\EA\auxone\CS RIC\the#1\EC
	\EA\let\EA\auxtwo\CS RIC\the#2\EC
	\EA\global\EA\let\CS RIC\the#2\EC\auxone
	\EA\global\EA\let\CS RIC\the#1\EC\auxtwo
}%

\let\cmp\cmpaw
\let\xch\xcholri


\n=\the\LD@Lectures\relax
\loop
	\global\advance\LD@Lectures by-1\relax
\ifnum\LD@Lectures>0\relax
	\EA\let\EA\LD@Temp\CS RIL\the\LD@Lectures\EC
	\EA\let\CS\the\LD@Lectures\EC\LD@Temp
\repeat

\prooffalse
\heapsort

% index par lectures

\goodbreak
\bigskip
\centerline{%
	\Arial{20}Index par lecture (\the\n)\strut
}%
\gutter=1pc
\sixcolumns
\global\LD@Lectures=0\relax
\loop
	\global\advance\LD@Lectures by1\CS RIC\the\LD@Lectures\EC
	\ifnum\LD@Lectures<\n\relax
\repeat

\singlecolumn
\write16{\the\LD@Entry@Count Entries}%
\bye