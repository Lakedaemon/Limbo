\magnification 1200
\hsize180truemm\vsize 270truemm\hoffset=-10truemm\voffset=-13truemm
\pretolerance=500\tolerance=1000\brokenpenalty=5000
\parindent3mm


\input color.txs
\input pstricks
\input pst-plot
\input eplaingt
\input Macrols
%\def\exo#1. #2\par{\expandafter\def\csname exo#1\endcsname{\global\advance\numeroexo by 1%
%\definexref{labelexo#1}{\the\numeroexo}{exo}\noindent\eqrefn{labelexo#1}) #2\medskip}}

\input ExosPTSI

%%% num�rotation automatique des questions

\newcount\numeroqprob
\numeroqprob=0
\def\pbq#1\par{\global\advance\numeroqprob by 1\noindent\the\numeroqprob. #1\medskip}
\overfullrule=0pt



%%% TD 1.
\vglue-10mm\rightline{PTSI\hfill Td 1 : Récurrences\hfill\date}
\bigskip
\vfill
\centerline{\fourteenbf Principe du raisonnement par récurrence}
Le raisonnement par récurrence consiste à démontrer une suite de propositions logiques $\{\sc P_0, \sc P_1, \sc P_2, \sc P_3, \cdots\}$ En procédant en deux étapes : 
\medskip
\noindent{\bf L'initialisation}. On établit la propriété au(x) premier(s) rang(s), de sorte à enclencher le méchanisme de transmission. 
\medskip
\noindent{\bf La transmission}. On prouve que la propriété se transmet du (des) rang(s) précédent(s) au rang suivant. 
\medskip
\noindent
Le raisonnement par récurrence s'appuie donc sur le schéma (très simple) suivant : 
\medskip
$$\Q\{
\eqalign{
&\sc P_0\cr
\sc P_n&\Rightarrow\sc P_{n+1}}\W.
\qquad\Longrightarrow \Q\{\eqalign{\sc P_0\cr
\sc P_1\cr
\sc P_2\cr
\sc P_3\cr
\vdots}\W.\leqno{\hbox{Recurence forte}}
$$

\centerline{\fourteenbf Démonstration type recommandée \rm (cas simple)}
\medskip
\noindent
Pour $n\in\ob N$,\footnote1{quantification de l'entier $n$ : les entiers $n$ pour lesquels vous voulez prouver $\sc P_n$.} prouvons par récurrence la propriété $\sc P_n$\footnote2{vous annoncez ce que vous voulez dérmontrer}:
{\bf énoncé au rang n.}\footnote3{il faut absolument énoncer clairement la propriété au rang $n$}\medskip\noindent
$\LuciferStar$ La propriété $\sc P_0$ est vraie\footnote4{initialisation} car {\bf justifications...}\medskip\noindent
$\LuciferStar$ Soit $n\ge0$ un entier tel que $\sc P_n$  soit vraie,\footnote5{vous supposez que $\sc P_n$ est vraie} 
établissons la propriété $\sc P_{n+1}$.$^{(4)}$ \smallskip\noindent
{\bf Raisonnement...d'après $\sc P_n$...raisonnement} donc la propriété $\sc P_{n+1}$ est vérifiée\footnote6{vous achevez ici la démonstration de l'implication $\sc P_n\Longrightarrow\sc P_{n+1}$}
\medskip\noindent
Comme $\sc P_0$ est vraie et comme $\sc P_n\Longrightarrow\sc P_{n+1}$ pour chaque $n\ge0$\footnote7{facultatif, mais apprécié par le correcteur}, 
par récurrence, nous concluons que  
la propriété $\sc P_n$ est vérifiée\footnote8{La récurrence est enfin démontrée} pour $n\in\ob N^{(1)}$. 
\bigskip

\centerline{\fourteenbf Remarques}

\Remarque{ \it 1.} Faire une démonstration par récurrence est en général très facile, sauf lorsque l'enoncé ne donne pas la propriété de récurrence. 
Auquel cas, il faut d'abord l'intuiter (il y a des techniques pour cela)  puis la prouver.   
\medskip 

\Remarque{ \it 2.} Il existe d'autres shemas de démonstration (les variations sont infinies). C'est pourquoi il est essentiel d'en comprendre le principe de récurrence pour l'adapter à des cas plus compliqués.  
$$\Q\{
\eqalign{
&\sc P_0\cr
\sc P_0\hbox{ et }\sc P_1 \hbox{et }\cdots\hbox{et }\sc P_n&\Rightarrow\sc P_{n+1}}\W.
\qquad\Longrightarrow \Q\{\eqalign{\sc P_0\cr
\sc P_1\cr
\sc P_2\cr
\sc P_3\cr
\vdots}\W.\leqno{\hbox{Récurrence faible}}
$$
$$\Q\{
\eqalign{
&\sc P_0, \sc P_1\cr
\sc P_{n-1}\hbox{ et }\sc P_n&\Rightarrow\sc P_{n+1}}\W.
\qquad\Longrightarrow \Q\{\eqalign{\sc P_0\cr
\sc P_1\cr
\sc P_2\cr
\sc P_3\cr
\vdots}\W.\leqno{\hbox{Recurrence à plusieurs pas}}
$$
$$\Q\{
\eqalign{
&\sc P_1\cr
\hbox{Pour $1\le n< k$, }\sc P_n&\Rightarrow\sc P_{n+1}}\W.
\qquad\Longrightarrow \Q\{\eqalign{\sc P_1\cr
\sc P_2\cr
\vdots\cr
\sc P_{k-1}\cr
\sc P_k}\W.\leqno{\hbox{Recurrence finie commençant au rang 1}}
$$
On peut même démontrer plusieurs propriétés (éventuellement inter-dépendantes) par récurence à la fois... 
\bigskip 

\Remarque{ \it 3.} Pour énoncer la propriété de récurrence, il est recomandé d'utiliser les signes sommes $\sum_a^b$ et produits $\prod_a^b$ ainsi que les factorielles plut\^ot que des points de suspension. 
On rappelle que le symbole 
$$
\sum_{n=a}^bf(n)=\sum_{a\le n\le b}f(n)
$$
vaut 
$$
f(a)+f(a+1)+\cdots+f(b-1)+f(b)
$$
si $a\le b$ et vaut $0$ si $a> b$ (c'est une convention).
De même, le symbole 
$$
\prod_{n=a}^bf(n)=\prod_{a\le n\le b}f(n)
$$
vaut 
$$
f(a)f(a+1)\cdots f(b-1)f(b)
$$
si $a\le b$ et vaut $1$ si $a> b$ (c'est une convention).\goodbreak
\vfill
\exocs
\vfill
\exoct
\vfill
\input ExosPT
\exoij
\vfill
\input ExosPTSI
\exocv
\vfill
\exopx
\vfill
\exocu
\vfill
\input ExosPT
\exoik
\vfill\noindent
{\it Adresse internet du site web de maths : }
$$
\hbox{\bf http://www.lakedaemon.org/Res\_Mathematica.php}
$$

\bye