%2multibyte Version: 5.50.0.2953 CodePage: 65001

\documentclass{article}
%%%%%%%%%%%%%%%%%%%%%%%%%%%%%%%%%%%%%%%%%%%%%%%%%%%%%%%%%%%%%%%%%%%%%%%%%%%%%%%%%%%%%%%%%%%%%%%%%%%%%%%%%%%%%%%%%%%%%%%%%%%%%%%%%%%%%%%%%%%%%%%%%%%%%%%%%%%%%%%%%%%%%%%%%%%%%%%%%%%%%%%%%%%%%%%%%%%%%%%%%%%%%%%%%%%%%%%%%%%%%%%%%%%%%%%%%%%%%%%%%%%%%%%%%%%%
%TCIDATA{OutputFilter=LATEX.DLL}
%TCIDATA{Version=5.50.0.2953}
%TCIDATA{Codepage=65001}
%TCIDATA{<META NAME="SaveForMode" CONTENT="1">}
%TCIDATA{BibliographyScheme=Manual}
%TCIDATA{Created=Sunday, September 29, 2013 14:47:20}
%TCIDATA{LastRevised=Wednesday, June 29, 2016 08:56:29}
%TCIDATA{<META NAME="GraphicsSave" CONTENT="32">}
%TCIDATA{<META NAME="DocumentShell" CONTENT="Standard LaTeX\Blank - Standard LaTeX Article">}
%TCIDATA{CSTFile=40 LaTeX article.cst}
%TCIDATA{PageSetup=28,28,28,28,0}
%TCIDATA{AllPages=
%H=36
%F=36
%}


\newtheorem{theorem}{Theorem}
\newtheorem{acknowledgement}[theorem]{Acknowledgement}
\newtheorem{algorithm}[theorem]{Algorithm}
\newtheorem{axiom}[theorem]{Axiom}
\newtheorem{case}[theorem]{Case}
\newtheorem{claim}[theorem]{Claim}
\newtheorem{conclusion}[theorem]{Conclusion}
\newtheorem{condition}[theorem]{Condition}
\newtheorem{conjecture}[theorem]{Conjecture}
\newtheorem{corollary}[theorem]{Corollary}
\newtheorem{criterion}[theorem]{Criterion}
\newtheorem{definition}[theorem]{Definition}
\newtheorem{example}[theorem]{Example}
\newtheorem{exercise}[theorem]{Exercise}
\newtheorem{lemma}[theorem]{Lemma}
\newtheorem{notation}[theorem]{Notation}
\newtheorem{problem}[theorem]{Problem}
\newtheorem{proposition}[theorem]{Proposition}
\newtheorem{remark}[theorem]{Remark}
\newtheorem{solution}[theorem]{Solution}
\newtheorem{summary}[theorem]{Summary}
\newenvironment{proof}[1][Proof]{\noindent\textbf{#1.} }{\ \rule{0.5em}{0.5em}}
\input{tcilatex}
\begin{document}


\subsubsection{ \ \ \ \ \ \ \ \ \ \ \ \ \ \ \ \ \ \ \ \ \ \ \ \ \ \ \ \ \ \
\ \ \ \ \ \ \ \ \protect\underline{Chapitre 1 : Utilisation de la console.}}

\paragraph{\protect\underline{1. Introduction.}}

A l'ouverture de Scilab, une fen\^{e}tre de commande ou console s'ouvre et
une invite de commande $\ $\frame{$-\rightarrow $} indique que le logiciel
attend les instructions.

Cette premi\`{e}re fen\^{e}tre fonctionne comme une calculatrice. On valide
les instructions en tapant sur "entr\'{e}e"

Si l'instruction se termine par un point virgule, l'instruction est effectu%
\'{e}e mais n'est pas affich\'{e}e.

Pour effacer la console , on peut taper "clc" ( clear console") ou aller
dans le menu "edition" puis "effacer console".

Remarques :

- Les signes \frame{//} indiquent \`{a} Scilab que tout ce qui est \`{a}
droite de ces signes ne doit pas \^{e}tre lu. On l'utilisera donc pour \'{e}%
crire des commentaires.

- On a possibilit\'{e} d'\'{e}crire plusieurs instructions sur une m\^{e}me
ligne en les s\'{e}parant par des "," ou des ";".

Le point virgule bloque l'affichage du r\'{e}sultat le pr\'{e}c\'{e}dant.

\paragraph{\protect\underline{2. Variables, affectations, constantes,}}

\underline{2.1. Variables et affectations.}

Les variables correspondent \`{a} un espace de stockage.

Elles sont cr\'{e}\'{e}es d\`{e}s qu'elles sont introduites par
l'instruction d'affectation \frame{"="} et ces variables prennent le type de
l'expression.

\frame{Affectation : \ Nom \ = \ expression .}

L'expression peut-\^{e}tre du type num\'{e}rique ( r\'{e}el ou complexe ),
bool\'{e}en, matriciel ou de type cha\^{\i}ne de caract\`{e}re.

Si un calcul n'est pas affect\'{e} \`{a} une variable, Scilab place
automatiquement le r\'{e}sultat dans une variable nomm\'{e}e \fbox{"ans"}. \
Le type de cette variable est celui de l'expression calcul\'{e}e.

\underline{2.2. Constantes}

Scilab conna\^{\i}t deux constantes r\'{e}elles pr\'{e}d\'{e}finies $\pi $
et $e$ not\'{e}es $\%pi$ et $\%e.$

Le nombre imaginaire $i$ est not\'{e} $\%i$.

Scilab conna\^{\i}t aussi deux constantes bool\'{e}ennes pr\'{e}d\'{e}finies 
$vrai$ et $faux$ not\'{e}es $\%t$ et $\%f.$

\underline{2.3 Cha\^{\i}nes de caract\`{e}res }

Une cha\^{\i}ne de caract\`{e}res est une suite de caract\`{e}res \'{e}crits
entre apostrophes ( ') ou guillemets droits doubles \ ( " " ).

\underline{2.4. Variable logique.}

Une variable logique ou bool\'{e}enne correspond \`{a} une expression
logique. Elle peut prendre deux valeurs : vraie ( $\%t$ ) ou fausse ( $\%f$
).

\paragraph{\protect\underline{3. Les op\'{e}rateurs en Scilab.}}

\underline{3.1. Op\'{e}rateurs arithm\'{e}tiques .}

$%
\begin{array}{ccccc}
+ & - & \ast & / & \symbol{94} \\ 
\text{addition} & \text{soustraction} & \text{multiplication} & \text{%
division} & \text{puissance}%
\end{array}%
$

Exercice 1 : Taper :

$-\rightarrow x=5;y=2;$

$-\rightarrow x+y,$ $\ z=x-y$

$-\rightarrow x\ast z,$ $x/y,$ $\ x\symbol{94}y$

\underline{3.2. Op\'{e}rateurs relationnels et logiques.}

Logiques : $%
\begin{array}{ccc}
\& & | & \symbol{126} \\ 
et & ou\text{ } & non%
\end{array}%
$

Comparaisons et tests : $%
\begin{array}{cccccc}
== & < & > & >= & <= & <> \\ 
\text{{\small \'{e}galit\'{e}}} & \text{{\small inf\'{e}rieur strict}} & 
\text{{\small sup\'{e}rieur strict}} & \text{{\small sup\'{e}rieur ou \'{e}%
gal}} & \text{{\small inf\'{e}rieur ou \'{e}gal}} & \text{{\small diff\'{e}%
rent}}%
\end{array}%
$

Exercice 2 :

Taper :

$-\rightarrow a=2>3,b=2<3,c=\%t;$

$-\rightarrow a\&b,\ \ a|b$

$-\rightarrow d=\symbol{126}a\&c,d\&b$

\underline{3.3. Les fonctions usuelles pr\'{e}d\'{e}finies.}

$%
\begin{array}{ccccccc}
\log & \exp & floor & abs & sqrt & \sin & \cos \\ 
\text{logarithme} & \text{exponentielle} & \text{partie enti\`{e}re} & \text{%
valeur absolue} & \text{ }\sqrt{} & \text{sinus} & \text{cosinus}%
\end{array}%
$

\bigskip

\paragraph{\protect\underline{4. Instructions d'entr\'{e}e et de sortie.}}

\underline{4.1. Instruction d'entr\'{e}e.}

L'entr\'{e}e par l'utilisateur d'une variable num\'{e}rique ou d'une matrice
se fait par la commande \frame{$input$}$.$

L'instruction $x=input(^{\prime }$ Donner une valeur \`{a} x : $^{\prime })$
affiche le texte entre apostrophe puis Scilab attend une r\'{e}ponse.

Si elle est correcte, la r\'{e}ponse est affect\'{e}e \`{a} la variable x.

Cette instruction permet donc de demander avec une phrase la valeur \`{a}
attribuer \`{a} une variable.

Exemple :

Taper :

$-\rightarrow e=input("$ somme en euros : $");$

\underline{4.2. Instruction de sortie.}

L'instruction \frame{$disp(x)$} affiche le contenu de la variable $x.$

L'instruction $disp(^{\prime }texte")$ la cha\^{\i}ne de caract\`{e}res
entre apostrophes.

Exemple :

Taper

$-\rightarrow a=5;$

$-\rightarrow $disp( ' la variable $a$ est \'{e}gale \`{a} :',$a$)

$-\rightarrow $disp( $a$, ' la variable $a$ est \'{e}gale \`{a} : ')

\bigskip

\bigskip

\bigskip

\bigskip

\bigskip

\paragraph{ \ \protect\underline{\ 5. Exercices.}\ \ \ \ \ \ \ \ \ \ \ \ \ \
\ \ \ \ \ \ \ \ \ \ \ \ \ \ \ \ \ }

\underline{Exercice 1 : }Taper successivement les commandes suivantes et v%
\'{e}rifier ce qui se passe :

$%
\begin{array}{c}
16/(2\symbol{94}3) \\ 
2+3\ast 4 \\ 
2+3\ast \%i \\ 
pi%
\end{array}%
$ \ \ \ \ puis \ \ \ $%
\begin{array}{c}
\%pi \\ 
sqrt(2) \\ 
SQRT(2) \\ 
sqrt(2);%
\end{array}%
.$

\underline{Exercice 2 :} Taper successivement les commandes suivantes et v%
\'{e}rifier ce qui se passe :

$%
\begin{array}{c}
x=5 \\ 
x=7; \\ 
8x \\ 
8\ast x \\ 
3\ast 5 \\ 
x+12 \\ 
disp(x)%
\end{array}%
$ \ \ \ \ \ \ \ \ \ \ \ \ \ \ puis \ \ \ \ \ $%
\begin{array}{c}
x=x+7 \\ 
disp(^{\prime }x^{\prime }) \\ 
y=\sin (x\symbol{94}3) \\ 
disp(y,^{\prime }valeur\text{ }de\text{ }y\text{'}) \\ 
a=input(^{\prime }entrer\text{ }un\text{ }entier\text{ }naturel\text{ }non%
\text{ }nul^{\prime }) \\ 
\cos (1+\log (a)\symbol{94}2)%
\end{array}%
$

\bigskip

\underline{Exercice 3 :}

Caculer les valeurs suivantes : \ 1. $a=(2e^{\pi }-5)^{2}$ \ \ \ \ \ \ \ \ \
\ \ \ \ \ \ \ \ \ \ \ \ 2. $b=\dfrac{\sqrt{3}-2\sqrt[3]{8}+3\ln 5}{6}$ \ \ 

\ \ \ \ \ \ \ \ \ \ \ \ \ \ \ \ \ \ \ \ \ \ \ \ \ \ \ \ \ \ \ \ \ \ \ \ \ \
\ \ \ \ \ \ \ 3. \ $c=\sqrt{\left\vert \ln (\cos \QDOVERD( ) {\pi
}{5})\right\vert }$ \ \ \ \ \ \ \ \ \ \ \ 4. $d=(5+2i)(3-4i)$

\underline{Exercice 4 :}

On consid\`{e}re le script suivant :

$x=input(^{\prime }Entrer$ $x:^{\prime })$

$y=input(^{\prime }Entrer$ $y:^{\prime })$

$z=x+y;$

$x=x\ast z;$

$y=x-z;$

$disp(z,y,x)$

On suppose qu'on entre les valeurs 3 et 7 au clavier. Quelles sont alors les
valeurs affich\'{e}es \`{a} l'issue de l'ex\'{e}cution de la commande " $%
disp(z,y,x)"$ ?

\underline{Exercice 5 :}

Quelles sont les valeurs affich\'{e}es \`{a} l'issue de l'ex\'{e}cution des
instructions de la ligne suivante :

$-->x=1;$ $\ $ $\ $

$-->y=\%pi;$

$-->floor(y)$

$-->y=ans-x;$

$-->x=ans\symbol{94}y$

$-->y-6+x$

$-->y=ans-y$

\underline{Exercice 6 :}Calculer le prix TTC d'un article \`{a} partir de
son prix HT, avec un taux de TVA \`{a} 20\%.

\underline{Exercice 7 :} Convertir un temps donn\'{e} en secondes, en
heures, minutes et secondes.

( On utilisera la fonction partie enti\`{e}re "floor" )

\end{document}
