%2multibyte Version: 5.50.0.2953 CodePage: 65001

\documentclass{article}
%%%%%%%%%%%%%%%%%%%%%%%%%%%%%%%%%%%%%%%%%%%%%%%%%%%%%%%%%%%%%%%%%%%%%%%%%%%%%%%%%%%%%%%%%%%%%%%%%%%%%%%%%%%%%%%%%%%%%%%%%%%%%%%%%%%%%%%%%%%%%%%%%%%%%%%%%%%%%%%%%%%%%%%%%%%%%%%%%%%%%%%%%%%%%%%%%%%%%%%%%%%%%%%%%%%%%%%%%%%%%%%%%%%%%%%%%%%%%%%%%%%%%%%%%%%%
\usepackage{amssymb}
\usepackage{amsfonts}
\usepackage{amsmath}

\setcounter{MaxMatrixCols}{10}
%TCIDATA{OutputFilter=LATEX.DLL}
%TCIDATA{Version=5.50.0.2953}
%TCIDATA{Codepage=65001}
%TCIDATA{<META NAME="SaveForMode" CONTENT="1">}
%TCIDATA{BibliographyScheme=Manual}
%TCIDATA{Created=Tuesday, February 11, 2014 09:26:47}
%TCIDATA{LastRevised=Wednesday, June 29, 2016 08:58:27}
%TCIDATA{<META NAME="GraphicsSave" CONTENT="32">}
%TCIDATA{<META NAME="DocumentShell" CONTENT="Standard LaTeX\Blank - Standard LaTeX Article">}
%TCIDATA{CSTFile=40 LaTeX article.cst}
%TCIDATA{PageSetup=28,28,28,28,0}
%TCIDATA{AllPages=
%H=36
%F=36
%}


\newtheorem{theorem}{Theorem}
\newtheorem{acknowledgement}[theorem]{Acknowledgement}
\newtheorem{algorithm}[theorem]{Algorithm}
\newtheorem{axiom}[theorem]{Axiom}
\newtheorem{case}[theorem]{Case}
\newtheorem{claim}[theorem]{Claim}
\newtheorem{conclusion}[theorem]{Conclusion}
\newtheorem{condition}[theorem]{Condition}
\newtheorem{conjecture}[theorem]{Conjecture}
\newtheorem{corollary}[theorem]{Corollary}
\newtheorem{criterion}[theorem]{Criterion}
\newtheorem{definition}[theorem]{Definition}
\newtheorem{example}[theorem]{Example}
\newtheorem{exercise}[theorem]{Exercise}
\newtheorem{lemma}[theorem]{Lemma}
\newtheorem{notation}[theorem]{Notation}
\newtheorem{problem}[theorem]{Problem}
\newtheorem{proposition}[theorem]{Proposition}
\newtheorem{remark}[theorem]{Remark}
\newtheorem{solution}[theorem]{Solution}
\newtheorem{summary}[theorem]{Summary}
\newenvironment{proof}[1][Proof]{\noindent\textbf{#1.} }{\ \rule{0.5em}{0.5em}}
\input{tcilatex}
\begin{document}


\subsection{ \ \ \ \ \ \ \ \ \ \ \ \ \ \ \ \ \ \ \ \ \ \ \ \ \ \ \ \ \ \ \ \
\ \ \ \protect\underline{Chapitre 6 : l'instruction while.}}

\bigskip

\subsubsection{\protect\underline{1. D\'{e}finition de l'instruction while.}}

\paragraph{\protect\underline{1.1. R\'{e}daction.}}

La boucle while correspond \`{a} "tant que ".

R\'{e}daction : $\ 
\begin{array}{c}
while\text{ cond 1 } \\ 
Instruction\text{ }1; \\ 
\vdots \\ 
Instruction\text{ }p; \\ 
end \\ 
Inst\text{ }A%
\end{array}%
$

Elle fonctionne de la fa\c{c}on suivante :

Si la condition bool\'{e}enne cond 1 est vraie, le bloc d'instruction est ex%
\'{e}cut\'{e}, puis on recommence la boucle.

Sinon on arr\^{e}te d'effectuer la boucle et on effectue l'instruction A .

Il est clair que le bloc d'instructions doit pouvoir modifier
l'environnement de mani\`{e}re \`{a} ce que l'expression bool\'{e}enne
puisse au bout d'un nombre fini d'\'{e}tapes prendre la valeur fausse ( \%F).

\paragraph{\protect\underline{1.2. Propri\'{e}t\'{e}s.}}

\underline{Quelle boucle utiliser ?}

On utilise les boucles for lorsqu'on conna\^{\i}t \`{a} priori le nombre d'it%
\'{e}rations. Dans le cas contraire on utilise une boucle while.

\underline{Comment compter le nombre d'it\'{e}rations ?}

Lors de l'utilisation de la boucle while, m\^{e}me si l'on ne conna\^{\i}t
pas \`{a} l'avance le nombre d'it\'{e}rations \`{a} effectuer, on peut avoir
besoin de les d\'{e}compter ou d'en conna\^{\i}tre le nombre final.

C'est le cas lorsque l'on effectue des calculs sur des suites r\'{e}%
currentes.

Pour ce faire, il suffit d'introduire un compteur qui compte 1 \`{a} chaque
passage dans la boucle.

On d\'{e}finit donc une variable k, \`{a} initialiser en dehors de la
boucle, et on \'{e}crit dans la boucle l'instruction $"k=k+1".$

\subsubsection{\protect\underline{2. Exercices.}}

\paragraph{\protect\underline{Exercice 1 :}}

1. On consid\`{e}re le programme Scilab suivant :

v=input('donner un entier naturel v= ');

\ \ s = 1;

i = 1;

\ \ while $s<v$

\ \ \ \ \ \ \ \ \ \ \ \ \ \ \ \ \ \ \ \ \ \ \ \ \ \ \ i = 1+i;

\ \ \ \ \ \ \ \ \ \ \ \ \ \ \ \ \ \ \ \ \ \ \ \ \ \ \ s= s + 1/i;

\ \ \ end

disp(s, 'la valeur est ');

Pour v = 2, donner la valeur affich\'{e}e par l'ordinateur.

2. On d\'{e}montre que la suite $u$ d\'{e}finie pour tout $n\geq 1$ par $%
u_{n}=\underset{k=1}{\overset{n}{\sum }}\frac{1}{k}-\ln n$ est une suite d%
\'{e}croissante, minor\'{e}e par 0 et donc convergente.Sa limite est appel%
\'{e}e constante d'Euler $C$ avec $C\approx 0.577215.$

Modifier le programme pr\'{e}c\'{e}dent pour qu'il d\'{e}termine la valeur
de $n$ \`{a} partir de laquelle la valeur de $u_{n}$ est une approximation
de $C$ \`{a} $\varepsilon $ pr\`{e}s, la valeur de $\varepsilon $ \'{e}tant
choisie par l'utilisateur.

\paragraph{\protect\underline{Exercice 2 : }}

On consid\`{e}re la suite r\'{e}elle $\left( u_{n}\right) $ d\'{e}finie par $%
u_{0}=-0.01$ et \ $\forall n\in \mathbb{N},$ \ $u_{n+1}=u_{n}-u_{n}^{2}.$

1. Ecrire un programme Scilab qui demande \`{a} l'utilisateur de saisir un
entier N et qui affiche tous les termes de la suite jusqu'au rang N.

2. Ecrire un programme qui calcule et affiche le premier rang n \`{a} partir
duquel $u_{n}\leq -10.$

\bigskip

\paragraph{\protect\underline{Exercice 3 :}}

On consid\`{e}re la suite $\left( u_{n}\right) $ d\'{e}finie par : $u_{0}=1,$
$u_{1}=2$ et $u_{n+2}=u_{n+1}+2u_{n}.$ \ 

Ecrire un programme qui calcule et affiche le premier rang n \`{a} partir
duquel $u_{n}\geq 10.$

\paragraph{\protect\underline{Exercice 4 :}}

( Etude de la suite de Syracuse )

On consid\`{e}re la suite $u$ d\'{e}finie par : $u_{0}\in 
%TCIMACRO{\U{2115} }%
%BeginExpansion
\mathbb{N}
%EndExpansion
$ et $\forall n\in 
%TCIMACRO{\U{2115} }%
%BeginExpansion
\mathbb{N}
%EndExpansion
,$ \ $u_{n+1}=\left\{ 
\begin{array}{c}
\dfrac{u_{n}}{2}\text{ si }u_{n}\text{ est pair} \\ 
3u_{n}+1\text{ \ \ \ sinon}%
\end{array}%
\right. .$

1. Ecrire un programme qui demande deux entiers naturels $u_{0}$ et $n$,qui
renvoie la valeur de $u_{n}$ et qui repr\'{e}sente les termes de $u.$

2. Quel est le comportement de la suite $u$ si $u_{0}=1$ ?

3. a. Ecrire un script qui demande un entier naturel $u_{0}$ et qui calcule
la plus petite valeur de $n$ pour laquelle $u_{n}=1.$

\ \ \ b. Compl\'{e}ter ce script pour qu'il calcule et affiche la plus
grande valeur prise par la suite.

( On parle d'altitude de la suite )

\paragraph{\protect\underline{Exercice 5 :\thinspace }}

On consid\`{e}re la suite $u$ d\'{e}finie par $u_{0}=-1$ et $\forall n\in 
\mathbb{N},\quad u_{n+1}=\dfrac{e^{u_{n}}-3}{2}$ ainsi que la fonction $f$ d%
\'{e}finie sur $\mathbb{R}$ par : $f(x)=\dfrac{e^{x}-3}{2}.$

1. Montrer que l'application $f$ admet un unique point fixe $\alpha $ sur $%
%TCIMACRO{\U{211d} }%
%BeginExpansion
\mathbb{R}
%EndExpansion
^{-}$. Justifier que $-2\leqslant \alpha \leqslant -1.$

2. Justifier que $f(]-\infty ,0])\subset ]-\infty ,0]$ et que $\forall x\in
]-\infty ,0],\quad \left\vert f^{\prime }(x)\right\vert \leqslant \dfrac{1}{2%
}.$

3. Montrer que $\forall n\in \mathbb{N},\quad u_{n}\leqslant 0.$

4. Montrer que $\forall n\in \mathbb{N},\quad $ $\left\vert u_{n}-\alpha
\right\vert \leqslant \dfrac{1}{2^{n}}.$

5. En d\'{e}duire que la suite $u$ converge vers $\alpha .$

6.Ecrire un programme affichant une valeur approch\'{e}e de $\alpha $ \`{a} $%
\varepsilon $ pr\`{e}s, la pr\'{e}cision $\varepsilon $ \'{e}tant fournie
par l'utilisateur et indiquant le nombre d'it\'{e}rations n\'{e}cessaires.

\bigskip

\bigskip

\end{document}
