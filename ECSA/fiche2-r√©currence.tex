%2multibyte Version: 5.50.0.2953 CodePage: 65001

\documentclass{article}
%%%%%%%%%%%%%%%%%%%%%%%%%%%%%%%%%%%%%%%%%%%%%%%%%%%%%%%%%%%%%%%%%%%%%%%%%%%%%%%%%%%%%%%%%%%%%%%%%%%%%%%%%%%%%%%%%%%%%%%%%%%%%%%%%%%%%%%%%%%%%%%%%%%%%%%%%%%%%%%%%%%%%%%%%%%%%%%%%%%%%%%%%%%%%%%%%%%%%%%%%%%%%%%%%%%%%%%%%%%%%%%%%%%%%%%%%%%%%%%%%%%%%%%%%%%%
\usepackage{amssymb}
\usepackage{amsfonts}
\usepackage{amsmath}

\setcounter{MaxMatrixCols}{10}
%TCIDATA{OutputFilter=LATEX.DLL}
%TCIDATA{Version=5.50.0.2953}
%TCIDATA{Codepage=65001}
%TCIDATA{<META NAME="SaveForMode" CONTENT="1">}
%TCIDATA{BibliographyScheme=Manual}
%TCIDATA{Created=Wednesday, August 29, 2007 15:11:43}
%TCIDATA{LastRevised=Tuesday, June 28, 2016 19:33:47}
%TCIDATA{<META NAME="GraphicsSave" CONTENT="32">}
%TCIDATA{<META NAME="DocumentShell" CONTENT="Standard LaTeX\Blank - Standard LaTeX Article">}
%TCIDATA{CSTFile=40 LaTeX article.cst}
%TCIDATA{PageSetup=43,28,28,28,0}
%TCIDATA{AllPages=
%H=36
%F=36,\PARA{038<p type="texpara" tag="Body Text" > \ \ \ \ \ \ \ \ \ \ \ \ \ \ \ \ \ \ \ \ \ \ \ \ \ \ \ \ \ \ \ \ \ \ \ \ \ Fiche n$\U{b0}2-$\thepage }
%}


\newtheorem{theorem}{Theorem}
\newtheorem{acknowledgement}[theorem]{Acknowledgement}
\newtheorem{algorithm}[theorem]{Algorithm}
\newtheorem{axiom}[theorem]{Axiom}
\newtheorem{case}[theorem]{Case}
\newtheorem{claim}[theorem]{Claim}
\newtheorem{conclusion}[theorem]{Conclusion}
\newtheorem{condition}[theorem]{Condition}
\newtheorem{conjecture}[theorem]{Conjecture}
\newtheorem{corollary}[theorem]{Corollary}
\newtheorem{criterion}[theorem]{Criterion}
\newtheorem{definition}[theorem]{Definition}
\newtheorem{example}[theorem]{Example}
\newtheorem{exercise}[theorem]{Exercise}
\newtheorem{lemma}[theorem]{Lemma}
\newtheorem{notation}[theorem]{Notation}
\newtheorem{problem}[theorem]{Problem}
\newtheorem{proposition}[theorem]{Proposition}
\newtheorem{remark}[theorem]{Remark}
\newtheorem{solution}[theorem]{Solution}
\newtheorem{summary}[theorem]{Summary}
\newenvironment{proof}[1][Proof]{\noindent\textbf{#1.} }{\ \rule{0.5em}{0.5em}}
\input{tcilatex}
\begin{document}


\subsection{ \protect\bigskip\ \ \ \ \ \ \ \ \ \ \ \ \ \ \ \ \ \ \ \ \ \ \ \
\ \ \ \ \ \ \protect\underline{\textbf{Fiche n}$%
%TCIMACRO{\U{b0}}%
%BeginExpansion
{{}^\circ}%
%EndExpansion
2$\textbf{\ : \ R\'{e}currence et sommation.}}}

\underline{Exercice 1 :} On consid\`{e}re la suite $(u_{n})$ d\'{e}finie par
: $\left\{ 
\begin{array}{c}
u_{0}=3 \\ 
\forall n\in 
%TCIMACRO{\U{2115} }%
%BeginExpansion
\mathbb{N}
%EndExpansion
,u_{n+1}=\sqrt{u_{n}}%
\end{array}%
\right. .$

1) Montrer que la suite $(u_{n})$ est bien d\'{e}finie et v\'{e}rifie : $%
\forall n\epsilon \mathbb{N},$ \ $u_{n}>0.$

2) Montrer que : $\forall n\epsilon \mathbb{N},$ \ $u_{n}=\left( 3\right)
^{\left( \frac{1}{2}\right) ^{n}}.$

\underline{Exercice 2* :} On consid\`{e}re la suite $(u_{n})$ d\'{e}finie
par : $\left\{ 
\begin{array}{c}
u_{0}=1 \\ 
\forall n\in 
%TCIMACRO{\U{2115} }%
%BeginExpansion
\mathbb{N}
%EndExpansion
,u_{n+1}=\frac{u_{n}+1}{u_{n}+2}%
\end{array}%
\right. .$

Montrer par r\'{e}currence que : $(u_{n})$ est bien d\'{e}finie et v\'{e}%
rifie $\forall n\epsilon \mathbb{N},$ $0\leq u_{n}\leq 1.$

\underline{Exercice 3 :} Soit la suite $u$ d\'{e}finie par : $\left\{ 
\begin{array}{c}
u_{0}=2\text{ et }u_{1}=3 \\ 
\forall n\in 
%TCIMACRO{\U{2115} }%
%BeginExpansion
\mathbb{N}
%EndExpansion
,\text{ }u_{n+2}=3u_{n+1}-2u_{n}%
\end{array}%
.\right. $

V\'{e}rifier que $\forall n\in 
%TCIMACRO{\U{2115} }%
%BeginExpansion
\mathbb{N}
%EndExpansion
,$ $u_{n}=1+2^{n}.$

\underline{Exercice 4* :}On pose $u_{0}=2,$ $u_{1}=4$ et $\forall n\epsilon 
\mathbb{N},u_{n+2}=\dfrac{u_{n+1}^{4}}{u_{n}^{3}}.$

1. Montrer que $(u_{n})$ est bien d\'{e}finie et v\'{e}rifie : $\forall
n\epsilon \mathbb{N},$ \ $u_{n}>0.$

2. D\'{e}montrer, pour tout entier naturel $n$, l'\'{e}galit\'{e} $%
u_{n}=\left( \sqrt{2}\right) ^{3^{n}+1}.$

\underline{Exercice 5* :} D\'{e}montrer que pour tout entier $n\geq 1,$ $\
3^{2n+2}-2^{n+1}$ est divisible par 7.

\underline{Exercice 6 :}D\'{e}montrer par r\'{e}currence que :

pour tout entier naturel $n>0,$ \ $\ 1.2.3+2.3.4+3.4.5+...+n(n+1)(n+2)=\frac{%
n(n+1)(n+2)(n+3)}{4}.$

\underline{Exercice 7 :}Calculer les sommes suivantes :

1. $S_{n}=\overset{n-1}{\underset{k=1}{\sum }}(5.3^{-k}-4.2^{2k-1})$ \ \ \ \
\ \ \ 2. $A_{n}=1+\left( \frac{1}{4}\right) ^{2}+\left( \frac{1}{4}\right)
^{4}+...+\left( \frac{1}{4}\right) ^{2n}$ \ \ \ \ \ \ \ \ \ 3. \ $B_{n}=%
\overset{n}{\underset{k=1}{\sum }}\frac{n^{2}.k+n.k^{2}-3k^{3}}{n^{4}}$ \ \
\ \ \ \ \ \ \ \ \ \ \ \ \ \ \ \ \ \ \ 

\ 4*. $I_{n}=\overset{n}{\underset{k=0}{\sum }}(\frac{k(k-1)3^{k}-5^{k-1}}{%
3^{k}})$ \ \ \ \ \ \ \ \ \ 5*. $J_{n}=\overset{2n-1}{\underset{k=0}{\sum }}%
\frac{5}{6^{k-1}}.$

\bigskip

\underline{Exercice 8 :}1) D\'{e}terminer deux r\'{e}els $a$ et $b$ tq $%
\frac{1}{n(n+1)}=\frac{a}{n}+\frac{b}{n+1}.$

2) En d\'{e}duire la valeur de la somme $S_{n}=\overset{n}{\underset{k=1}{%
\sum }}\frac{3}{k(k+1)}.$

\underline{Exercice 9* :}1) D\'{e}terminer trois r\'{e}els $a,b$ et $c$ tq $%
\frac{1}{n(n^{2}-1)}=\frac{a}{n}+\frac{b}{n-1}+\frac{c}{n+1}$

2) En d\'{e}duire la valeur de la somme $S_{n}=\overset{n}{\underset{k=3}{%
\sum }}\frac{1}{k(k^{2}-1)}.$

\underline{Exercice 10 :}

1.D\'{e}montrer par r\'{e}currence que pour tout entier naturel $n\in 
%TCIMACRO{\U{2115} }%
%BeginExpansion
\mathbb{N}
%EndExpansion
^{\ast },$ $\overset{n}{\underset{k=1}{\sum }}k\times k!=\left( n+1\right)
!-1.$

2. En utilisant la m\'{e}thode du t\'{e}lescopage, retrouver la valeur de la
somme $\overset{n}{\underset{k=1}{\sum }}k\times k!.$

\underline{Exercice 11* :}

1. D\'{e}montrer par r\'{e}currence que pour tout entier naturel $n\in 
%TCIMACRO{\U{2115} }%
%BeginExpansion
\mathbb{N}
%EndExpansion
^{\ast },$ $\overset{n}{\underset{k=1}{\sum }}\dfrac{k}{\left( k+1\right) !}%
=1-\dfrac{1}{\left( n+1\right) !}.$

2. Red\'{e}montrer ce r\'{e}sultat en utilisant la m\'{e}thode du t\'{e}%
lescopage.

\underline{Exercice 12 :}Soit $q\neq 1.$

1. Calculer $\left( 1-q\right) \overset{n}{\underset{k=0}{\sum }}q^{k}.$ En d%
\'{e}duire $\overset{n}{\underset{k=0}{\sum }}q^{k}.$

2. Application : calculer \ les sommes suivantes : \ \ \ \ - $\overset{2n}{%
\underset{k=1}{\sum }}q^{k}$ \ \ \ \ \ \ \ - $\overset{n}{\underset{k=0}{%
\sum }}q^{2k+1}$

\underline{Exercice 13* :}Soit $n\in 
%TCIMACRO{\U{2115} }%
%BeginExpansion
\mathbb{N}
%EndExpansion
.$

1. En utilisant l'\'{e}galit\'{e} $\overset{n+1}{\underset{k=1}{\sum }}%
k^{2}= $ $\overset{n+1}{\underset{k=1}{\sum }}\left( (k-1)+1\right) ^{2}$,
et en d\'{e}veloppant le second membre, retrouvez la valeur de la somme $%
S_{1}=\overset{n}{\underset{k=1}{\sum }}k.$

2. Utiliser une m\'{e}thode analogue pour retrouver la valeur de la somme $%
S_{2}=\overset{n}{\underset{k=1}{\sum }}k^{2}.$

\underline{Exercice 14 :} D\'{e}montrer par r\'{e}currence que pour tout
entier naturel $n\in 
%TCIMACRO{\U{2115} }%
%BeginExpansion
\mathbb{N}
%EndExpansion
^{\ast },$ $\overset{n}{\underset{k=1}{\dprod }}(n+k)=2^{n}\overset{n}{%
\underset{k=1}{\dprod }}\left( 2k-1\right) .$

\underline{Exercice 15* :}

Calculer les sommes et les produits qui suivent : 1. 
\begin{tabular}[t]{llll}
$\overset{n}{\underset{k=1}{\sum }}\ln k$ & 2. $\overset{n}{\underset{k=1}{%
\dprod }}e^{k}$ & 3. $\overset{2p}{\underset{k=p}{\sum }}k^{2}$ & 4. $%
\overset{n}{\underset{k=1}{\dprod }}ke^{-2k}$%
\end{tabular}%
.

\underline{Exercice 16* :}Pour tout entier n, on pose : $u_{n}=\overset{n}{%
\underset{k=0}{\sum }}\left( -1\right) ^{n-k}k^{2}.$

1. Calculer $u_{n}+u_{n+1}$ pour tout entier naturel $n.$

2. D\'{e}montrer que \ $\forall n\in 
%TCIMACRO{\U{2115} }%
%BeginExpansion
\mathbb{N}
%EndExpansion
,$ $u_{n}=\dfrac{n\left( n+1\right) }{2}.$

\underline{Exercice 17* :}Soient $\left( x_{i}\right) _{i\in 
%TCIMACRO{\U{2115} }%
%BeginExpansion
\mathbb{N}
%EndExpansion
}$ et $\left( f_{i}\right) _{i\in 
%TCIMACRO{\U{2115} }%
%BeginExpansion
\mathbb{N}
%EndExpansion
}$ deux suites de r\'{e}els. On suppose que tous les $f_{i}$ sont
strictement positifs et on pose :

$\forall n\in 
%TCIMACRO{\U{2115} }%
%BeginExpansion
\mathbb{N}
%EndExpansion
,\forall x\in 
%TCIMACRO{\U{211d} }%
%BeginExpansion
\mathbb{R}
%EndExpansion
,$ $v_{n}(x)=\dfrac{\overset{n}{\underset{i=1}{\sum }}f_{i}(x_{i}-x)^{2}}{%
\overset{n}{\underset{i=1}{\sum }}f_{i}}.$

1. Pour tout entier $n\geq 1,$ d\'{e}terminer le r\'{e}el $m_{n}$ tel que la
fonction $v_{n}$ admet un minimum absolu sur $%
%TCIMACRO{\U{211d} }%
%BeginExpansion
\mathbb{R}
%EndExpansion
.$

2. On pose alors : $V_{m}=v_{n}(m_{n}).$ Montrer que $%
V_{n}=v_{n}(0)-m_{n}^{2}.$

\underline{Exercice 18*: In\'{e}galit\'{e} de Cauchy-Schwarz.}Le but de
l'exercice est de d\'{e}montrer pour tout entier naturel n non nul, et pour
tout n-uplets r\'{e}els $(a_{1},...,a_{n})\in 
%TCIMACRO{\U{211d} }%
%BeginExpansion
\mathbb{R}
%EndExpansion
^{n},(b_{1},...,b_{n})\in 
%TCIMACRO{\U{211d} }%
%BeginExpansion
\mathbb{R}
%EndExpansion
^{n}$ l'in\'{e}galit\'{e} : $\left( \overset{n}{\underset{k=1}{\sum }}%
a_{i}b_{i}\right) ^{2}\leq \left( \overset{n}{\underset{k=1}{\sum }}%
a_{i}^{2}\right) \left( \overset{n}{\underset{k=1}{\sum }}b_{i}^{2}\right) .$

On pose $A=\overset{n}{\underset{k=1}{\sum }}a_{i}^{2},$ $B=\overset{n}{%
\underset{k=1}{\sum }}b_{i}^{2}$ et $C=\overset{n}{\underset{k=1}{\sum }}%
a_{i}b_{i}.$

1. Exprimer $P(x)=\overset{n}{\underset{k=1}{\sum }}\left(
a_{i}x+b_{i}\right) ^{2}$ en fonction de $A,B,C.$

2. En d\'{e}duire que $C^{2}\leq AB.$

3. Si $a_{1},a_{2},...,a_{n}$ sont des nombres r\'{e}els strictement
positifs, montrer que $n^{2}\leq \left( \overset{n}{\underset{k=1}{\sum }}%
a_{i}\right) \left( \overset{n}{\underset{k=1}{\sum }}\dfrac{1}{a_{i}}%
\right) .$

\underline{Exercice 19 :}Calculer : \ a. $\sum\limits_{k=1}^{n}\dfrac{1}{%
k(k+1)(k+2)}.$ \ \ \ \ \ \ b. $\overset{n}{\underset{k=1}{\dprod }}%
ke^{-k^{2}}$ c. $J_{n}=\overset{n}{\underset{k=1}{\sum }}\left( \frac{%
k3^{k}+2^{k}}{3^{k-1}}\right) $

\underline{Exercice 20 :} D\'{e}montrer que $\forall n\in 
%TCIMACRO{\U{2115} }%
%BeginExpansion
\mathbb{N}
%EndExpansion
,$ \ $\overset{n}{\underset{k=0}{\dprod }}\left( 1+\dfrac{1}{2k+1}\right) >%
\sqrt{2n+3}.$

\bigskip

\bigskip

\bigskip

\qquad

\bigskip

.

\end{document}
