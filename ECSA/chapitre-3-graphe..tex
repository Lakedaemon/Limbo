%2multibyte Version: 5.50.0.2953 CodePage: 65001

\documentclass{article}
%%%%%%%%%%%%%%%%%%%%%%%%%%%%%%%%%%%%%%%%%%%%%%%%%%%%%%%%%%%%%%%%%%%%%%%%%%%%%%%%%%%%%%%%%%%%%%%%%%%%%%%%%%%%%%%%%%%%%%%%%%%%%%%%%%%%%%%%%%%%%%%%%%%%%%%%%%%%%%%%%%%%%%%%%%%%%%%%%%%%%%%%%%%%%%%%%%%%%%%%%%%%%%%%%%%%%%%%%%%%%%%%%%%%%%%%%%%%%%%%%%%%%%%%%%%%
\usepackage{amssymb}

%TCIDATA{OutputFilter=LATEX.DLL}
%TCIDATA{Version=5.50.0.2953}
%TCIDATA{Codepage=65001}
%TCIDATA{<META NAME="SaveForMode" CONTENT="1">}
%TCIDATA{BibliographyScheme=Manual}
%TCIDATA{Created=Wednesday, November 20, 2013 15:35:53}
%TCIDATA{LastRevised=Wednesday, June 29, 2016 08:57:19}
%TCIDATA{<META NAME="GraphicsSave" CONTENT="32">}
%TCIDATA{<META NAME="DocumentShell" CONTENT="Standard LaTeX\Blank - Standard LaTeX Article">}
%TCIDATA{CSTFile=40 LaTeX article.cst}
%TCIDATA{PageSetup=28,28,28,28,0}

\newtheorem{theorem}{Theorem}
\newtheorem{acknowledgement}[theorem]{Acknowledgement}
\newtheorem{algorithm}[theorem]{Algorithm}
\newtheorem{axiom}[theorem]{Axiom}
\newtheorem{case}[theorem]{Case}
\newtheorem{claim}[theorem]{Claim}
\newtheorem{conclusion}[theorem]{Conclusion}
\newtheorem{condition}[theorem]{Condition}
\newtheorem{conjecture}[theorem]{Conjecture}
\newtheorem{corollary}[theorem]{Corollary}
\newtheorem{criterion}[theorem]{Criterion}
\newtheorem{definition}[theorem]{Definition}
\newtheorem{example}[theorem]{Example}
\newtheorem{exercise}[theorem]{Exercise}
\newtheorem{lemma}[theorem]{Lemma}
\newtheorem{notation}[theorem]{Notation}
\newtheorem{problem}[theorem]{Problem}
\newtheorem{proposition}[theorem]{Proposition}
\newtheorem{remark}[theorem]{Remark}
\newtheorem{solution}[theorem]{Solution}
\newtheorem{summary}[theorem]{Summary}
\newenvironment{proof}[1][Proof]{\noindent\textbf{#1.} }{\ \rule{0.5em}{0.5em}}
\input{tcilatex}
\begin{document}


\subsection{ \ \ \ \ \ \ \ \ \ \ \ \ \ \ \ \ \ \ \ \ \ \ \ \ \ \ \ \ \ 
\protect\underline{Chapitre 3 : graphe en dimension deux.}}

\paragraph{\protect\underline{1. Instruction plot et plot2d.}}

Pour tracer la courbe repr\'{e}sentative d'une fonction $f$ de une variable,
Scilab trace une ligne bris\'{e}e entre les points $%
(x_{1},f(x_{1})),...,(x_{2},f(x_{2})).$

Il faut donc d'une mani\`{e}re ou d'une autre fournir :

- un vecteur colonne $x=[x_{1},...,x_{n}]^{\prime }$ \ ( ou un vecteur ligne 
$x=[x_{1},...,x_{n}]$ )

- un vecteur colonne $y=[f(x_{1}),...,f(x_{n})]^{\prime }$ ( ou un vecteur
ligne $y=[f(x_{1}),...,f(x_{n})]$ )

Remarque :

Pour repr\'{e}senter une courbe, il est plus simple d'utiliser des vecteurs
lignes mais dans le cas o\`{u} l'on doit repr\'{e}senter plusieurs courbes
il est pr\'{e}f\'{e}rable de travailler avec des vecteurs colonnes.

-\underline{ Pour d\'{e}clarer le vecteur $x$ des abscisses} , on a deux
possibilit\'{e}s ( voir chapitre 2, paragraphe 3).

Par exemple pour d\'{e}couper le segment [0,1] en 100 sous-segments, on
dispose des solutions suivantes :

$x=[0:0.01:1]^{\prime }$

$x=linspace(0,1,101)^{\prime }.$

-\underline{ Pour d\'{e}clarer le vecteur $y$ des ordonn\'{e}es} , on
construit ce vecteur en utilisant les op\'{e}rations coefficients par
coefficients ( voir chapitre 2, paragraphe 4.2 )

- \underline{Pour tracer le graphe}, on utilise ensuite les fonctions plot
ou plot2d.

L'instruction $plot(x,y)$ ou $plot2d(x,y)$ trace donc une courbe en reliant
les points $(x_{k},y_{k})$ entre eux.

Exemple : Tracer la courbe de la fonction $f:x\mapsto x^{2}$ sur $\left[ 0;1%
\right] $ avec un pas de $0,1.$

$-\rightarrow x=[0:0.1:1]^{\prime };$

$-\rightarrow y=x\cdot \symbol{94}2;$

$-\rightarrow plot2d(x,y)$

Exercice 1 : Tracer la courbe de la fonction $f:x\mapsto \cos (x)$ sur $%
\left[ 0;2\pi \right] $ avec un pas de $0,01.$

Exercice 2 : Tracer la courbe de la fonction $f:x\mapsto \dfrac{1}{1+x^{2}}$
sur $\left[ -3,3\right] $ pour 11 valeurs.

\paragraph{\protect\underline{2. Trac\'{e}s de plusieurs courbes sur un m%
\^{e}me graphique.}}

Par d\'{e}faut, les diff\'{e}rents trac\'{e}s viennent se superposer les uns
par dessus les autres.

Pour n'avoir qu'un seul trac\'{e}, il faut effacer le pr\'{e}c\'{e}dent avec
la commande : $clf.$

A chaque appel de $plot2d$ les valeurs maximales et minimales sont \'{e}valu%
\'{e}es pour offrir un affichage qui permet de voir la courbe en entier.

Dans le cas de plusieurs appels successifs de la commande plot2d, les
courbes sont superpos\'{e}es mais de plus, en cas de changement de valeurs
extr\^{e}mes, les courbes sont r\'{e}-affich\'{e}es pour permettre une
visualisation simultan\'{e}e.

Exercice 3 :

Tracer la fonction $f:x\mapsto \QDOVERD( ) {x}{1+x^{3}}$ sur $\left[ 0;1%
\right] $ et $g:x\mapsto \dfrac{x}{\exp (x)}$ sur $\left[ -1;2\right] .$
(avec un pas de 0.01)

Mais si on veut imposer les coordonn\'{e}es de la fen\^{e}tre d'affichage,
on utilise l'instruction : $rect=[x_{\min },y_{\min },x_{\max },y_{\max }],$
qu'on place dans l'instruction plot2d : l'appel est fait par $%
plot2d(x,y,rect=...).$

Exercice 4 :

Tracer sur une m\^{e}me figure la courbe de $f:x\mapsto \dfrac{x^{2}}{2x-1}$
et la courbe de l'asymptote d'\'{e}quation $y=\frac{1}{2}x+\frac{1}{4}$ sur $%
\left[ 1;4\right] .$(avec un pas de 0.01)

On d\'{e}finit pour fen\^{e}tre d'affichage : $x$ varie de 1 \`{a} 4 et $y$
de 0 \`{a} 3.

\bigskip

L'instruction $plot2d$ permet aussi d'afficher simultan\'{e}ment plusieurs
courbes , pour cela il suffit d'utiliser en argument des matrices \`{a} la
place des vecteurs $x$ et $y.$

L'appel est fait par $plot2d(xx,yy)$ o\`{u} $xx$et $yy$ sont deux matrices
de m\^{e}me dimension $m\times n$ et revient \`{a} superposer sur un m\^{e}%
me graphique $n$ courbes repr\'{e}sentant chacune des $n$ colonnes de $yy$
en fonction de la colonne correspondante de $xx.$

On a deux m\'{e}thodes possibles :

- si le vecteur colonne d'abscisses est commun \`{a} l'ensemble des ordonn%
\'{e}es : $plot2d(x,\left[ f_{1}(x),f_{2}(x),...\right] )$

- s'il y a autant de vecteurs colonnes d'abscisses et d'ordonn\'{e}es : $%
plot2d(\left[ x_{1},x_{2},...\right] ,\left[ f_{1}(x_{1}),f_{2}(x_{2}),...%
\right] ).$

Exemple :

Taper les instructions :

$-\rightarrow x=linspace(0,\%pi,101)^{\prime };$

$-\rightarrow plot2d(x,[\sin (x),\cos (x),\sin (x).\ast \cos (x)])$

Exercice 5 :

1. Tracer les courbes des fonctions $x\mapsto \sin (x),x\mapsto \sin (2x)$ $%
\ et$ $x\mapsto \sin (3x)$ sur $\left[ 0;2\pi \right] .$

2. Tracer les courbes des fonctions $x\mapsto \sin (x),x\mapsto \sin (2x)$ $%
\ et$ $x\mapsto \sin (3x)$ respectivement sur $\left[ 0;2\pi \right] ,\left[
2\pi ;4\pi \right] $ et $\left[ 4\pi ,6\pi \right] .$

\bigskip

\paragraph{\protect\underline{3. Choix de la couleur et du trait.}}

Pour obtenir une couleur ou un mode de trac\'{e} diff\'{e}rent, on utilise
l'instruction $style=\left( vecteur\text{ }ligne\right) :$ l'appel est fait
par $plot2d(\left[ x1,x2\right] ,\left[ y_{1},y_{2}\right] ,style=\left[
n_{1},n_{2}\right] ).$

La taille du vecteur ligne correspond au nombre de vecteurs des ordonn\'{e}%
es.

Ce tableau doit imp\'{e}rativement \^{e}tre en troisi\`{e}me position et on
peut omettre d'\'{e}crire $style$.

Les valeurs num\'{e}riques de ce vecteur ligne correspondent \`{a} une
couleur ou \`{a} un symbole.

Plus pr\'{e}cis\'{e}ment une valeur positive correspond \`{a} une couleur et
une valeur n\'{e}gative ou nulle remplace l'affichage continu par un symbole.

$\ \ \ \ \ \ \ \ \ \ \ \ \ \ \ \ \ \ \ \ \ \ \ \ \ \ \ \ \ \ \ \ \ \ \ 
\begin{array}{ccccccc}
1 & 2 & 3 & 4 & 5 & 6 & 7 \\ 
noir & bleu & vert & cyan & rouge & magenta & jaune \\ 
0 & -1 & -2 & -3 & -4 & -5 & -6 \\ 
\cdot & + & \times & \oplus & \ast & \Diamond & \bigtriangleup%
\end{array}%
$

\underline{Exercice 6 :}

Tracer la courbe de la fonction $x\mapsto x^{2}$ en rouge et de $x\mapsto
\ln (1+2x)$ en vert sur $\left[ 0;1\right] $( pas de 0.01 )

\underline{Exercice 7 :} Tracer la courbe de la fonction $f$ d\'{e}finie sur 
$I$ avec un pas de $0,01:$

1. $f:x\mapsto x-\left\lfloor x\right\rfloor $ sur $I=\left[ -10;10\right] .$

2. $f:x\mapsto \left\vert \ln (x+2)-1\right\vert $ sur $I=\left[ -1;1\right]
.$

3. $f:x\mapsto \ln (\sqrt{x^{4}+2})$ sur $I=\left[ 0;4\right] .$

\bigskip

Lors du trac\'{e} de $y=f(x)$, si le vecteur $x$ est omis, alors les points
trac\'{e}s sont $(k,y_{k}).$

\underline{Exercice 8 :} Tracer les 21 premi\`{e}res valeurs de la suite $u$
de terme g\'{e}n\'{e}ral $u_{n}=(-2+0,2\times n)^{2}.$

\end{document}
