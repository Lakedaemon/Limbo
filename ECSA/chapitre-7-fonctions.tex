%2multibyte Version: 5.50.0.2953 CodePage: 65001

\documentclass{article}
%%%%%%%%%%%%%%%%%%%%%%%%%%%%%%%%%%%%%%%%%%%%%%%%%%%%%%%%%%%%%%%%%%%%%%%%%%%%%%%%%%%%%%%%%%%%%%%%%%%%%%%%%%%%%%%%%%%%%%%%%%%%%%%%%%%%%%%%%%%%%%%%%%%%%%%%%%%%%%%%%%%%%%%%%%%%%%%%%%%%%%%%%%%%%%%%%%%%%%%%%%%%%%%%%%%%%%%%%%%%%%%%%%%%%%%%%%%%%%%%%%%%%%%%%%%%
\usepackage{amsfonts}
\usepackage{amsmath}

\setcounter{MaxMatrixCols}{10}
%TCIDATA{OutputFilter=LATEX.DLL}
%TCIDATA{Version=5.50.0.2953}
%TCIDATA{Codepage=65001}
%TCIDATA{<META NAME="SaveForMode" CONTENT="1">}
%TCIDATA{BibliographyScheme=Manual}
%TCIDATA{Created=Tuesday, February 25, 2014 14:29:33}
%TCIDATA{LastRevised=Wednesday, June 29, 2016 08:59:27}
%TCIDATA{<META NAME="GraphicsSave" CONTENT="32">}
%TCIDATA{<META NAME="DocumentShell" CONTENT="Standard LaTeX\Blank - Standard LaTeX Article">}
%TCIDATA{CSTFile=40 LaTeX article.cst}
%TCIDATA{PageSetup=28,28,28,28,0}

\newtheorem{theorem}{Theorem}
\newtheorem{acknowledgement}[theorem]{Acknowledgement}
\newtheorem{algorithm}[theorem]{Algorithm}
\newtheorem{axiom}[theorem]{Axiom}
\newtheorem{case}[theorem]{Case}
\newtheorem{claim}[theorem]{Claim}
\newtheorem{conclusion}[theorem]{Conclusion}
\newtheorem{condition}[theorem]{Condition}
\newtheorem{conjecture}[theorem]{Conjecture}
\newtheorem{corollary}[theorem]{Corollary}
\newtheorem{criterion}[theorem]{Criterion}
\newtheorem{definition}[theorem]{Definition}
\newtheorem{example}[theorem]{Example}
\newtheorem{exercise}[theorem]{Exercise}
\newtheorem{lemma}[theorem]{Lemma}
\newtheorem{notation}[theorem]{Notation}
\newtheorem{problem}[theorem]{Problem}
\newtheorem{proposition}[theorem]{Proposition}
\newtheorem{remark}[theorem]{Remark}
\newtheorem{solution}[theorem]{Solution}
\newtheorem{summary}[theorem]{Summary}
\newenvironment{proof}[1][Proof]{\noindent\textbf{#1.} }{\ \rule{0.5em}{0.5em}}
\input{tcilatex}
\begin{document}


\subsubsection{ \ \ \ \ \ \ \ \ \ \ \ \ \ \ \ \ \ \ \ \ \ \ \ \ \ \ \ \ \ \
\ \ \protect\underline{Chapitre 7 : les fonctions en Scilab.}}

\paragraph{\protect\underline{1. Les fonctions.}}

\underline{1.1. Fonctions renvoyant une seule valeur.}

Scilab dispose d'un langage qui permet de d\'{e}finir de nouvelles commandes
gr\^{a}ce \`{a} l'\'{e}criture des fonctions.

Une fonction est une suite d'instructions qui, apr\`{e}s ex\'{e}cution ,
rend une valeur.

La d\'{e}claration d'une fonction renvoyant une seule valeur se pr\'{e}sente
de la mani\`{e}re suivante :

\textit{function }$y$\textit{=nom-de-la fonction(}$x_{1},x_{2},...,x_{m})$

\textit{\ \ Bloc d'instructions}

\textit{endfunction}

Toutes les variables $y$, $x_{1},...,x_{m}$ utilis\'{e}es \`{a} l'int\'{e}%
rieur de la fonction sont par d\'{e}faut locales. Ces variables sont des
variables muettes et leurs noms peuvent \^{e}tre r\'{e}utilis\'{e}s dans la d%
\'{e}finition d'autres fonctions ou dans Scilab.

L'identificateur $y$ joue un r\^{o}le particulier .Lors de l'appel d'une
fonction , le bloc d'instructions est ex\'{e}cut\'{e} et, \`{a} la fin,on
affecte \`{a} la variable $y$ une valeur .C'est cette valeur qui sera rendue
par la fonction, apr\`{e}s avoir affect\'{e} aux variables $x_{1},...,x_{m}$
des expressions.

\bigskip

Exemple 1 :

Pour d\'{e}finir la fonction $f:f(x)=\dfrac{3}{8+\exp (-x)},$ nous d\'{e}%
finissons une fonction Scilab dans la console dont le nom est $f$, avec un
argument

$x$ et une valeur de retour $y$ ( l'image $f(x)$ ).

$-\longrightarrow function$ $y=f(x)$

$-\longrightarrow y=3/(8+\exp (-x));$

$-\longrightarrow endfunction;$

Pour appeler la fonction et calculer une valeur, il suffit par exemple de
taper :

$-\longrightarrow f(0)$

$ans=$

$0.3333333$

\fbox{ou }

d'utiliser la commande $f\func{eval}:$

$-\longrightarrow f\func{eval}(0,f)$

$ans=$

$0.3333333$

Remarque sur la fonction $f\func{eval}:$

l'instruction $y=f\func{eval}(x,f)$ renvoie le vecteur $y$ d\'{e}finie par $%
y(i)=f(x(i)).$

\underline{1.2. Fonctions retournant plusieurs valeurs.}

R\'{e}daction :

\textit{function }$[y_{1},...,y_{p}]$\textit{=nom-de-la fonction(}$%
x_{1},x_{2},...,x_{m})$

\textit{\ \ Bloc d'instructions}

\textit{endfunction}

Pour r\'{e}cup\'{e}rer toutes les valeurs, on utilise les crochets " $\left[
{}\right] $ ". Sinon on ne r\'{e}cup\`{e}re qu'une seule valeur, c'est la
premi\`{e}re qui est retourn\'{e}e. ( la plus \`{a} gauche )

Exemple 2 :

$-\longrightarrow function$ $\left[ m,M\right] =\min \max (x,y)$

$-\longrightarrow m=\min (x,y);M=\max (x,y);$

$-\longrightarrow endfunction;$

Pour appeler la fonction, on utilise l'instruction :\ 

$\left[ m,M\right] =\min \max (x,y).$

ou

$feval([x,y],minmax)$;

\underline{1.3. Charger une fonction.}

Pour que la fonction soit reconnue et ex\'{e}cut\'{e}e par Scilab, il faut
la "charger". En effet les fonctions sous Scilab ont un statut de variables
et donc n\'{e}cessitent d'\^{e}tre initialis\'{e}es.

M\^{e}me s'il est possible de d\'{e}finir les fonctions dans la console
comme dans les deux exemples pr\'{e}c\'{e}dents, il vaut toujours mieux les 
\'{e}crire dans Scinotes , les sauver dans un fichier et les charger ensuite
dans Scilab.

Il sera ainsi beaucoup plus facile de r\'{e}aliser des corrections ou des
modifications.

\underline{1.4. Fonction \ " $deff$ ".}

La fonction $"deff"$ permet de construire simplement des fonctions math\'{e}%
matiques. Elle prend en argument deux cha\^{\i}nes de caract\`{e}res la premi%
\`{e}re correspondant \`{a} l'en-t\^{e}te de la fonction d\'{e}finie et la
seconde correspondant \`{a} sa " d\'{e}finition math\'{e}matique ".

Exemple 3 :

Soit $f:x\rightarrow \exp (x)+\ln (x+1).$

$-\rightarrow deff(^{\prime }y=f(x)^{\prime },^{\prime }y=\exp (x)+\log
(x+1)^{\prime });$

\bigskip

\subsubsection{\protect\underline{2. Exercices .}}

\bigskip

\paragraph{\protect\underline{Exercice 1 :}}

1. Cr\'{e}er une fonction factorielle ( nomm\'{e}e $fact$ ) qui prend en
argument $n$ et qui renvoie la valeur $n!.$

2. Cr\'{e}er ensuite un programme qui demande deux entiers $n$ et $p$ et qui
affiche la valeur $\dbinom{n}{p}.$

\bigskip

\paragraph{\protect\underline{Exercice 2* :}}

On consid\`{e}re la suite $u$ d\'{e}finie par : $\left\{ 
\begin{array}{c}
u_{0}=1 \\ 
u_{n+1}=\frac{1}{2}\ln (1+2u_{n})%
\end{array}%
\right. .$

1. Cr\'{e}er une fonction $h:x\mapsto \frac{1}{2}\ln (1+2x)$ sous scilab
afin de calculer tout terme de rang $n$ de la suite $u$, $n$ \'{e}tant saisi
par l'utilisateur.

2. Repr\'{e}senter graphiquement les 20 premiers termes de la suite.
Conjecturer sur le comportement de cette suite.

3. \ D\'{e}terminer le rang $N$ \`{a} partir duquel $\left\vert
u_{N}\right\vert \leq 10^{-4}.$

\paragraph{\protect\underline{Exercice 3 :}}

Crit\`{e}re des s\'{e}ries altern\'{e}es.

Si $u$ est une suite d\'{e}croissante et convergente de limite nulle, alors
la s\'{e}rie $S=\sum \left( -1\right) ^{n}u_{n}$ converge.

De plus , si on note , pour tout entier $p\in \mathbb{N}$ $R_{p}=\underset{%
n=p+1}{\overset{+\infty }{\sum }}\left( -1\right) ^{n}u_{n}$ le reste
d'ordre $p$ de la s\'{e}rie $S,$ on a l'encadrement : $\left\vert
R_{p}\right\vert \leq u_{p+1}.$

1. Justifier la convergence de la s\'{e}rie de terme g\'{e}n\'{e}ral : $%
v_{n}=\left( -1\right) ^{n}\ln (1+\exp (-n)).$

2. Ecrire un programme qui donne une approximation \`{a} $10^{-6}$ pr\`{e}s
de la somme de cette s\'{e}rie. La fonction $k:x\mapsto \ln (1+\exp (-x))$
sera d\'{e}finie en t\^{e}te de programme.

\bigskip

\paragraph{\protect\underline{Exercice 4 : Tracer des graphes en Scilab.}}

Objectif : Repr\'{e}senter la courbe de la fonction $g:x\mapsto x\exp (-x)$
sur $\left[ -1;5\right] $ ( avec un pas de 0.01 ).

Il faut dans un premier temps d\'{e}clarer le vecteur ligne ou colonne $x$
des abscisses$:x=\left[ -1:0.01:5\right] ;$

Nous allons maintenant voir deux m\'{e}thodes pour d\'{e}clarer le vecteur $%
y $ des ordonn\'{e}es .

1. \underline{En utilisant les op\'{e}rations point\'{e}es sur les vecteurs,}
construire le vecteur $y$ des ordonn\'{e}es puis repr\'{e}senter la courbe
de $g.$

2. \underline{En utilisant la d\'{e}finition d'une fonction r\'{e}elle sous
Scilab et la fonction $f\func{eval}.$}

Cr\'{e}er la fonction r\'{e}elle $g$ sous $S$cilab, d\'{e}finir le vecteur $%
y $ des ordonn\'{e}es en utilisant $f\func{eval}$ puis repr\'{e}senter la
courbe de $g.$

Remarque : une derni\`{e}re m\'{e}thode serait d'utiliser les op\'{e}rations
point\'{e}es \`{a} l'int\'{e}rieur de la fonction et de d\'{e}finir le
vecteur $y$ des ordonn\'{e}es par l'instruction : $y=g(x).$

\paragraph{\protect\underline{Exercice 5 :}}

On consid\`{e}re la fonction num\'{e}rique $f$ d\'{e}finie sur $\left[ 1,2%
\right] $ par:

\ \ \ \ \ \ \ \ \ \ \ \ \ \ \ \ \ \ \ \ $\ f(x)=2(1-e^{-x})-x.$

1.Etudier les variations de $f$ sur $\left[ 1,2\right] .$

Repr\'{e}senter la courbe de la fonction $f$ sur $\left[ 1;2\right] $ ( avec
un pas de 0.01 ).

2.Montrer que l'\'{e}quation $f(x)=0$, admet une unique solution dans $\left[
1,2\right] $ , not\'{e}e $L$.

On d\'{e}finit deux suites num\'{e}riques $(a_{n})_{n\geq 0}$ et $%
(b_{n})_{n\geq 0}$ par :$a_{0}=1$,$b_{0}=2$

et pour tout entier naturel n, si \ $f(a_{n})f(\frac{a_{n}+b_{n}}{2})>0,$
alors, $a_{n+1}=\frac{a_{n}+b_{n}}{2}$ \ \ \ \ \ \ \ \ \ \ \ \ \ \ \ \ \ \ \
sinon, $a_{n+1}=a_{n}$

\ \ \ \ \ \ \ \ \ \ \ \ \ \ \ \ \ \ \ \ \ \ \ \ \ \ \ \ \ \ \ \ \ \ \ \ \ \
\ \ \ \ \ \ \ \ \ \ \ \ \ \ \ \ \ \ \ \ \ \ \ \ \ \ \ \ \ \ \ \ \ \ \ \ \ \
\ \ \ \ \ \ \ \ \ \ \ \ \ $b_{n+1}=b_{n}$ \ \ \ \ \ \ \ \ \ \ \ \ \ \ \ \ \
\ \ \ \ \ \ \ \ \ $b_{n+1}=\frac{a_{n}+b_{n}}{2}$\ \ \ \ \ \ \ \ \ \ \ \ \ \
\ \ \ \ \ \ \ \ \ \ \ \ \ \ \ \ \ \ \ \ \ \ \ \ \ \ \ \ \ \ \ \ \ \ \ \ \ \
\ \ \ \ \ \ \ \ \ \ \ \ \ \ \ \ \ \ \ \ \ \ \ \ \ \ \ \ \ \ \ \ \ \ \ \ \ \
\ \ \ \ \ \ \ \ \ \ \ \ 

3. On donne les valeurs de $f$ suivantes : $f(\frac{3}{2})\simeq 0,05$ et $f(%
\frac{7}{4})\simeq -0,1.$ \ \ D\'{e}terminer les valeurs de $a_{1}$ et $%
b_{1} $ puis de $a_{2}$ et $b_{2}.$

4. Montrer , par r\'{e}currence que, pour tout entier n : $a_{n}\leq L\leq
b_{n}.$

5. Montrer que, pour tout entier $n$, $b_{n}-a_{n}=(\frac{1}{2})^{n}$ et 
\'{e}tudier le sens de variation des suites $(a_{n})_{n\geq 0}$ et $%
(b_{n})_{n\geq 0}$.

6. En d\'{e}duire la convergence de ces deux suites,et pr\'{e}ciser leur
limite.

7.\ Ecrire un programme en Scilab, donnant une valeur approch\'{e}e de $L$ 
\`{a} 10$^{-5}$ pr\`{e}s et indiquant le nombre d'it\'{e}rations n\'{e}%
cessaires. La fonction $f$ sera d\'{e}finie en t\^{e}te de programme.

\end{document}
