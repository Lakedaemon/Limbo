%2multibyte Version: 5.50.0.2953 CodePage: 65001

\documentclass{article}
%%%%%%%%%%%%%%%%%%%%%%%%%%%%%%%%%%%%%%%%%%%%%%%%%%%%%%%%%%%%%%%%%%%%%%%%%%%%%%%%%%%%%%%%%%%%%%%%%%%%%%%%%%%%%%%%%%%%%%%%%%%%%%%%%%%%%%%%%%%%%%%%%%%%%%%%%%%%%%%%%%%%%%%%%%%%%%%%%%%%%%%%%%%%%%%%%%%%%%%%%%%%%%%%%%%%%%%%%%%%%%%%%%%%%%%%%%%%%%%%%%%%%%%%%%%%
\usepackage{amsfonts}
\usepackage{amssymb}

%TCIDATA{OutputFilter=LATEX.DLL}
%TCIDATA{Version=5.50.0.2953}
%TCIDATA{Codepage=65001}
%TCIDATA{<META NAME="SaveForMode" CONTENT="1">}
%TCIDATA{BibliographyScheme=Manual}
%TCIDATA{Created=Wednesday, December 18, 2013 11:12:08}
%TCIDATA{LastRevised=Wednesday, June 29, 2016 08:57:39}
%TCIDATA{<META NAME="GraphicsSave" CONTENT="32">}
%TCIDATA{<META NAME="DocumentShell" CONTENT="Standard LaTeX\Blank - Standard LaTeX Article">}
%TCIDATA{CSTFile=40 LaTeX article.cst}
%TCIDATA{PageSetup=28,28,28,28,0}

\newtheorem{theorem}{Theorem}
\newtheorem{acknowledgement}[theorem]{Acknowledgement}
\newtheorem{algorithm}[theorem]{Algorithm}
\newtheorem{axiom}[theorem]{Axiom}
\newtheorem{case}[theorem]{Case}
\newtheorem{claim}[theorem]{Claim}
\newtheorem{conclusion}[theorem]{Conclusion}
\newtheorem{condition}[theorem]{Condition}
\newtheorem{conjecture}[theorem]{Conjecture}
\newtheorem{corollary}[theorem]{Corollary}
\newtheorem{criterion}[theorem]{Criterion}
\newtheorem{definition}[theorem]{Definition}
\newtheorem{example}[theorem]{Example}
\newtheorem{exercise}[theorem]{Exercise}
\newtheorem{lemma}[theorem]{Lemma}
\newtheorem{notation}[theorem]{Notation}
\newtheorem{problem}[theorem]{Problem}
\newtheorem{proposition}[theorem]{Proposition}
\newtheorem{remark}[theorem]{Remark}
\newtheorem{solution}[theorem]{Solution}
\newtheorem{summary}[theorem]{Summary}
\newenvironment{proof}[1][Proof]{\noindent\textbf{#1.} }{\ \rule{0.5em}{0.5em}}
\input{tcilatex}
\begin{document}


\subsubsection{ \ \ \ \ \ \ \ \ \ \ \ \ \ \ \ \ \ \ \ \ \ \ \ \ \ \ \ \ \ \
\ \ \ \ \ \protect\underline{Chapitre 4 : l'instruction r\'{e}p\'{e}titive $%
for.$}}

\paragraph{\protect\underline{1. Utilisation de l'\'{e}diteur Scinotes.}}

D\`{e}s que l'on a besoin d'\'{e}crire des programmes un peu d\'{e}velopp%
\'{e}s, on utilisera la fen\^{e}tre d'\'{e}dition Scinotes, qui permet d'%
\'{e}crire, de sauvegarder, de modifier des programmes complets.

Pour ouvrir la fen\^{e}tre Scinotes, on clique sur l'ic\^{o}ne situ\'{e}e en
haut \`{a} gauche de l'\'{e}cran " d\'{e}marrer Scinotes" qui ouvre une
seconde fen\^{e}tre dans laquelle on pourra \'{e}crire le programme consid%
\'{e}r\'{e}.

On tape alors les commandes que l'on veut dans la fen\^{e}tre Scinotes, en
enregistrant r\'{e}guli\`{e}rement gr\^{a}ce au menu "fichier" puis
"enregistrer sous" pour le premier enregistrement, puis "enregistrer" pour
les suivants ( fichier enregistr\'{e} avec l'extension \ $\cdot sce).$

\bigskip

Afin de rendre un programme plus compr\'{e}hensible, il est judicieux de
placer des commentaires qui apparaissent \`{a} droite des caract\`{e}res " $%
//$ ".

\paragraph{ \protect\underline{2. R\'{e}daction de la boucle $for.$}}

L'instruction $for$ permet de faire une it\'{e}ration sur les composantes
d'un vecteur ligne donn\'{e}.

\underline{R\'{e}daction de cette instruction :}

Consid\`{e}rons le programme suivant :

$for$ \ $x=v;$ \ \ \ \ $//$ $v$ est un vecteur ligne

\ \ \ \ \ \ \ \ suite d'instructions portant ou non sur la variable $x;$

$end$

Dans la commande ci-contre , l'instruction sera ex\'{e}cut\'{e}e pour $x$
prenant successivement comme valeurs les composantes de $v$.

\bigskip

Exemple :

$For$ \ $i=1:5;$ \ \ $//$ \ $i$ prend successivement les valeurs 1,2,3,4 et
5.

\ \ \ \ \ \ $disp(i.\symbol{94}2)$ \ \ $//$ afficher $i^{2}$ \ pour \ $i$
variant de 1 \`{a} 5.

$end$

\bigskip

\paragraph{\protect\underline{3. Exercices sur l'instruction $for.$}}

\underline{Exercice 1 :}

Ecrire un programme qui calcule et affiche \ la somme des N premiers nombres
entiers pour un N non nul choisi par l'utilisateur.

\underline{Exercice 2 :}

Ecrire un programme qui calcule et affiche le factoriel \ $n!$ \ pour un $n$
choisi par l'utilisateur.

\underline{Exercice 3 :}

Ecrire un programme qui calcule et affiche $\sum_{k=1}^{n}\frac{1}{k!}$ \
pour un $n$ non nul choisi par l'utilisateur.

\underline{Exercice 4 :}

Soit la suite u d\'{e}finie par : $u_{n+1}=3u_{n}-2.$

Ecrire un programme qui calcule et affiche\ \ les termes successifs de cette
suite pour une valeur initiale $u_{1}$ et jusqu'\`{a} un rang n , ces deux
valeurs \'{e}tant entr\'{e}es par l'utilisateur et repr\'{e}senter les
termes successifs de la suite .

\underline{Exercice 5:}

On consid\'{e}re la suite de Fibonacci d\'{e}finie par: $\ \
u_{n}=u_{n-1}+u_{n-2}$ \ \ \ \ \ ,\ \ $u_{1}=1$ \ \ et\ \ \ \ $u_{2}=1.$

Ecrire un programme qui calcule et affiche la valeur de $u_{n}$ pour un n
choisi par l'utilisateur.

\underline{Exercice 6 :}

On consid\`{e}re les suites $\left( a_{n}\right) $ et $\left( b_{n}\right) $
d\'{e}finies par : $a_{1}=1$ , $b_{1}=1$ et $\forall n\in \mathbb{N}^{\ast
},a_{n+1}=\dfrac{a_{n}}{n+1}-\dfrac{b_{n}}{\left( n+1\right) ^{2}}$ et $%
b_{n+1}=\dfrac{b_{n}}{n+1}\smallskip $

Ecrire un programme qui calcule et affiche les $n$ premiers termes de
chacune des suites $\left( a_{n}\right) $ et $\left( b_{n}\right) $ pour une
valeur de $n$ entr\'{e}e par l'utilisateur et qui permet de repr\'{e}senter
sur un m\^{e}me graphique les termes successifs de ces suites .

\underline{Exercice 7 :}

On consid\`{e}re les suites d\'{e}finies par : $\left\{ 
\begin{array}{cc}
u_{1}=1 & v_{1}=2 \\ 
\forall n\in \mathbb{N}^{\ast }\text{ }v_{n+1}=\dfrac{u_{n}+v_{n}}{2} & 
u_{n+1}=\sqrt{u_{n}v_{n}}%
\end{array}%
\right. .$

1. Ecrire un programme qui calcule et affiche les $n$ premiers termes de
chacune des suites $\left( u_{n}\right) $ et $\left( v_{n}\right) $ pour une
valeur de $n$ entr\'{e}e par l'utilisateur et qui permet de repr\'{e}senter
sur un m\^{e}me graphique les termes successifs de ces suites .

2. En d\'{e}duire le comportement de ces suites ( monotonie, convergence,
limite...)

\bigskip

\bigskip \underline{Exercice 8 :}Soit la suite $(u_{n})_{n\in \mathbb{N}}$ d%
\'{e}finie par $u_{0}=1$ et $\forall n\in \mathbb{N}:u_{n+1}=u_{n}+\dfrac{1}{%
4}(2-u_{n}^{2})$. \newline
On pose $f:x\mapsto x+\dfrac{1}{4}(2-x^{2})$.\thinspace

1. Etudier les variations de $f$ et d\'{e}terminer ses points fixes.

Tracer sur une m\^{e}me figure la courbe de $f$ et la courbe de la droite d'%
\'{e}quation $y=x$ sur $\left[ 1;\sqrt{2}\right] .$

2. Ecrire un programme qui calcule et affiche les $n$ premiers termes de la
suite $\left( u_{n}\right) $ pour une valeur de $n$ entr\'{e}e par
l'utilisateur et qui permet de repr\'{e}senter sur un m\^{e}me graphique les
termes successifs de cette suite .

3. Etudier le comportement asymptotique de la suite $\left( u_{n}\right) $
et donner en cas de convergence la limite de $u.$

\underline{Exercice \ 9 :}

Ecrire un programme qui demande \`{a} l'utilisateur un entier naturel non
nul $n$ et qui permet d'afficher une matrice triangulaire sup\'{e}rieure de $%
M_{n}\left( 
%TCIMACRO{\U{211d} }%
%BeginExpansion
\mathbb{R}
%EndExpansion
\right) $ dont les coefficients diagonaux sont \'{e}gaux \`{a} 4 et tous les
coefficients au-dessus de la diagonale \'{e}gaux \`{a} -1.

\bigskip

\end{document}
