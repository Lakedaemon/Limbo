%2multibyte Version: 5.50.0.2953 CodePage: 65001

\documentclass{article}
%%%%%%%%%%%%%%%%%%%%%%%%%%%%%%%%%%%%%%%%%%%%%%%%%%%%%%%%%%%%%%%%%%%%%%%%%%%%%%%%%%%%%%%%%%%%%%%%%%%%%%%%%%%%%%%%%%%%%%%%%%%%%%%%%%%%%%%%%%%%%%%%%%%%%%%%%%%%%%%%%%%%%%%%%%%%%%%%%%%%%%%%%%%%%%%%%%%%%%%%%%%%%%%%%%%%%%%%%%%%%%%%%%%%%%%%%%%%%%%%%%%%%%%%%%%%
\usepackage{amsfonts}
\usepackage{amssymb}
\usepackage{graphicx}
\usepackage{amsmath}

\setcounter{MaxMatrixCols}{10}
%TCIDATA{OutputFilter=LATEX.DLL}
%TCIDATA{Version=5.50.0.2953}
%TCIDATA{Codepage=65001}
%TCIDATA{<META NAME="SaveForMode" CONTENT="1">}
%TCIDATA{BibliographyScheme=Manual}
%TCIDATA{Created=Wed Oct 13 14:25:44 2004}
%TCIDATA{LastRevised=Wednesday, June 29, 2016 08:58:02}
%TCIDATA{<META NAME="GraphicsSave" CONTENT="32">}
%TCIDATA{<META NAME="DocumentShell" CONTENT="General\Blank Document">}
%TCIDATA{CSTFile=LaTeX article (bright).cst}
%TCIDATA{PageSetup=28,28,28,14,0}

\newtheorem{theorem}{Theorem}
\newtheorem{acknowledgement}[theorem]{Acknowledgement}
\newtheorem{algorithm}[theorem]{Algorithm}
\newtheorem{axiom}[theorem]{Axiom}
\newtheorem{case}[theorem]{Case}
\newtheorem{claim}[theorem]{Claim}
\newtheorem{conclusion}[theorem]{Conclusion}
\newtheorem{condition}[theorem]{Condition}
\newtheorem{conjecture}[theorem]{Conjecture}
\newtheorem{corollary}[theorem]{Corollary}
\newtheorem{criterion}[theorem]{Criterion}
\newtheorem{definition}[theorem]{Definition}
\newtheorem{example}[theorem]{Example}
\newtheorem{exercise}[theorem]{Exercise}
\newtheorem{lemma}[theorem]{Lemma}
\newtheorem{notation}[theorem]{Notation}
\newtheorem{problem}[theorem]{Problem}
\newtheorem{proposition}[theorem]{Proposition}
\newtheorem{remark}[theorem]{Remark}
\newtheorem{solution}[theorem]{Solution}
\newtheorem{summary}[theorem]{Summary}
\newenvironment{proof}[1][Proof]{\textbf{#1.} }{\ \rule{0.5em}{0.5em}}
\input{tcilatex}
\begin{document}


\subsection{ \ \ \ \ \ \ \ \ \ \ \ \ \ \ \ \ \ \ \ \ \ \ \ \ \ \ \ \ \ 
\protect\underline{Chapitre 5 $:$ Instruction \ conditionnelle If}.}

\subsubsection{\protect\bigskip \protect\underline{1. R\'{e}daction de l'
instruction conditionnelle $if.$}}

L'instruction conditionnelle \ de base permet de donner des instructions diff%
\'{e}rentes selon qu'un test est vrai ou faux.

\bigskip

\underline{Premi\`{e}re r\'{e}daction :} $%
\begin{array}{c}
if\text{ cond 1 }then \\ 
Instruction\text{ A} \\ 
end%
\end{array}%
$

Si la condition bool\'{e}enne 1 est vraie alors l'instruction A est r\'{e}%
alis\'{e}e.

\bigskip

\underline{Deuxi\`{e}me r\'{e}daction : } $%
\begin{array}{c}
if\text{ cond 2 }then \\ 
\text{ \ \ \ \ \ \ \ \ \ \ \ \ \ \ \ \ \ \ \ \ \ \ \ \ \ \ \ \ }Instruction%
\text{ A } \\ 
\text{ \ }else \\ 
\text{ \ \ \ \ \ \ \ \ \ \ \ \ \ \ \ \ \ \ \ \ \ \ \ \ }Instruction\text{ B}
\\ 
end%
\end{array}%
$

Si la condition bool\'{e}enne 2 est vraie, l'instruction A est ex\'{e}cut%
\'{e}e sinon l'instruction B est ex\'{e}cut\'{e}e.

\bigskip

Remarques :

- \ Il existe une autre synthaxe de $if$ \ utile pour \'{e}crire des
instructions tenant sur une ligne , qui utilise des virgules "," :

\ \ \ \ \ \ \ \ \ \ \ \ \ \ \ \ \ \ \ \ \ \ \ \ \ \ if condition \ then,
Instruction A, else, Instruction B, end

- Quand $else$ est suivi d'un $if$ , on peut utiliser la forme $elseif$ de
la fa\c{c}on suivante :

$\ \ \ \ \ \ \ \ \ \ \ \ \ \ \ \ \ \ \ \ \ \ \ \ \ \ \ \ \ \ \ \ \ \ \ \ \ \
\ \ \ \ \ 
\begin{array}{c}
if\text{ cond 1 }then \\ 
\text{ \ \ \ \ \ \ \ \ \ \ \ \ \ \ \ \ \ \ \ \ \ \ \ \ \ \ \ \ }Instruction%
\text{ A1 } \\ 
\text{ \ }elseif\text{ cond 2 \ }then \\ 
\text{ \ \ \ \ \ \ \ \ \ \ \ \ \ \ \ \ \ \ \ \ \ \ \ \ }Instruction\text{ A2}
\\ 
else\text{ \ \ \ }Instruction\text{ A3} \\ 
end%
\end{array}%
$

L'utilisation de $if$ \ $elseif$ permet de sp\'{e}cifier plusieurs
conditions. Seules les instructions correspondant \`{a} la premi\`{e}re
condition v\'{e}rifi\'{e}e sont ex\'{e}cut\'{e}es.

- La seule contrainte est que chaque mot-cl\'{e} $then$ doit \^{e}tre sur la
m\^{e}me ligne que le $if$ ou le $elseif$ correspondant.

\bigskip \underline{Rappels : op\'{e}rateurs relationnels et logiques.}

Une variable logique ou bool\'{e}enne correspond \`{a} une expression
logique. Elle peut prendre deux valeurs : vraie ( $\%t$ ) ou fausse ( $\%f$
).

Logiques : $%
\begin{array}{ccc}
\& & | & \symbol{126} \\ 
et & ou\text{ } & non%
\end{array}%
$

Comparaisons et tests : $%
\begin{array}{cccccc}
== & < & > & >= & <= & <> \\ 
\text{{\small \'{e}galit\'{e}}} & \text{{\small inf\'{e}rieur strict}} & 
\text{{\small sup\'{e}rieur strict}} & \text{{\small sup\'{e}rieur ou \'{e}%
gal}} & \text{{\small inf\'{e}rieur ou \'{e}gal}} & \text{{\small diff\'{e}%
rent}}%
\end{array}%
$

\subsubsection{\protect\underline{2. Exercices .}}

\paragraph{\protect\underline{Exercice 1 :}}

On consid\`{e}re le programme suivant :

a=input("donner la valeur de a");

b=input("donner la valeur de b");

\ \ \ if a \TEXTsymbol{<} b then

\ \ \ \ \ \ \ \ \ \ \ \ \ \ \ \ \ \ \ \ \ \ aux = a

\ \ \ \ \ \ \ \ \ \ \ \ \ \ \ \ \ \ \ \ \ \ \ \ \ \ a = b

\ \ \ \ \ \ \ \ \ \ \ \ \ \ \ \ \ \ \ \ \ \ \ \ \ \ b= aux

\ \ \ \ \ \ \ \ \ \ \ \ \ \ \ \ \ \ end

disp(a,"a=");

disp("b=",b);

1. Pour a = 2 et b = 6, donner les valeurs de a et b \`{a} la fin du
programme. Expliquer l'action de ce programme sur a et b.

2. Idem pour a = 9 et b = 3.

\paragraph{\protect\underline{Exercice 2 :}}

On consid\`{e}re le programme suivant :

a=input("donner la valeur de a");

b=input("donner la valeur de b");

c=input("donner la valeur de c");

\ \ \ \ if a \TEXTsymbol{>} b then \ d = a

\ \ \ \ \ \ \ \ \ \ \ \ \ \ \ else \ \ d = b

\ \ \ end

\ \ \ \ if c \TEXTsymbol{>} d then d =c

\ \ \ end

disp(d)

1. Pour a =-1, b=6, c=9, donner la valeur affich\'{e}e par l'ordinateur.
Expliquer l'objectif de ce programme.\ \ \ 

2. Idem pour a =8, b=17, c=-4.

\paragraph{\protect\underline{Exercice 3 :}}

Soit $f$ la fonction d\'{e}finie sur $\mathbb{R}$ par :$f(x)=\dfrac{\exp (x)%
}{x-1}$ si $x\leq 0$ et $f(x)=x\ln x-1$ \ si $x>0$

1. Etudier la continuit\'{e} de $f$ sur $%
%TCIMACRO{\U{211d} }%
%BeginExpansion
\mathbb{R}
%EndExpansion
.$

2. \ Ecrire un programme qui demande \`{a} l'utilisateur une valeur $x$ et
qui affiche la valeur de $f(x).$

\paragraph{\protect\underline{Exercice 4 :}}

Soit $f$ la fonction d\'{e}finie sur $\mathbb{R}$ par :$\left\{ 
\begin{array}{c}
f(x)=\exp (-x)+x-2\text{ si }x\leq 0 \\ 
f(x)=x\cos x-1\text{ \ si }x\epsilon \left] 0;3\right] \\ 
f(x)=\frac{x^{2}+2}{x-1}\text{ \ si }x>3%
\end{array}%
\right. .$

Ecrire un programme qui demande \`{a} l'utilisateur une valeur $x$ et qui
affiche la valeur de $f(x).$

\bigskip

\paragraph{\protect\underline{Exercice \ 5 :}}

Ecrire un programme qui demande \`{a} l'utilisateur un entier naturel non
nul $n$ et qui permet d'afficher la matrice de $M_{n}\left( 
%TCIMACRO{\U{211d} }%
%BeginExpansion
\mathbb{R}
%EndExpansion
\right) $ de la forme suivante :

$A_{n}=\left( 
\begin{array}{ccccc}
2 & -1 & -1 & \cdots & -1 \\ 
3 & 2 & \ddots &  & \vdots \\ 
3 & \ddots & \ddots & \ddots & -1 \\ 
\vdots &  & \ddots & 2 & -1 \\ 
3 & \cdots & 3 & 3 & 2%
\end{array}%
\right) .$

\paragraph{\protect\underline{Exercice 6 :}}

Ecrire un programme prenant en entr\'{e}e un trin\^{o}me $aX^{2}+bX+c,$ avec 
$a\neq 0$, repr\'{e}sent\'{e} par un tableau de ses coefficients et
affichant ses racines r\'{e}elles si elles existent ou "aucune racine r\'{e}%
elle" sinon.

\end{document}
