\startcomponent component_DS1
\project project_Res_Mathematica
\environment environment_Maths
\environment environment_Inferno
\xmlprocessfile{exo}{xml/Limbo_Exercices.xml}{}
\iffalse
\setupitemgroup[List][1][R,inmargin][after=,before=,left={\bf Exo },symstyle=bold,inbetween={\blank[big]}]
\setupitemgroup[List][2][n,joineup][after=,before=,inbetween={\blank[small]}]
\setupitemgroup[List][3][a,joineup][after=,before=,inbetween={\blank[small]}]
\setupitemgroup[List][4][1,joineup,nowhite]
\fi

%\setupitemgroup[List][1][A,inmargin][after=,before=,left={\bf Exo },symstyle=bold,inbetween={\blank[big]}]
%\setupitemgroup[List][1][R,joineup][after=,before=,inbetween={\blank[small]}]
%\setupitemgroup[List][1][n,inmargin][after=,before=,left={\bf Exo },symstyle=bold,inbetween={\%blank[big]}]
%\setupitemgroup[List][2][n,joineup][after=,before=,inbetween={\blank[small]}]
%\setupitemgroup[List][3][a,joineup][after=,before=,inbetween={\blank[small]}]
%\setupitemgroup[List][4][1,joineup,nowhite]
%\setupitemgroup[List][4][a,joineup,nowhite]
\definecolor[myGreen][r=0.55, g=0.76, b=0.29]%
\setuppapersize[A4]
\setuppagenumbering[location=]
\setuplayout[header=0pt,footer=0pt]
\def\conseil#1{{\myGreen\it #1}}%


\starttext
\setupheads[alternative=middle]
%\showlayout
\def\gah#1{\margintext{Exercice #1}}

\iftrue
\page
\centerline{\bfb DEVOIR MAISON 9}
\blank[big]

\setupitemgroup[List][1][n,joineup][after=,before=,inbetween={\blank[small]}]
\setupitemgroup[List][2][a,joineup][after=,before=,inbetween={\blank[small]}]
\setupitemgroup[List][3][a,joineup][after=,before=,inbetween={\blank[small]}]
\setupitemgroup[List][4][1,joineup,nowhite]

\centerline{\bf EXERCICE 1}
{\it Un grand classique : polynômes de Tchebycheff}\crlf
On considère la suite des polynômes $(P_n)_{n∈ℕ}$ définie par $P_0 = 1$, $P_1 = X$ et
\startformula
P_{n+2}=2XP_{n+1}-P_n\qquad(n∈ℕ).
\stopformula
\startList
\item Préciser $P_2$, $P_3$ et $P_4$.
\item Déterminer le coefficient dominant de $P_n$ ainsi que son degré.
\item Etudier la parité des polynômes $P_n$.
\item Montrer l'identité fondamentale des polynômes de Tchebycheff de 1\high{ère} espèce : 
\startformula
P_n(cos(x)) = cos(nx)\qquad(n∈ℕ, x∈ℝ).
\stopformula
\item {\it Bonus} En déduire les racines de $P_n$ ainsi que sa forme factorisée. 
\stopList

\centerline{\bf EXERCICE 2}
{\it Un autre classique : polynômes de Lagrange}\crlf
\startList
\item D'abord un exemple : Déterminer, en raisonnant par analyse et synthèse, 
l’unique polynôme $P$ de degré $3$ tel que $P (1) = 0$, $P (2) = 0$, $P (3) = 0$ et $P (4) = 1$.
\item Cas général : soit $n∈ℕ$ et $a_0, a_1, ⋯, a_n$ des nombres réels distincts deux à deux.
\startList
\item Par analyse et synthèse, déterminer l’unique polynôme de degré $n$, noté $L_0$, tel que
$L_0 (a_0) = 1$ et $L_0(a_k) = 0$ pour $1⩽k⩽n$.
\item {\it bonus}. Plus généralement, pour $1⩽k⩽n$, déterminer l’unique polynôme de degré $n$, noté $L_k$, tel que $L_k(a_k ) = 1$ et $L_k (a_j ) = 0$ pour $j∈⟦0,n⟧\ssm\{k\}$.
\stopList
\stopList

\centerline{\bf EXERCICE 3}
Soit la suite de polynômes $(P_n)$ définie par : $P_1=1$ et 
\startformula
P_n=1+{X\F 1!}+⋯+{X(X+1)⋯(X+n-1)\F n!}\qquad (n⩾2)
\stopformula
\startList
\item Ecrire $P_2$ et $P_3$ sous forme factorisée.
\item En déduire une factorisation de $P_n$ en polynômes de degré $1$.
\stopList

\centerline{\bf EXERCICE 4}
Soit $P$ et $Q$ deux polynômes de $ℝ[X]$ vérifiant : 
\startformula
∀x ∈ ℝ,\qquad P(x)\cos(x) + Q(x)\sin(x) = 0.
\stopformula
Montrer que $P=Q=0$.

\stoptext
\stopcomponent
\endinput
