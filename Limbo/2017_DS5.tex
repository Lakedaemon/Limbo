\startcomponent component_DS1
\project project_Res_Mathematica
\environment environment_Maths
\environment environment_Inferno
\xmlprocessfile{exo}{xml/Limbo_Exercices.xml}{}
\iffalse
\setupitemgroup[List][1][n,inmargin][after=,before=,left={\bf Exo },symstyle=bold,inbetween={\blank[big]}]
\setupitemgroup[List][2][a,joineup][after=,before=,inbetween={\blank[small]}]
\setupitemgroup[List][3][a,joineup][after=,before=,inbetween={\blank[small]}]
\setupitemgroup[List][4][1,joineup,nowhite]
\fi
\let\ds\displaystyle
\setupitemgroup[List][1][A,inmargin][after=,before=,left={\bf Exo },symstyle=bold,inbetween={\blank[big]}]
\setupitemgroup[List][2][n,joineup][after=,before=,inbetween={\blank[small]}]
\setupitemgroup[List][3][i,joineup,nowhite]
\setupitemgroup[List][4][a,joineup,nowhite]
\definecolor[myGreen][r=0.55, g=0.76, b=0.29]%
\setuppapersize[A4]
\setuppagenumbering[location=]
\setuplayout[header=0pt,footer=0pt]
\def\conseil#1{{\myGreen\it #1}}%
\def\tr{\text{tr}}

\starttext
\setupheads[alternative=middle]
%\showlayout
\def\gah#1{\margintext{Exercice #1}}

\iftrue
\page
\centerline{\bfb DEVOIR SURVEILLE 5}
\blank[big]


%\setupitemgroup[List][1][A,inmargin][after=,before=,left={\bf Exo },symstyle=bold,inbetween={\blank[big]}]
%\setupitemgroup[List][2][n,joineup][after=,before=,inbetween={\blank[small]}]
%\setupitemgroup[List][3][a,joineup,nowhite]
%\setupitemgroup[List][4][a,joineup,nowhite]

\setupitemgroup[List][1][A,inmargin][after=,before=,left={\bf Exo },symstyle=bold,inbetween={\blank[big]}]



\startList




\setupitemgroup[List][2][n,joineup][after=,before=,inbetween={\blank[small]}]
\setupitemgroup[List][3][a,joineup,nowhite]
\setupitemgroup[List][4][i,joineup,nowhite]


\item%
On considère la suite $(T_n)_{n∈ℕ}$ de polynômes de $ℝ[X]$ définie par 
\startformula
T_0=1,\qquad T_1=2X\qquad\Et\qquad T_n=2XT_{n-1}-T_{n-2}\qquad(n⩾2)
\stopformula
On pourra confondre polynôme et fonction polynomiale. Ainsi, pour $n⩾2$, on a 
\startformula
T_n(x)=2xT_{n-1}(x)-T_{n-2}(x)\qquad(x∈ℝ)
\stopformula
\startList
\item Calculer $T_2$ et $T_3$
\item\startList
\item Pour $n∈ℕ$, démontrer que $T_n$ est un polynôme de degré $n$, dont on déterminera le coefficient du terme de degré $n$.
\item Etablir que si $n$ est un entier pair (resp. impair), alors $T_n$ est un polynôme pair (resp. impair).
\stopList
\item Pour $n∈ℕ$, calculer $T_n(1)$ en fonction de $n$.
\item\startList
\item Pour $n∈ℕ$ et $ϑ∈]0,π[$, établir que $\ds
T_n(\cos ϑ)={\sin\big((n+1)ϑ\big)\F\sin ϑ}$.
\item Pour $n∈ℕ^*$, en déduire que $T_n$ admet $n$ racines dans $]-1,1[$, à expliciter.
\item Pour $n∈ℕ^*$, établir que 
$\ds 
T_n=2^n∏_{k=1}^n\Q(X-\cos{kπ\F n+1}\W)$
\item Pour $n∈ℕ^*$, en déduire la valeur de $\ds ∏_{k=1}^n\sin{kπ\F 2(n+1)}$ en fonction de $n$
\stopList
\item\startList
\item Pour $n∈ℕ$, démontrer que 
\startformula
\sin^2(ϑ)T_n''(\cos ϑ)-3\cos(ϑ)T_n'(\cos ϑ)+(n^2+2n)T_n(\cos ϑ)=0\qquad(0<ϑ<π)
\stopformula
{\it Indication :} On pourra dériver deux fois la fonction (nulle) définie par 
\startformula
g(ϑ)=\sin(ϑ)T_n(\cos ϑ)-\sin\big((n+1)ϑ\big)\qquad(ϑ∈ℝ)
\stopformula
\item Pour $n∈ℕ$, en déduire que 
\startformula
(X^2-1)T_n''+3XT_n'-(n^2+2n)T_n=0
\stopformula
\stopList
\stopList

\item%Edhec S 2013
{\bf Extrait EDHEC S}. Pour $n∈ℕ$, on pose $\ds u_n=\int_0^{π\F 2}(\sin t)^n\d t$
\startList
\item \startList
\item Calculer $u_0$ et $u_1$.
\item Montrer que la suite $u=(u_n)_{n∈ℕ}$ est décroissante
\item Montrer que la suite $u$ est convergente
\stopList
\item\startList
\item En remarquant que $(\sin t)^{n+2}=(\sin t)^n-\cos(t)×\cos(t)(\sin t)^n$ 
et en procédant à une intégration par partie, montrer que
\startformula
(n+2)u_{n+2}=(n+1)u_n\qquad(n∈ℕ)
\stopformula
\item En déduire que $\ds u_{2n}={(2n)!\F (2^nn!)^2}{π\F 2}$ pour $n∈ℕ$.
\item Pour $n∈ℕ$, montrer que $(n+1)u_{n+1}u_n={π\F 2}$.
\item En déduire la valeur de $u_{2n+1}$ en fonction de $n$.
\stopList
\item\startList
\item Calculer la limite de ${u_{n+2}\F u_n}$ lorsque $n$ tend vers $+∞$.
\item En remarquant que $u_n⩽u_{n+1}⩽u_{n+2}$, en déduire que $\ds \lim_{n→+∞}{u_{n+1}\F u_n}=1$.
\item Enfin, montrer que l'on a $u_n∼\sqrt{π\F 2n}$
\stopList
\stopList


\setupitemgroup[List][2][I,joineup][after=,before=,inbetween={\blank[small]}]
\setupitemgroup[List][3][n,joineup,nowhite]
\setupitemgroup[List][4][a,joineup,nowhite]
\item%Exo matrices
On dit qu’une matrice $M∈\mc M_3(ℝ)$ est semi-magique \ssi on obtient la même somme des coefficients sur chaque ligne et sur chaque colonne de $M$.\crlf
Par exemple, comme la somme des coefficients sur chaque ligne et sur chaque colonne est $-1$ pour $\ds W=\Matrix{
\NC -1\NC -1\NC 1\NR
\NC 1\NC 1\NC -3\NR
\NC -1\NC -1\NC 1
}$, la matrice $W$ est semi-magique\crlf
On note $\mc SM_3(ℝ)$ l'ensemble des matrices semi-magiques de $\mc M_3(ℝ)$ et on admet que 
\startformula
\mc SM_3(ℝ)=\{M∈\mc M_3(ℝ):JM=MJ\}\quad\text{avec}\quad J=\Matrix{
\NC 1\NC 1\NC 1\NR
\NC 1\NC 1\NC 1\NR
\NC 1\NC 1\NC 1
}
\stopformula

On pose $G=\Vect(E_1,E_2,E_3,E_4)$ avec $E_1=\Matrix{
\NC 0\NC -1\NC 1\NR
\NC 1\NC 0\NC -1\NR
\NC -1\NC 1\NC 0\NR
}$, $E_2=\Matrix{
\NC 1\NC 0\NC 1\NR
\NC 0\NC 0\NC 2\NR
\NC 1\NC 2\NC -1\NR
}$, $E_3=\Matrix{
\NC 0\NC 1\NC 1\NR
\NC 1\NC 0\NC 1\NR
\NC 1\NC 1\NC 0\NR
}$ et $E_4=\Matrix{
\NC 0\NC 0\NC 2\NR
\NC 0\NC 1\NC 1\NR
\NC 2\NC 1\NC -1\NR
}$.

\startList
\item{\bf ETUDE de $\mc SM_3(ℝ)$}
\startList
\item\startList
\item Prouver que $\mc SM_3(ℝ)$ est un sous-espace vectoriel de $E$ contenant $J$.
\item Pour $M∈SM_3(ℝ)$, montrer que $\strut^{t}M∈SM_3(ℝ)$.
\stopList
\item\startList
\item Montrer que $E_1∈\mc SM_3(ℝ)$. On admet qu'on montrerait de même que $E_2$, $E_3$ et $E_4$ appartiennent à $\mc SM_3(ℝ)$.
En déduire que $G$ est un sous-espace vectoriel de $\mc SM_3(ℝ)$ dont on précisera la dimension
\item Montrer que si $M∈G$, alors $\tr(M)=0$. On admet pour la suite que 
\startformula
G=\{M∈\mc SM_3(ℝ):\tr(M)=0\}
\stopformula
\item Montrer que $G∩\Vect(J)=\{0\}$. 
\stopList
\item Soit $M∈\mc SM_3(ℝ)$. Justifier que la matrice $N=M-{\tr(M)\F 3}J$ appartient à $\mc SM_3(ℝ)$ et calculer $\tr(N)$.
\item\startList
 \item Montrer que $\mc SM_3(ℝ)=G⊕\Vect(J)$. 
\item En déduire que $\dim(\mc SM_3(ℝ))=5$ et que $(E_1,E_2,E_3,E_4,J)$ forme une base de $\mc SM_3(ℝ)$.
\stopList

\stopList
\item {\bf Etude d'un sous-espace vectoriel de $\mc SM_3(ℝ)$}\crlf
Pour $M=(m_{i,j})_{1⩽i⩽3\atop1⩽j⩽3}∈\mc SM_3(ℝ)$, on pose
\startformula
φ(M)=m_{1,1}+m_{1,2}+m_{1,3}\quad\Et\quad ψ(M)=m_{3,1}+m_{2,2}+m_{1,3}
\stopformula
On admet que $H=\{M∈\mc SM_3(ℝ):φ(M)=ψ(M)=\tr(M)\}$ est un sous espace vectoriel de $\mc SM_3(ℝ)$ de dimension $3$.
\startList
\item On pose $D=\Matrix{
\NC -1\NC 2\NC λ\NR
\NC 0\NC 0\NC 0\NR
\NC 1\NC -2\NC 1
}$
\startList
\item Déterminer $λ∈ℝ$ pour que l'on ait $D∈H$
\item Justifier que $\strut^{t}D∈H$
\item Montrer que $(J,D, \strut^{t}D)$ forme une base de $H$
\stopList
\iffalse
\item On désigne par $\mc S_3(ℝ)$ l'ensemble des matrices $3×3$ symétriques et par $\mc A_3(ℝ)$ celui des matrices $3×3$ anti-symétriques et on pose 
\startformula 
G_A=Q∩\mc A_3(ℝ)\quad\Et\quad G_S=G∩\mc S_3(ℝ).
\stopformula
\startList
\item Soit $M∈G$. On pose $N_1={1\F 2}\Q(M+\strut^tM\W)$ et $N_2={1\F 2}\Q(M-\strut^tM\W)$. Montrer que $N_1∈G_S$ et que $N_2∈G_A$.
\item En déduire que $G=G_S⊕G_A$ 
\stopList
\fi
\iffalse
\item\startList
\item Soit $M$ une matrice anti-symétrique avec $M=\Matrix{
\NC 0\NC α\NC β\NR
\NC -α\NC 0\NC γ\NR
\NC -β\NC -γ\NC 0}$ où $(α, β, γ)∈ℝ^3$
Montrer que si $M∈G_A$, alors $M∈\Vect(E_1)$
\item En déduire une base de $G_A$
\stopList
\item Déterminer une base de $G_S$
\fi
\stopList
\stopList


\setupitemgroup[List][2][n,joineup][after=,before=,inbetween={\blank[small]}]
\setupitemgroup[List][3][a,joineup,nowhite]
\setupitemgroup[List][4][i,joineup,nowhite]
\item%Edhec 2000
{\bf Extrait EDHEC E}. On lance $n$ fois, de façon indépendante, une pièce donnant pile avec la probabilité $p∈]0,1[$ et face avec la probabilité $q=1-p$. \crlf 
Pour $k⩾2$, on dit que le $k\high{ième}$ lancer est un {\it changement} s'il amène un résultat différent du $(k-1)\high{ième}$~lancer.
Pour $n⩾2$, on note $X_n$ la variable égale au nombre de changements survenus durant les $n$ premiers lancers\crlf
On note $P_k$ l'événement \quote{on obtient pile au $k\high{ième}$ lancer}. 
\startList
\item \startList
\item Donner la loi de $X_2$
\item Donner la loi de $X_3$ et calculer son espérance et sa variance
\stopList
\item Dans cette question, $n$ désigne un entier supérieur ou égal à $2$.
\startList
\item \startList
\item Que vaut $X_n(Ω)$ ?
\item Exprimer $P(X_n=0)$ en fonction de $p$, $q$ et $n$.
\item En décomposant l'événement $(X_n=1)$ en une réunion d'événements incompatibles, montrer que 
\startformula
P(X_n=1)={2pq\F q-p}\Q(q^{n-1}-p^{n-1}\W)
\stopformula
\item En distinguant les cas $n$ pair et $n$ impair, exprimer $P(X_n=n-1)$ en fonction de $p$ et $q$
\item Grace aux questions précédentes, démontrer que 
\startformula
P(X_4=1)=P(X_4=2)=2pq(1-pq)
\stopformula
et préciser la loi de $X_4$. Calculer l'espérance de $X_4$. 
\stopList
\item Pour $2⩽k⩽n$, on note $Z_k$ la variable aléatoire qui vaut $1$ si le $k\high{ième}$ lancer est un changement et $0$ sinon, de sorte que $Z_k$ est une variable de Bernoulli.
Exprimer $X_n$ à l'aide des variables $Z_k$ et en déduire $E(X_n)$.
\stopList
\item Dans cette question, on suppose que $p=q={1\F 2}$.
\startList
\item Vérifier en utilisant les résultats des questions précédentes que $X_3$ et $X_4$ suivent chacune une loi binomiale.
\item Soit $n⩾2$. 
\startList
\item Pour $1⩽k⩽n$, justifier que 
\startformula
\Align{
\NC P\big((X_n=k)∩P_n∩P_{n+1}\big)={1\F 2}P\big((X_n=k)∩P_n\big)\NR
\NC P\big((X_n=k)∩\overline{P_n}∩\overline{P_{n+1}}\big)={1\F 2}P\big((X_n=k)∩\overline{P_n}\big)
}
\stopformula
et en déduire que 
\startformula
P\big((X_n=k)∩(X_{n+1}=k)\big)={1\F 2}P(X_n=k)
\stopformula
On admettra que l'on démontrerait de même que
\startformula
P\big((X_n=k-1)∩(X_{n+1}=k)\big)={1\F 2}P(X_n=k-1)
\stopformula
\item Démontrer par récurrence que la variable $X_n$ suit la loi $\mc B(n-1,{1\F 2})$
\stopList
\stopList
\stopList
\stopList





\stoptext
\stopcomponent
\endinput
