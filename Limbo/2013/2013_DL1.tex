\startcomponent component_DS1
\project project_Res_Mathematica
\environment environment_Maths
\environment environment_Inferno
\xmlprocessfile{exo}{xml/Limbo_Exercices.xml}{}
\setupitemgroup[List][1][R,inmargin][after=,before=,left={\bf Exo },symstyle=bold,inbetween={\blank[big]}]
\setupitemgroup[List][2][n,joineup][after=,before=,inbetween={\blank[small]}]
\setupitemgroup[List][3][a,joineup][after=,before=,inbetween={\blank[small]}]
\setupitemgroup[List][4][1,joineup,nowhite]


\setuppapersize[A4]
\setuppagenumbering[location=]
\setuplayout[header=0pt,footer=0pt]

\starttext
\setupheads[alternative=middle]
%\showlayout
\def\gah#1{\margintext{Exercice #1}}
\centerline{\bfb DEVOIR LIBRE 2}
\vskip-2em
%\blank[big]
%\item\exo{C1}
%\item\exo{C2}
%\item\exo{C3}
%{\it \hint{C3}}
 \filterpages[xml/2013/S3_DM_sommes_doubles.pdf][1][height=18cm]

%\item\exo{C6}
%\sym{Problème}\exo{C5}

\iftrue
\page
\centerline{\bfb CORRECTION DU DEVOIR LIBRE 2}
\blank[big]
\startList%\item\solution{C1}
%\item\solution{C2}
%\item\solution{C3}
\item\startList\item En remarquant que nous avons affait à une somme géomètrique de raison $2$, nous obtenons que 
\startformula 
 ∑_{k=p}^q2^k={2^{q+1}-2^p\F 2-1}= 2^{q+1}-2^p
\stopformula
\item A fortiori, nous en déduisons que 
\startformula 
\Align{
\NC u_n\NC = ∑_{j=0}^n ∑_{k=j}^n2^k= ∑_{j=0}^n\Q(2^{n+1}-2^j\W)= ∑_{j=0}^n2^{n+1}- ∑_{j=0}^n2^j\NR
\NC \NC =(n+1)2^{n+1}-{2^{n+1}-1\F 2-1}=n2^{n+1}+1.\NR
}
\stopformula
\stopList
\item Il résulte alors du théorème de Fubini que 
\startformula 
u_n= ∑_{j=0}^n ∑_{k=j}^n2^k=u_n= ∑_{k=0}^n ∑_{j=0}^k2^k= ∑_{k=0}^n(k+1)2^k.
\stopformula
En procédant au changement d'indice $ℓ=k+1$, il suit
\startformula 
u_n= ∑_{k=1}^{n+1}ℓ2^{ℓ-1}.
\stopformula
De sorte que 
\startformula 
∑_{k=0}^nℓ2^{ℓ-1}=u_n-0-(n+1)2^n=n2^{n+1} + 1 -(n+1)2^n=(n-1)2^n+1.
\stopformula
\item Nous obtenons alors que 
\startformula 
\Align{
\NC ∑_{i=0}^n ∑_{k=0}^i(k+1)2^k\NC =∑_{i=0}^nu_i=∑_{i=0}^n\Q(i2^{i+1}+1\W)=4∑_{i=0}^ni2^{i-1}+∑_{i=0}^n1\NR
\NC \NC=4\Q((n-1)2^n+1\W)+n+1\NR
}
\stopformula
\item Il résulte alors du théorème de Fubini que 
$$
\Align{
\NC  4\Q((n-1)2^n+1\W)+n+1 \NC = ∑_{i=0}^n ∑_{k=0}^i(k+1)2^k = ∑_{k=0}^n ∑_{i=k}^n(k+1)2^k\NR 
\NC \NC = ∑_{k=0}^n (n+1-k)(k+1)2^k \NR
\NC\NC = (n+1)∑_{k=0}^n (k+1)2^k -  ∑_{k=0}^n k(k+1)2^k\NR
\NC\NC =(n+1)u_n- ∑_{k=0}^n k(k+1)2^k\NR
}
$$
En conclusion, nous avons trouvé que 
$$
\Align{
\NC A_n\NC =∑_{k=0}^n k(k+1)2^k= (n+1)u_n-\Q(4(n-1)2^n+1)+n+1\W)\NR
\NC\NC =(n+1)(n2^{n+1}+1)-\Q(4(n-1)2^n+1)+n+1\W)\NR}
$$
La flemme de simplifier....
\item Enfin, nous remarquons que 
$$
∑_{k=0}^n k^22^k=∑_{k=0}^n k(k+1)2^k-∑_{k=0}^n k2^k=A_n-2((n-1)2^n+1)
$$
%\sym{Problème}\solution{C5}
\stopList
\fi
\stoptext
\stopcomponent
\endinput
