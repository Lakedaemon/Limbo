\startenvironment environment_Maths
\unprotect
%%% todo

% ? context way to type
%
%\startformula
%\eqalign{
%ℜe(λs+μz)&=λℜe(s)+μℜe(z)\cr
%ℑm(λs+μz)&=λℑm(s)+μℑm(z)\cr
%}\qquad(λ, μ, s, z) ∈ ℝ^2×ℂ^2
%\stopformula
%

% ? use \startalign inside \Eq

% ? fix CDATA

% trim spaces



% setup
\mainlanguage[fr]
\fr

\definefloat[centeredTable][centeredTables][table]
\setupfloat[centeredTable][default={fixed,none}]
\setupinmargin[style=\bfxx\setupinterlinespace]
\usemodule[tikz]








\startuseMPgraphic{FunnyFrame}
picture p ; numeric w, h, o ;
p := textext.rt(\MPstring{FunnyFrame}) ;
w := OverlayWidth ; h := OverlayHeight ; o := BodyFontSize ;
p := p shifted (2o,h-ypart center p) ; draw p ;
drawoptions (withpen pencircle scaled 1pt withcolor .625red) ;
draw (2o,h)--(0,h)--(0,0)--(w,0)--(w,h)--(xpart urcorner p,h) ;
draw boundingbox p ;
setbounds currentpicture to unitsquare xyscaled(w,h) ;
\stopuseMPgraphic



\startuseMPgraphic{GraphicDefinition}
picture p ; numeric o ; path a, b ; pair c ;
p := textext.rt(\MPstring{OverlayDefinition}) ;
o := BodyFontSize ;
a := unitsquare xyscaled (OverlayWidth,OverlayHeight) squeezed 2pt ;
p := p shifted (2o,OverlayHeight-ypart center p) ;
pickup pencircle scaled OverlayLineWidth ;
b := a squeezed 2pt;
fill b withcolor OverlayColor ;
draw b withcolor OverlayLineColor ;
b := (boundingbox p) squeezed 2pt;
fill b withcolor OverlayColor ;
draw b withcolor OverlayLineColor ;
draw p withcolor black ;
setbounds currentpicture to a ;
\stopuseMPgraphic
                                                                   
\defineoverlay[OverlayDefinition][\useMPgraphic{GraphicDefinition}]

\startuseMPgraphic{GraphicProperty}
picture p ; numeric o ; path a, b ; pair c ;
p := textext.rt(\MPstring{OverlayProperty}) ;
o := BodyFontSize ;
a := unitsquare xyscaled (OverlayWidth,OverlayHeight);
p := p shifted (2o,OverlayHeight-ypart center p) ;
pickup pencircle scaled OverlayLineWidth ;
b := a superellipsed .95;
fill b withcolor OverlayColor ;
draw b withcolor OverlayLineColor ;
b := (boundingbox p) superellipsed .95;
fill b withcolor OverlayColor ;
draw b withcolor OverlayLineColor ;
draw p withcolor black ;
setbounds currentpicture to a ;
\stopuseMPgraphic
                                                                   
\defineoverlay[OverlayProperty][\useMPgraphic{GraphicProperty}]

%fill p withcolor .850white ; draw p withcolor .625yellow ;



%\startuniqueMPgraphic{GraphicDefinitio}
%path p ; p := fullsquare
%xyscaled (\overlaywidth,\overlayheight) squeezed 2pt ;
%pickup pencircle scaled 1pt ;
%fill p withcolor .850white ; draw p withcolor .625yellow ;
%\stopuniqueMPgraphic






\setupframedtexts
[Definition]
[backgroundcolor=lightgray,
framecolor=darkred,
rulethickness=2pt,
offset=\bodyfontsize,
before={\blank[big,medium]},
after={\blank[big]},
width=\textwidth]


\setupframedtexts
[Property]
[backgroundcolor=lightgray,
framecolor=darkred,
rulethickness=2pt,
offset=\bodyfontsize,
before={\blank[big,medium]},
after={\blank[big]},
width=\textwidth]



\defineframedtext[Definition][frame=off,background=OverlayDefinition,before={}]
\defineframedtext[Property][frame=off,background=OverlayProperty,before={}]

\def\FrameTitle#1#2{\setMPtext{#1}{\hbox spread 1em{\hss\strut#2\hss}}}\setMPtext{#1}{} % initialize the text variable


%\def\App#1:#2→#3:#4→#5\ppA{\Eq[C,C,C,C]{
%\NC #1:\NC #2\NC \to\NC #3\NC\NR
%\NC\NC #4\NC \mapsto\NC #5\NC\NR}
%}


\FrameTitle{OverlayDefinition}{Zapf (1)}
\FrameTitle{OverlayProperty}{Zapf (2)}
%\FrameTitle{Zapf (1)}
%\StartFrame
%Coming back to the use of typefaces in electronic
%publishing: many of the new typographers receive their
%knowledge and information about the rules of typography from
%books, from computer magazines or the instruction manuals
%which they get with the purchase of a PC or software.
%\StopFrame


%definition

%%  obsolete
\let\ob\mathbb

\long\def\Application : #1\par{\noindent{\eightpts{\it Application : }#1}\medskip}

\def\Center#1{\placecenteredTable{}{#1}}%
\def\Conseil : #1\par{{%
	\eightpoint
	\noindent
	\llap{%
		$\underline{\mbox{Conseil}}$ : 
	}%
	#1\medskip
}}%
\def\CTab#1{\Center{\Tab{#1}}}%


\def\Définition{\dosingleargument\Définition@do}%
\def\Hypothèses@empty{}%
\def\Définition@do[#1]#2\par{\def\Hypotheses{#1}\unless\ifx\Hypotheses\Hypothèses@empty\FrameTitle{OverlayDefinition}{#1}\fi\startDefinition\ignorespaces #2\stopDefinition\medskip}%

\long\def\Démonstration. #1\CQFD{\noindent{\eightpoint Démonstration. #1}\bigskip}

\def\Eq{\dosingleargument\Eq@do}%
\def\Eq@process#1{%
	\def\c{}%
	\processaction[#1][
		r=>\def\c{rh\FramedCell|},%
		c=>\def\c{c|},
		C=>\def\c{cm|},
		*=>\let\Eq@Remember\relax]%
   %  default=>\relax,
    % unknown=>\unknown{... \commalistelement ...}
    %
\@EA\@EA\@EA\def\@EA\@EA\@EA\Eq@columns\@EA\@EA\@EA{\@EA\Eq@columns\c}
}%
\def\Eq@do[#1]#2{{\def\Eq@columns{|}\processcommalist[#1]\Eq@process
\ifx\Eq@Remember\undefined
	\@EA\CTab\@EA{\@EA[\Eq@columns]#2}\else
	\@EA\R\@EA{\@EA\CTab\@EA{\@EA[\Eq@columns]#2}}\fi
}}%

\def\Et{\text{ et }}%
\def\EtQ{\quad\Et\quad}
\def\EtQQ{\qquad\Et\qquad}

\def\Exemple{\def\Exemple@text{Exemple}\dosingleargument\Exemple@do}%
\def\Exemple@do[#1]#2\par{{\getparameters[??LD][#1]}{\eightpoint\Exemple@text. \ignorespaces #2}\medskip}%
\def\Exemples{\def\Exemple@text{Exemples}\dosingleargument\Exemple@do}%

\let\F\over
%\def\FramedCell@command{\inframed[offset=-0.2ex]}%
\def\FramedCell#1{\inframed[offset=-0.2ex]{ #1 }}%

\def\Notation{\noindent{\it Notation}}%

\def\Propriété{\dosingleargument\Propriété@do}%
\long\def\Propriété@do[#1]#2\par{\def\Hypothèses{#1}\unless\ifx\Hypotheses\Hypothèses@empty\FrameTitle{OverlayProperty}{#1}\fi\startProperty\ignorespaces #2\stopProperty\medskip}%
\def\Définition@do[#1]#2\par{\def\Définition@hypotheses{#1}\unless\ifx\Définition@hypotheses\Définition@hypothèses@empty\FrameTitle{OverlayDefinition}{#1}\fi\startDefinition\ignorespaces #2\stopDefinition\medskip}%

\let\Q\left

\def\Rappel : #1\par{{%
	\eightpoint
	\noindent
	\llap{%
		$\underline{\mbox{Rappel}}$ : 
	}%
	#1\medskip
}}%
\long\def\Rappels. #1\par{\noindent{\eightpoint{\it Rappels. }#1}\medskip}


\def\Remarque{\noindent{\it Remarque. }}%
\def\Remarques{\noindent{\it Remarques. }}%

\def\Tab#1{\starttabulate#1\stoptabulate}%

\def\Théorème{\dosingleargument\Théorème@do}%
\def\Théorème@do[#1]#2\par{{\getparameters[??LD][#1]}Théorème. \ignorespaces #2\par\medskip}%

\def\VTab#1{\vcenter{\Tab{#1}}}%
\let\W\right

\protect
\stopenvironment
