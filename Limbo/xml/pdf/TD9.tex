\magnification 1200
\hsize180truemm\vsize 270truemm\hoffset=-10truemm\voffset=-13truemm
\pretolerance=500\tolerance=1000\brokenpenalty=5000
\parindent3mm


\input color.txs
\input pstricks
\input pst-plot
\input eplaingt
\input Macrols

\input ExosPTSI

%%%  Num�rotation automatique des questions 
\newcount\numeroexo
\newcount\numeroqprob
\numeroqprob=0
\def\pbq#1\par{\global\advance\numeroqprob by 1\noindent\the\numeroexo.\the\numeroqprob. #1\medskip}

%%% Probleme 1.
\vglue-10mm\rightline{PTSI\hfill TD 7 : Courbes param\'etr\'ees cart\'esiennes\hfill \date}
\bigskip 
\exoet
\vfill
\exoqn
{\it Rappels : Si $a$ est une racine (\'evidente) d'un polyn\^ome en $P$ de la variable~$t$, on~peut factoriser $t-a$ dans l'expression de $P$. }
\medskip\noindent
{\it Etudier la position d'une courbe param\'etr\'ee $t\mapsto \b(x(t),y(t)\b)$ par rapport \`a une droite d'\'equation $y=ax+b$ consiste \`a \'etudier le signe de $y(t)-ax(t)-b$. }
\vfill
\exoex
\vfill
\exoqo
\vfill
\exoeu
\vfill
\exoer
\bye
 