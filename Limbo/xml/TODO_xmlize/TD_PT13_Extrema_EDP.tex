\catcode`@=11\relax

\input LD@Maths@TD.tex

\vglue-10mm\rightline{Spé PT\hfill TD 13 : Extrema et équations aux dérivées partielles\hfill}%\date}
\bigskip

\centerline{\bf Pour localiser les extrema locaux, cherchez les points critiques ! }
\hrule
\bigskip\noindent

\Definition [$a\in D\subset\ob R^n$] 
Le point $a$ est un point critique d'une fonction $f:D\to \ob R$, si et~seulement~si $f$ admet des dérivées partielles en $a$, 
qui sont nulles, i. e. 
$$
\forall k\in\{1,\cdots, n\}, \qquad {\partial f\F\partial x_k}(a)=0.
$$

\Theoreme [$a\in U$ ouvert de $\ob R^n$] 
Si $f\in\sc C^1(U,\ob R)$ admet un extremum relatif en un point $a$ de l'{\bf ouvert} $U$, 
alors $a$ est un point critique de $f$.   
\medskip

\centerline{\bf Pour tester si un point critique est un extemum local}
\hrule
\medskip\noindent
{\bf Utiliser un DL, des inégalités, l'astuce, etc... mais d'abord le théorème suivant  : }
\bigskip

\Theoreme [$U$ ouvert de $\ob R^2$, $a\in U$ point critique de $f\in\sc C^2(U,\ob R)$] 
Soient $r$, $s$ et  $t$ les nombres réels définis par 
$$
\ds r={\partial^2f\F\partial x^2}(a),\qquad 
\ds s={\partial^2f\F\partial x\partial y}(a)\quad \hbox{ et }\quad 
\ds t={\partial^2f\F\partial y^2}(a). \leqno{(*)}
$$
Alors, quatre cas se présentent : 
\medskip
\noindent 
1) Si $rt-s^2>0$ et $r<0$, la fonction $f$ admet un maximum relatif en $a$. \pn
2) Si $rt-s^2>0$ et $r>0$, la fonction $f$ admet un minimum relatif en $a$. \pn
3) Si $rt-s^2<0$, la fonction $f$ n'admet pas d'extremum relatif en $a$ {\it (selle de cheval).} \pn
4) Si $rt-s^2=0$, on ne peut conclure. {\it Il faudrait affiner l'estimation de $f$ en $a$.} 



\vfill\noindent
\Exercice{PTajt}
\vfill\noindent
\Exercice{PTajq}
\vfill\noindent
\Exercice{PTajw}
\vfill \noindent
\Exercice{PTajs}
\vfill \noindent
\Exercice{PTajx}
\vfill \noindent
\Exercice{PTajr}
\vfill \noindent
\Exercice{PTakb}
\vfill \noindent
\Exercice{PTakc}
\vfill \noindent
\Exercice{PTajy}
\vfill\vfill\vfill\vfil\vfill\vfill\vfill\vfill\vfill\vfill\vfill\vfill\null
\bye