\catcode`@=11\relax

\input LD@Maths@TD.tex

\vglue-10mm\rightline{Spé PT\hfill TD : Séries II\hfill}
\bigskip
\vfill
\input LD@Inferno@Macros.tex
%\LD@Exo@Label@Show
\def\LD@List{\Séries\SériesNumériques}
\def\LD@Font@Arial{}

\hrule
\smallskip
\centerline{Inégalités de factorielles ou formule de Stirling $n!\sim\Q({n\F\e}\W)^n\sqrt{2\pi n}$}
\smallskip
\hrule
\bigskip
\Exercice{PTaly}
\bigskip
\Exercice{PTos}
\bigskip
\Exercice{PTot}




\hrule
\smallskip
\centerline{Comparaison série-intégrale}
\smallskip
\hrule

%\Theoreme 
%Si l'application $f$ est positive, décroissante et continue sur $[0, \infty[$, alors  la série $\ds\sum_{n=0}^\infty f(n)$ et la limite $\ds\lim\limits_{X\to+\infty}\int_0^X f(t)\d t$ 
%ont la même nature. 

\bigskip
\Exercice{PTov}

\hrule
\smallskip
\centerline{Série alternée}
\smallskip
\hrule
\medskip

\Exercice{PTaff}


\hrule
\smallskip
\centerline{La base : Equivalents et inégalités}
\smallskip
\hrule

\bigskip
\Exercice{PTpc}
\bigskip
\Exercice{PTpd}
\bigskip
\Exercice{PTou}
\bigskip
\Exercice{PTsl}
\bigskip
\Exercice{PTalz}
\bigskip
\hrule
\smallskip
\centerline{Exercice théorique fun sur les séries}
\smallskip
\hrule
\bigskip
\Exercice{PTux}

\bye
\bigskip
\hrule
\smallskip
\centerline{Trigo}
\smallskip
\hrule
\bigskip
\Exercice{PTpo}






\bigskip
\Exercice{PTagi}
\bigskip
\Exercice{PTpa}





\bigskip
\Exercice{PTaff}
\bigskip


\Exercice{PTagh}
\bigskip
\Exercice{PTal}










\LD@Exo@Theme@Display{2}\LD@List{%
	\TravauxDirigés,\Exercices,\Colles%,\Problèmes,\Others,\Mathematica,\Maple,\LD@Empty
}%

























\centerline{Séries télescopiques}
\hrule
\medskip\noindent

\Propriete [$(u_n)_{n\in\ob N}$ suite complexe]
La série $u_0+\sum_{n=1}^\infty(u_n-u_{n-1})$ converge $\ssi$ la suite 
$(u_n)_{n\in\ob N}$ converge. De plus, en~cas de convergence, on a 
\Equation [\bf Séries télescopiques]
$$
u_0+\sum_{n=1}^\infty(u_n-u_{n-1})=\lim_{n\to\infty}u_n. 
$$
{\it lorsque les sommes partielles sont téléscopiques, on en trouve facilement une expression et la limite}. 

\noindent
\Exercice{PTpm}
\vfill
\noindent
\Exercice{PTpn}
\vfill
\noindent
\Exercice{PTpl}
\vfill

\centerline{Comparaison avec des séries de références}
\hrule
\medskip\noindent
\Propriete [$(a_n)_{n\in\ob N}$ et $(b_n)_{n\in\ob N}$ suites de nombres]
La série $u_0+\sum_{n=1}^\infty(u_n-u_{n-1})$ converge $\ssi$ la suite 
$(u_n)_{n\in\ob N}$ converge. De plus, en~cas de convergence, on a 
\Equation [\bf Séries télescopiques]
$$
u_0+\sum_{n=1}^\infty(u_n-u_{n-1})=\lim_{n\to\infty}u_n. 
$$
{\it lorsque les sommes partielles sont téléscopiques, on en trouve facilement une expression et la limite}. 

\medskip\noindent
\bigskip
\Exercice{PTama}
\bigskip
\Exercice{PTaff}
\bigskip
\Exercice{PTor}
\bigskip
\Exercice{PTot}
\bigskip
\Exercice{PTpj}
\bigskip
\Exercice{PTagh}
\bigskip
\Exercice{PTal}
\bigskip
\hrule
\smallskip
\centerline{Exercices fun/originaux sur les séries}
\smallskip
\hrule
\bigskip
\Exercice{PTagi}
\bigskip
\Exercice{PTux}
\bigskip
\Exercice{PTpa}


\vfill\null











\bye