\catcode`@=11\relax
\def\Api{Mathematicon@Api}%
\input LD@Header.tex
\input LD.tex

\input LD@Exercices.tex
\DefineRGBcolor F0F9E3=VLGreen.
\DefineRGBcolor E5F9D1=LGreen.
\DefineRGBcolor DAF9BE=TGreen.
\DefineRGBcolor 5DA93B=Green.
\DefineRGBcolor F6DCCA=VLRed.
\DefineRGBcolor F6D4BD=LRed.
\DefineRGBcolor DAF9BE=TRed.
\DefineRGBcolor B5F9A1=TTRed.
\DefineRGBcolor F6B080=Red.
\DefineRGBcolor F9F5E3=VLOrange.
\DefineRGBcolor F9F5D0=LOrange.
\DefineRGBcolor DAF9BE=TOrange.
\DefineRGBcolor B5F9A1=TTOrange.
\DefineRGBcolor D7A93B=Orange.
\DefineRGBcolor EEEEEE=VLBlack.
\DefineRGBcolor DDDDDD=LBlack.
\DefineRGBcolor CCCCCC=TBlack.
\DefineRGBcolor B5F9A1=TTBlack.
\DefineRGBcolor 000000=Black.

%\DefineRGBcolor 000000=Green.
%\definecolor{ColorVLGreen}{rgb}{1,1,1}%
%\definecolor{ColorLGreen}{rgb}{1,1,1}%
%\definecolor{ColorTGreen}{rgb}{1,1,1}%
%\expandafter\definecolor\temp
%\DefineRGBcolor 000000=Red.
%\definecolor{ColorVLRed}{rgb}{1,1,1}%
%\definecolor{ColorLRed}{rgb}{1,1,1}%
%\definecolor{ColorTRed}{rgb}{1,1,1}%

\catcode`@=11\relax


%%%%%%%%%%%%%%%%%%%%%%%%%%%%%%%%%%%%%%%%%%%%%%%%%%%%%%%%%%%%%%%%%%
%															%
%						TD 04 : Diagonalisation simultanée					%
%															%
%%%%%%%%%%%%%%%%%%%%%%%%%%%%%%%%%%%%%%%%%%%%%%%%%%%%%%%%%%%%%%%%%%

\vglue-10mm\rightline{PT\hfill TD 4 :  Approfondissement Th\'eorique\hfill}
\bigskip
\bigskip
Les concepts (espaces stables, commutter, ...) et les propri\'et\'es que nous allons utiliser et d\'emontrer au cours de ce TD sont tr\`es importants et permettent de mieux comprendre comment fonctionnent les endomorphismes et les matrices.
\bigskip\bigskip

\centerline{\bf Matrices qui commutent}
\hrule
\bigskip

\Exercice{PTanr}%
\bigskip

\centerline{\bf Diagonalisation dans une m\^eme base}
\hrule
\bigskip

Dans cette section, on admet que deux matrices diagonalisables  $A$ et $B$ qui commuttent peuvent \^etre 
diagonalis\'ees dans une m\^eme base, i.e. qu'il existe une matrice inversible $P$ 
et deux matrices diagonales $D_1$ et $D_2$ telles que 
$$
A=PD_1P^{-1}\qquad \hbox{et}\qquad B=PD_2P^{-1}.
$$
Si $A$ est une matrice carr\'ee inversible diagonalisable sur $\ob C$, on admet \'egalement que $A^2$ et $A^3$ sont diagonalisables sur $\ob C$ (prouv\'e \`a l'exercice 7). 
\hrule
\bigskip
\Exercice{PTft}%
\bigskip
\Exercice{PTfr}%
\bigskip
\Exercice{PTbp}%

\noindent
{\eightpts \it Indication : pour l'implication $\Longrightarrow$, faire intervenir la matrice de Vandermond. } 
\bigskip

\centerline{\bf Espaces propres}
\hrule
\bigskip
\Exercice{PTans}%
\bigskip

\centerline{\bf Espaces stables}
\hrule
\bigskip
\Exercice{PTant}%
\bigskip
\goodbreak
\centerline{\bf 2nde propri\'et\'e admise}
\hrule
\bigskip
\Exercice{PTanq}%
\bigskip
\centerline{\bf 1ere propri\'et\'e admise}
\hrule

\Exercice{PTbr}%
\bye









