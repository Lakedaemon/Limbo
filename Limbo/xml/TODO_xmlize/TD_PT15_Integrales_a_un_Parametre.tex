%\def\Variables{MathsVariables}%
\catcode`@=11\relax
\def\Api{Mathematicon@Api}%
%%%% Newif
%\def\@firstofone#1{#1}

\newif\ifexonumber
%%%% Switches
\exonumberfalse
\catcode`@=11\relax
\input LD@Header.tex
\input LD.tex
\input LD@Typesetting.tex
\input LD@Exercices.tex
\input LD@Exercices.tex
\def\LD@Exercice@Display@Code{\eightpts}%
% debug tikz
\let\@firstofone\pgfutil@firstofone
\let\@ifnextchar\pgfutil@ifnextchar

\olspept
\DefineRGBcolor F0F9E3=VLGreen.
\DefineRGBcolor E5F9D1=LGreen.
\DefineRGBcolor DAF9BE=TGreen.
\DefineRGBcolor 5DA93B=Green.
\DefineRGBcolor F6DCCA=VLRed.
\DefineRGBcolor F6D4BD=LRed.
\DefineRGBcolor DAF9BE=TRed.
\DefineRGBcolor B5F9A1=TTRed.
\DefineRGBcolor F6B080=Red.
\DefineRGBcolor F9F5E3=VLOrange.
\DefineRGBcolor F9F5D0=LOrange.
\DefineRGBcolor DAF9BE=TOrange.
\DefineRGBcolor B5F9A1=TTOrange.
\DefineRGBcolor D7A93B=Orange.
\DefineRGBcolor EEEEEE=VLBlack.
\DefineRGBcolor DDDDDD=LBlack.
\DefineRGBcolor CCCCCC=TBlack.
\DefineRGBcolor B5F9A1=TTBlack.
\DefineRGBcolor 000000=Black.

\def\transparent{%
	\CS long\EC\def\Demonstration##1\CQFD{}%
}%
%\transparent
\def\Students{%
	\DefineRGBcolor FFFFFF=VLGreen.
	\DefineRGBcolor FFFFFF=LGreen.
	\DefineRGBcolor FFFFFF=TGreen.
	\DefineRGBcolor 000000=Green.
	\DefineRGBcolor FFFFFF=VLRed.
	\DefineRGBcolor FFFFFF=LRed.
	\DefineRGBcolor FFFFFF=TRed.
	\DefineRGBcolor FFFFFF=TTRed.
	\DefineRGBcolor 000000=Red.
	\DefineRGBcolor FFFFFF=VLOrange.
	\DefineRGBcolor FFFFFF=LOrange.
	\DefineRGBcolor FFFFFF=TOrange.
	\DefineRGBcolor FFFFFF=TTOrange.
	\DefineRGBcolor 000000=Orange.
	\DefineRGBcolor FFFFFFF=VLBlack.
	\DefineRGBcolor FFFFFF=LBlack.
	\DefineRGBcolor FFFFFF=TBlack.
	\DefineRGBcolor FFFFFF=TTBlack.
	\DefineRGBcolor 000000=Black.
}
\Students
\def\red{}
\def\blue{}
\def\Red#1{#1}%%%% Fix this !
\def\Blue#1{#1}%
\def\Font #1@#2pt{\font\olbi=cmr10\olbi}
\font\SvgText=cmr10\relax
%
%
%\catcode`@=11\relax
%\def\Api{Mathematicon@Api}%
%
%\input LD@Header.tex
%\input LD.tex
%\input LD@Exercices.tex
%\input LD@Typesetting.tex
%
%\catcode`@=11\relax
\font\LD@Font@Arial="Arial" at 10pt
%%%%%%%%%%%%%%%%%%%%%%%%%%%%%%%%%%%%%%%%%%%%%%%%%%%%%%%%%%%%%%%%%%
%															%
%					 Intégrales généralisées à un paramètre					%
%															%
%%%%%%%%%%%%%%%%%%%%%%%%%%%%%%%%%%%%%%%%%%%%%%%%%%%%%%%%%%%%%%%%%%
\newcount\LD@Count@Temp
\def\LD@Exercice@Display@Code{}%%\LD@Option@@Label\qquad\eightpts}%
\def\LD@Exercice@Display@Code@Post{%
	\ifcsname LD@Exo@@Solution\endcsname
		\unless\ifx\LD@Exo@@Solution\LD@Empty
			\pn{\eightpts Solution : \eightpts \LD@Exo@@Solution}%
		\fi
	\fi
}%
\def\LD@Display#1{%
	\LD@Count@Temp=#1\relax
	\ifcase\LD@Count@Temp
	\or
	Math. Sup.
	\or
	Math. Sp\'e
	\else
	\fi
}%
\newcount\LD@Exo@Total\LD@Exo@Total=0\relax

%%% TD 14.
\vglue-10mm\rightline{Sp\'e PT\hfill TD 14 :  Int\'egrales g\'en\'eralis\'ees \`a un param\`etre\hfill}%\date}
\bigskip

{\eightpts
Dans tout ce TD, on va \'etudier une fonction $F$ d\'efinie par une int\'egrale \`a un param\`etre, i.e du type 
$$
\forall x\in I, \qquad F(x):=\int_a^bf(x,t)\d t.
$$
\bigskip

\centerline{Pour prouver qu'une fausse int\'egrale impropre \`a un param\`etre est $\sc C^k$}
\hrule
\medskip\noindent

\Propriete [$k\in\ob N\cup\{\infty\}$, $a<b$ dans $\ob R$, $I$ intervalle]
Si la fonction $f:(t,x)\mapsto f(t,x)$ est de classe $\sc C^k$ sur $[a,b]\times I$, alors l'application 
$$
\eqalign{
F:&I\to\ob C\cr
x&\mapsto\int_a^bf(x,t)\d t}
$$
est de classe~$\sc C^k$ sur~l'intervalle~$I$. De plus, pour $0\le n\le k$, on a  
$$
\forall x\in I, \qquad {\d^n\F\d x^n}\underbrace{\int_a^bf(t,x)\d t}_{F(x)}=\int_a^b{\partial^n f\F\partial x^n}(t,x)\d t. 
$$

\centerline{Pour prouver qu'une int\'egrale impropre \`a un param\`etre est continue}
\hrule
\medskip\noindent

\Theoreme [$a<b$ dans $\ol{\ob R}$, $I$ intervalle, $f:(t,x)\mapsto f(t,x)$ application]  
Si l'application $t\mapsto f(t,x)$ est continue  par morceaux sur $]a,b[$ pour chaque $x\in I$ fix\'e, \pn
si l'application $x\mapsto f(t,x)$ est continue sur $I$ pour chaque nombre r\'eel $t\in[a,b]$ fix\'e et \pn
s'il existe  $g:]a,b[\to\ob R^+$ continue par morceaux telle que $\int_a^bg(t)\d t$ converge et telle que 
$$
\big|f(t,x)\big|\le g(t)\qquad(a<t<b, x\in I), \leqno{(\hbox{\it hypoth\`ese de domination de } f)}
$$
alors l'int\'egrale $F(x):=\int_a^bf(t,x)\d t$ converge absolument pour chaque nombre $x\in I$ et 
la fonction $F:I\to\ob C$ est continue sur $I$. En particulier, pour chaque $x_0\in I$, on a  
\Equation [\hbox{\bf Convergence domin\'ee}] 
$$
\lim_{x\to x_0}\underbrace{\int_a^bf(t,x)\d t}_{F(x)}=\underbrace{\int_a^b\lim_{x\to x_0}f(t,x)\d t}_{F(x_0)}. 
$$

\bigskip

\centerline{Pour prouver qu'une int\'egrale impropre \`a un param\`etre est de classe $\sc C^k$}
\hrule
\medskip\noindent

\Theoreme  [$k\in\ob N^*$, $a<b$ dans $\ol{\ob R}$, $I$ intervalle, $f:(t,x)\mapsto f(t,x)$ application]  
Si l'application $x\mapsto f(x,t)$ est de classe $\sc C^k$ sur $I$ pour $t\in[a,b]$ fix\'e,  \pn
si  $t\mapsto\ds{\partial f^n\F\partial x^n}(x,t)$ est $\sc C^0$ P.M. et si $\int_a^b\Q|{\partial f^n\F\partial x^n}(x,t)\W|\d t$ converge  pour $x\in I$ et $0\le n\le k$ fix\'es et 
s'il~existe   $g:]a,b[\to\ob R^+$continue par morceaux telle que~$\int_a^bh(t)\d t$ converge et telle que 
$$
\Q|{\partial f^k\F\partial x^k}(t,x)\W|\le g(t)\qquad(a<t<b, x\in I), \leqno{(\hbox{\it hypoth\`ese de domination de }{\partial f^k\F\partial x^k})}
$$
alors l'int\'egrale $F(x):=\int_a^bf(t,x)\d t$ converge absolument pour chaque nombre $x\in I$ et 
la fonction $F:I\to\ob C$ est de classe~$\sc C^1$ sur $I$ et v\'erifie
\Equation 
$$
\forall n\in\{0, \cdots, k\}, \quad \forall x\in I, \qquad {\d^n\F\d x^n}\underbrace{\int_a^bf(t,x)\d t}_{F(x)}=\int_a^b{\partial^n f\F\partial x^n}(t,x)\d t. 
$$

}

\eject
\centerline{\fourteenbf Probl\`eme}
\bigskip
\medskip \noindent
On pose $\ds A(x):=\int_0^\infty{\sin^2(tx)\F t^2}\d t$ et $\ds F(x):=\int_0^\infty{\sin^2(tx)\F t^2(1+t^2)}\d t$. 
\medskip
\noindent 1) D\'emontrer que $A$ et $F$ sont d\'efinies sur $\ob R$. 
\medskip
\noindent 2) Donner une expression de $A(x)$ en fonction de $x$ et de $A(1)$. 
\medskip
\noindent 3) En \'etudiant $F(x)-A(x)$, donner un \'equivalent de $F$ en $+\infty$ et en $-\infty$. 
\medskip
\noindent 4) Montrer rexpectivement que $F$ est continue, de classe $\sc C^1$, de classe $\sc C^2$ sur $\ob R$ et 
exprimer les d\'eriv\'ees $F'(x)$ et $F''(x)$ \`a l'aide d'int\'egrales impropres. 
\medskip
\noindent 5) Exprimer $F(x)-A(x)$ en fonction de $F''(x)$. 
\medskip
\noindent 6) En d\'eduire que $F$ est solution d'une \'equation diff\'erentielle 
du type $y''-4y=a|x|+b$ o\`u $a$ et $b$ sont des constantes r\'eelles qu'on d\'eterminera. 
\medskip
\noindent 6) D\'eterminer $F(x)$ au moyen des fonctions \'el\'ementaires et du nombre $A(1)$. 
\medskip
\noindent 7) Que vaut $A(1)$ ? 
\vfill
\centerline{\fourteenbf Exercices}
\bigskip\bigskip \noindent
\Exercice{PTjz}
\vfill
\Exercice{PTw}
\vfill
\Exercice{PTka}
\vfill
\bye