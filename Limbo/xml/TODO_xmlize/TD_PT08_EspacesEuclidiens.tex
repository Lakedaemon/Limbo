\catcode`@=11\relax

\input LD@Maths@TD.tex
\vglue-10mm\rightline{Spé PT\hfill TD 8 : Espaces euclidiens 2\hfill}%\date}
\bigskip

\Exercice{PTacq}%
\bigskip

\Exercice{PTvr}%
\bigskip

\Exercice{PTtr}%
\bigskip

\Exercice{PTvw}%

\bigskip
\centerline{$\underline{\hbox{\fourteenbf Propriété méga-importante du cours}}$}%
\bigskip

\noindent
Un endomorphisme symétrique $u:E\to E$ est diagonalisable dans une base orthonormale, i.e. 
il au moins une base orthonormale constituée de vecteurs propres de $u$. 
\medskip
\noindent
Une matrice symétrique réelle $A$ est diagonalisable dans  une base orthonormale, autrement dit :  il~existe une matrice orthogonale $P$ et une matrice diagonale $D$ telle que 
$$
A=PDP^{-1}=PD^{\hbox{\rm t}}P. 
$$


\Exercice{PTww}%
\bigskip

\Exercice{PTxg}%
\bigskip
\bye