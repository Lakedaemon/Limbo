\catcode`@=11\relax

\input LD@Maths@TD.tex

\vglue-10mm\rightline{PTSI/PT\hfill TD :  Courbes paramétrées\hfill}
\bigskip

\overfullrule0pt
\Methode [Courbe paramétrée cartésienne {$\vec {OM}(t) = x(t)\vec i+y(t)\vec j$}]
\medskip\noindent
1) Recherche de l'ensemble de définition, de dérivabilité de $x$ et $y$. \PAR\noindent
2) Etude de la périodicité et de la parité (symétries de la courbe). \PAR\noindent 
3) Etudier le signe des dérivées $x'(t)$ et $y'(t)$ pour dresser un tableau de variation. \PAR\noindent
4) Compléter le tableau de variation en faisant apparaitre points singuliers, tangentes, asymptotes,
branches paraboliques, points doubles...\PAR\noindent
5) Tracer la courbe.

\Exercice{PTSIet}
\def\LD@Maths@Solution@Text{%
%	{\bf Corrigé de l'exercice \refn{labelexo\LD@Maths@Label@Internal}.}\PAR
}%
\Solution{PTSIet}

\Exercice{PTSIey}
\Solution{PTSIey}


\Methode [Branche infinie en $t_0$ d'une courbe  paramétrée cartésienne {$\vec {OM} (t)= x(t)\vec i+y(t)\vec j$}]
\medskip\noindent'
1) $\ds\lim_{t\to t_0}x(t)=\pm\infty$ et $\ds\lim_{t\to t_0}y(t)=a\in\ob R$ : asymptote horizontale~$y=a$. 
\PAR\noindent
2) $\ds\lim_{t\to t_0}x(t)=a\in\ob R$ et $\ds\lim_{t\to t_0}y(t)=\pm\infty$ : asymptote verticale~$x=a$. 
\PAR\noindent
\smallskip\noindent\hrule\smallskip
\centerline{$\ds\lim_{t\to t_0}x(t)=\pm\infty$ et $\ds\lim_{t\to t_0}y(t)=\pm\infty$}
\PAR\noindent\hrule\smallskip\noindent
3a) $\ds \lim_{t\to t_0}{y(t)\F x(t)}=0$, branche parabolique de direction $(Ox)$. 
\PAR\noindent
3b) $\ds \lim_{t\to t_0}{y(t)\F x(t)}=\pm\infty$ : branche parabolique de direction $(Oy)$.
\PAR\noindent
3c) $\ds \lim_{t\to t_0}{y(t)\F x(t)}=a\neq0$ et $\ds\lim_{t\to t_0}\b(y(t)-ax(t)\b)=b\in\ob R$ : asymptote d'équation $y=ax+b$. 
\PAR\noindent
3d) $\ds \lim_{t\to t_0}{y(t)\F x(t)}=a\neq0$ et $\ds\lim_{t\to t_0}\b(y(t)-ax(t)\b)=\pm\infty$, : branche  parabolique de direction $y=ax$. 
\medskip

\Exercice{PTSIeu}
\Solution{PTSIeu}

\Methode [Courbe paramétrée polaire {$\vec {OM}(\theta) = \rho(\theta)\vec u$}]
\medskip\noindent
1) Recherche de l'ensemble de définition, de dérivabilité de $\rho$. \PAR\noindent 
2) Etude de la périodicité et de la parité de $\rho$ (symétries de la courbe). \PAR\noindent 
3) Résoudre l'équation $\rho(\theta)=0$ (passage au pôle).\PAR\noindent 
4) Etudier le signe des dérivées pour dresser un tableau de variation. \PAR\noindent 
5) Compléter le tableau de variation en faisant apparaitre  tangentes, points singuliers, asymptotes,
branches paraboliques (utiliser le repère polaire), points doubles.\PAR\noindent 
6) Tracer la courbe polaire.

\Methode [Branche infinie en $\theta_0$ d'une courbe polaire {$\vec {OM}(\theta) = \rho(\theta)\vec u$}]
\PAR\noindent
1) $\theta_0=\pm\infty$ et $\ds\lim_{\theta\to\theta_0}\rho(\theta)=\pm\infty$ : branche spirale.
\PAR\noindent
2) $\theta_0=\pm\infty$ et $\ds\lim_{\theta\to\theta_0}\rho(\theta)=a\in\ob R$ : 
cercle asymptote de centre $0$ et de rayon $|a|$. 
\smallskip\noindent\hrule\smallskip
\centerline{$\theta_0\in\ob R$ et $\ds\lim_{\theta\to\theta_0}\rho(\theta)=\pm\infty$ :}
\PAR\noindent\hrule\smallskip\noindent
3a) $\ds\lim_{\theta\to\theta_0}\rho(\theta)\sin(\theta-\theta_0)=\pm\infty$ :  branche parabolique de direction $\vec u(\theta_0)$. 
\PAR\noindent
3b) $\ds\lim_{\theta\to\theta_0}\rho(\theta)\sin(\theta-\theta_0)=a\in\ob R$ : asymptote dirigée par $\vec u(\theta_0)$ et passant par $P$ avec $\vec{OP}=a\vec v(\theta_0)$. 

\Exercice{PTabm}
\Solution{PTabm}

\Exercice{PTabo}
\Solution{PTabo}

\Exercice{PTabp}

%\Solution{PTabp}% bad solution

%\Solution{PTSIfa}
%\Solution{PTSIfb}
%\Solution{PTSIez}
\bye