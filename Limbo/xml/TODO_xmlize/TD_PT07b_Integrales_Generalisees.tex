\catcode`@=11\relax

\input LD@Maths@TD.tex

\vglue-10mm\rightline{Spé PT\hfill TD 6 :  Intégrales généralisées\hfil }


\Chapter Outils, Nature des intégrales généralisées.

\Section 1, Algorithme d'étude.

\overfullrule0pt%
\Methode [Nature d'une intégrale $\int_a^bf(t)\d t$]
\noindent A) Repérer les points à problème potentiel en répondant à la question : 
$$\hbox{
Pour quels $t\in[a,b]$, la quantité $f(t)$ est-elle continue ou prolongeable par continuité ? 
}
$$
B) Trouver la nature des petites intégrales avec un seul point à problème potentiel via : 
\PAR\noindent
1) Une primitive
\PAR\noindent
2) Un changement de variable
\PAR\noindent
3) Une intégration par partie
\PAR\noindent
4) La convergence absolue (permet souvent de se ramener à une fonction de signe constant)
\PAR\noindent
5) Lorsque $f(t)$ est de signe constant autour du point à problème potentiel $b$, on utilise ausi 
\PAR
a) Un équivalent $\sim$ de $f(t)$ en $b$.
\PAR
b) Une inégalité avec $\le$. 
\PAR
b') Une inégalité avec $o$ en $b$.
\PAR\noindent
6) Chasles
\PAR\noindent
7) L'addition

\Section 2, Changement de variable.

\medskip
\Theoreme [$a<b$ dans $\overline{\ob R}$, $\varphi$ difféomorphisme de classe $\sc C^1$ de {$\Q]a,b\W[$} dans un intervalle $I$]
Si $f:I\to\ob C$ est continue par morceaux sur $I$. 
Alors $\int_{\varphi(a)}^{\varphi(b)}f(x)\d x$ et $\int_a^bf\big(\varphi(t)\big)\varphi'(t)\d t$ 
ont même nature. En cas de convergence, on a 
\Equation [\bf Changement de variable]
$$
\int_{\varphi(a)}^{\varphi(b)}f(x)\d x=\int_a^bf\big(\varphi(t)\big)\varphi'(t)\d t.
$$

\Section 3, Intégration par partie.

\medskip
\Theoreme[$a$ et $b$ dans $\ol{\ob R}$, $f$ et $g$ deux fonctions de classe $\sc C^1$ sur {$\Q]a,b\W[$}] 
Si deux des trois nombres $\int_a^bf'(t)g(t)\d t$, $\int_a^bf(t)g'(t)\d t$ et 
$$
\Big[f(t)g(t)\Big]_a^b:=\lim\limits_{t\to b}\Big(f(t)g(t)\Big)-\lim\limits_{t\to a}\Big(f(t)g(t)\Big)
$$
sont définis, alors le troisième l'est aussi et l'on a 
\Equation [\bf Intégration par partie]
$$
\int_a^bf'(t)g(t)\d t=\Big[f(t)g(t)\Big]_a^b-\int_a^bf(t)g'(t)\d t. 
$$

\Section 4, Convergence absolue.

\medskip
\Propriete [$(a,b)\in\ol{\ob R}^2$ et {$f:\Q]a,b\W[\to\ob C$ continue par morceaux sur~$\Q]a,b\W[$}]
Si l'intégrale $\int_a^b\big|f(t)\big|\d t$ converge, alors l'intégrale $\int_a^bf(t)\d t$ converge et l'on a 
\Equation [\bf convergence absolue]
$$
\Q|\int_a^bf(t)\d t\W|\le \int_a^b\b|f(t)\b|\d t. 
$$


\Subsection 41, Intégrales absolument-convergentes.

\Propriete L'intégrale $\int_a^bf(t)\d t$ converge absolument $\ssi$ l'intégrale $\int_a^b\big|f(t)\big|\d t$ converge. 

\Subsection 42, Intégrales semi-convergentes.

\Propriete L'intégrale $\int_a^bf(t)\d t$ est ``semi-convergente'' $\ssi$ $\int_a^bf(t)\d t$ converge et $\int_a^b\big|f(t)\big|\d t$ diverge.


\Section 5, Outils de comparaison.

\Subsection 51, Equivalents.

\Theoreme [$a\in\ob R$ et $b>a$ un élément de $\ol R$] 
Soient $f$ et $g$ des fonctions continues par morceaux et {\bf positives} sur $\Q[a,b\W[$ telles que 
$$
f(x)\mathop{\sim}\limits_bg(x).
$$ 
Alors,~$\int_a^bf(t)\d t$ et $\int_a^bg(t)\d t$ ont même nature. 

\Subsection 52, Inégalités.

\Propriete [$a\in\ob R$, $b>a$ un élément de $\ol R$] 
Soient $f:\Q[a,b\W[\to\ob R$ et $g:\Q[a,b\W[\to\ob R$ deux applications continues par morceaux avec  
$$
0\le f(x)\le g(x)\qquad (a\le x<b). 
$$
Si $\int_a^bf(t)\d t$ diverge, alors $\int_a^bg(t)\d t$ diverge. \medskip\noindent
Si $\int_a^bg(t)\d t$ converge alors $\int_a^bf(t)\d t$ converge et l'on a 
\Equation [\bf Intégration des inégalités]
$$
0\le \int_a^bf(t)\d t\le \int_a^bg(t)\d t. 
$$ 

\Subsection 53, Fonctions négligeables.

\Theoreme 
[$a\in\ob R$ et $b>a$ élément de $\ol R$]
Soient  $f$ et $g$ des fonctions continues par morceaux et {\bf positives} sur $\Q[a,b\W[$ telles que 
$$
f(x)=o_b\b(g(x)\b).
$$ 
Si $\int_a^b f(t)\d t$ diverge, alors $\int_a^bg(t)\d t$ diverge. \pn
Si $\int_a^b g(t)\d t$ converge, alors $\int_a^bf(t)\d t$ converge.  

\goodbreak

\Chapter Exos, Exercices.

\Section 1, Comparaison.

\Exercice{PTacr}
\vfill
\Exercice{PTuh}
\vfill
\Exercice{PTug}
\vfill
\Exercice{PTuf}
\vfill
\Exercice{PTtp}
\vfill
\Exercice{PTlj}
\vfill
\Exercice{PTapl}
\vfill
\Exercice{PTaix}
\vfill
\Exercice{PTaiu}
\vfill
\Exercice{PTapm}
\vfill
\Exercice{PTapk}

\Section 1, Primitives.

\Exercice{PTagf}
\vfill
\Exercice{PTapi}
\vfill
\Exercice{PTaph}
\vfill
\Exercice{PTapt}

\Section 1, Changerment de variable.

\Exercice{PTapo}
\vfill
\Exercice{PTapn}
\vfill
\Exercice{PTalu}







% 
\Chapter Indications, Indications.
% 
 \LD@Exo@Indication@Display
% 
%\Chapter Notions, Notions.
% 
\LD@Exo@Notion@Display

\Chapter Solutions, Solutions.

\LD@Exo@Sol@Display

\bye