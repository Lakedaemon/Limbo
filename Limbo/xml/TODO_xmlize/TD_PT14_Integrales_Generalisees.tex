%\def\Variables{MathsVariables}%
\catcode`@=11\relax
\def\Api{Mathematicon@Api}%
%%%% Newif
%\def\@firstofone#1{#1}

\newif\ifexonumber
%%%% Switches
\exonumberfalse
\catcode`@=11\relax
\input LD@Header.tex
\input LD@Library.tex
\input LD@Typesetting.tex
\input LD@Exercices.tex
\def\LD@Exercice@Display@Code{\eightpts}%
% debug tikz
\let\@firstofone\pgfutil@firstofone
\let\@ifnextchar\pgfutil@ifnextchar

\olspept
\DefineRGBcolor F0F9E3=VLGreen.
\DefineRGBcolor E5F9D1=LGreen.
\DefineRGBcolor DAF9BE=TGreen.
\DefineRGBcolor 5DA93B=Green.
\DefineRGBcolor F6DCCA=VLRed.
\DefineRGBcolor F6D4BD=LRed.
\DefineRGBcolor DAF9BE=TRed.
\DefineRGBcolor B5F9A1=TTRed.
\DefineRGBcolor F6B080=Red.
\DefineRGBcolor F9F5E3=VLOrange.
\DefineRGBcolor F9F5D0=LOrange.
\DefineRGBcolor DAF9BE=TOrange.
\DefineRGBcolor B5F9A1=TTOrange.
\DefineRGBcolor D7A93B=Orange.
\DefineRGBcolor EEEEEE=VLBlack.
\DefineRGBcolor DDDDDD=LBlack.
\DefineRGBcolor CCCCCC=TBlack.
\DefineRGBcolor B5F9A1=TTBlack.
\DefineRGBcolor 000000=Black.

\def\transparent{%
	\CS long\EC\def\Demonstration##1\CQFD{}%
}%
%\transparent
\def\Students{%
	\DefineRGBcolor FFFFFF=VLGreen.
	\DefineRGBcolor FFFFFF=LGreen.
	\DefineRGBcolor FFFFFF=TGreen.
	\DefineRGBcolor 000000=Green.
	\DefineRGBcolor FFFFFF=VLRed.
	\DefineRGBcolor FFFFFF=LRed.
	\DefineRGBcolor FFFFFF=TRed.
	\DefineRGBcolor FFFFFF=TTRed.
	\DefineRGBcolor 000000=Red.
	\DefineRGBcolor FFFFFF=VLOrange.
	\DefineRGBcolor FFFFFF=LOrange.
	\DefineRGBcolor FFFFFF=TOrange.
	\DefineRGBcolor FFFFFF=TTOrange.
	\DefineRGBcolor 000000=Orange.
	\DefineRGBcolor FFFFFFF=VLBlack.
	\DefineRGBcolor FFFFFF=LBlack.
	\DefineRGBcolor FFFFFF=TBlack.
	\DefineRGBcolor FFFFFF=TTBlack.
	\DefineRGBcolor 000000=Black.
}
\Students
\def\red{}
\def\blue{}
\def\Red#1{#1}%%%% Fix this !
\def\Blue#1{#1}%
\def\Font #1@#2pt{\font\olbi=cmr10\olbi}
\font\SvgText=cmr10\relax
%
%
%\catcode`@=11\relax
%\def\Api{Mathematicon@Api}%
%
%\input LD@Header.tex
%\input LD.tex
%\input LD@Exercices.tex
%\input LD@Typesetting.tex
%
%\catcode`@=11\relax
\font\LD@Font@Arial="Arial" at 10pt
%%%%%%%%%%%%%%%%%%%%%%%%%%%%%%%%%%%%%%%%%%%%%%%%%%%%%%%%%%%%%%%%%%
%															%
%					Intégrales généralisées							%
%															%
%%%%%%%%%%%%%%%%%%%%%%%%%%%%%%%%%%%%%%%%%%%%%%%%%%%%%%%%%%%%%%%%%%
\newcount\LD@Count@Temp
\def\LD@Exercice@Display@Code{}%%\LD@Option@@Label\qquad\eightpts}%
\def\LD@Exercice@Display@Code@Post{%
	\ifcsname LD@Exo@@Solution\endcsname
		\unless\ifx\LD@Exo@@Solution\LD@Empty
			\pn{\eightpts Solution : \eightpts \LD@Exo@@Solution}%
		\fi
	\fi
}%
\def\LD@Display#1{%
	\LD@Count@Temp=#1\relax
	\ifcase\LD@Count@Temp
	\or
	Math. Sup.
	\or
	Math. Sp\'e
	\else
	\fi
}%
\newcount\LD@Exo@Total\LD@Exo@Total=0\relax



\Theoreme[$a$ et $b$ dans $\ol{\ob R}$, $f$ et $g$ deux fonctions de classe $\sc C^1$ sur {$\Q]a,b\W[$}] 
Si deux des trois nombres $\int_a^bf'(t)g(t)\d t$, $\int_a^bf(t)g'(t)\d t$ et 
$$
\Big[f(t)g(t)\Big]_a^b:=\lim\limits_{t\to b}\Big(f(t)g(t)\Big)-\lim\limits_{t\to a}\Big(f(t)g(t)\Big)
$$
sont d\'efinis, alors le troisi\`eme l'est aussi et l'on a 
\Equation [\bf Int\'egration par partie]
$$
\int_a^bf'(t)g(t)\d t=\Big[f(t)g(t)\Big]_a^b-\int_a^bf(t)g'(t)\d t. 
$$

\medskip
\Theoreme [$a<b$ dans $\overline{\ob R}$, $\varphi$ diff\'eomorphisme de classe $\sc C^1$ de {$\Q]a,b\W[$} dans un intervalle $I$]
Si $f:I\to\ob C$ est continue par morceaux sur $I$. 
Alors $\int_{\varphi(a)}^{\varphi(b)}f(x)\d x$ et $\int_a^bf\big(\varphi(t)\big)\varphi'(t)\d t$ 
ont m\^eme nature. En cas de convergence, on a 
\Equation [\bf Changement de variable]
$$
\int_{\varphi(a)}^{\varphi(b)}f(x)\d x=\int_a^bf\big(\varphi(t)\big)\varphi'(t)\d t.
$$

\medskip
\Propriete L'int\'egrale $\int_a^bf(t)\d t$ est ``semi-convergente'' $\ssi$ $\int_a^bf(t)\d t$ converge et $\int_a^b\big|f(t)\big|\d t$ diverge.

\medskip
\Propriete L'int\'egrale $\int_a^bf(t)\d t$ converge absolument $\ssi$ l'int\'egrale $\int_a^b\big|f(t)\big|\d t$ converge. 

\medskip
\Propriete [$(a,b)\in\ol{\ob R}^2$ et {$f:\Q]a,b\W[\to\ob C$ continue par morceaux sur~$\Q]a,b\W[$}]
Si l'int\'egrale $\int_a^b\big|f(t)\big|\d t$ converge, alors l'int\'egrale $\int_a^bf(t)\d t$ converge et l'on a 
\Equation [\bf convergence absolue]
$$
\Q|\int_a^bf(t)\d t\W|\le \int_a^b\b|f(t)\b|\d t. 
$$

\medskip
\Theoreme [$a\in\ob R$ et $b>a$ un \'el\'ement de $\ol R$] 
Soient $f$ et $g$ des fonctions continues par morceaux et {\bf positives} sur $\Q[a,b\W[$ telles que 
$$
f(x)\mathop{\sim}\limits_bg(x).
$$ 
Alors,~$\int_a^bf(t)\d t$ et $\int_a^bg(t)\d t$ ont m\^eme nature. 

\medskip
\Propriete [$a\in\ob R$, $b>a$ un \'el\'ement de $\ol R$] 
Soient $f:\Q[a,b\W[\to\ob R$ et $g:\Q[a,b\W[\to\ob R$ deux applications continues par morceaux sur~$\Q[a,b\W[$ telles que 
$$
0\le f(x)\le g(x)\qquad (a\le x<b). 
$$
Si $\int_a^bf(t)\d t$ diverge, alors $\int_a^bg(t)\d t$ diverge. \medskip\noindent
Si $\int_a^bg(t)\d t$ converge alors $\int_a^bf(t)\d t$ converge et l'on a 
\Equation [\bf Int\'egration des in\'egalit\'es]
$$
0\le \int_a^bf(t)\d t\le \int_a^bg(t)\d t. 
$$ 

\medskip
\Theoreme 
[$a\in\ob R$ et $b>a$ \'el\'ement de $\ol R$]
Soient  $f$ et $g$ des fonctions continues par morceaux et {\bf positives} sur $\Q[a,b\W[$ telles que 
$$
f(x)=o_b\b(g(x)\b).
$$ 
Si $\int_a^b f(t)\d t$ diverge, alors $\int_a^bg(t)\d t$ diverge. \pn
Si $\int_a^b g(t)\d t$ converge, alors $\int_a^bf(t)\d t$ converge.  

\eject
%\centerline{semie et absolue convergence}
%\hrule
%\medskip
%\vskip -2cm


%%% TD 14. 
\vglue-10mm\rightline{Sp\'e PT\hfil TD 14 : Int\'egrales g\'en\'eralis\'ees\hfil }

\bigskip\bigskip
\Exercice{PTkx}
\vfill
\Exercice{PTkw}
\vfill
\Exercice{PTky}
\vfill
\Exercice{PTkz}
\vfill
\Exercice{PTla}
\vfill
%\Exercice{PTlb}
%\vfill
\Exercice{PTlc}
\vfill
%\Exercice{PTld}
%\vfill
%\Exercice{PTle}
%\vfill
\Exercice{PTlf}
\vfill
\Exercice{PTlh}
\vfill
\Exercice{PTli}
\vfill
\Exercice{PTlj}
\vfill
\Exercice{PTkq}
\vfill
\Exercice{PTkr}
\vfill
\Exercice{PTks}


\bye