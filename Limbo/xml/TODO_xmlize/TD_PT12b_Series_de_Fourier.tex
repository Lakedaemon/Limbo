\catcode`@=11\relax

\input LD@Maths@TD.tex

\vglue-10mm\rightline{Spé PT\hfill TD : Séries de Fourier\hfill}
\bigskip
\vfill

\Concept Coefficients de Fourier $a_n$ et $b_n$. 

\Definition [$T>0$, $f:\ob R\to\ob R$ continue par morceaux et $T$-périodique] 
$$
\eqalign{
a_0(f)&:={1\F T}\int_0^Tf(x)\d x
\cr
a_n(f)&:={2\F T}\int_0^Tf(x)\cos(n\omega x)\d x\qquad(n\in\ob N^*)
\cr
b_n(f)&:={2\F T}\int_0^Tf(x)\sin(n\omega x)\d x\qquad(n\in\ob N^*)
}
$$

\Concept Théorème de Parseval

\Theoreme [$f:\ob R\to\ob R$ fonction $T$ périodique et continue par morceaux] 
Les séries $\sum_{n=1}^\infty\b|a_n(f)\b|^2$ et $\sum_{n=1}^\infty\b|b_n(f)\b|^2$ convergent et l'on a 
$$
{1\F T}\int_0^T\b|f(x)\b|^2\d x=\b|a_0(f)\b|^2+{1\F 2}\sum_{n=1}^\infty\b|a_n(f)\b|^2
+{1\F 2}\sum_{n=1}^\infty\b|b_n(f)\b|^2. 
$$

\Concept Théorème de Dirichlet

\Theoreme [$T>0$, $f:\ob R\to\ob R$ fonction $T$-périodique de classe $\sc C^1$ par morceaux]
Pour chaque nombre réel $x$, la série numérique $S[f](x)$ converge et l'on a 
\Equation [\bf Dirichlet]
$$
\underbrace{f(x^-)+f(x^+)\F2}_{\tilde f(x)}=\underbrace{a_0+\sum_{n=1}^\infty\B(a_n(f)\cos(n\omega x)+b_n(f)\sin(n\omega x)\B)}
_{S[f](x)}
$$

\Concept Théorème de convergence normale des séries de Fourier

\Theoreme [$f:\ob R\to\ob R$ fonction $T$-périodique, {\bf continue} et de classe $\sc C^1$ par morceaux] 
Pour chaque nombre réel $x$, la série numérique $S[f](x)$ converge et 
$$
f(x)=\underbrace{a_0+\sum_{n=1}^\infty\B(a_n(f)\cos(n\omega x)+b_n(f)\sin(n\omega x)\B)}
_{S[f](x)}.
$$
On peut intégrer terme à terme $S$ sur chaque segment. \PAR\noindent
Enfin, les séries numériques $\ds\sum_{n=0}^\infty\b|a_n(f)\b|$ 
et $\ds\sum_{n=1}^\infty\b|b_n(f)\b|$ convergent. 

\goodbreak
\Exercice{PTaou}
\vfill
\Exercice{PTtf}
\vfill
\Exercice{PTsv}
\vfill
\Exercice{PTtg}
\vfill
\Exercice{PTtd}
\vfill
\Exercice{PTte}
\bye