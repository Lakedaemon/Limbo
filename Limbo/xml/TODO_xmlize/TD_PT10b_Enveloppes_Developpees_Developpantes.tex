\catcode`@=11\relax

\input LD@Maths@TD.tex

\vglue-10mm\rightline{Spé PT\hfill TD 10 :  Enveloppes, développées, dévelopantes\hfill}
\bigskip

\Exercice{PTaab}

\Exercice{PTaaa}

\Exercice{PTaap}

\Chapter dev, Développée. 

\Theoreme La développée d'une courbe paramétrée est l'enveloppe 
de ses droites normales. 

\Theoreme La développée d'une courbe paramétrée $OM(t)\ (t\in I)$ est le lieu de ses centres de courbures. 
$$
\vec{OC}(t)=\vec{OM}(t)+r_c(t)\vec N(t)\qquad(t\in I). 
$$

\Exercice{PTaet}

\Exercice{PTaan}

\Chapter dev, Développante. 
\bigskip

\Theoreme [$f:I\to\sc P$ arc plan birégulier de classe $\sc C^2$ et $O\in\sc P$] 
Les développantes d'une courbe paramétrée $\vec{OM}(t)\ (t\in I)$ sont les courbes paramétrées
$$
\vec {OP(t)}=\vec{OM}(t)+(k-s(t))\vec T
$$ 
où $s(t)$ est l'abscisse curviligne et $k$ est une constante. 

\Exercice{PTaes}

\Exercice{PTabl}

\bye
