\catcode`@=11\relax

\input LD@Maths@TD.tex

\vglue-10mm\centerline{Spé PT\hfill TD : Fonctions de classe $\sc C^k$\hfill}
\bigskip

\hrule
\centerline{Comment montrer qu'une fonction $f$ est de classe $\sc C^0$ sur un domaine $D$ ?}
\hrule
\bigskip
\noindent
A) Trouver le plus grand ensemble $E\subset D$ sur lequel $f$ peut être obtenu en faisant des sommes, différences, produits, quotients dont le dénominateur ne s'annule pas, ou compositions de fonctions de classe~$\sc C^1$. 
\medskip
\noindent
B) {\bf Si $D\neq E$}, prouver que $f$ est continue en chaque point à problème $a\in D\smallsetminus E$, en montrant que  
$$
\lim_{X\to a\atop X\neq a}f(X)=f(a).
$$

\hrule
\centerline{Comment montrer qu'une fonction $f$ est de classe $\sc C^1$ sur un domaine $D$ ?}
\hrule

\bigskip\noindent
I) Trouver le plus grand ensemble $E\subset D$ sur lequel $f$ peut être obtenu en faisant des sommes, différences, produits, quotients dont le dénominateur ne s'annule pas, ou compositions de fonctions de classe~$\sc C^1$, 
\medskip\noindent
II) {\bf Si $D\neq E$}, effectuer en chaque point à problème $a\in D\smallsetminus E$ les opérations suivantes : \pn
i) prouver que $f$ est $\sc C^0$ sur $D$ (faire l'étape B) de l'algorythme précédent). \pn
ii) Montrer que les dérivées partielles de $f$ existent en $a$ et les calculer en faisant des taux d'accroissement :
$$
{\partial f\F\partial x}(a)=\lim_{h\to 0}{f(a+(h,0))-f(a)\F h}\qquad {\partial\F\partial y}(a)=\lim_{h\to 0}{f(a+(0,h))-f(a)\F h}. 
$$
iii) Calculer les dérivées partielles de $f$, à l'aide des formules traditionnelles de dérivation en un point $X\in D$. 
Puis, prouver que ces dérivées partielles sont continues en $a$, en montrant que  
$$
\lim_{X\to a\atop X\neq a}{\partial f\F\partial x}(X)={\partial f\F\partial x}(a)\qquad \hbox{et} \lim_{X\to a\atop X\neq a}{\partial f\F\partial y}(X)={\partial f\F\partial y}(a)
$$

\hrule
\centerline{Comment montrer qu'une fonction $f$ est de classe $\sc C^k$ sur un domaine $D$ avec $k>1$ ?}
\hrule

\bigskip\noindent
1) Trouver le plus grand ensemble $E\subset D$ sur lequel $f$ peut être obtenu en faisant des sommes, différences, produits, quotients dont le dénominateur ne s'annule pas, ou compositions de fonctions de classe~$\sc C^k$. 
\medskip\noindent
{\bf Si $D\neq E$}, effectuer en chaque point à problème $a\in D\smallsetminus E$ les opérations suivantes : \pn
2) Prouver que $f$ est de classe $\sc C^0$ (en effectuant l'étape B). \pn
3) prouver que $f$ est de clase $\sc C^1$ (en effectuant les étapes i, ii et iii). \pn
4) prouver que $f$ est $\sc C^2$ (en prouvant à l'aide des étapes i, ii et iii que ${\partial f\F\partial x}$ et  ${\partial f\F\partial x}$ sont de classe $\sc C^1$).\pn
5) prouver que $f$ est $\sc C^3$ (en prouvant que les dérivées secondes sont de classe $\sc C^1$).\pn
6) ...etc...\pn
On ne vous donnera jamais d'exercice faisant intervenir les étapes 5+. 
\medskip
\hrule
\centerline{Exercices}
\hrule
\bigskip
\Exercice{PTir}\par\noindent
({\it D'après le théorème de Schwarz ${\partial^2f\F\partial x\partial y}(a)={\partial^2f\F\partial y\partial x}(a)$ si $f$ est de classe $\sc C^2$ autour de $a$}).
\goodbreak
\Exercice{PTaoo}
\vfill
\Exercice{PTis}
\vfill
\Exercice{PTaon}
\vfill
\Exercice{PTcv}
\vfill
\Exercice{PTgs}
\vfill
\Exercice{PTanh}
\vfill
\Exercice{PTani}
\vfill
\Exercice{PTanj}
\vfill
\Exercice{PTjd}
\vfill\null\eject
\bye
