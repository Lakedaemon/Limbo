%\def\Variables{MathsVariables}%
\catcode`@=11\relax
\def\Api{Mathematicon@Api}%
%%%% Newif
%\def\@firstofone#1{#1}

\newif\ifexonumber
%%%% Switches
\exonumberfalse
\catcode`@=11\relax
\input LD@Header.tex
\input LD.tex
\input LD@Typesetting.tex
\input LD@Exercices.tex
\input LD@Exercices.tex
\def\LD@Exercice@Display@Code{\eightpts}%
% debug tikz
\let\@firstofone\pgfutil@firstofone
\let\@ifnextchar\pgfutil@ifnextchar

\olspept
\DefineRGBcolor F0F9E3=VLGreen.
\DefineRGBcolor E5F9D1=LGreen.
\DefineRGBcolor DAF9BE=TGreen.
\DefineRGBcolor 5DA93B=Green.
\DefineRGBcolor F6DCCA=VLRed.
\DefineRGBcolor F6D4BD=LRed.
\DefineRGBcolor DAF9BE=TRed.
\DefineRGBcolor B5F9A1=TTRed.
\DefineRGBcolor F6B080=Red.
\DefineRGBcolor F9F5E3=VLOrange.
\DefineRGBcolor F9F5D0=LOrange.
\DefineRGBcolor DAF9BE=TOrange.
\DefineRGBcolor B5F9A1=TTOrange.
\DefineRGBcolor D7A93B=Orange.
\DefineRGBcolor EEEEEE=VLBlack.
\DefineRGBcolor DDDDDD=LBlack.
\DefineRGBcolor CCCCCC=TBlack.
\DefineRGBcolor B5F9A1=TTBlack.
\DefineRGBcolor 000000=Black.

\def\transparent{%
	\CS long\EC\def\Demonstration##1\CQFD{}%
}%
%\transparent
\def\Students{%
	\DefineRGBcolor FFFFFF=VLGreen.
	\DefineRGBcolor FFFFFF=LGreen.
	\DefineRGBcolor FFFFFF=TGreen.
	\DefineRGBcolor 000000=Green.
	\DefineRGBcolor FFFFFF=VLRed.
	\DefineRGBcolor FFFFFF=LRed.
	\DefineRGBcolor FFFFFF=TRed.
	\DefineRGBcolor FFFFFF=TTRed.
	\DefineRGBcolor 000000=Red.
	\DefineRGBcolor FFFFFF=VLOrange.
	\DefineRGBcolor FFFFFF=LOrange.
	\DefineRGBcolor FFFFFF=TOrange.
	\DefineRGBcolor FFFFFF=TTOrange.
	\DefineRGBcolor 000000=Orange.
	\DefineRGBcolor FFFFFFF=VLBlack.
	\DefineRGBcolor FFFFFF=LBlack.
	\DefineRGBcolor FFFFFF=TBlack.
	\DefineRGBcolor FFFFFF=TTBlack.
	\DefineRGBcolor 000000=Black.
}
\Students
\def\red{}
\def\blue{}
\def\Red#1{#1}%%%% Fix this !
\def\Blue#1{#1}%
\def\Font #1@#2pt{\font\olbi=cmr10\olbi}
\font\SvgText=cmr10\relax
%
%
%\catcode`@=11\relax
%\def\Api{Mathematicon@Api}%
%
%\input LD@Header.tex
%\input LD.tex
%\input LD@Exercices.tex
%\input LD@Typesetting.tex
%
%\catcode`@=11\relax
\font\LD@Font@Arial="Arial" at 10pt
%%%%%%%%%%%%%%%%%%%%%%%%%%%%%%%%%%%%%%%%%%%%%%%%%%%%%%%%%%%%%%%%%%
%															%
%						Métrique des courbes							%
%															%
%%%%%%%%%%%%%%%%%%%%%%%%%%%%%%%%%%%%%%%%%%%%%%%%%%%%%%%%%%%%%%%%%%
\newcount\LD@Count@Temp
\def\LD@Exercice@Display@Code{}%%\LD@Option@@Label\qquad\eightpts}%
\def\LD@Exercice@Display@Code@Post{%
	\ifcsname LD@Exo@@Solution\endcsname
		\unless\ifx\LD@Exo@@Solution\LD@Empty
			\pn{\eightpts Solution : \eightpts \LD@Exo@@Solution}%
		\fi
	\fi
}%
\def\LD@Display#1{%
	\LD@Count@Temp=#1\relax
	\ifcase\LD@Count@Temp
	\or
	Math. Sup.
	\or
	Math. Sp\'e
	\else
	\fi
}%
\newcount\LD@Exo@Total\LD@Exo@Total=0\relax


%%% TD 11.

\vglue-10mm\rightline{Sp\'e PT\hfill TD 11 : Fonctions de classe $\sc C^k$}%\hfill 8/10/2002}
\bigskip

\hrule
\centerline{Comment montrer qu'une fonction $f$ est de classe $\sc C^0$ sur un domaine $D$ ?}
\hrule
\bigskip
\noindent
A) Trouver le plus grand ensemble $E\subset D$ sur lequel $f$ peut \^etre obtenu en faisant des sommes, diff\'erences, produits, quotients dont le d\'enominateur ne s'annule pas, ou compositions de fonctions de classe~$\sc C^1$. 
\medskip
\noindent
B) {\bf Si $D\neq E$}, prouver que $f$ est continue en chaque point \`a probl\`eme $a\in D\smallsetminus E$, en montrant que  
$$
\lim_{X\to a\atop X\neq a}f(X)=f(a).
$$

\hrule
\centerline{Comment montrer qu'une fonction $f$ est de classe $\sc C^1$ sur un domaine $D$ ?}
\hrule

\bigskip\noindent
I) Trouver le plus grand ensemble $E\subset D$ sur lequel $f$ peut \^etre obtenu en faisant des sommes, diff\'erences, produits, quotients dont le d\'enominateur ne s'annule pas, ou compositions de fonctions de classe~$\sc C^1$, 
\medskip\noindent
II) {\bf Si $D\neq E$}, effectuer en chaque point \`a probl\`eme $a\in D\smallsetminus E$ les op\'erations suivantes : \pn
i) prouver que $f$ est $\sc C^0$ sur $D$ (faire l'\'etape B) de l'algorythme pr\'ec\'edent). \pn
ii) Montrer que les d\'eriv\'ees partielles de $f$ existent en $a$ et les calculer en faisant des taux d'accroissement :
$$
{\partial f\F\partial x}(a)=\lim_{h\to 0}{f(a+(h,0))-f(a)\F h}\qquad {\partial\F\partial y}(a)=\lim_{h\to 0}{f(a+(0,h))-f(a)\F h}. 
$$
iii) Calculer les d\'eriv\'ees partielles de $f$, \`a l'aide des formules traditionnelles de d\'erivation en un point $X\in D$. 
Puis, prouver que ces d\'eriv\'ees partielles sont continues en $a$, en montrant que  
$$
\lim_{X\to a\atop X\neq a}{\partial f\F\partial x}(X)={\partial f\F\partial x}(a)\qquad \hbox{et} \lim_{X\to a\atop X\neq a}{\partial f\F\partial y}(X)={\partial f\F\partial y}(a)
$$

\hrule
\centerline{Comment montrer qu'une fonction $f$ est de classe $\sc C^k$ sur un domaine $D$ avec $k>1$ ?}
\hrule

\bigskip\noindent
1) Trouver le plus grand ensemble $E\subset D$ sur lequel $f$ peut \^etre obtenu en faisant des sommes, diff\'erences, produits, quotients dont le d\'enominateur ne s'annule pas, ou compositions de fonctions de classe~$\sc C^k$. 
\medskip\noindent
{\bf Si $D\neq E$}, effectuer en chaque point \`a probl\`eme $a\in D\smallsetminus E$ les op\'erations suivantes : \pn
2) Prouver que $f$ est de classe $\sc C^0$ (en effectuant l'\'etape B). \pn
3) prouver que $f$ est de clase $\sc C^1$ (en effectuant les \'etapes i, ii et iii). \pn
4) prouver que $f$ est $\sc C^2$ (en prouvant \`a l'aide des \'etapes i, ii et iii que ${\partial f\F\partial x}$ et  ${\partial f\F\partial x}$ sont de classe $\sc C^1$).\pn
5) prouver que $f$ est $\sc C^3$ (en prouvant que les d\'eriv\'ees secondes sont de classe $\sc C^1$).\pn
6) ...etc...\pn
On ne vous donnera jamais d'exercice faisant intervenir les \'etapes 5+. 
\medskip
\hrule
\centerline{Exercices}
\hrule
\bigskip
\Exercice{PTir}\par\noindent
({\it D'apr\`es le th\'eor\`eme de Schwarz ${\partial^2f\F\partial x\partial y}(a)={\partial^2f\F\partial y\partial x}(a)$ si $f$ est de classe $\sc C^2$ autour de $a$}).
\goodbreak
\Exercice{PTaoo}
\vfill
\Exercice{PTis}
\vfill
\Exercice{PTaon}
\vfill
\Exercice{PTcv}
\vfill
\Exercice{PTgs}
\vfill
\Exercice{PTanh}
\vfill
\Exercice{PTani}
\vfill
\Exercice{PTanj}
\vfill
\Exercice{PTjd}
\vfill\null\eject
\bye
