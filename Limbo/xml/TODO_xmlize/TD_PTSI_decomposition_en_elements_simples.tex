\catcode`@=11\relax

\input LD@Maths@TD.tex

\vglue-10mm\rightline{PTSI/PT\hfill TD : Décomposition en éléments simples et primitives\hfill}
\bigskip\noindent
\vfill
Rappel : Pour décomposer en éléments simples : \pn
a) on  effectue une division euclidienne si le degré de la fraction rationnelle est positif ou nul : 
$$
{x^4+1\F x^3+2x^2+x}=x-2+{3x^2+2x+1\F x^3+2x^2+x}. 
$$
b) on factorise le dénominateur
$$
{3x^2+2x+1\F x^3+2x^2+x}={3x^2+2x+1\F x(x+1)^2}.
$$
c) on écrit la décomposition en éléments simples
$$
{3x^2+2x+1\F x(x+1)^2}={a\F x}+{b\F x+1}+{c\F(x+1)^2}.\qquad(*)
$$
d) on obtient les constantes associées aux facteurs $\ds {1\F (x-a)^n},\cdots, {1\F (x-a)}$ en~multipliant par $(x-a)^n$, puis en calculant les dérivées $0^{\hbox{\sevenrm ième}}, \cdots, (n-1)^{\hbox{\sevenrm ième}}$ en $x=a$. \pn
\bigskip
Ainsi, pour trouver les constantes $b$ et $c$, on multiplie (*) par $(x+1)^2$ pour obtenir que
$$
g(x):={3x^2+2x+1\F x}=(x+1)^2{a\F x}+\underbrace{b(x+1)+c}_{\hbox{partie intéressante}}
$$
puis, en prenant les valeurs en $-1$ de $g$ et de $g'$, on trouve que
$$
c=g(-1)=-2\qquad\hbox{et}\qquad b=g'(-1)=2.
$$ 
Pour trouver la constante $a$, on multiplie (*) par $x$ pour obtenir que
$$
h(x):={3x^2+2x+1\F (x+1)^2}=\underbrace{a}_{\hbox{partie intéressante}}+x\Q({b\F x+1}+{c\F(x+1)^2}\W),
$$
puis, en prenant la valeur en $0$ de $h$, on trouve que
$$
a= h(0)=1.
$$
d) on peut également trouver d'autres relations entre les constantes $a$, $b$, $c$, etc en faisant subir des opérations 
à la relation (*), on peut par exemple : \pn
d1) substituer  à $x$  dans (*) d'autres valeurs que les pôles. \pn
d2) calculer des limites de $(*)$, $x.(*)$, $x^2.(*)$, etc...lorsque $x$ tends vers $+\infty$. \pn
d3) utiliser la parité\pn
d4) calculer $b$ et $c$ puis en déduire $a$ en faisant passer ${b\F x+1}+{c\F(x+1)^2}$ 
de l'autre coté de l'égalité dans (*) et en mettant sous le même dénominateur.\pn
\eject


\Exercice{PTSItd}
\vfill
\Exercice{PTSItg}
\vfill
\Exercice{PTSIte}
\vfill
\Exercice{PTSItf}
\vfill
\Exercice{PTSIth}
\bye
 