\catcode`@=11\relax

\input LD@Maths@TD.tex

\vglue-10mm\rightline{Spé PT\hfill TD 17C : Séries Entières\hfill}%\date}
\bigskip

\Exercice{PTsa}
\bigskip
\Exercice{PTqw}
\bigskip
\Exercice{PTsf}
\bigskip
\Exercice{PTco}
\eject
\Exercice{PTas}


\bigskip
\Exercice{PTde}
\bigskip
\Exercice{PTry}
\bigskip
\Exercice{PTbi}
\bigskip


\bigskip
\vfill\vfill\vfill\vfil\vfill\vfill\vfill\vfill\vfill\vfill\vfill\vfill\null
\bye