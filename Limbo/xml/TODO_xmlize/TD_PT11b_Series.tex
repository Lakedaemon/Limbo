\catcode`@=11\relax

\input LD@Maths@TD.tex

\vglue-10mm\rightline{Spé PT\hfill TD : Séries\hfill}
\bigskip
\vfill

\centerline{Séries télescopiques}
\hrule
\medskip\noindent

\Propriete [$(u_n)_{n\in\ob N}$ suite complexe]
La série $u_0+\sum_{n=1}^\infty(u_n-u_{n-1})$ converge $\ssi$ la suite 
$(u_n)_{n\in\ob N}$ converge. De plus, en~cas de convergence, on a 
\Equation [\bf Séries télescopiques]
$$
u_0+\sum_{n=1}^\infty(u_n-u_{n-1})=\lim_{n\to\infty}u_n. 
$$
{\it lorsque les sommes partielles sont téléscopiques, on en trouve facilement une expression et la limite}. 

\noindent
\Exercice{PTpm}
\vfill
\noindent
\Exercice{PTpn}
\vfill
\noindent
\Exercice{PTpl}
\vfill

\centerline{Théorème special des séries alternées}
\hrule
\medskip\noindent


\Theoreme [$(u_n)_{n\in\ob N}$ suite réelle alternée]
Si la suite $|u_n|$ est décroissante, de limite nulle et si $(-1)^nu_n$ est de signe constant à partir d'un certain rang, 
alors la série $\sum_{n=0}^\infty u_n$ converge et 
\Equation [\bf Théorème spécial]
$$
\forall k\ge0, \qquad  \sum_{n=k}^\infty u_n \sbox { est  compris entre } 0 \sbox{ et }u_k \sbox{ inclus}.
$$ 

\bigskip
\Exercice{PTpk}
\bigskip
\noindent
\Exercice{PTpj}
\bigskip
\noindent
\Exercice{PTon}
\bigskip
\Exercice{PTom}
\bigskip
\Exercice{PToq}
\bigskip
\Exercice{PTama}
\medskip\noindent
\centerline{Convergence absolue}
\hrule
\medskip\noindent

\Propriete [$(u_n)_{n\in\ob N}$ suite complexe] 
Si la série $\sum_{n=0}^\infty|u_n|$ converge, alors la série $\sum_{n=0}^\infty u_n$ converge et l'on a 
\Equation [\bf Convergence absolue]
$$
\Q|\sum_{n=0}^\infty u_n\W|\le \sum_{n=0}^\infty|u_n|. 
$$

\Theoreme [$f:[0,\infty[\to\ob R$ continue par morceaux sur $[0,\infty[$]
Si $f$ est positive et décroissante sur $[0, \infty[$, alors l'intégrale $\int_0^\infty f(t)\d t$ 
et la série $\sum_{n=0}^\infty f(n)$ ont la m\^eme nature. 

\centerline{Exercices fun sur les séries}
\hrule
\medskip\noindent
\bigskip
\Exercice{PTux}
\bigskip
\Exercice{PTpi}
\vfill\null











\bye