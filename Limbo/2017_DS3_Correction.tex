\startcomponent component_DS1
\project project_Res_Mathematica
\environment environment_Maths
\environment environment_Inferno
\xmlprocessfile{exo}{xml/Limbo_Exercices.xml}{}
\iffalse
\setupitemgroup[List][1][n,inmargin][after=,before=,left={\bf Exo },symstyle=bold,inbetween={\blank[big]}]
\setupitemgroup[List][2][a,joineup][after=,before=,inbetween={\blank[small]}]
\setupitemgroup[List][3][a,joineup][after=,before=,inbetween={\blank[small]}]
\setupitemgroup[List][4][1,joineup,nowhite]
\fi

\setupitemgroup[List][1][A,inmargin][after=,before=,left={\bf Exo },symstyle=bold,inbetween={\blank[big]}]
%\setupitemgroup[List][1][R,joineup][after=,before=,inbetween={\blank[small]}]
%\setupitemgroup[List][1][n,inmargin][after=,before=,left={\bf Exo },symstyle=bold,inbetween={\blank[big]}]
%\setupitemgroup[List][1][n,joineup][after=,before=,inbetween={\blank[small]}]
\setupitemgroup[List][2][n,joineup][after=,before=,inbetween={\blank[small]}]
\setupitemgroup[List][3][a,joineup,nowhite]
\setupitemgroup[List][4][a,joineup,nowhite]
\definecolor[myGreen][r=0.55, g=0.76, b=0.29]%
\setuppapersize[A4]
\setuppagenumbering[location=]
\setuplayout[header=0pt,footer=0pt]
\def\conseil#1{{\myGreen\it #1}}%


\starttext
\setupheads[alternative=middle]
%\showlayout
\def\gah#1{\margintext{Exercice #1}}

\iffalse
\page
\centerline{\bfb DEVOIR SURVEILLE 3}
\blank[big]





\startList


\setupitemgroup[List][1][A,inmargin][after=,before=,left={\bf Exo },symstyle=bold,inbetween={\blank[big]}]
\setupitemgroup[List][2][n,joineup][after=,before=,inbetween={\blank[small]}]
\setupitemgroup[List][3][a,joineup,nowhite]
\setupitemgroup[List][4][a,joineup,nowhite]


\item%Exo
{\it Les flèches (optionnelles) sont utilisées pour aider les étudiants à distinguer les vecteurs, dans cet exercice mélangeant suites et espaces vectoriels }\crlf
On considère l’ensemble de suites suivant :
\startformula
E = \{\vec u=(u_n)_{n∈ℕ} : ∀ n ∈ ℕ , 8u_{n+3} −12u_{n+2} + 6u_{n+1} − u_n = 0\} 
\stopformula
Soient $\vec i=(i_n)_{n∈ℕ}$, $\vec j=(j_n)_{n∈ℕ}$, $\vec k=(k_n)_{n∈ℕ}$ les suites définies par 
\startformula
\Align{
\NC i_n=\Q({1\F 2}\W)^n\qquad (n∈ℕ)\NR
\NC j_n=n\Q({1\F 2}\W)^n\qquad (n∈ℕ)\NR
\NC k_n=n^2\Q({1\F 2}\W)^n\qquad (n∈ℕ)
}
\stopformula
\startList%1
\item Prouver que $E$ forme un sous-espace vectoriel de l’espace $ℝ^ℕ$ des suites réelles
\item\startList%2
\item Prouver que $\vec i∈E$, que $\vec j∈E$ et que $\vec k∈E$ (sur 6 points)
\item Prouver que $(\vec i,\vec j,\vec k)$ est une famille libre de $E$
\stopList%2
\item Soit $\vec u=(u_n)_{n∈ℕ}$ une suite de $E$. On définit une suite $\vec w=(w_n)_{n∈ℕ}$ en posant
\startformula
w_k=2^{k+1}u_{k+1}-2^ku_k\qquad(k∈ℕ)
\stopformula
\startList%2
\item On admet qu'il existe $(λ,μ)∈ℝ^2$ tel que 
\startformula
w_k=λ+μk\qquad(k∈ℕ)
\stopformula
Pour $n∈ℕ^*$, calculer de deux manière la somme $\D S_{n-1}=∑_{k=0}^{n-1}w_k$.
\item En déduire que $\vec u$ est une combinaison linéaire des suites $\vec i$, $\vec j$, $\vec k$.  
\stopList%2
\item A l'aide de ce qui précède, déterminer une base de $E$.
\item {\bf On va démontrer ici le résultat admis au 3a).}\crlf
On considère une suite $\vec u=(u_n)_{n∈ℕ}$ de $E$ et on pose $\vec w=(w_n)_{n∈ℕ}$ avec 
\startformula
w_k = 2^{k+1}u_{k+1}-2^ku_k\qquad(k∈ℕ)
\stopformula
\startList%2
\item Prouver que $∀k∈ℕ, w_{k+2}-2w_{k+1}+w_k=0$
\item En déduire l'expression de $w_k$ en fonction de $k$ admise au 3a)
\stopList%2
\item On pose $F=\{\vec u=(u_n)_{n∈ℕ}∈E:u_0-2u_1=0\}$
\startList%2
\item Montrer que $F$ est un sous-espace vectoriel de $E$
\item Montrer que $\vec i∈F$.
\item Soit $\vec u=(u_n)_{n∈ℕ}∈F$. Justifier que $∃!(α, β, γ)∈ℝ^3$ tel que 
\startformula
u_n=αi_n+βj_n+γk_n\qquad(n∈ℕ)
\stopformula
puis montrer que $γ=-β$.
\item En déduire une base de $F$
\stopList%2
\stopList%1

\goodbreak

\item%Exo(dénombrement)
Un site internet demande de choisir un mot de passe de $8$ caractères 
parmi les $26$ lettres minuscules de l’alphabet, les $10$ chiffres et $6$ signes de ponctuation (,;!.:?).
\startList 
\item Combien y a-t-il de mots de passe possibles ?
\item Combien y a-t-il de mots de passe possibles commençant par $5$ minuscules et se terminant par un nombre avec $3$ chiffres distincts
\item Combien y a-t-il de mots de passe possibles avec au moins un chiffre ou un signe de ponctuation ? 
\item Combien y a t-il de mots de passe avec au moins deux signes de ponctuation ? 
\item Combien y a t-il de mots de passe avec exactement trois chiffres et un signe de ponctuation
\item Combien y a-t'il de mots de passe constitués uniquement de $8$ chiffres distincts placés dans l'ordre croissant
\stopList


\setupitemgroup[List][1][A,inmargin][after=,before=,left={\bf Exo },symstyle=bold,inbetween={\blank[big]}]
\setupitemgroup[List][2][A,joineup][after=,before=,inbetween={\blank[small]}]
\setupitemgroup[List][3][n,joineup,nowhite]
\setupitemgroup[List][4][a,joineup,nowhite]

\item%Exo(proba)
Soit $n ∈ ℕ^∗$. On considère trois urnes : l’urne n° $1$ contient deux boules rouges et trois boules bleues, 
l’urne n° $2$ contient une boule rouge et aucune boule bleue et l’urne n° $3$ contient une 
boule bleue et aucune boule rouge.
On choisit d’abord une de ces trois urnes au hasard avec équiprobabilité. Une fois cette urne choisie, 
on effectue dans cette urne et sans jamais en changer un nombre fini de n de tirages successifs 
d’une boule, avec remise dans cette urne. Lorsque l’urne a été choisie, les tirages sont 
considérés comme indépendants.
Pour $i∈\{1, 2, 3\}$ on note $U_i$ l’événement \quote{l’urne choisie pour les tirages est l’urne n° $i$}.
Pour tout entier naturel non nul $k$, on note $R_k$ l’événement \quote{le $k\high{ième}$ tirage a amené une boule
rouge}.
\startList%Partie
\item%A
\startList%1
\item Donner les probabilités conditionnelles $P_{U_1}(R_k)$, $P_{U_2}(R_k)$, $P_{U_3}(R_k)$ et en déduire que $P(R_k)={7\F 15}$.
\item\startList%22
\item Justifier que $P_{U_1}(R_1∩R_2∩⋯∩R_n)=\Q({2\F 5}\W)^n$.
\item Préciser les valeurs de $P_{U_2}(R_1∩R_2∩⋯∩R_n)$ et $P_{U_3}(R_1∩R_2∩⋯∩R_n)$.
\item En déduire que $P(R_1∩R_2∩⋯∩R_n)={1\F 3}\Q({2\F 5}\W)^n+{1\F 3}$
\stopList%22
\item Montrer que les événements $R_1$ et $R_2$ ne sont pas indépendants pour la probabilité $P$
\stopList%1
\item%Partie B
\startList%1
\item
Pour $2⩽k⩽n$, montrer que 
\startformula
P_{R_1∩R_2∩⋯∩R_{k-1}}(R_k)={1+\Q({2\F 5}\W)^k\F 1+\Q({2\F 5}\W)^{k-1}}
\stopformula
\item Pour $1⩽k⩽n$, on note $A_k$ l’événement \quote{une boule bleue apparaît pour la première fois 
au tirage n° $k$} et $A$ l’événement \quote{aucune boule bleue n’apparaît jamais lors des $n$ tirages}.
\startList%22
\item Calculer $P(A_1)$
\item Soit $k⩾2$. Exprimer $A_k$ en fonction des événements $R_k$.\crlf
En déduire, avec les questions précédentes, pour $k⩾2$, la valeur de $P(A k)$ en fonction de $k$.
\item Calculer la probabilité $P(A)$. 
\stopList%22
\stopList%1
\stopList%Partie
\setupitemgroup[List][1][A,inmargin][after=,before=,left={\bf Exo },symstyle=bold,inbetween={\blank[big]}]
\setupitemgroup[List][2][n,joineup][after=,before=,inbetween={\blank[small]}]
\setupitemgroup[List][3][a,joineup,nowhite]
\setupitemgroup[List][4][a,joineup,nowhite]

\goodbreak
\item%Exo
On pose $∀x∈ℝ,\quad g(x)=\e^x-x$. 
\startList%1
\item\startList%11
\item Etudier les variations de $g$
\item Pour $n⩾2$, prouver que l'équation $g(x)=n$ admet une unique solution strictement positive, que l'on notera $β_n$, et une unique solution strictement négative, que l'on notera $α_n$. 
\stopList%11
\item%2
\startList%22
\item Pour $n⩾2$, montrer que $1⩽g(\ln n)⩽n$.
\item Pour $n⩾2$, montrer que $\D g(\ln(2n))=n+g(\ln(n))-\ln 2$ puis que 
\startformula g(\ln(2n))⩾n
\stopformula
\item Pour $n⩾2$, en déduire que $\D\ln(n)⩽β_n⩽\ln(2n)$ puis que $\D\lim_{n→+∞}β_n$.
\item Déterminer $\D\lim_{n→+∞}{β_n\F\ln(n)}$
\stopList%22
\item {\bf Dans cette question, on cherche une valeur approchée de $α_2$}.\crlf
On considère la suite $u$ définie par $u_0=-1$ et 
\startformula
u_{n+1}=\e^{u_n}-2\qquad(n∈ℕ)
\stopformula
\startList
\item Calculer $g(-2)$ et $g(-1)$. En déduire que $α_2∈]-2,-1[$
\item Montrer que $α_2⩽u_k⩽-1$ pour $k∈ℕ$. 
\item On pose $∀t∈[α_2, -1]$, $h(t)=\e^t-2$. Montrer que 
\startformula
h(x)-α_2⩽{1\F \e}(x-α_2)\qquad(α_2⩽x⩽-1)
\stopformula
\item En déduire que $\D 0⩽u_{k+1}-α_2⩽{1\F\e}(u_k-α_2)$ pour $k∈ℕ$.
\item Montrer que $\D 0⩽u_k-α_2⩽\Q({1\F\e}\W)^k$ pour $k∈ℕ$
\item En déduire que la suite $u$ converge et préciser sa limite.
\stopList
\stopList





\stopList%Exos

\else
Au final ce ds (plutôt facile) comportait
\startList
\item Espaces vectoriels : questions faciles(1, 2a, 4, 5a, 5b, 6a, 6b), questions moyennes (2b,3a,6c,6d), questions difficiles (3b)
\item Dénombrement : questions faciles(1, 3), questions moyennes (2, 4, 5), questions difficiles (6)
\item Probabilités : questions faciles(A1, A2a, A2b,A3, B1, B2a), questions moyennes (A2c), questions difficiles (B2b, B2c)
\item Fonctions : questions faciles(1a,2b,2d,3d,3e,3f), questions moyennes (1b,2a,2c,3a,3c), questions difficiles (3b)
\stopList
Les résultats étant donnés dans la plupart des questions (ce n'est pas toujours le cas). C'est rassurant et cela permet d'admettre le resultat pour traiter simplement les questions suivantes


L'exo de dénombrement est le plus court et simple (rentable) à rédiger, suivi de l'exo sur les fonctions.\crlf
L'exo de proba est plus simple que ce qui tombe en moyenne dans le même style.
Comme d'habitude, des points classiques du cours : sommes géométriques, sommes (et produits) telescopiques, récurrences linéaires, systèmes rodent dans le sujet...  
\blank[big]
\startList
\item%Exo A
\startList%A1a
\item%1
\startitemize[1]
\item $E⊂ℝ^ℕ$qui est un espace vectoriel de référence, par définition de $E$.
\item $E≠\emptyset$. En effet, la suite nulle $v=(0)_{n∈ℕ}$ appartient à $E$ car 
\startformula
8v_{n+3}-12v_{n+2}+6v_{n+1}-v_n=8×0-12×0+6×0-0=0\qquad(n∈ℕ)
\stopformula
\item Soient $(λ,μ)∈ℝ^2$ et soit $(u,v)∈E^2$. Montrons que $λu+μv∈E$.
\startitemize[2]
\item Comme $u∈E⊂ℝ^ℕ$,  comme $v∈E⊂ℝ^ℕ$ et comme $ℝ^ℕ$ est stable par combinaisons linéaires, en tant qu'espace vectoriel, 
on a $λu+μv∈ℝ^ℕ$
\item Pour $n∈ℕ$, nous remarquons que 
\startformula
\Align{
\NC \NC 8(λu+μv)_{n+3}-12(λu+μv)_{n+2}+6(λu+μv)_{n+1}-(λu+μv)_n\NR
\NC = \NC 8λu_{n+3}+8μv_{n+3}-12λu_{n+2}-12μv_{n+2}+6λu_{n+1}+6μv_{n+1}-λu_n-μv_n\NR
\NC = \NC λ(8u_{n+3}-12u_{n+2}+6u_{n+1}-u_n)+μ(8v_{n+3}-12v_{n+2}+6v_{n+1}-v_n)
}
\stopformula
Comme $u∈E$ et $v∈E$, nous remarquons alors que 
\startformula
8(λu+μv)_{n+3}-12(λu+μv)_{n+2}+6(λu+μv)_{n+1}-(λu+μv)_n
= λ×0+μ×0=0
\stopformula
\stopitemize%2
A fortiori, $λu+μv∈E$
\stopitemize%1
En conclusion, $E$ forme un $ℝ$ sous-espace vectoriel de $ℝ^ℕ$.
\item%2
\startList
\item%2a
\startitemize[1]%2a-
\item Comme la suite $\vec i$ appartient à $ℝ^ℕ$ et satisfait 
\startformula
\Align{[align={left, left}]
\NC\NC 8i_{n+3}-12i_{n+2}+6i_{n+1}-i_n\NR
\NC =\NC 8\Q({1\F 2}\W)^{n+3}-12\Q({1\F 2}\W)^{n+2}+6\Q({1\F 2}\W)^{n+1}-\Q({1\F 2}\W)^n\NR
\NC =\NC\Q({1\F 2}\W)^n\Q({8\F 2^3}-{12\F 2^2}+{6\F 2}-1\W)=\Q({1\F 2}\W)^n(1-3+3-1)=0,
}
\stopformula
nous remarquons que $\vec i∈E$.
\item Comme la suite $\vec j$ appartient à $ℝ^ℕ$ et satisfait 
\startformula
\Align{[align={left, left}]
\NC\NC 8j_{n+3}-12j_{n+2}+6j_{n+1}-j_n\NR
\NC =\NC 8(n+3)\Q({1\F 2}\W)^{n+3}-12(n+2)\Q({1\F 2}\W)^{n+2}+6(n+1)\Q({1\F 2}\W)^{n+1}-n\Q({1\F 2}\W)^n\NR
\NC =\NC \Q({1\F 2}\W)^n\Q({8(n+3)\F 2^3}-{12(n+2)\F 2^2}+{6(n+1)\F 2}-n\W)\NR
\NC=\NC\Q({1\F 2}\W)^n(n+3-3(n+2)+3(n+1)-n)=\Q({1\F 2}\W)^n×0=0,
}
\stopformula
nous remarquons que $j∈E$.
\item Comme la suite $k$ appartient à $ℝ^ℕ$ et satisfait 
\startformula
\Align{[align={left, left}]
\NC\NC 8k_{n+3}-12k_{n+2}+6k_{n+1}-k_n\NR
\NC =\NC 8(n+3)^2\Q({1\F 2}\W)^{n+3}-12(n+2)^2\Q({1\F 2}\W)^{n+2}+6(n+1)^2\Q({1\F 2}\W)^{n+1}-n^2\Q({1\F 2}\W)^n\NR
\NC =\NC\Q({1\F 2}\W)^n\Q({8(n+3)^2\F 2^3}-{12(n+2)^2\F 2^2}+{6(n+1)^2\F 2}-n^2\W)\NR
\NC=\NC\Q({1\F 2}\W)^n\underbrace{\big((n+3)^2-3(n+2)^2+3(n+1)^2-n^2)}_{(1-3+3-1)^2+n(6-12+6)+3^2-3×2^2+3}=\Q({1\F 2}\W)^n×0=0,
}
\stopformula
nous remarquons que $\vec k∈E$.
\stopitemize%2a-
\item%A2b
Les vecteurs $\vec i$, $\vec j$ et $\vec k$ appartienent à $E$. 
Supposons qu'il existe des saclaires $x$, $y$ et $z$ dans $ℝ$ 
tels que $x\vec i+y\vec j+z\vec k=\vec0$, où $\vec 0$ désigne le vecteur nul de l'espace des suites, 
c'est à dire la suite constante nulle.
Alors, en choisssant les indices, $n=0$, n$=1$ et $n=2$, on obtient le système
\startformula
\Align{[align={left, left}]
\NC \NC \System{[align={left, left}]
\NC {\red 0=}\NC xi_0+yj_0+zk_0=x+y×0+z×0^2={\red x}\NR
\NC {\red 0=}\NC xi_1+yj_1+zk_1={x\F 2}+{y\F 2}×1+{z\F 2}×1^2={\red {x\F 2}+{y\F 2}+{z\F 2}}\NR
\NC {\red 0=}\NC xi_2+yj_2+zk_2={x\F 4}+{y\F 4}×2+{z\F 4}×2^2={\red {x\F 4}+{2y\F 4}+{4z\F 4}}\NR
}\NR
\NC ⟺\NC \System{[align={left, left}]
\NC 0=x\NR
\NC 0=y+z\NR
\NC 0=2y+4z
}\NR
\NC ⟺\NC \System{[align={left, left}]
\NC 0=x\NR
\NC 0=y\NR
\NC 0=z
}}
\stopformula
A fortiori, $x=y=z=0$.Et la famille $(\vec i,\vec j,\vec k)$ est libre. En conclusion, c'est une famille libre de vecteurs de $E$.
\stopList%A2
\item%3
\startList%A3
\item En remarquant que l'on somme les termes d'une suite arithmétique, nous obtenons que 
\startformula
S_{n-1}=∑_{k=0}^{n-1}(λ+μk)= n×{λ+λ+μ(n-1)\F 2}=n×\Q(λ+μ{n-1\F 2}\W)
\stopformula
D'autre part, nous déduisons de la relation $w_k=2^{k+1}u_{k+1}-2^ku_k$ que $S_{n-1}$ est une somme telescopique de sorte que 
\startformula
\Align{
\NC S_{n-1}\NC =\D∑_{k=0}^{n-1}(2^{k+1}u_{k+1}-2^ku_k)\NC\NR
\NC \NC = \D∑_{k=0}^{n-1}2^{k+1}u_{k+1}-∑_{k=0}^{n-1}2^ku_k\NC\qquad\text{(séparation)}\NR
\NC \NC = \D∑_{ℓ=1}^n2^ℓu_ℓ-∑_{k=0}^{n-1}2^ku_k\NC\qquad\text{(changement d'indice  $ℓ=k+1$)}\NR
\NC \NC = \D∑_{ℓ=1}^{n-1}2^ℓu_ℓ+2^nu_n-2^0u_0-∑_{k=1}^{n-1}2^ku_k\NC\qquad\text{(Chasles)}\NR
\NC \NC = 2^nu_n-u_0
}
\stopformula
\item Nous déduisons des résultats de la question précédente que 
\startformula
2^nu_n-u_0=n\Q(λ+μ{n-1\F 2}\W)\qquad(n⩾1)
\stopformula
En particulier, nous remarquons que 
\startformula
\Align{
\NC u_n\NC ={u_0\F 2^n} + λ{n\F 2^n}+μ{n^2-n\F 2^{n+1}}\NR
\NC \NC = u_0 {1\F 2^n}+ \Q(λ-{μ\F 2}\W){n\F 2^n}+ {μ\F 2}{n^2\F 2^n}\NR
\NC\NC =  u_0i_n+ \Q(λ-{μ\F 2}\W)j_n+ {μ\F 2}k_n
}\qquad(n⩾1)
\stopformula
Comme cette relation est également vérifiée pour $n=0$, nous concluons que
\startformula
\vec u=u_0\vec i+\Q(λ-{μ\F 2}\W)\vec j+ {μ\F 2}\vec k
\stopformula
Ainsi, $\vec u$ est est combinaison linéaire des suites $\vec i$, $\vec j$ et $\vec k$.
\stopList%A3
\item La famille $\mc B=(\vec i, \vec j, \vec k)$ est une famille libre de $E$ d'après 1.2b et génératrice de $E$ d'après 1.3b 
puisque   toute suite $\vec u$ de $E$ est combinaison linéaire des suites $\vec i$, $\vec j$ et $\vec k$. 
A fortiori, $\mc B$ est une base de $E$.
\item Pour $k∈ℕ$, nous remarquons que 
\startformula
\Align{
\NC w_{k+2}-2w_{k+1}+w_k\NC =\underbrace{2^{k+3}u_{k+3}-2^{k+2}u_{k+2}}_{w_{k+2}}-2\Q(\underbrace{2^{k+2}u_{k+2}-2^{k+1}u_{k+1}}_{w_{k+1}}\W)+\underbrace{2^{k+1}u_{k+1}-2^ku_k}_{w_k}\NR
\NC\NC =2^k\Q(\underbrace{8u_{k+3}-12u_{k+2}+6u_{k+1}-u_k}_{0\text{ car $u∈E$}}\W)=0}
\stopformula
\item D'après la question précédente, la suite $w$ satisfait une récurrence linéaire homogène d'ordre $2$ d'équation caractéristique
\startformula
x^2-2x+1=0⟺(x-1)^2=0,
\stopformula
admettant la racine double $1$. En particuier, $∃!(λ,μ)∈ℝ^2$ tel que 
\startformula
w_k=λ1^k+μk1^k=λ+μk\qquad(k⩾0)
\stopformula
\item%6
\startList%6a
\item \startitemize[1]
\item $F⊂E$ qui est un espace vectoriel.
\item $F≠\emptyset$. En effet, la suite nulle $v=(0)_{n∈ℕ}$ appartient à $F$ car 
\startformula
v_0-2v_1=0-2×0=0
\stopformula
\item Soient $(λ,μ)∈ℝ^2$ et soit $(u,v)∈F^2$. Montrons que $λu+μv∈F$.
\startitemize[2]
\item Comme $u∈F⊂E$,  comme $v∈F⊂E$ et comme $E$ est stable par combinaisons linéaires, en tant qu'espace vectoriel, 
on a $λu+μv∈E$
\item Nous remarquons que 
\startformula
(λu+μv)_0-2(λu+μv)_1=λu_0+μv_0-2λu_1-2μv_1=  λ(u_0-2u_1)+μ(v_0-2v_1)
\stopformula
Comme $u∈F$ et $v∈F$, nous remarquons alors que 
\startformula
(λu+μv)_0-2(λu+μv)_1= λ×0+μ×0=0
\stopformula
\stopitemize%2
A fortiori, $λu+μv∈F$
\stopitemize%1
En conclusion, $F$ forme un $ℝ$ sous-espace vectoriel de $E$.
\item%6b
Nous avons déjà montré au 2a que $\vec i∈E$ et nous remarquons que 
\startformula
i_0-2i_1={1\F 2^0}-2{1\F 2^1}=0.
\stopformula
De sorte que $\vec i∈F$.
\item Comme $\vec u∈F$ et comme $F$ est un sous-espace vectoriel de $E$, nous avons $\vec u∈F$. 
Or nous avons montré au 4 que la famille $\mc E=(\vec i, \vec j, \vec k)$ forme une base de $E$. 
Alors, il existe un unique triplet $(α, β, γ)∈ℝ^3$ tel que 
$\vec u = α\vec i+β\vec j+γ\vec k$.
{\it Comme la famille $\mc E$ est génératrice, le triplet existe. Comme cetre famille est libre, le triplet est unique}\crlf
Comme $\vec u∈F$, nous remarquons alors que 
\startformula
0=u_0-2u_1=αi_0+βj_0+γk_0-2(αi_1+βj_1+γk_1)=α-2\Q({α\F 2}+β{1\F 2}+γ{1\F 2}\W)={β+γ\F 2}
\stopformula
A fortiori, nous avons $γ=-β$. 
\item Il résulte des calculs effectués dans la question précédente que 
\startformula
\vec u = α\vec i+β\vec j+γ\vec k= α\vec i+β\vec j-β\vec k= α\vec i+β(\vec j-\vec k)
\stopformula
En particulier, une suite $\vec u$ quelconque de $F$ est combinaison linéaires des suites $\vec i∈F$ et $\vec j-\vec k$. 
Comme la suite $\vec j-\vec k$ appartient à $F$ d'après la relation
\startformula
(j-k)_0-2(j-k)_1=j_0-k_0-2j_1+2k_1={0\F 1}-{0^2\F 1}-2{1\F 2}+2{1^2\F 2}=0
\stopformula
Nous remarquons que la famille $\mc F=(\vec i,\vec j-\vec k)$ est une famille génératrice de $F$, qui est également libre car
Si $x∈ℝ$ et $y∈ℝ$ vérifient $x\vec i+y(\vec j-\vec k)=\vec 0=x\vec i+y\vec j-y\vec k$, alors nous remarquons que nous avons une combinaison linéaire nulle 
de la famille libre $(\vec i, \vec j, \vec k)$ (montré en 2b), de sorte que $x=y=-y=0$.
En conclusion la famille $\mc F$ forme une base de $F$.
\stopList%6a
\stopList%A

\item%ExoB
\startList%B
\item Il y a autant de mots de passe que de $8$-listes d'éléments de l'ensemble $E$ constitué des $26$ minuscules, $10$ chiffres et $6$ signes de ponctuation.
Il y en a 
\startformula
(26+10+6)^8=42^8
\stopformula
\item Pour constituer un tel mot de passe, on commence par constituer une $5$-liste avec les minuscules (il y en a $26^5$) que l'on complete avec une $3$ liste sans répétition des $10$ chiffres (il y en a $A_{10}^3$). 
Au total, il y a $26^5×A_{10}^3$ mots de passe.
\item L'ensemble de ces mots de passe est constitué de tous les mots de passe ($42^8$), privé des mots de passe ne comportant ni chiffre, ni signe de ponctuation ($26^8$). Au~total, il y en a $42^8-26^8$.
\item L'ensemble de ces mots de passe est cnstitué de tous les mots de passe ($42^8$), privé des mots de passe sans signe de ponctuation (il y en a $(26+10)^8=36^8$) et des mots de passe comportant exactement un signe de ponctuation (on choisit le signe de ponctuation, sa position, et on complete les 7 cases qui restent avec des minuscules ou des chiffres : il y en a ${6\choose 1}×{8\choose 1}×36^7$).
Au total, il y en~a 
\startformula
42^8-36^8-{6\choose 1}{8\choose 1}36^7
\stopformula
\item Pour constituer un mot de passe avec exactement trois chiffres et un signe de ponctuation,  
\startitemize[n]
\item On choisit $3$ positions pour les trois chiffres \big(${8\choose 3}$ possibilités\big)
\item On choisit les chiffres pour ces trois positions ($10^3$ possibilités)
\item On choisit la position du signe de ponctuation \big(${5\choose 1}$ possibilités\big)
\item On choisit le signe de ponctuation \big(${6\choose 1}$ possibilités\big)
\item On complete les quatre cases restantes avec des minuscules ($26^4$ possibilités)
\stopitemize
Au total, il y en a 
\startformula
{8\choose 3}×10^3×{5\choose 1}×{6\choose 1}×26^4
\stopformula
\item Pour constituer un tel mot de passe, on choisit huit chiffres distincts (${10\choose 8}$), puis on les dispose dans l'ordre croissant.
Au total, il y en a ${10\choose 8}=40$\crlf 
{\it spéciale dédicace : K., tu peux les écrire tous, si tu le souhaites, je valide}
\stopList%B

\setupitemgroup[List][1][A,inmargin][after=,before=,left={\bf Exo },symstyle=bold,inbetween={\blank[big]}]
\setupitemgroup[List][2][A,joineup][after=,before=,inbetween={\blank[small]}]
\setupitemgroup[List][3][n,joineup,nowhite]
\setupitemgroup[List][4][a,joineup,nowhite]
\item%Exo C
\startList%Parties
\item%PartieA
\startList%L1
\item Comme les tirages de boules sont indépendants et comme les boules ont les mêmes probabilités d'être tirées, 
on utilise la probabilité uniforme pour calculer nos probabilités conditionnelles.
En particulier, on a
\startformula
\Align{[align={right, left}]
\NC P_{U_1}(R_k)\NC ={2\F 5}\NR
\NC P_{U_2}(R_k)\NC ={1\F 1} =1\NR
\NC P_{U_3}(R_k)\NC ={0\F 1} =0
}
\stopformula
Il résulte alors de la formule des probabilités totales appliquée au système complet d'événements (non négligeables) $\{U_1,U_2,U_3\}$ que 
\startformula
\Align{
\NC P(R_k)\NC =P(U_1)×P_{U_1}(R_k)+P(U_2)×P_{U_2}(R_k)+P(U_3)×P_{U_3}(R_k)\NR
\NC \NC ={1\F 3}×{2\F 5}+{1\F 3}×1+{1\F 3}×0={2+5\F 15}={7\F 15}
}
\stopformula
\item%2
\startList%L22
\item%2a
D'après l'indépendance des tirages des boules (une fois l'urne choisie),  
\startformula
P_{U_1}(R_1∩⋯∩R_n)=P_{U_1}(R_1)×⋯×P_{U_1}(R_n)=\Q({2\F 5}\W)^n
\stopformula
\item De même, on a
\startformula
\Align{[align={left, left}]
\NC P_{U_2}(R_1∩⋯∩R_n)\NC =P_{U_2}(R_1)×⋯×P_{U_2}(R_n)=1^n=1\NR
\NC P_{U_3}(R_1∩⋯∩R_n)\NC =P_{U_3}(R_1)×⋯×P_{U_3}(R_n)=0^n=0
}
\stopformula
\item Les résultats précédents et la formule des probabilités totales appliquée au système complet d'événements (non négligeables) $\{U_1,U_2,U_3\}$ induisent~que
\startformula
\Align{
\NC P(R_1∩⋯∩R_n)\NC =P(U_1)×P_{U_1}(R_1∩⋯∩R_n)+P(U_2)×P_{U_2}(R_1∩⋯∩R_n)\NR
\NC\NC \quad +P(U_3)×P_{U_3}(R_1∩⋯∩R_n)\NR
\NC \NC = {1\F 3}×\Q({2\F 5}\W)^n+{1\F 3}×1+{1\F 3}×0={1\F 3}\Q({2\F 5}\W)^n+{1\F 3}
}
\stopformula
\stopList%L1
\item D'après les résultas des questions 2a et 2c, nous avons 
\startformula
\Align{
\NC P(R_1∩R_2)\NC ={1\F 3}\Q({2\F 5}\W)^2+{1\F 3}={4+25\F 3×5^2}={29\F 3×5^2}\NR
\NC P(R_1)P(R_2)\NC =\Q({7\F 15}\W)^2={49\F 3^2×5^2}}
\stopformula
Comme $P(R_1∩R_2)≠P(R_1)P(R_2)$, les événements $R_1$ et $R_2$ ne sont pas indépendants pour la probabilité $P$.
\stopList%L22
\item%PartieB
\startList%L1
\item Soit $k∈⟦2,n⟧$. Alors, il résulte du résultat de la question 2a et de la définition ds probabilités conditionnelles que 
\startformula
P_{R_1∩⋯∩R_{k-1}}(R_k)={P(R_1∩⋯∩R_{k-1}∩R_k)\F P(R_1∩⋯∩R_{k-1})}={{1\F 3}\Q({2\F 5}\W)^k+{1\F 3}\F {1\F 3}\Q({2\F 5}\W)^{k-1}+{1\F 3}}={\Q({2\F 5}\W)^k+1\F \Q({2\F 5}\W)^{k-1}+1}
\stopformula
\item%B2
\startList%Bl2
\item Nous déduisons du résultat de la question A1a que 
\startformula
P(A_1)=P(\overline{R_1})=1-P(R_1)=1-{7\F15}={8\F 15}
\stopformula
\item Pour $⩾2$, nous remarquons que $A_k=R_1∩⋯∩R_{k-1}∩\overline{R_k}$. En particulier, nous déduisons de la formule des probabilités composées (vu qu'il n'y a pas indépendance entre les $R_i$) que 
\startformula
\Align{
\NC P(A_k)\NC =P(R_1)×P_{R_1}(R_1∩R_2)×⋯×P_{R_1∩⋯∩R_{k-2}}(R_{k-1})×P_{R_1∩⋯∩R_{k-1}}(\overline{R_k})\NR
\NC \NC = {7\F 15}×\Q(∏_{ℓ=2}^{k-1}P_{R_1∩⋯∩R_{ℓ-1}}(R_ℓ)\W)×\Q(1-P_{R_1∩⋯∩R_{k-1}}(R_k)\W)\NR
\NC\NC = {7\F 15}×∏_{ℓ=2}^{k-1}{\Q({2\F 5}\W)^ℓ + 1\F \Q({2\F 5}\W)^{ℓ-1}+1}×\Q(1-{\Q({2\F 5}\W)^k + 1\F \Q({2\F 5}\W)^{k-1}+1}\W)\qquad\qquad\text{(produits telescopiques)}\NR
\NC\NC = {7\F 15}×\Q({\Q({2\F 5}\W)^{k-1}+1\F \Q({2\F 5}\W)^1+1}-{\Q({2\F 5}\W)^k+1\F \Q({2\F 5}\W)^1+1}\W)\NR
\NC\NC = {7\F 15}×{\Q({2\F 5}\W)^{k-1}-\Q({2\F 5}\W)^k\F {7\F 5}}={5\F 15}\Q({2\F 5}\W)^{k-1}×\Q(1-{2\F 5}\W)\NR
\NC\NC = {1\F 5}×\Q({2\F 5}\W)^{k-1}={1\F 2}×\Q({2\F 5}\W)^k
}
\stopformula
\item On remarque que $\overline{A}=∪_{k=1}^nA_k$ de sorte que l'on déduit de la formule des sommes géométriques de raison différente de $1$ que  
\startformula
\Align{
\NC P(A)\NC =1-P(∪_{k=1}^nA_k)=1-∑_{k=1}^nP(A_k)\NC\text{(événements mutuellement incompatibles)}\NR
\NC \NC =1-∑_{k=1}^n{1\F 2}×\Q({2\F 5}\W)^k\NC\text{(Somme géométrique de raison $≠1$)}\NR
\NC\NC =1-{1\F 2}{\Q({2\F 5}\W)^1-\Q({2\F 5}\W)^{n+1}\F 1-{2\F 5}}=1-{1\F3}\Q(1-\Q({2\F 5}\W)^n\W)\NR
\NC\NC ={2\F 3}+{1\F3}\Q({2\F 5}\W)^n
}
\stopformula
\stopList%Bl2
\stopList%L1
\stopList%Parties
\setupitemgroup[List][1][A,inmargin][after=,before=,left={\bf Exo },symstyle=bold,inbetween={\blank[big]}]
\setupitemgroup[List][2][n,joineup][after=,before=,inbetween={\blank[small]}]
\setupitemgroup[List][3][a,joineup,nowhite]
\setupitemgroup[List][4][a,joineup,nowhite]

\item%Exo D
\startList%L1
\item \startList%1L
\item {\it Réaliser un tableau de variation de $g$ aide beaucoup pour la suite}\crlf
La fonction $g$ est dérivable sur $ℝ$ en tant que différence de fonctions dérivables. De plus, on a 
\startformula
g'(x)=\e^x-1\qquad(x∈ℝ)
\stopformula
En particulier, il résulte de la croissance stricte de la fonction exponentielle et de l'égalité $\e^0=1$ que  
\startformula
g'(x)>0⟺x>0\qquad g'(x)=0⟺ x=0\qquad g'(x)<0⟺x<0
\stopformula
La fonction $g$ est donc strictement décroissante sur $]-∞,0[$ (sur $]-∞,0]$ d'après la propriété 10.76 du cours) et strictement croissante sur $]0,+∞[$ (idem sur $[0,+∞[$).
\item La fonction $g$ est continue, strictement croissante sur $[0,+∞[$ donc est une bijection $\tilde g$ de $[0,+∞[$ dans $g([0,+∞[)=[1,+∞[$ (il résulte en effet du théorème de croissance comparée que $\D\lim\L_{x→+∞}g(x)=+∞$.
	En particulier, il existe une unique solution strictement positive $β_n$ de l'équation $g(x)=n$ (car $g(0)=1\Le n$), il s'agit de $β_n=\tilde g^{-1}(n)$. \crlf
	De même, il existe une unique solution $α_n$ stritement négative à l'équation $g(x)=n$ (pour les mêmes raisons, car $\D\lim_{x→-∞}g(x)=+∞$.	
\stopList%1L
\item%2
\startList%2L
\item Pour $n⩾2$, nous déduisons l'inégalité $g(n)⩾1$ de la croissance de $g$ sur l'intervalle $[0,+∞[$ 
et de $g(0)=\e^0-0=1$. 
De plus, nous remarquons que 
\startformula
g(\ln n)=\e^{\ln n}-\ln(n)=n-\underbrace{\ln(n)}_{⩾0}⩽n.
\stopformula
\item Pour $n⩾2$, nous remarquons que 
\startformula
g(\ln(2n))=\e^{\ln(2n)}-\ln(2n)=2n-\ln(2)-\ln(n)=n+(n-\ln(n))-\ln(2)=n+g(\ln(n))-\ln(2).
\stopformula
Comme $g(\ln(n))⩾1⩾\ln (2)$ (car il est bien connu que $2<\e$ et donc que $\ln(2)<\ln(\e)=1$), nous en déduisons que 
\startformula
g(\ln(2n))=n+\underbrace{g(\ln(2b))-\ln(2)}_{⩾0}⩾n
\stopformula
\item Pour $n⩾2$, nous avons $g(\ln(n))⩽n=g(β_n)⩽g(\ln(2n))$, d'après les résultats précédents. 
La fonction $g$ étant strictement croissante sur $[0,+∞[$, nous avons alors nécéssairement
\startformula
\ln(n)⩽β_n⩽\ln(2n)\qquad(n⩾2).
\stopformula
Comme $β_n⩾\ln(n)$ pour $n⩾2$ et comme $\lim_{n→+∞}\ln(n)=+∞$, il résulte du théorème d'encadrement (ou du principe des gendarmes, etc...) que 
\startformula
\lim_{n→+∞}β_n=+∞
\stopformula
\item En divisant par $\ln(n)$ l'inégalité obtenue précédemment, il vient 
\startformula
1={\ln(n)\F \ln(n)}⩽{β_n\F\ln(n)}⩽{\ln(2n)\F \ln(n)}=1+{\ln(2)\F\ln(n)}\qquad(n⩾2).
\stopformula
Comme les suites à gauche et à droite convergent vers $1$, il résulte du principe des gendarmes que 
\startformula
\lim_{n→+∞}{β_n\F\ln(n)}=1
\stopformula
Ce qui revient à dire que $β_n∼\ln(n)$
\stopList%2L
\item \startList%3L
\item Comme $\e>1$, nous remarquons que 
\startformula
g(-1)=\e^{-1}+1 \Le 2=g(α_2)\Le \e^{-2}+2=g(-2)
\stopformula
La fonction $g$ étant strictement décroissante sur l'intervalle $]-∞,0[$, qui contient $-2$, $α_2$ et $-1$, nous avons nécéssairement 
$-2\Le α_2\Le -1$.
\item Pour $k∈ℕ$, prouvons par récurrence la propriété $\mc P_n:α_2⩽u_k⩽-1$.
\startitemize[1]
\item $\mc P_0$ est vraie d'après l'égalité $u_0=-1$ et le résultat de la question 3a : $-2\Le α_2\Le -1$.
\item Supposnons $\mc P_n$ pour un entier $n∈ℕ$ et montrons $\mc P_{n+1}$.
Commençons par remarquer que $u_{n+1}=\e^{u_n}-2=g(u_n)+u_n-2$. Or, d'après $\mc P_n$, 
nous avons $α_2⩽u_n⩽-1$ et il résulte de la décroissance de $g$ sur $]-∞,0]$ que $2=g(α2)⩽g(u_n)⩽g(-1)=1+\e^{-1}$. De sorte que  
\startformula
α_2=2+α_2-2⩽u_{n+1}=g(u_n)+u_n-2⩽1+\e^{-1}-1-2⩽-1
\stopformula
En particulier $\mc P_{n+1}$ est vraie
\stopitemize
En conclusion, la proposition $\mc P_n$ est vraie pour $n∈ℕ$.
\item {\it ARGH le sujet n'utilise pas une nouvelle lettre pour la nouvelle fonction $g$.}
Pour éviter les confusions dans le corrigé, je vais poser 
\startformula
h(t)=\e^t-2\qquad(α_2⩽t⩽-1)
\stopformula
{\red \it Cette question sent à plein nez l'inégalité des acrroissements finis.}\crlf
La fonction $h$ est évidemment de classe $\sc C^1$ sur le segment $[α_2, -1]$ et on a 
\startformula
h'(t)={\darkgray \underbrace{ 0⩽}_{\text{ne servira pas}}}\e^t⩽{\red 1\F \e}\qquad(α_2⩽t⩽-1)
\stopformula
Il résulte alors de l'inégalité des accroissements finis appliquée entre $x∈[α_2,-1]$ et $α_2$ que 
\startformula
{\darkgray 0=0×(x-α_2)}⩽h(x)-α_2⩽{\red 1\F\e}(x-α_2)\qquad(α_2⩽x⩽-1)
\stopformula
\item Soit $k∈ℕ$. Pour le nombre réel $x=u_k$, qui appartient à l'intervalle $[α_2,-1]$ d'après la question 3b, il résulte de l'égalité $u_{n+1}=h(u_n)$ et de l'inégalité établie au 3c que 
\startformula
u_{k+1}-α_2=h(u_k)-α_2⩽{ 1\F\e}(u_k-α_2)\qquad(k∈ℕ)
\stopformula
\item Pour $k∈ℕ$, prouvons par récurrence la proposition $\mc P_k:0⩽u_k-α_2⩽\Q({1\F \e}\W)^k$
\startitemize[1]
\item $\mc P_0$ est vraie car $0=-1-(-1)⩽u_0-α_2=-1-α_2⩽1-(-2)=1$ puisque $-2⩽α_2⩽-1$ et $u_0=-1$.
\item Supposons la proposition $\mc P_k$ pour un entier $k⩾0$ et montrons $\mc P_{k+1}$
Nous remarquons que 
\startformula
\Align{
\NC \overbrace{0⩽}^{\text{d'après 3b}}u_{k+1}-α_2\NC \overbrace{⩽}^{\text{d'après 3d}}{ 1\F\e}(u_k-α_2)=\overbrace{⩽}^{\text{d'après $\mc P_k$}}{ 1\F\e}\Q({1\F \e}\W)^k\NR
\NC\NC ⩽\Q({1\F \e}\W)^{k+1}
}
\stopformula
Donc $\mc P_{k+1}$ est vraie.
\stopitemize

\item Il résulte de la question précédente et du principe des gendarmes que la suite $u_n-α_2$ converge vers $0$. 
A fortiori, la suite $u$ converge vers $α_2$.
\stopList%3L
\stopList%L1
\stopList
 \fi

\stoptext
\stopcomponent
\endinput
