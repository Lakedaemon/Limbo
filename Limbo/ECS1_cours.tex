\startproduct Limbo
\project project_Cours




%\definereferenceformat[eqref][left=(,right=)]
%\definehighlight[avoid][color=darkgray]
\definecolor[myGreen][r=0.55, g=0.76, b=0.29] 
\definecolor[lighterYellow][r=0.98, g=0.75, b=0.18] 
\definecolor[myYellow][r=0.98, g=0.66, b=0.15] 
\definecolor[darkerYellow][r=0.96, g=0.50, b=0.09] 
\definecolor[myBlue][r=0, g=0.51, b=0.56] 
\def\rare#1{{\darkgray #1}}
\defineenumeration[definition][command=\myGreen,left={\somenamedheadnumber{chapter}{current}.},location=serried,width=fit,text=Définition,location=serried,counter=property,alternative=left,title=yes,style=normal,list=all,listtext={Définition }]
\defineenumeration[theorem][command=\darkerYellow,left={\somenamedheadnumber{chapter}{current}.},location=serried,width=fit,text=Théorème,counter=property,title=yes,style=normal,list=all,listtext={Théorème }]
\defineenumeration[property][command=\myYellow,before={\startaxiomframe},after={\stopaxiomframe},left={\somenamedheadnumber{chapter}{current}.},location=serried,width=fit,title=yes,style=normal,list=all,text=Propriété,listtext={Propriété }]
\defineenumeration[corollary][command=\lighterYellow,left={\somenamedheadnumber{chapter}{current}.},location=serried,width=fit,text=Corollaire,title=yes,style=normal,list=all,listtext={Corollaire }]
\defineenumeration[notation][command=\myBlue,left={\somenamedheadnumber{chapter}{current}.},location=serried,width=fit,text=Notation,counter=definition,title=yes,style=normal,list=all,listtext={Notation }]
`\catcode`𝕂=\active
\def𝕂{ℝ}

\definetextbackground
[axiomframe]
[
mp=background:random,
location=paragraph,
rulethickness=1pt,
width=broad,
leftoffset=1em,
rightoffset=1em,
before={\testpage[3]\blank[3*big]},
after={\blank[3*big]}
]

\startuseMPgraphic{background:random}
path p;
for i = 1 upto nofmultipars :
p = (multipars[i]
topenlarged 10pt
bottomenlarged 10pt) randomized 4pt ;
fill p withcolor lightgray ;
draw p withcolor \MPvar{linecolor}
withpen pencircle scaled \MPvar{linewidth};
endfor;
\stopuseMPgraphic


\starttext
\centerline{\bfd Mathématiques}
\centerline{\tf (document de cours)}
\blank[big]
\centerline{\bfb ECS 1ère année}
\blank[big]
\centerline{Olivier Binda}
\blank[medium]
\centerline{\currentdate}
\stoptext
%\ctxlua{lxml.load("calcul", 'xml/Limbo_Méthodes_de_calcul.xml')}%
%\ctxlua{L.insertSubsection("calcul", 'section[attribute("t")=="Sommes"]')}%
%\meaning\xmlprocessfile
%\meaning\doxmlprocess
%\ctxlua{L.test()}%
\def\applySettups#1{
\startxmlsetups xml:#1
\xmlsetsetup{#1}{*}{-}
% sectioning
\xmlsetsetup{#1}{part|chapter|section|subsection|subsubsection|concept}{xml:#1:group}
% grouping 
\xmlsetsetup{#1}{group}{xml:groupl}
%\xmlsetsetup{suites}{combination}{xml:combination}
%\xmlsetsetup{maths}{combination/property}{xml:combination/statement}
% statements (primary objects)
\xmlsetsetup{#1}{corollary|definition|lemma|notation|property|theorem|}{xml:#1:statement} 
% best practice (secondary info)
\xmlsetsetup{#1}{craft|uses|item}{xml:#1:*}
% illustration
\xmlsetsetup{#1}{figure|example}{xml:#1:*} 
% text (tertiary info)
\xmlsetsetup{#1}{remark|remarks|text}{xml:#1:*} 
%\xmlsetsetup{suites}{remarks/remark}{xml:remarks:remark} 
% exercises
\xmlsetsetup{#1}{exercise}{xml:#1:*}
% proofs
\xmlsetsetup{#1}{proof}{xml:#1:*}
\stopxmlsetups 

\xmlregistersetup{xml:#1} 

\startxmlsetups xml:#1:group
\ctxlua{L.processGroup("##1")}%
\stopxmlsetups
\startxmlsetups xml:#1:statement
\ctxlua{L.processNewStatement("##1")}%
%\placeleftfigure{}{\useMPgraphic{Frame}}%
\stopxmlsetups

\startxmlsetups xml:#1:combination
\ctxlua{L.processCombination("##1")}%
\stopxmlsetups

\startxmlsetups xml:#1:example
\ctxlua{L.processExample("##1")}%
\stopxmlsetups


% remarks
\startxmlsetups xml:#1:remark
\ctxlua{L.processRemark("##1")}%
\stopxmlsetups

\startxmlsetups xml:#1:remarks 
{\tenpoint\startitemize[packed,intext,joinedup,2,nowhite,after]%
\xmlflush{##1}%
\stopitemize}%
\stopxmlsetups

\startxmlsetups xml:#1:remarks:remark
\item\ctxlua{L.processRawContent("##1")}%
\stopxmlsetups

% texts
\startxmlsetups xml:#1:text
\ctxlua{L.processRawContent("##1")}%
\stopxmlsetups

\startxmlsetups xml:#1:uses
{\tenpoint
\startitemize[packed,intext,joinedup,4,nowhite,after]
\xmlflush{##1}%
\stopitemize}
\stopxmlsetups


\startxmlsetups xml:#1:item
\ctxlua{L.processContent("##1","\\item")}%
\stopxmlsetups

}

\def\pn{\medskip}

% defines unnumbered domain
% see http://wiki.contextgarden.net/Titles#Unnumbered_titles_in_table_of_contents
% see http://wiki.contextgarden.net/Table_of_Contents#Including_unnumbered_heads_in_the_ToC
\definehead     [domain] [chapter]
\setuphead      [domain] [incrementnumber=list]
\setuplist[domain][width=0em,style=bold]  
\setuplist[chapter][width=1.5em]  
\setuplist[section][width=2.5em]                                              
\setuplist[subsection][width=3.5em, margin=2.5em]   
\setuplist[subject][margin=1.5em]
\setupcombinedlist[content][list={domain, chapter,section,subsection}]
\applySettups{suites}
\applySettups{sommes}
\applySettups{logique}
\applySettups{ensembles}
\applySettups{systèmes}
\applySettups{matrices}
\applySettups{ev}
\applySettups{polynomes}
\applySettups{probabilités}
\applySettups{VAR}

% Fondements théoriques
% alphabet grec
\startdomain[title=Bases théoriques]
% A1.1 (propositions)
\xmlprocessfile{maths}{xml/ECS_Logique.xml}{}
% A1.2 (suites)
\xmlprocessfile{maths}{xml/ECS_sommes_produits_récurrences.xml}{}
% A1.3 (fonctions)
\xmlprocessfile{maths}{xml/ECS_Ensembles_et_applications.xml}{}
% A6.4 combinatoire

% A4.1 ℝ 
\stopdomain

\startdomain [title=Analyse]
% Analyse (suites) 
%A4.2 exemples suites réelles
\xmlprocessfile{maths}{xml/ECS_Suites_fondamentales.xml}{}
%A4.3 suites réelles 
\xmlprocessfile{maths}{xml/ECS_Suites.xml}{}
%B2.1 etude asymptotique
%B2.3 Séries

% Analyse (fonctions)
% A5.1 limites et continuité fonctions réelles
% A5.2 etude globale sur un intervalle
\xmlprocessfile{maths}{xml/ECS_Continuité_limites.xml}{}
\xmlprocessfile{maths}{xml/Limbo_Continuité.xml}{}
% A5.3 dérivation
% B2.8 extrema
% B2.9 fonctions convexes
\xmlprocessfile{maths}{xml/Limbo_Dérivation.xml}{}
% B2.5 dérivées successives
% A5.4 intégration sur un segment + sommes de Riemann
\xmlprocessfile{maths}{xml/ECS_Dérivées_et_primitives.xml}{}
\xmlprocessfile{maths}{xml/Limbo_Intégration.xml}{}
%B2.6 Formule de Taylor
%B2.2 comparaison des fonctions
%B2.7 développements limités
%B2.4 integrales generalisées

\xmlprocessfile{maths}{xml/Limbo_Développements_limités.xml}{}


\stopdomain 
\startdomain[title=Algèbre] 

% Algèbre (non-linéaire)
%A2.1
\xmlprocessfile{maths}{xml/Limbo_Nombres_complexes.xml}{}
%A2.2
\xmlprocessfile{maths}{xml/ECS_Polynômes.xml}{}

% Algèbre (linéaire)
%A3.2
\xmlprocessfile{maths}{xml/ECS_Systèmes.xml}{}
%A3.1
\xmlprocessfile{maths}{xml/ECS_Matrices.xml}{}
%A3.3
\xmlprocessfile{maths}{xml/ECS_Espaces_vectoriels.xml}{}
%B1.1 Espaces de dim finie
%B1.2 sommes directes supplémentaires
%B1.3 Applications linéaires 



\stopdomain
\startdomain[title=Probabilités]

% Probabilités
% A6.1 Univers fini : Expériences aléatoires, événements, probabilité, probabilité conditionnelles, indépendance
\xmlprocessfile{maths}{xml/ECS_Probabilités.xml}{}
% A6.2 VAR réelle (finie)
\xmlprocessfile{maths}{xml/ECS_Variables_aléatoires.xml}{}
% A6.3 loi usuelles


%B3.1 espace probabilisé quelconque
%B3.2 VAR réelles
%B3.3 VAR discretes
%B3.4 loi discretes usuelles
%B3.5 VAR à densité
%B3.6 loi à densité usuelles
%B3.7 convergence et approximation


\stopdomain







%\xmlprocessfile{maths}{xml/Limbo_Structures_algèbriques.xml}{}

\starttext
\chapter{Fiches}
\completeRecords
\stopcolumns
\stoptext

\starttext
\completecontent[criterium=all]
\stoptext

\iffalse
\starttext
\completeindex
\stoptext
\fi
\stopproduct
