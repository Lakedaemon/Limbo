\startcomponent component_DS1
\project project_Res_Mathematica
\environment environment_Maths
\environment environment_Inferno
\xmlprocessfile{exo}{xml/Limbo_Exercices.xml}{}
\iffalse
\setupitemgroup[List][1][R,inmargin][after=,before=,left={\bf Exo },symstyle=bold,inbetween={\blank[big]}]
\setupitemgroup[List][2][n,joineup][after=,before=,inbetween={\blank[small]}]
\setupitemgroup[List][3][a,joineup][after=,before=,inbetween={\blank[small]}]
\setupitemgroup[List][4][1,joineup,nowhite]
\fi

%\setupitemgroup[List][1][A,inmargin][after=,before=,left={\bf Exo },symstyle=bold,inbetween={\blank[big]}]
%\setupitemgroup[List][1][R,joineup][after=,before=,inbetween={\blank[small]}]
%\setupitemgroup[List][1][n,inmargin][after=,before=,left={\bf Exo },symstyle=bold,inbetween={\%blank[big]}]
%\setupitemgroup[List][2][n,joineup][after=,before=,inbetween={\blank[small]}]
%\setupitemgroup[List][3][a,joineup][after=,before=,inbetween={\blank[small]}]
%\setupitemgroup[List][4][1,joineup,nowhite]
%\setupitemgroup[List][4][a,joineup,nowhite]
\definecolor[myGreen][r=0.55, g=0.76, b=0.29]%
\setuppapersize[A4]
\setuppagenumbering[location=]
\setuplayout[header=0pt,footer=0pt]
\def\conseil#1{{\myGreen\it #1}}%


\starttext
\setupheads[alternative=middle]
%\showlayout
\def\gah#1{\margintext{Exercice #1}}

\iftrue
\page
\centerline{\bfb DEVOIR MAISON 7}
\blank[big]

\setupitemgroup[List][1][A,joineup][after=,before=,inbetween={\blank[small]}]
\setupitemgroup[List][2][n,joineup][after=,before=,inbetween={\blank[small]}]
\setupitemgroup[List][3][a,joineup][after=,before=,inbetween={\blank[small]}]
\setupitemgroup[List][4][1,joineup,nowhite]

\centerline{\bf PROBLÈME 1 - UNE ÉLECTION EN DEUX TOURS ... OU TROIS}
\blank[big]
\startList%Partie
\item%A
{\bf Les candidats en lice}\crlf
Lors d’une élection, 7 candidats se présentent : 6 hommes et une femme.
On suppose qu’il n’y jamais d’ex-aequo.
\startList
\item De combien de façons ces candidats peuvent-ils être classés à l’issue du scrutin s’il n’y a pas d’ex-aequo ?
\item Combien y a-t-il de trios de tête possibles (c’est-à-dire de possibilités pour les premier, deuxième et troisième) ?
\item Combien y a-t-il de trios de tête possibles avec la candidate à l’une des trois places ?
\stopList
\item {\bf Les reports de voix pour le second tour}\crlf
À l’issue du premier tour, le candidat A a obtenu $40\%$ des suffrages, le candidat B a obtenu 30\% et les 30\%
restant se répartissent entre les autres candidats, les votes blancs et nuls.
Seuls les candidats A et B restent en lice pour le second tour et on suppose que les votants du second tour
sont exactement ceux ayant voté au premier (on ne considère donc pas les abstentionnistes du premier ou
du second tour).\crlf
Après une relecture  des programmes des deux candidats, il s’avère que :
\startitemize[1]
\item un individu ayant voté pour A au premier tour a une probabilité égale à $0,7$ de confirmer son vote au
second tour, une probabilité égale à $0,2$ de finalement voter pour B au second tour et une probabilité égale
à $0,1$ de voter blanc ou nul au second tour ;
\item un individu ayant voté pour B au premier tour a une probabilité égale à $0,85$ de confirmer son vote au
second tour, une probabilité égale à $0,05$ de finalement voter pour A au second tour et une probabilité
égale à $0,1$ de voter blanc ou nul au second tour ;
\item un individu ayant voté pour un autre candidat, voté blanc ou voté nul au premier tour a une probabilité
égale à $0,3$ de voter pour A au second tour, une probabilité égale à $0,4$ de voter pour B au second tour et
une probabilité égale à $0,3$ de voter blanc ou nul au second tour.
\stopitemize
\startList%B
\item On choisit un électeur au hasard.\crlf
Quelle est la probabilité qu’il vote pour A au second tour ? Quelle est la probabilité qu’il vote pour B au
second tour ?
\item On choisit un électeur au hasard.\crlf
Sachant que cet électeur a voté pour B au second tour, quelle est la probabilité qu’il ait également voté
pour B au premier tour ?
\item Quelle est la probabilité qu’un électeur, choisi au hasard, ait voté de la même façon aux deux tours
(c’est-à-dire deux fois pour A, deux fois pour B ou deux fois ni pour A, ni pour B) ?
\stopList
\item {\bf La campagne sur internet}
Lors des derniers jours de la campagne de \quote{l’entre deux tours}, l’équipe de communiquants de l’un des deux
candidats décide d’être très active sur les réseaux sociaux en postant régulièrement des messages tantôt
valorisant le programme de leur candidat, tantôt dénigrant celui de leur adversaire. On suppose que :
\startitemize[1]
\item si le $k\high{ième}$ message valorise le programme de son candidat alors il y a une probabilité 
égale à $0,6$ que le suivant valorise à nouveau ce programme (donc une probabilité égale à $0,4$ 
	qu’il dénigre le programme du candidat adverse) 
\item si le $k\high{ième}$ message dénigre le programme du candidat adverse 
alors il y a une probabilité égale à $0,8$ que le suivant valorise 
le programme de son candidat (donc une probabilité égale à $0,2$ qu’il dénigre à nouveau
le programme du candidat adverse).\crlf
On note $V_k$ l’événement \quote{le $k\high{ième}$ message valorise le programme de son candidat}.
\stopitemize
\startList
\item Quelles sont les probabilités directement données par l’énoncé ?
\item Soit $k∈ℕ$. Exprimer $P(V_{k+1})$ en fonction de $P(V_k)$ (en justifiant bien).
\item Déterminer l’expression de $P(V_k)$ en fonction de $k∈ℕ$ et de $P(V_0)$.
\item Quelle est la limite de la suite $\big(P(V_k)\big)_{k∈ℕ}$
\stopList
\item {\bf Épilogue}\crlf
Le scrutin du second tour ayant été largement entaché de fraudes, les candidats A et B décident, dans la plus
pure tradition française, de régler leur différend par un duel au pistolet.\crlf
Les candidats vont tirer à tour de rôle, le premier qui touche son adversaire a gagné. Le candidat A tire en
premier et a, à chaque tir, la probabilité $p_1$ de toucher son adversaire alors que, pour le candidat B, cette
probabilité est égale à $p_2$ (avec $p_1 > 0$ et $p_2 > 0$).\crlf
On supposera les différents tirs mutuellement indépendants (bien que le succès d’un tir mette un terme à l’expérience).\crlf
Pour tout $n∈ℕ^∗$ , on note :
\startitemize[1]
\item $A_n$ l’événement \quote{le tireur A effectue un $n\high{ième}$ tir et touche son adversaire}
\item  $B_n$ l’événement \quote{le tireur B effectue un $n\high{ième}$ tir et touche son adversaire}
\item $C_n$ l’événement \quote{le $n\high{ième}$ tir est réussi}.
\stopitemize
\startList
\item Calculer $P(C_1)$, $P(C_2)$ et $P(C_3)$.
\item Soit $n ∈ℕ^*$, calculer la probabilité de $P(C_n)$.
\item Déterminer la limite quand $N$ tend vers $+∞$ de $\D∑_{n=1}^NP(C_n)$. 
\stopList
\stopList
\blank[big]
\centerline{\bf PROBLÈME 2 - TIRAGES SANS REMISE PUIS AVEC REMISE}
\blank[big]
Soit $n$ et $b$ deux entiers avec $n⩾1$ et $b⩾2$. 
On considère une urne contenant $n$ boules noires et $b$ boules blanches, toutes indiscernables.
Un joueur A effectue des tirages successifs d’une boule sans remise dans l’urne jusqu’à obtenir 
une boule blanche.\crlf
Il laisse alors la place au joueur B qui effectue des tirages successifs d’une boule avec remise 
dans l’urne jusqu’à obtenir une boule blanche.\crlf
On suppose qu’il existe un espace probabilisé, d’univers $Ω$ et de probabilité $P$, 
permettant de modéliser cette expérience. \crlf
Pour tout $k ∈ℕ$ , on note :
\startitemize[1]
\item $X_k$ l’événement \quote{le joueur A tire $k$ boules noires avant de tirer une boule blanche}
\item $Y_k$ l’événement \quote{le joueur B tire $k$ boules noires avant de tirer une boule blanche}.
\stopitemize
Par exemple, si $n = 3$ et $b = 7$ et si les tirages successifs ont donné successivement des boules noire, 
blanche, noire, noire, noire, noire et blanche, alors :
\startitemize
\item A a effectué deux tirages, il a tiré une boule noire puis une boule blanche
\item l’urne contenait alors 8 boules dont deux noires et six blanches ;
\item B a alors effectué cinq tirages successifs dans cette urne, il a pioché 4 boules noires qu’il a reposées dans
l’urne puis il a pioché une boule blanche
\item les événements $X_1$ et $Y_4$ sont donc réalisés.
\stopitemize

\startList
\item {\bf Étude du cas particulier où $b$ et $n$ valent $2$}\crlf
On suppose donc ici que l’urne contient initialement $2$ boules blanches et $2$ boules noires.
\startList
\item Calculer les probabilités des événements $X_0$ , $X_1$ et $X_2$.
\item Montrer que $P(Y_0)={1\F 2}$.
\item Pour tout $i∈ℕ$, calculer les probabilités suivantes :
\startformula
P(X_0∩Y_i), \qquad P(X_1∩Y_i),\qquad P(X_2∩Y_i)
\stopformula
\item En déduire, pour tout $i ∈ℕ$, l’expression de $P(Y_i)$ puis déterminer la limite 
(éventuelle) quand $N$ tend vers $+∞$ de $\D∑_{i=0}^NP(Y_i)$
\stopList
\item {\bf Retour au cas général}
\startList
\item Pour tout $k∈ ⟦1, n⟧$, calculer $P(X_k)$ et montrer que : $\D P(X_k)={{n-k+b-1\choose b-1}\F {n+b\choose b}}$
\item Utiliser la question qui précède pour justifier que $\D∑_{k=0}^n{k+b-1\choose b-1}={n+b\choose b}$\crlf
Par conséquent on vient de démontrer la formule suivante :
\placeformula[gah]
\startformula
∀N∈ℕ, ∀a∈ℕ,∑_{k=0}^N{k+a\choose a}={N+a+1\choose a+1}
\stopformula
\item Soit $k⩾1$, $N⩾1$ et $a∈ℕ$. Comparer $\D k{k+a\choose a}$ et $\D (a+1){k+1\choose a+1}$ puis justifier que 
\startformula
∑_{k=0}^N k{k+a\choose a}=(a+1)∑_{k=0}^{N-1}{k+a+1\choose a+1}
\stopformula
\item A l'aide des questions précédentes, calculer $\D\sum_{k=0}^n(n-k)P(X_k)$ puis $\D∑_{k=0}^nkP(X_k)$
\item Pour tout $k∈⟦0,n⟧$ et tout $i∈ℕ$, calculer $P(X_k∩Y_i)$.\crlf
Les événements $X_k$ et $Y_i$ sont-ils indépendants ?
\stopList
\stopList








\stoptext
\stopcomponent
\endinput
