startcomponent component_DS1
\project project_Res_Mathematica
\environment environment_Maths
\environment environment_Inferno
\xmlprocessfile{exo}{xml/Limbo_Exercices.xml}{}
\iffalse
\setupitemgroup[List][1][n,inmargin][after=,before=,left={\bf Exo },symstyle=bold,inbetween={\blank[big]}]
\setupitemgroup[List][2][a,joineup][after=,before=,inbetween={\blank[small]}]
\setupitemgroup[List][3][a,joineup][after=,before=,inbetween={\blank[small]}]
\setupitemgroup[List][4][1,joineup,nowhite]
\fi


\definecolor[myGreen][r=0.55, g=0.76, b=0.29]%
\setuppapersize[A4]
\setuppagenumbering[location=]
\setuplayout[header=0pt,footer=0pt]
\def\conseil#1{{\myGreen\it #1}}%


\starttext
\setupheads[alternative=middle]
%\showlayout
\def\gah#1{\margintext{Exercice #1}}

\iftrue
\centerline{\bfb CORRECTION DU DEVOIR MAISON 9}
\blank[big]

\setupitemgroup[List][1][n,joineup][after=,before=,inbetween={\blank[small]}]
\setupitemgroup[List][2][a,joineup][after=,before=,inbetween={\blank[small]}]
\setupitemgroup[List][3][a,joineup][after=,before=,inbetween={\blank[small]}]
\setupitemgroup[List][4][1,joineup,nowhite]

\centerline{\bf EXERCICE 1}
{\it Un grand classique : polynômes de Tchebycheff}\crlf
On considère la suite des polynômes $(P_n)_{n∈ℕ}$ définie par $P_0 = 1$, $P_1 = X$ et
\startformula
P_{n+2}=2XP_{n+1}-P_n\qquad(n∈ℕ).
\stopformula
\startList
\item On a 
\startformula
\Align{
\NC P_2\NC =2XP_1-P_0=2X×X-1=2X^2-1\NR 
\NC P_3\NC = 2XP_2-P_1=2X×(2X^2-1)-X= 4X^3-3X\NR
\NC P_4\NC = 2XP_3-P_2=2X×(4X^3-3X)-(2X^2-1)=8X^4-8X^2+1
}
\stopformula
\item {\it Cette question se traite par récurrence. La difficulté consite à écrire une bonne propriété à démontrer (par soucis d'efficacité, cela requiert un peu d'expérience)}
Pour $n∈ℕ^*$, prouvons par récurrence la proposition
\startformula
\mc P_n:\qquad ∃Q_n∈ℝ_{n-1}[X]: P_n=2^{n-1}X^n + Q_n
\stopformula
\startitemize[1]
\item D'après le calcul de $P_1$ et $P_2$, les propositions $\mc P_1$ et $\mc P_2$ sont vraies
\item Supposons que les propositions $\mc P_n$ et $\mc P_{n+1}$ sont vérifiées pour un entier $n⩾1$ (et prouvons $\mc P_{n+2}$). 
Nous remarquons alors que 
\startformula
\Align{
\NC P_{n+2}\NC =2XP_{n+1}-P_n\NR
\NC\NC =2X(2^nX^{n+1} + Q_{n+1})-(2^{n-1}X^n + Q_n)\NR
\NC\NC = 2^{n+1}X^{n+2}+\underbrace{2XQ_{n+1}-2^{n-1}X^n-Q_n}_{Q_{n+2}∈ℝ_{n+1}[X]}
}
\stopformula
Comme $\deg(Q_n)⩽n-1$ et $\deg(Q_{n+1})⩽n$, nous remarquons en effet que $Q_{n+2}∈ℝ_{n+1}[X]$ et nous concluons que $\mc P_{n+2}$ est vraie  
\stopitemize
En conclusion, la proposition $\mc P_n$ est vraie pour $n∈ℕ^*$, de sorte que $P_n$ est de degré $n$ pour $n∈ℕ$ (car $P_0=1$)
et de coefficient dominant $2^{n-1}$ pour $n⩾1$ et de coeeficient dominant $1$ pour $n=0$.
\item {\it cela est encore plus vrai ici}. Pour $n∈ℕ$, prouvons par récurrence la proposition 
\startformula
\mc P_n:\qquad  P_n(-X)=(-1)^nP(X)
\stopformula
\startitemize[1]
\item D'après la donnée de $P_0$ et $P_1$, les propositions $\mc P_0$ et $\mc P_1$ sont vraies
\item Supposons que les propositions $\mc P_n$ et $\mc P_{n+1}$ sont vérifiées pour un entier $n⩾0$ (et prouvons $\mc P_{n+2}$). 
Nous remarquons alors que 
\startformula
\Align{
\NC P_{n+2}(-X)\NC =(2XP_{n+1}-P_n)(-X)\NR
\NC\NC = -2XP_{n+1}(-X)-P_n(-X)\NR
\NC\NC = -2X(-1)^{n+1}P_{n+1}(X)-(-1)^nP_n(X)\NR
\NC\NC = 2X(-1)^{n+2}P_{n+1}(X)-(-1)^n(-1)^2P_n(X)\NR
\NC\NC = (-1)^{n+2}\big(2XP_{n+1}(X)-P_n(X)\big)\NR
\NC\NC = (-1)^{n+2}P_{n+2}(X)
}
\stopformula
En particulier, $\mc P_{n+2}$ est vraie  
\stopitemize
En conclusion, la proposition $\mc P_n$ est vraie pour $n∈ℕ$, de sorte que $P_n$ a la même parité de $n$. 
Si $n$ est pair, $P_n$ est un polynôme pair (avec uniquement des termes de degré pair) et 
si $n$ est pair, $P_n$ est un polynôme impair (avec uniquement des termes de degré impair)
\item {\it Encore une fois, il faut procéder par récurrence à deux pas, car la définition de $P_{n+2}$ à partir de $P_{n+1}$ et $P_n$ 
est la seule relation que l'on peut utiliser pour aboutir}. Pour $n∈ℕ$, prouvons par récurrence la proposition
\startformula
\mc P_n:\qquad P_n(cos(x)) = cos(nx)\qquad(x∈ℝ).
\stopformula
\startitemize[1]
\item D'après la donnée de $P_0$ et $P_1$, les propositions $\mc P_0$ et $\mc P_1$ sont vraies car
\startformula
\Align{
\NC P_0\big(\cos (x)\big)\NC =1=\cos(0×x)\NR
\NC P_1\big(\cos (x)\big)\NC =\cos(x)=\cos(1×x)
}\qquad( x∈ℝ)
\stopformula
\item Supposons que les propositions $\mc P_n$ et $\mc P_{n+1}$ sont vérifiées pour un entier $n⩾0$ (et prouvons $\mc P_{n+2}$). 
Pour $x∈ℝ$, nous remarquons alors que 
\startformula
\Align{
\NC P_{n+2}(\cos x)\NC =(2\cos(x)P_{n+1}(\cos x)-P_n(\cos x)\NR
\NC\NC =(2\cos(x)\cos\big((n+1)x\big)-\cos( nx)
	}
	\stopformula
Or, il résulte de la formule trigonométrique d'addition des angles que 
\startformula
\Align{
\NC \cos \big((n+2)x\big)\NC = \big((n+1)x + x\big)\NC=\cos(x)\cos\big((n+1)x\big)-\sin(x)\sin\big((n+1)x\big)\NR
\NC \cos \big(nx\big)\NC = \big((n+1)x - x\big)\NC =\cos(-x)\cos\big((n+1)x\big)-\sin(-x)\sin\big((n+1)x\big)\NR
\NC \NC \NC =\cos(x)\cos\big((n+1)x\big)+\sin(x)\sin\big((n+1)x\big)
}
\stopformula
A fortiori, nous remarquons d'une part que $\cos(nx)+\cos\big((n+2)x\big)=2cos(x)\cos\big((n+1)x\big)$
et d'autre part que 
\startformula
\Align{
\NC P_{n+2}(\cos x)\NC =(2\cos(x)\cos\big((n+1)x\big)-\cos( nx)=\cos(nx)+\cos\big((n+2)x\big)-\cos(nx)\NR
	\NC \NC =\cos\big((n+2)x\big)
}
	\stopformula
En particulier, la proposition $\mc P_{n+2}$ est vraie  
\stopitemize
En conclusion, la proposition $\mc P_n$ est vraie pour $n∈ℕ$.
\item {\it Maintenant, grace à l'identité établie à la question précédente, 
nous pouvons procéder autrement que par récurrence}. Pour $n⩾2$, commençons par trouver toutes les racines de $P_n$ de la forme $z=\cos(x)$
\startformula
\Align{
\NC P_n(\cos x)=0\NC ⟺\cos(nx)=0⟺\cos(nx)=\cos\Q({π\F 2}\W)⟺nx≡{π\F 2} \quad [π]\NR 
\NC \NC ⟺ x≡{π\F 2n}\quad[{π\F n}] }
\stopformula
En particulier, nous venons de trouver $n$ racines distinctes de $P_n$, les nombres de la forme
\startformula
\cos({π\F 2n}+{kπ\F n})\qquad (0⩽k⩽n-1)
\stopformula
(Comme la fonction $\cos$ est strictement décroissante sur $[0, 2π]$ et comme les ${π\F 2n}+{kπ\F n}$ 
	forment une suite strictement croissante 
	de $n$ nombres de cet intervalle, nous avons bien $n$ racines distinctes de $P_n$, qui est de degré $n$. 
	Nous avons donc bien trouvé toutes les racines de $P_n$. 
	De plus, comme $P_n$ est de coefficient dominant $2^{n-1}$, nous pouvons également décomposer $P_n$ pour écrire que 
	\startformula
	P_n=2^{n-1}∏_{k=0}^{n-1}\Q(X-\cos\Q({π\F 2n}+{kπ\F n}\W)\W)
	\stopformula
	{\it Voici une troisième relation fondamentale à propose des polynômes de Tchebycheff de première espèce}
\stopList

\centerline{\bf EXERCICE 2}
{\it Un autre classique : polynômes de Lagrange}\crlf
\startList
\item D'après les relations  $P(1)=0$, $P(2=0$ et $P(3)=0$, les nombres $1$, $2$ et $3$ sont les trois racines du polynôme $P$, qui est de degré $3$. 
Il résulte de l'unique décomposition de $P$ en facteurs, qu'il existe une unique constante $α∈ℝ^*$ telle que 
\startformula
P=α(X-1)(X-2)(X-3)
\stopformula
En substituant $4$ à $X$, il résulte alors de la relation $P(4)=1$ que 
\startformula
P(4)=α(4-1)(4-2)(4-3)=6α
\stopformula
En particulier $α={1\F6}$ de sorte que 
\startformula
P={(X-1)(X-2)(X-3)\F 6}
\stopformula
{\it nous avons fait une démonstration directe, sans analyse-synthèse (quand on sait comment procéder, pas besoin d'enqueter/de tatonner).}
\item Cas général : soit $n∈ℕ$ et $a_0, a_1, ⋯, a_n$ des nombres réels distincts deux à deux.
\startList
\item Traitons les deux questions en une seule fois. Soit $k∈⟦0,n⟧$. Le polynôme $L_n$ est de degré $n$ et admet $n$ racines distinctes dans $⟦0,n⟧\ssm\{k\}$ d'après les relations 
\startformula
L_k(a_j) = 0\qquad(j∈⟦0,n⟧\ssm\{k\})
\stopformula
A fortiori, il existe un unique nombre réel $α∈ℝ^*$ tel que 
\startformula
L_k=α∏_{j=0\atop j≠k}^n(X-j)
\stopformula
Et il résulte alors de la relation $L_k(k)=1$ que 
\startformula
1=L_k(k)=α∏_{j=0\atop j≠k}^n(k-j)
\stopformula
De sorte que $α=1/∏_{j=0\atop j≠k}^n(k-j)$. En particulier, nous obtenons que 
\startformula
L_k={∏_{j=0\atop j≠k}^n(X-j)\F ∏_{j=0\atop j≠k}^n(k-j)}=∏_{j=0\atop j≠k}^n{X-j\F k-j}
\stopformula
{\it Quel est l'intérêt des polynômes de Lagrange ? \crlf
les polynômes $L_0, ⋯, L_n$ constituent une base de $ℝ_n[X]$ qui est particulièrement utile pour résoudre la problème suivant : \crlf
Comment trouver un polynôme $P$ de degré $n$ vérifiant $P(k)=α_k$ pour $0⩽k⩽n$, les nombres $α_k$ étant arbitrairement choisis...\crlf
Il suffit de prendre
\startformula
P=∑_{k=0}^nα_kL_k
\stopformula
Par un procédé identique, on peut fabriquer aisément une fonction polynômiale qui passe par $n+1$ points quelconques du plan (d'abscisses distinctes 2 à 2, au lieu de $⟦0,n⟧$)}
\stopList
\stopList

\centerline{\bf EXERCICE 3}{\it Remarquez le léger changement d'énoncé (avec un décalage d'indice de 1)}
Soit la suite de polynômes $(P_n)$ définie par : $P_1=1$ et 
\startformula
P_{n+1}=1+{X\F 1!}+⋯+{X(X+1)⋯(X+n-1)\F n!}=∑_{k=0}^{n}{1\F k!}∏_{j=0}^{k-1}(X+j)\qquad (n⩾1)
\stopformula
\startList
\item On a $P_2=1+X$ et $P_3=1+X+{X(X+1)\F 2}={X^2+3X+2\F 2}={(X+1)(X+2)\F 2}$
\item Pour $n⩾2$, Prouvons par récurrence la proposition 
\startformula
\mc P_n:\qquad P_n={1\F(n-1)!}∏_{j=1}^{n-1}(X+j)
\stopformula
\startitemize[1]
\item La proposition $\mc P_2$ est vraie car $P_2=1+X={1\F1!}∏_{j=1}^1(X+j)$
\item Supposons la proposition $\mc P_n$ pour un entier $n⩾2$ (et montrons $\mc P_{n+1}$).
Alors, nous remarquons que 
\startformula
\Align{
\NC P_{n+1}\NC =∑_{k=0}^{n}{1\F k!}∏_{j=0}^{k-1}(X+j)\NR
\NC \NC =∑_{k=0}^{n-1}{1\F k!}∏_{j=0}^{k-1}(X+j) + {1\F n!}∏_{j=0}^{n-1}(X+j)\NR
\NC \NC =P_n + {1\F n!}∏_{j=0}^{n-1}(X+j)\NR
\NC \NC ={1\F(n-1)!}∏_{j=1}^{n-1}(X+j)+ {1\F n!}∏_{j=0}^{n-1}(X+j)\NR
\NC \NC =n×{1\F n!}∏_{j=1}^{n-1}(X+j)+ {1\F n!}∏_{j=1}^{n-1}(X+j)×X\NR
\NC \NC =(X+n)×{1\F n!}∏_{j=1}^{n-1}(X+j)\NR
\NC \NC ={1\F n!}∏_{j=1}^n(X+j)}
\stopformula
En particulier, la proposition $\mc P_{n+1}$ est vraie
\stopitemize
En conclusion, la proposition $\mc P_n$ est vraie pour $n⩾2$
\stopList

\centerline{\bf EXERCICE 4}
En substituant $x={π\F 2}+2kπ$ à $x$, nous déduisons de la $2π$-périodicité 
des fonctions cosinus et sinus et des relations $\sin{π\F 2}=1$ et $\cos{π\F 2}=0$ que 
\startformula
Q\Q({π\F 2}+2kπ\W)=0\qquad(k∈ℤ)
\stopformula
En particulier, le polynôme $Q$ admet une infinité de racines, c'est donc le polynôme $Q=0$.
De même, en substituant 
remarquons que $x=2kπ$ à $x$, nous obtebnons que le polynôme $P$ admet une infinité de racines et donc également que $P=0$.


\stoptext
\stopcomponent
\endinput