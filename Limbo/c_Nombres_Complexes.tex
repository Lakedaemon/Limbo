\startcomponent c_Nombres_Complexes
\project project_Res_Mathematica
%\definecolor [darkred] [r=.5, g=.0, b=.0]

% \startcolor[darkgreen]           
%\stopcolor




\starttext
\enablemode[book] 

\chapter{Nombres Complexes}

La construction de $\ob C$ étant hors-programme, nous admettrons les définitions suivantes.

\section{Forme additive}

\Définition Un nombre complexe est un nombre de la forme \R[title=forme additive]$z=x+iy$\R, avec $x$ et  $y$ réels et $i^2=-1$.

\Propriété L'ensemble $\ob C$ des nombres complexes est muni des opérations $+$ et $\times$ définies par 
\R[title=$+$ et $\times$]{
$$
\eqalign{
	(a + ib) + (c + id) &:= (a + c) + i(b + d)\Z\qquad\qquad\mbox{\bf Addition}\Z\cr
	(a + ib) \times (c + id)&:=(ac - bd) + i(ad + bc)\Z\qquad\mbox{\bf Multiplication}\Z\cr
}
$$}\R

Théorème L'ensemble $(\ob C, +, \times)$ est un corps commutatif, i.e pour $x$, $y$, $z$ dans $\ob C$
\startformula
\eqalign{
x\times y & \in \ob C
}
\stopformula

Un ensemble $E$ muni de deux opérations $+$ et $\times$ forme un corps $(E,+,\cdots)$ si, et~seulement si, les onze propriétés suivantes sont satisfaites :
i) L'ensemble $E$ n'est pas vide : 
$$
\exists a\in E.
$$ 
ii) L'opération $+$ est interne à $E$ : 
$$
\forall (a,b)\in E^2, \qquad a+b\mbox{ est défini}\quad\mbox{et}\quad a+b\in E
$$
iii) L'opération $+$ admet un élément neutre, noté $0$, dans $E$ : 
$$
\exists 0\in E:\qquad \forall a\in E,\qquad 0+a=a+0=a
$$
iv) Chaque élément de l'ensemble $E$ admet un inverse pour la loi $+$ : 
$$
\forall a\in E, \qquad \exists b\in E:\qquad a+b=0=b+a
$$
v) L'opération $+$ est associative dans $E$ :
$$
\forall (a,b,c)\in E^3, \qquad(a+b)+c=a+(b+c)
$$ 
vi) L'opération $+$ est commutative dans $E$ : 
$$
\forall (a,b)\in E^2, \qquad a+b=b+a
$$
vii) L'opération $\times$ est distributive sur l'opération $+$ dans $E$ : 
$$
\forall(a,b,c)\in E^3, \qquad a\times(b+c)=a\times b+a\times c\quad\mbox{et}\quad (b+c)\times a=b\times a+c\times a.
$$ 
viii) L'opération $\times$ est interne à $E$ : 
$$
\forall (a,b)\in E^2,\qquad a\times b\mbox{ est défini}\quad \mbox{et}\quad a\times b\in E
$$
ix) L'opération $\times$ admet un élément neutre non nul dans $E$, noté $1$ :
$$
\exists 1\in E, \qquad\mbox{ différent de }0: \qquad \forall a\in E, \qquad a\times1=1\times a=a.
$$ 
x) Chaque élément non nul de $E$ est inversible pour $\times$ : 
$$
\forall a\in E, \qquad\mbox{différent de }0, \qquad \exists b\in E:\qquad a\times b=b\times a=1. 
$$
xi) L'opération $\times$ est associative dans $E$
$$
\forall (a,b,c)\in E^3, \qquad(a\times b)\times c=a\times(b\times c)
$$ 



\chapter{Fiches}
\disablemode[book]
\eightpoint
\ShowR

\stoptext
\stopcomponent
\endinput

$(\ob C, +, \times)$ l'ensemble L'ensemble des nombres complexes $\ob C$ est 


De même que $\ob R$ est muni d'une addition et d'une multiplication, 
il est possible de munir de deux opérations l'ensemble $\ob C:=\ob R^2$ des couples de nombres réels $(x,y)$, 
en~posant : 
$$
\eqalign{
(a,b)+(c,d)&:=(a+c,b+d)\qquad\qquad\mbox{\bf Addition}\cr
(a,b)\times(c,d)&:=(ac-bd,ad+bc)\qquad\mbox{\bf Multiplication}\cr
}%\eqdef{Coperation}
$$
Muni de ces opérations $+$ et $\times$, l'ensemble $\ob C=\ob R^2$ forme un corps commutatif 
(une structure algébrique que nous détaillons plus loin) de nouveaux nombres du type $(x,y)$, appelés nombres~complexes. 
Manier ces~nombres sous leur forme $(x,y)$ est peu pratique. 
C'est~pourquoi la notation simplifiée suivante est presque toujours utilisée : 
\medskip
\item{$\bullet$}
Un nombre du type $(x,0)$, où $x$ désigne un nombre réel, sera simplement noté ``$x$'' et sera appelé nombre réel (un abus bien pratique).
\medskip
\item{$\bullet$} 
Le nombre $(0,1)$ sera noté ``$i$''. On remarque qu'il vérifie $i^2=-1$. 
\medskip
\item{$\bullet$} 
Un nombre du type $(0,y)$, où $y$ désigne un nombre réel, sera simplement noté ``$iy$'' et sera appelé nombre imaginaire pur. 
\medskip
\item{$\bullet$}
Plus généralement, un nombre complexe $(x,y)$, où $x$ et $y$ sont deux nombres réels, sera~noté ``$x+iy$''. 
\bigskip

\section{Forme additive}

\section{Forme multiplicative}

\stoptext

\endinput
\hautspages{Olus Livius Bindus}{Nombres complexes}

\Chapter nombres complexes, Nombres complexes.

\Section Introduction C, Introduction. 

Historiquement, la constatation que les nombres réels ne suffisent pas pour résoudre certaines équations polynômiales 
du second degré telle que
$$
x^2=-1
\eqdef{Ceq}
$$ 
a motivé l'invention des nombres complexes : les nombres $a+ib$ pour lesquels $a$ et $b$ sont des nombres réels 
et pour lesquels $i$ désigne une solution de l'équation \eqref{Ceq}. 
\bigskip
L'ensemble $\ob C$ des nombres complexes permet de résoudre toutes les équations po\-ly\-nô\-mia\-les 
du second degré et possède la propriété fondamentale suivante : 

\Theoreme [Index=Theoreme@Théorème!de d'Alembert;Title=Théorème de d'Alembert-Gauss] 
Toute équation polynômiale de degré supérieur ou égal à $1$ admet au moins une solution dans~$\ob C$. 

L'ensemble $\ob C$ des nombres complexes présente des avantages par rapport à l'ensemble $\ob R$ des nombres réels, 
qui ne résident pas uniquement dans ces propriétés liées aux polynômes. D'ailleurs, une méthode classique très employée pour résoudre un problème réel consiste à faire le détour dans l'ensemble $\ob C$ suivant : \medskip
 \noindent
a) transformer le problème réel en problème complexe, en général plus facile. 
\smallskip\noindent
b) résoudre le problème complexe. 
\smallskip\noindent
c) En déduire les solutions du problème réel, souvent par projection. 
\bigskip

Les nombres complexes sont employés quotidiennement par la plupart des ma\-thé\-ma\-ti\-ciens et constituent un des pilliers sur lesquels reposent les mathématiques. 
C'est pourquoi, il est impératif en pratique de maitriser les nombres complexes, c'est à dire en ce qui vous concerne d'apprendre à calculer ou à transformer n'importe quelle expression complexe ou trigonométrique sans faire d'erreur. Et raisonnablement vite si c'est possible...

Les nombres complexes se présentent sous deux formes fondamentales : la forme al\-gè\-bri\-que $z=x+iy$ 
et la forme multiplicative $z=r\e^{i\theta}$. La première forme s'emploie plutôt dans un contexte additif 
et la seconde forme s'emploie plutôt dans un contexte multiplicatif. 



\Section CorpsC, Forme algébrique des nombres complexes. 

\Subsection ConstructionC, Construction de $\ob C$ et lien géométrique. 

De même que $\ob R$ est muni d'une addition et d'une multiplication, 
il est possible de munir de deux opérations l'ensemble $\ob C:=\ob R^2$ des couples de nombres réels $(x,y)$, 
en~posant : 
$$
\eqalign{
(a,b)+(c,d)&:=(a+c,b+d)\qquad\qquad\mbox{\bf Addition}\cr
(a,b)\times(c,d)&:=(ac-bd,ad+bc)\qquad\mbox{\bf Multiplication}\cr
}%\eqdef{Coperation}
$$
Muni de ces opérations $+$ et $\times$, l'ensemble $\ob C=\ob R^2$ forme un corps commutatif 
(une structure algébrique que nous détaillons plus loin) de nouveaux nombres du type $(x,y)$, appelés nombres~complexes. 
Manier ces~nombres sous leur forme $(x,y)$ est peu pratique. 
C'est~pourquoi la notation simplifiée suivante est presque toujours utilisée : 
\medskip
\item{$\bullet$}
Un nombre du type $(x,0)$, où $x$ désigne un nombre réel, sera simplement noté ``$x$'' et sera appelé nombre réel (un abus bien pratique).
\medskip
\item{$\bullet$} 
Le nombre $(0,1)$ sera noté ``$i$''. On remarque qu'il vérifie $i^2=-1$. 
\medskip
\item{$\bullet$} 
Un nombre du type $(0,y)$, où $y$ désigne un nombre réel, sera simplement noté ``$iy$'' et sera appelé nombre imaginaire pur. 
\medskip
\item{$\bullet$}
Plus généralement, un nombre complexe $(x,y)$, où $x$ et $y$ sont deux nombres réels, sera~noté ``$x+iy$''. 
\bigskip

\noindent
Exercice : vérifier que $i\times i=-1$ puis que $x+i\times y$ est bien le nombre $(x,y)$. 
\bigskip

Structurellement, l'ensemble des nombres complexes est étroitement lié au plan affine : 
\bigskip
\Definition [$\sc P$ plan affine muni d'un repère orthonormal $(O,\vec i,\vec j)$] 
\item{$\bullet$} L'affixe d'un point $M$ de coordonnées $(x,y)$ est le nombre complexe $z=x+yi$. 
\item{$\bullet$} L'affixe d'un vecteur $\vec v=x\vec i+y\vec j$ est le nombre complexe $z=x+yi$. 
\item{$\bullet$} L'image d'un nombre complexe $z=x+iy$ est le point $M(z)$ de coordonnées $(x,y)$. 

\centerline{
\tikzpicture
\clip (-0.5,-0.5) rectangle (5.4,3.1);
\draw[blue!20,ultra thin] (-0.5,-0.5) grid (5.4,3.1);
\draw[-] (-0.5,0) -- (5,0) coordinate (x axis);
\draw[-] (0,-0.5) -- (0,3) coordinate (y axis);
\draw[-triangle 60,thick] (0,0) -- node [anchor=south,sloped] {$\vec v=x\vec i+y\vec j$} node [anchor=north,sloped,pos=0.6,color=red] {$z=x+yi$} (4.3,2.4) node [anchor=south] {$M (x,y)$} node [anchor=north west,color=red] {$M (z)$} ;
\draw[-triangle 45,thick] (0,0) -- node [anchor=east,pos=0.65] {$y\vec j$} (0,2.4);
\draw[-triangle 45,thick] (0,0) -- node [anchor=north,pos=0.7] {$x\vec i$} (4.3,0);
\draw[dashed] (4.3,2.4) -- (0,2.4) node [color=red,anchor=south west] {$yi$};
\draw[dashed] (4.3,2.4) -- (4.3,0) node [color=red,anchor=south west] {$x$};
\draw[-open triangle 60,thick] (0,0) -- node [pos=0.4,anchor=east] {$\vec j$} (0,1);
\draw[-open triangle 60,thick] (0,0) -- node [pos=0.4,anchor=north] {$\vec i$} (1,0);
\node [anchor=north east] (O) at (0,0) {$O$};
\endtikzpicture}

\Remarque : Ne pas confondre le nombre complexe $i$ avec le vecteur $\vec i$. 

Géométriquement, l'addition de deux nombres complexes $s=a+ib$ et $z=c+id$ s'interprètre 
par la règle du parallèlogramme : en effet, les points $O$, $M$, $P$ et $Q$, d'affixes~respectives $0$, $s$, $z$ et $s+z$, 
forment un parallèlogramme. 

\centerline{
\tikzpicture[scale=0.8]
\draw[-] (-0.2,0) -- (6.5,0) coordinate (x axis);
\draw[-] (0,-0.2) -- (0,4.5) coordinate (y axis);
\draw[-,thick,dotted] (2,2.5) -- (6,4) ;
\draw[-,thick,dotted] (4,1.5) -- (6,4);
\draw[-open triangle 60,thick] (0,0) -- node [pos=0.4,anchor=east] {$\vec j$} (0,1);
\draw[-open triangle 60,thick] (0,0) -- node [pos=0.4,anchor=north] {$\vec i$} (1,0);
\draw[-triangle 60,thick] (0,0) --  (2,2.5) node [anchor=south east] {$M (s)$};
\draw[dashed] (2,2.5) -- (0,2.5) node [color=red,anchor=east] {$b$};
\draw[dashed] (2,2.5) -- (2,0) node [color=red,anchor=north] {$a$};
\draw[-triangle 60,thick] (0,0) --  (4,1.5) node [anchor=north west] {$P (z)$};
\draw[dashed] (4,1.5) -- (0,1.5) node [color=red,anchor=east] {$d$};
\draw[dashed] (4,1.5) -- (4,0) node [color=red,anchor=north] {$c$};
\draw[-triangle 60,thick] (0,0) --  (6,4) node [anchor=south] {$Q (s+z)$};
\draw[dashed] (6,4) -- (0,4) node [color=red,anchor=east] {$b+d$};
\draw[dashed] (6,4) -- (6,0) node [color=red,anchor=north] {$a+c$};
\node [anchor=north east] (O) at (0,0) {$O$};
\endtikzpicture}

\Section DefinitionC, Définition algébrique de $\ob C$. 

En utilisant la notation $x+iy$, la définition de l'ensemble $\ob C$ des nombres complexes et de ses opérations $+$ et $\times$ 
est légèrement plus simple et plus intuitive : 

\Definition L'ensemble $\ob C$ des nombres complexes est l'ensemble des nombres de la forme $z=x+iy$, 
où $x$ et $y$ sont des nombres réels et où $i^2=-1$, muni des opérations 
$$
\eqalign{(a+ib)+(c+id)&:=(a+c)+i(b+d)\qquad\qquad\mbox{\bf Addition}\cr
(a+ib)\times(c+id)&:=(ac-bd)+i(ad+bc)\qquad\mbox{\bf Multiplication}}
\eqdef{Coperation2}
$$


\noindent
Exercice : Calculer $(1-i)^3$ ainsi que $(1+i)^8$, en essayant de limiter les opérations. 
\bigskip

Ces régles de calcul des nombres complexes sont assez semblables aux règles de calcul des nombres réels (développement, factorisation par $i$) auxquelles on ajouterait la relation 
$$
i^2=-1. \eqdef{Ceqi}
$$ 
Ceci est du au fait que l'ensemble $\ob C$ des nombres complexes et l'ensemble $\ob R$ des nombres réels 
sont régis par la même structure algèbrique : ce sont des corps commutatifs. 

\Theoreme 
L'ensemble $\ob C$ des nombres complexes muni des opérations $+$~et~$\times$ définies par \eqref{Coperation2} 
forme un corps commutatif, que l'on note $(\ob C,+,\times)$. 

Avant de rappeler succintement les structures de corps et de corps commutatifs, prenons le temps d'introduire quelques abbréviations mathématiques standards : 
\medskip

\item{$\bullet$}
Le symbole ``$\in$'' signifie ``élément de", ``dans" ou ``appartient à'' ,
\smallskip
\item{$\bullet$}
Le symbole ``$\forall$'' se lit ``pour chaque" ou``pour tout",
\smallskip
\item{$\bullet$}
Le symbole ``$\exists$'' se traduit par ``il existe'' ou par ``on peut trouver'',
\smallskip
\item{$\bullet$}
Le signe de ponctuation ``:'' signifie très souvent ``tel que'' ou ``pour lequel''. 
\medskip

En particulier, on peut traduire la phrase ``on peut trouver un nombre x dans $\ob R$ tel que la quantité $x^2+3$ soit égale à $4$'' par la proposition mathématique 
$$
\exists\ x\in\ob R:\qquad x^2+3=4. 
$$
Réciproquement, il est possible de prendre une assertion symbolique et de la mettre sous une forme plus intélligible, utilisant la langue fran\c caise. En pratique, 
cela en facilite la compréhension et c'est ce que nous vous engageons à faire pour la définition suivante : 
\bigskip

\Concept [Index=Structure!Corps] Définition d'un Corps

Un ensemble $E$ muni de deux opérations $+$ et $\times$ forme un corps $(E,+,\cdots)$ si, et~seulement si, les onze propriétés suivantes sont satisfaites :
i) L'ensemble $E$ n'est pas vide : 
$$
\exists a\in E.
$$ 
ii) L'opération $+$ est interne à $E$ : 
$$
\forall (a,b)\in E^2, \qquad a+b\mbox{ est défini}\quad\mbox{et}\quad a+b\in E
$$
iii) L'opération $+$ admet un élément neutre, noté $0$, dans $E$ : 
$$
\exists 0\in E:\qquad \forall a\in E,\qquad 0+a=a+0=a
$$
iv) Chaque élément de l'ensemble $E$ admet un inverse pour la loi $+$ : 
$$
\forall a\in E, \qquad \exists b\in E:\qquad a+b=0=b+a
$$
v) L'opération $+$ est associative dans $E$ :
$$
\forall (a,b,c)\in E^3, \qquad(a+b)+c=a+(b+c)
$$ 
vi) L'opération $+$ est commutative dans $E$ : 
$$
\forall (a,b)\in E^2, \qquad a+b=b+a
$$
vii) L'opération $\times$ est distributive sur l'opération $+$ dans $E$ : 
$$
\forall(a,b,c)\in E^3, \qquad a\times(b+c)=a\times b+a\times c\quad\mbox{et}\quad (b+c)\times a=b\times a+c\times a.
$$ 
viii) L'opération $\times$ est interne à $E$ : 
$$
\forall (a,b)\in E^2,\qquad a\times b\mbox{ est défini}\quad \mbox{et}\quad a\times b\in E
$$
ix) L'opération $\times$ admet un élément neutre non nul dans $E$, noté $1$ :
$$
\exists 1\in E, \qquad\mbox{ différent de }0: \qquad \forall a\in E, \qquad a\times1=1\times a=a.
$$ 
x) Chaque élément non nul de $E$ est inversible pour $\times$ : 
$$
\forall a\in E, \qquad\mbox{différent de }0, \qquad \exists b\in E:\qquad a\times b=b\times a=1. 
$$
xi) L'opération $\times$ est associative dans $E$
$$
\forall (a,b,c)\in E^3, \qquad(a\times b)\times c=a\times(b\times c)
$$ 

\Definition 
Un corps $(E,+,\times)$ est commutatif si, et seulement si, 
\bigskip
xii) L'opération $\times$ est commutative dans $E$ : 
$$
\forall (a,b)\in E^2, \qquad a\times b=b\times a
$$

Ces douze propriétés constituent les règles de calculs à appliquer dans chaque corps~commutatif et en particulier dans $\ob R$ et $\ob C$. 
Pour résumer sommairement, on peut retenir que 
\medskip
\centerline
{
On calcule dans $\ob C$ comme dans $\ob R$ en utilisant la relation \eqref{Ceqi}.
}
\bigskip

Exercice. En admettant que $(\ob R,+,\times)$ est un corps commutatif, vérifier que $(\ob C,+,\times)$ en est un également. 
\bigskip

\goodbreak
L'identité suivante est très importante et sert en particulier à calculer la somme d'une suite géomètrique. 
\bigskip

\Propriete [$n\in\ob N$ et $(a,b)\in\ob C^2$]
$$
a^n-b^n=(a-b)\sum_{k=0}^{n-1}a^kb^{n-1-k}=(a-b)\sum_{k+\ell=n-1}a^kb^\ell.
$$
En particulier, l'identité remarquable $a^2-b^2=(a-b)(a+b)$ est une conséquence immédiate 
de la relation précédente pour l'entier $n=2$. 
\bigskip

\Definition 
Pour des entiers $n\ge k\ge 0$, on pose 
$$
{n\choose k}:={n!\F k!(n-k)!}.
$$

Le symbôle ${n\choose k}$ se lit ``$n$ et $k$'' et le nombre qu'il représente se note parfois $c_n^k$ 
et satisfait la propriété suivante : 

\Propriete
Pour $0\le k\le n$, le nombre ${n\choose k}$ est un entier. De plus, si $0\le k<n$, on a 
$$
{n\choose k}+{n\choose k+1}={n+1\choose k+1}.
$$

L'identité suivante est fondamentale. On l'emploie par exemple pour linéariser des expressions trigonométriques. 

\Propriete [Title=Binôme de Newton;$n\in\ob N$ et $(a,b)\in\ob C^2$] 
$$
(a+b)^n=\sum_{k=0}^n{n\choose k}a^kb^{n-k}=\sum_{k+\ell=n}{n\choose k}a^kb^\ell.
$$

\noindent
Les identités remarquables $(a+b)^2=a^2+2ab+b^2$ et $(a-b)^2=a^2-2ab+b^2$ sont des conséquences 
immédiates du binôme de Newton pour l'entier $n=2$. 
\bigskip

\Subsection CProjections, Parties réelles et imaginaires. 

La définition du corps $\ob C$ implique que chaque nombre complexe $z$ peut se mettre sous la forme $z=x+iy$, où $x$ et $y$ sont des nombres réels, 
et que cette écriture est unique, ce~qui nous permet de définir les parties réelles et imaginaires du nombre $z$. 
\bigskip

\Definition [$z=x+iy$ nombre complexe, avec $x$ et $y$ nombres réels] 
La partie réelle $\re(z)$ et la partie imaginaire $\im(z)$ du nombre $z$ sont définis par 
$$
\re(z):=x\qquad\mbox{et}\qquad\im(z):=y. 
$$

\Remarque : malgré son nom, la partie imaginaire est un nombre réel. 

\noindent
Une égalité entre nombres complexes se traduit par un système de deux égalités entre nombres réels. 
\medskip

\Propriete 
Deux nombres complexes $s$ et $z$ sont égaux si et seulement s'ils ont même parties réelles et même parties imaginaires. 
$$
s=z\qquad\Longleftrightarrow\qquad
\Q\{\eqalign{
\re(s)=\re(z),\cr
\im(s)=\im(z).
}\W.
$$


\Propriete La partie réelle et la partie imaginaire sont des applications $\ob R$-linéaires. Autrement dit, elles vérifient 
$$
\forall(s,z)\in\ob C^2, \qquad\forall(\lambda,\mu)\in\ob R^2, \qquad\Q\{
\eqalign{\re(\lambda s+\mu z)&=\lambda \re(s)+\mu\re(z),\cr
\im(\lambda s+\mu z)&=\lambda\im(s)+\mu\im(z).}\W.
$$

\noindent
Exercice : Prouver la propriété précédente. 

\Subsection CConjugaison, Conjugué d'un nombre complexe. 


Le corps $\ob C$ est naturellement muni d'une application remarquable : la conjugaison $z\mapsto\overline z$. 
\medskip

\Definition [$z=x+iy$ nombre complexe avec $x$ et $y$ nombres réels]
Le conjugé du nombre $z$ est le nombre complexe $\overline z$ défini par 
$$
\overline z:=x-iy. 
$$

Géomètriquement, la conjugaison est une symétrie par rapport à la droite réelle. 
Les~points $M$ et $P$ d'affixes respectives $z$ et $\overline z$ sont symétriques par rapport à l'axe $(O,\vec i)$. 

\centerline{
\tikzpicture
\clip (-1,-1.9) rectangle (5.5,1.9);
\draw[-] (-0.5,0) -- (5,0) coordinate (x axis);
\draw[-] (0,-1.7) -- (0,1.7) coordinate (y axis);
\node [anchor=north east] (O) at (0,0) {$O$};
\draw[-triangle 45,thick] (0,0) -- node [anchor=south,sloped,pos=0.4] {$x\vec i+y\vec j$} %node [anchor=north,sloped,color=red,pos=0.6] {$z=x+iy$}
 (4.3,1.4) node [anchor=south west] {$M (z)$};
\draw[dashed] (4.3,1.4) -- (0,1.4) node [anchor=east] {$iy$};
\draw[-triangle 45,thick] (0,0) -- node [anchor=north,sloped,pos=0.4] {$x\vec i-y\vec j$} % node [anchor=north,sloped,color=red,pos=0.4] {$\overline z=x-yi$} 
 (4.3,-1.4) node [anchor=north west] {$P (\overline z)$};
\draw[dashed,color=red] (4.3,-1.4) -- (0,-1.4) node [anchor=east] {$-iy$};
\draw[-triangle 45,thick] (0,0) -- node [anchor=south,pos=0.6] {$x\vec i$} node [color=red,anchor=north,pos=0.6] {$x$} (4.3,0) ;
\draw[-triangle 45,thick] (4.3,0) -- node [anchor=west] {$y\vec j$} (4.3,1.4);
\draw[|-triangle 45,thick] (4.3,0) -- node [anchor=west] {$-y\vec j$} (4.3,-1.4);
\draw[-open triangle 60,thick] (0,0) -- node [pos=0.4,anchor=east] {$\vec j$} (0,1);
\draw[-open triangle 60,thick] (0,0) -- node [anchor=north,pos=0.2] {$\vec i$} (1,0);
\endtikzpicture}%

\noindent
Il est possible d'exprimer un nombre $z$ et son conjugué $\overline z$ en fonction des parties réelles et imaginaires $\re(z)$ et $\im(z)$
$$
\forall z\in\ob C, \qquad z=\re(z)+i\ \im(z)\quad\mbox{et}\quad \overline z=\re(z)-i\ \im(z)
$$
et vice versa
$$
\forall z\in\ob C, \qquad \re(z)={z+\overline z\F 2}\quad\mbox{et}\quad\im(z)={z-\overline z\F2i}. 
$$
La conjugaison et les parties réelles et imaginaires permettent de caractériser 
simplement les nombres complexes qui sont des nombres réels et ceux qui sont imaginaires purs. 
\medskip

\Propriete 
Un nombre complexe $z$ est un nombre réel si, et seulement si, sa partie imagnaire est nulle. 
$$
z\mbox{ est réel}\quad\Longleftrightarrow\quad \overline z=z\quad\Longleftrightarrow\quad z=\re(z)\quad\Longleftrightarrow\quad\im(z)=0
$$

\Propriete 
Un nombre complexe $z$ est imaginaire pur si, et seulement si, sa partie réelle est nulle. 
$$
z\mbox{ est imaginaire pur}\quad\Longleftrightarrow\quad \overline z=-z\quad\Longleftrightarrow\quad z=i\ \im(z)\quad\Longleftrightarrow\quad\re(z)=0.
$$

On ne change pas un nombre complexe en le conjuguant successivement deux fois. 
\medskip

\Bullet Le nombre $\overline z$ est le conjugué de $z$ et le nombre $z$ est le conjugué de $\overline z$. Autrement~dit, les nombres $z$ et $\overline z$ sont conjugués. 
$$
\forall z\in\ob C,\qquad \overline{\thinspace\overline z\thinspace}=z.
$$

Le conjugué d'une somme est la somme des conjugués ; le conjugué d'une différence est la différence des conjugués et 
le conjugué d'un nombre multiplié par une constante réelle est le produit de son conjugué par la constante réelle. 
\medskip

\Bullet
La conjuguaison $z\mapsto\overline z$ est une application $\ob R$-linéaire. Autrement dit 
$$
\forall (s,z)\in\ob C^2,\qquad\forall(\lambda,\mu)\in\ob R^2, 
\qquad \lambda s+\mu z=\lambda\overline s+\mu\overline z.
$$


\Bullet Le conjugué d'un produit est le produit des conjugués. 
$$
\forall (s,z)\in\ob C^2, \qquad \overline{s\times z}=\overline s\times \overline z.
$$


\Bullet Le conjugué d'un quotient est le quotient des conjugués. 
$$
\forall (s,z)\in\ob C\times\ob C^*, \qquad 
\overline{\Q({s\F z}\W)}={\thinspace\overline s\thinspace\F \overline z}.
$$

Une conséquence immédiate de ces propriétés est que le conjugué d'une puissance est la puissance du conjugué : 
$$
\forall z\in\ob C^*, \qquad\forall n\in\ob Z, \qquad \overline{z^n}=\overline z^n.
$$

\Section CModule, Module d'un nombre complexe. 

\Definition 
Pour chaque nombre complexe $z=x+iy$, avec $x$ et $y$ nombres réels, le mo\-du\-le du nombre $z$ est le nombre réel positif $|z|$ défini par 
$$
|z|:=\sqrt{z\overline z}=\sqrt{x^2+y^2}. 
$$
\medskip
Dans un plan affine muni d'un repère orthonormé $(O,\vec i,\vec j)$, la distance de l'origine $O$ à un point $M$ d'affixe $z=x+iy$ 
est le module $|z|=\sqrt{x^2+y^2}$ de l'affixe du point $M$. 

\centerline{
\tikzpicture
\clip (-1,-0.7) rectangle (6.5,2.7);
\draw[-] (-0.5,0) -- (6,0) coordinate (x axis);
\draw[-] (0,-0.5) -- (0,2.7) coordinate (y axis);
\node [anchor=north east] (O) at (0,0) {$O$};
\draw[-triangle 45,thick] (0,0) -- node [anchor=south,sloped,pos=0.4] {$OM=|z|=\sqrt{x^2+y^2}$} 
 (5.3,2.4) node [anchor=west] {$M (z)$};
\draw[-triangle 45,thick] (0,0) -- node [anchor=south] {$x\vec i$} (5.3,0) ;
\draw[-triangle 45,thick] (5.3,0) -- node [anchor=east] {$y\vec j$} (5.3,2.4);
\draw[-open triangle 60,thick] (0,0) -- node [pos=0.4,anchor=east] {$\vec j$} (0,1);
\draw[-open triangle 60,thick] (0,0) -- node [anchor=north] {$\vec i$} (1,0);
\endtikzpicture}%

Le module et la conjugaison étant liés par la relation 
$$
\forall z\in\ob C, \qquad |z|^2=z\overline z. 
$$

\goodbreak
Une méthode classique pour mettre le quotient $s/z$ sous sa forme algébrique est : 

\Methode [Pour mettre l'inverse $1/z$ d'un nombre complexe {$z=x+iy$} non nul sous la~forme $a+ib$] 
On multiplie numérateur et dénominateur par le conjugué de $z$. 
$$
\forall z\in\ob C, \qquad {1\F z}={\overline z\F z\overline z}={\overline z\F |z|^2}={x-iy\F x^2+y^2}={x\F x^2+y^2}-i {y\F x^2+y^2}.
$$

\noindent
Exercice : Mettre sous la forme algèbrique $x+iy$ les nombres complexes suivants :
$$
{2+i\sqrt3\F\sqrt3-2i}\qquad\mbox{et}\qquad {(3+i)(2-i)\F2+i}.
$$

\noindent Le module possède les propriétés suivantes 
\bigskip

\Bullet L'application $z\mapsto|z|$ est bien définie et à valeurs réelles positives :
$$
\forall z\in\ob C, \qquad|z|\ge0.
$$

\Bullet L'application $z\mapsto|z|$ est dite ``définie'' :
$$
\forall z\in\ob C, \qquad|z|=0\ \Longleftrightarrow\ z=0.
$$

\Remarque : Ne pas confondre cette propriété ``définie'' avec la propriété standard ``défini=existe". 
\medskip 

\Bullet Inégalités triangulaires :
$$
\forall (s,z)\in\ob C^2, \qquad\B| |s|-|z|\B| \le |s+z|\le |s|+|z|.\eqdef{eqtriangulaire}
$$

Dans un plan affine muni d'un repère orthonormé $(O,\vec i,\vec j)$, la distance d'un point $M$ d'affixe $z=x_1+iy_1$ 
à un point $Q$ d'affixe $a=x_2+iy_2$ est 
$$
QM=\sqrt{(x_1-x_2)^2+(y_1-y_2)^2}=|z-a|.
$$




\Definition [$a$ nombre complexe et $r>0$ nombre réel] 
\item{$\bullet$} Le disque ouvert de centre $a$ et de rayon $r$ est l'ensemble 
$$
\sc D(a,r):=\{z\in\ob C:|z-a|<r\}. $$
\item{$\bullet$} Le disque fermé de centre $a$ et de rayon $r$ est l'ensemble 
$$
\overline{\sc D}(a,r):=\{z\in\ob C:|z-a|\le r\}. $$
\item{$\bullet$} Le cercle de centre $a$ et de rayon $r$ est l'ensemble
$$
\sc C(a,r):=\{z\in\ob C:|z-a|=r\}. 
$$


\centerline{
\tikzpicture
%\clip (-1.3,-0.5) rectangle (7,6);
\node [circle,fill=gray,pattern=north west lines,pattern color=gray,draw=red,thick] at (1,4.5) {$\sc D(a,r)$};
\node [circle,fill=gray,pattern=north west lines,pattern color=gray,draw=gray,thick] at (0,3) {$\overline{\sc D}(a,r)$};
\node [circle,draw=gray,thick] at (1,1.5) {$\sc C(a,r)$};
\draw (4,3) circle (2cm);
\node [label={[label distance=-0.3cm]-45:$a$}] at (4,3) {$\times$};
\draw [->] (4,3) -- node [anchor=south,sloped] {rayon $r$}++(200:2cm);
\draw [->] (4,3) -- node [anchor=south,sloped,pos=0.4] {$|z-a|$}++(50:2.5cm) node [label={[label distance=-0.3cm]-45:$z$}] {$\times$};
\endtikzpicture}


La relation \eqref{eqtriangulaire} étant également satisfaite par le nombre $z'=-z$, elle est équivalente 
à l'inégalité triangulaire
$$
\forall (s,z)\in\ob C^2, \qquad\Q| |s|-|z|\W| \le |s-z|\le |s|+|z|, 
$$
qui s'interprète géométriquement de la fa\c con suivante : le coté d'un triangle est plus petit que la somme 
des deux autres cotés et plus grand que leur différence. 


\medskip
\hfill
\pspicture*[](-1,-0.7)(8.8,3.5)
\psaxes*[labels=none,ticks=none]{-}(0,0)(-.5,-.5)(8.8,3.3)
\psline[linewidth=1pt,arrowsize=6pt]{->}(0,2)
\psline[linewidth=1.5pt,arrowsize=6pt]{->}(2,0)
\rput{0}(-.2,-.2){$O$}
\rput{0}(1,-.3){$\vec i$}
\rput{0}(-.3,1){$\vec j$}
\psline[linewidth=1.5pt,arrowsize=6pt,linecolor=blue]{->}(6,3)
\rput{0}(5.8,3.3){$M(z)$}
\rput{26.5651}(3,1.7){\blue {\bf distance }$OM=|z|$}
\psline[linewidth=1.5pt,arrowsize=6pt,linecolor=red]{->}(8,1)
\rput{0}(8.4,.8){$P(s)$}
\rput{7.12502}(4,0.7){\red {\bf distance }$OP=|s|$}
\psline[linewidth=1.5pt,arrowsize=6pt]{-}(6,3)(8,1)
\rput{-45}(7.1,2.2){{\bf distance}}
\rput{-45}(6.7,2){$MP=|s-z|$}
\endpspicture
\hfill\null
\medskip

\noindent
Le module d'un produit est le produit des modules. 
$$
{
\forall (s,z)\in\ob C^2}, \qquad{|s\times z|=|s|\times|z|}.
$$

\noindent
Un nombre complexe a même module que son conjugué. 
$$
{\forall z\in\ob C}, \qquad
{|z|=|\overline z|}.
$$

\noindent
Le module d'un quotient est le quotient des modules. 
$$
{\forall (s,z)\in\ob C\times \ob C^*}, \qquad
{\Q|{s\F z}\W|={\thinspace|s|\thinspace\F|z|}}.
$$
En particulier, le module d'une puissance est la puissance du module. 
$$
{\forall z\in\ob C^*}, \qquad
{\forall n\in\ob Z}, \qquad 
{\Q|z^n\W|=|z|^n}.
$$
Les parties réelles et imaginaires d'un nombre sont majorées par son module : 
$$
{
\forall z\in\ob C}, \qquad 
{\b|\re(z)\b|\le |z|}\quad\mbox{et}\quad
{\b|\im(z)\b|\le|z|}. 
$$
Le module d'un nombre réel est égal à sa valeur absolue. 
$$
\forall x\in\ob R, \qquad \underbrace{|x|}_{\mbox{module}}=\underbrace{|x|}_{\vtop{\mbox{valeur absolue}}}
$$

\Section CTrigonometrie, Forme trigonométrique des nombres complexes. 

\Subsection CGroupeU, Groupe $\ob U$ des nombres complexes de module $1$. 

Commen\c cons par introduire une nouvelle structure algèbrique : la structure de groupe. 
\medskip


\Definition [] Un ensemble $E$ muni d'une opération $\otimes$ est un groupe si, et seulement, si 
les cinq propriétés suivantes sont satisfaites : 
\item{i)}L'ensemble $E$ n'est pas vide : 
$$
\exists a\in E.
$$ 
\item{ii)}L'opération $\otimes$ est interne à $E$ : 
$$
\forall (a,b)\in E^2, \qquad a\otimes b\mbox{ est défini}\quad\mbox{et}\quad a\otimes b\in E
$$
\item{iii)}L'opération $\otimes$ admet un élément neutre, noté $e$, dans $E$ : 
$$
\exists e\in E:\qquad \forall a\in E,\qquad e\otimes a=a\otimes e=a
$$
\item{iv)}Chaque élément de l'ensemble $E$ admet un inverse pour la loi $\otimes$ : 
$$
\forall a\in E, \qquad \exists b\in E:\qquad a\otimes b=e=b\otimes a
$$
\item{v)}L'opération $\otimes$ est associative dans $E$ :
$$
\forall (a,b,c)\in E^3, \qquad(a\otimes b)\otimes c=a\otimes (b\otimes c)
$$ 

\Definition [] Un groupe $(E,\otimes)$ est commutatif (ou abélien) si, et seulement si, 
\item{vi)}L'opération $\otimes$ est commutative dans $E$ : 
$$
\forall (a,b)\in E^2, \qquad a\otimes b=b\otimes a
$$
\bigskip

%Exemples

Cette nouvelle structure permet de mémoriser plus facilement les propriétés de~$\ob C$. 
Ainsi, au paragraphe \xref{secDefinitionC}, % Fix that
 les propriétés i) à vi) signifient que {$(\ob C,+)$ est un groupe abélien}
et les propriétés viii) à xii) signifient que {$(\ob C^*,\times)$ est un groupe commutatif}. Il reste alors à~mémoriser la~propriété viii) 
affirmant que l'opération $\times$ est distributive sur la loi $+$. 
\medskip

{{ \noindent
{\bf Propriété.} L'ensemble $\ob U:=\{z\in\ob C:|z|=1\}$ des nombres complexes de module $1$ forme un groupe abélien 
pour la multiplication $\times$. 
}}
\bigskip

L'ensemble $\ob U$ forme un cercle de centre $0$ et~de~rayon~$1$, 
appelé cercle unité et parfois appelé cercle trigonométrique pour des raisons, qui seront détaillées plus loin. 
\par\noindent\hfill
\pspicture*[](-2,-2)(2,2)
\psaxes*[labels=none,ticks=none]{<->}(0,0)(-2,-2)(2,2)
\pscircle[linewidth=1pt,linecolor=red](0,0){1.5}
\rput{0}(1.7,-.2){$1$}
\rput{0}(-1.8,-.2){$-1$}
\rput{0}(-.2,1.7,){$i$}
\rput{0}(-.3,-1.7){$-i$}
\rput{0}(-.2,-.2){$0$}
\rput{0}(1.5,0){$+$}
\rput{0}(-1.5,0){$+$}
\rput{0}(0,1.5){$+$}
\rput{0}(0,-1.5){$+$}
\psline[linewidth=.5pt]{-}(1.06066,1.06066)(1.3,1.3)
\rput{0}(1.5,1.5){\red$\ob U$}
\endpspicture
\hfill\null\medskip

{{ \quad Soit $(E,\oplus)$ un groupe. Pour montrer qu'un sous-ensemble $F$ 
de l'ensemble $E$

\noindent 
est un groupe pour la loi $\oplus$, 
c'est-à-dire que $(F,\otimes)$ est un sous-groupe de $(E,\otimes)$, 

\noindent\ 
il suffit de prouver que :

\noindent
\quad $\bullet$ $F$ est non vide

\noindent
\quad $\bullet$ $F$ est inclus dans $E$

\noindent
\quad{$\bullet$} $\forall (x,y)\in F^2,x\otimes y^{-1}\in F$\quad(où $y^{-1}$ désigne l'inverse de $y$ pour la loi $\otimes$ de $E$)}}
\bigskip

\noindent 
Exercice. Prouver que $(\ob Z,+)$ et que $(\ob R^*,\times)$ sont des groupes (abéliens).

\Subsection CArgument, Argument d'un nombre complexe non nul. 

\Definition [] L'argument d'un nombre complexe $z$ de module $1$ est la longueur $\theta$ en~radian 
de l'arc du cercle trigonométrique allant de $1$ à $z$ dans le sens direct. Cette~longueur, qui est unique modulo $2\pi$, 
est notée 
$$
\arg(z)\equiv\theta\quad[2\pi].
$$


\pspicture*[](-2,-2)(2,2)
\psaxes*[labels=none,ticks=none]{<->}(0,0)(-2,-2)(2,2)
\rput{0}(1.7,-.2){$1$}
\rput{0}(-.2,-.2){$0$}
\rput{0}(1.5,0){$+$}
\psarc[linecolor=red,arrowsize=6pt]{->}(0,0){1.5}{0}{110}
\psarc{-}(0,0){1.5}{110}{360}
\rput{110}(-0.51303,1.40954){+}
\rput{0}(-0.7,1.6){z}
\rput{0}(1.3,1.3){\red$\theta$}
\endpspicture


Pour chaque nombre complexe $z$ {\bf non nul}, le nombre $z/|z|$ est bien défini 
et appartient au cercle unité $\ob U$. En effet, il est de module $1$ car 
$$
\Q|{z\F |z|}\W|={|z|\F|z|}=1. 
$$

\Definition [] L'argument d'un nombre complexe $z$ {\bf non nul} est l'argument de $z/|z|$. 
Notant $O$, $P$, $M$ les points du plan complexe d'affixe respective $0$, $1$ et $z$, 
c'est la mesure $\theta$ de l'angle orienté $(\vec {OP},\vec {OM})$ en radian, qui est unique modulo $2\pi$, notée 
$$
\arg(z)\equiv\theta\quad[2\pi].
$$

\medskip
\hfill
\pspicture*[](-.5,-.5)(3,3)
\psaxes*[labels=none,ticks=none]{-}(0,0)(-.5,-.5)(3,3)
\psline[linewidth=1.5pt,arrowsize=6pt]{->}(2,0)
\rput{0}(-.2,-.2){$0$}
\rput{0}(2,-.3){$1$}
\rput{0}(2,0){$+$}
\rput{0}(1,-.3){$\vec{OP}$}
\rput{0}(0,2){$+$}
\rput{0}(-.3,2){$i$}
\psline[linewidth=1.5pt,arrowsize=6pt]{->}(2.7,2.4)
\rput{0}(2.7,2.4){$+$}
\rput{41.6335}(.6,.9){$\vec{OM}$}
\rput{0}(2.9,2.2){$z$}
\psarc[linecolor=red]{->}(0,0){2}{0}{41.6335}
\rput{0}(2.1,.8){\red $\theta$}
\rput{0}(1.5,1.8){\red${ z\F |z|}$}
\rput{41.6335}(1.49482,1.32873){$+$}
\endpspicture
\hfill\null
\medskip

\Propriete Deux nombres complexes non nuls sont égaux si, et seulement s'ils ont même module et même argument. 
$$
{\forall (s,z)\in(\ob C^*)^2}, \qquad {s=z\ \Longleftrightarrow\ \Q\{\eqalign{
|s|&=|z|
\cr
\arg(s)&\equiv\arg(z)\quad[2\pi]
}\W.}
$$

\Propriete [$z$ et $s$ nombres complexes non nuls] 
$$
\arg(z\times s)=\arg(z)+\arg(s)\quad[2\pi]\qquad\mbox{et} \arg\Q({z\F s}\W)=\arg(z)-\arg(s)\quad[2\pi].
$$

La propriété suivante est une conséquence immédiate des relations précédente.
\medskip


\Propriete [$z$ nombre complexe non nul] 
$$
\arg(\overline z)\equiv-\arg(z)\quad[2\pi]\qquad\mbox{et}\qquad \arg\Q({1\F z}\W)\equiv-\arg(z)\quad[2\pi]
$$

\Subsection Ccosinus, Cosinus et sinus. 

Parce qu'il permet de définir les fonctions trigonométriques cosinus et sinus, 
le cercle de centre $0$ et de rayon $1$ est parfois appelé cercle trigonométrique. 
\medskip


\Definition [] Pour chaque nombre réel $\theta$, les parties réelles et imaginaires 
de l'unique nombre complexe $z$ de module $1$ et d'argument 
$\theta$ sont notées $\cos(\theta)$ et $\sin(\theta)$. 

%\pspicture*[](-3,-3)(5.2,3)
%\psaxes*[labels=none,ticks=none]{<->}(0,0)(-3,-3)(3,3)
%\pscircle[linewidth=1pt,fillcolor=white](0,0){2.5}
%\rput{40}(0.7,.8){rayon$=1$}
%\rput{40}(1.91511,1.60697){$+$}
%\rput{0}(3.6,1.8){$z=\cos(\theta)+i\sin(\theta)$}
%\rput{0}(0,1.60697){$+$}
%\rput{0}(-.7,1.60697){\red$i\ \sin(\theta)$}
%\psline[linewidth=1.5pt,linecolor=red,linestyle=dotted]{-}(0,1.60697)(1.91511,1.60697)
%\psline[linewidth=1.5pt,linecolor=red,linestyle=dotted]{-}(1.911511,0)(1.91511,1.60697)
%\psline[linewidth=1.5pt,arrowsize=6pt]{->}(1.91511,1.60697)
%\psarc[linecolor=red]{->}(0,0){1}{0}{40}
%\rput{0}(1.1,0.4){\red $\theta$}
%\rput{0}(1.91511,0){$+$}
%\rput{0}(1.91511,-0.3){\red$\cos(\theta)$}
%\psline[linewidth=.5pt]{-}(-1.76777,1.76777)(-2,2)
%\rput{0}(-2.2,2.2){$\ob U$}
%\rput{0}(-.2,2.7){$i$}
%\rput{0}(0,2.5){$+$}
%\rput{0}(-.2,-.2){$0$}
%\rput{0}(2.5,0){$+$}
%\rput{0}(2.7,.2){$1$}
%\endpspicture

\noindent
Les fonctions cosinus $x\mapsto\cos(x)$ et sinus $x\mapsto\sin(x)$ sont $2\pi$-périodiques 
$$
\forall x\in\ob R, \quad \cos(x+2\pi)=\cos(x)\quad\mbox{et} \quad\sin(x+2\pi)=\sin(x)
$$
et bornées 
$$
\forall x\in\ob R, \qquad-1\le \cos(x)\le 1\quad\mbox{et}\quad-1\le\sin(x)\le1.
$$
De plus, en pivotant le cercle trigonométrique d'un quart de tour, on peut exprimer le cosinus en fonction du sinus et vice versa, 
$$
\forall x\in\ob R, \quad \cos\Q(x+{\pi\F2}\W)=-\sin(x) \quad\mbox{et} \quad
\sin\Q(x+{\pi\F2}\W)=\cos(x), \eqdef{translationFTC}
$$ 
comme on peut le voir sur le graphe $y=\cos(x)$ et $y=\sin(x)$ suivants : 


\pspicture*[](-6.8,-1.2)(6.8,1.2)
%\psset{xunit=1cm, yunit=2cm}
\psplot[linecolor=red,plotpoints=1000]{-6.8}{6.8}{x 3.1415 div 180 mul sin}
\psplot[linecolor=blue,plotpoints=1000]{-6.8}{6.8}{x 3.1415 div 180 mul cos}
\psaxes*[labels=none,ticks=none]{<->}(0,0)(-6.8,-1.2)(6.8,1.2)
\rput{0}(3.1415,0){$+$}
\rput{0}(3.18,.2){$\pi$}
\rput{0}(-1.5708,0){$+$}
\rput{0}(-1.79,.2){$-{\pi\F2}$}
\rput{0}(1.5708,0){$+$}
\rput{0}(1.65,.2){${\pi\F2}$}
\rput{0}(-.2,.15){$0$}
\rput{0}(-3.1415,0){$+$}
\rput{0}(-3,.2){$-\pi$}
\rput{0}(4.71239,0){$+$}
\rput{0}(4.5,.2){${3\pi\F2}$}
\rput{0}(-4.71239,0){$+$}
\rput{0}(-4.5,.2){$-{3\pi\F2}$}
\rput{0}(-6.283,0){$+$}
\rput{0}(-6.6,.2){$-2\pi$}
\rput{0}(6.283,0){$+$}
\rput{0}(6.1,.2){$2\pi$}
\rput{0}(0,1){$+$}
\rput{0}(-.15,.9){$1$}
\rput{0}(0,-1){$+$}
\rput{0}(-.4,-1){$-1$}
\rput{0}(.2,1.1){$y$}
\rput{0}(6.7,-.1){$x$}
\psline[linewidth=.5pt]{-}(-4.85,-1)(-4,-.65)
\rput{0}(-5.5,-1){\blue cosinus}
\psline[linewidth=.5pt]{-}(-6,-.6)(-6.7,-.40485)
\rput{0}(-5.5,-.6){\red sinus}
\endpspicture

\centerline{
	\tikzpicture[domain=-4:4,samples=66]
		\draw[very thin,color=gray,step={(1.570796327,1)}] (-4.1,-1.1) grid (4.1,1.1);
		\draw[->] (0,0) -- (4.2,0) node[below] {$x$};
		\draw[->] (0,0) -- (0,1.2) node[left] {$y$};
		\draw[color=red,smooth] plot (\x,{cos(\x r)}) ;
		\node[rotate=-45,color=red] at (1.45,0.4) {$\cos x$};
		\draw[color=blue,smooth] plot (\x,{sin(\x r)});
		\node[rotate=-45,color=blue] at (3.1,0.4) {$\sin x$};
	\endtikzpicture
}%
\Figure [Index=Courbes!cosinus et sinus] Graphes $y=\cos(x)$ et $y=\sin(x)$.

\noindent
Les fonctions {cosinus et sinus sont continues et indéfiniment dérivables sur $\ob R$} et on a 
$$
{
\forall x\in\ob R,\quad
\cos'(x)=-\sin(x)\quad\mbox{et}\quad \sin'(x)=\cos(x).}\eqdef{deriverFTC}
$$
Il résulte de \eqref{deriverFTC} et \eqref{translationFTC} que l'on peut dériver $n$ fois 
ces fonctions de~la fa\c con suivante : 
$$
{\forall n\in\ob N}, \quad\forall x\in\ob R, \quad {\cos^{(n)}(x)=\cos\Q(x+n{\pi\F2}\W)}
\quad\mbox{et}\quad{\sin^{(n)}(x)=\sin\Q(x+n{\pi\F2}\W)}.
$$
Exercice. Prouver l'identité précédente. 
\bigskip

\Subsection CExp, Exponentielle d'un nombre complexe.

La fonction exponentielle est initialement définie sur l'ensemble $\ob R$, 
en tant que bijection réciproque du logarithme népérien $\ln:x\mapsto\int_1^x{\d t\F t}$. 
On la prolonge dans un premier temps à l'ensemble des nombres imaginaires purs pour la prolonger, 
dans un second temps, à~l'ensemble des nombres complexes. 
\bigskip
 
\Definition [] Pour chaque {$\theta\in\ob R$}, l'exponentielle du nombre imaginaire pur $i\theta$ est 
$$
{\e^{i\theta}:=\cos\theta+i\sin\theta}.
$$
\medskip
\noindent
L'exponentielle transforme un nombre imaginaire pur en nombre complexe de $\ob U$ : 
$$
{\forall \theta\in\ob R}, \qquad {\Q|\e^{i\theta}\W|=1}.
$$
De plus, la fonction $\theta\mapsto\e^{i\theta}$ est un morphisme du groupe $(\ob R,+)$ dans le groupe $(\ob U,\times)$. 
Autrement dit, l'exponentielle satisfait 
$$
{\forall (\theta,\varphi)\in\ob R^2}, \qquad {\e^{i(\theta+\varphi)}=\e^{i\theta}\times\e^{i\varphi}}.
$$

\noindent
Chaque nombre complexe peut se mettre sous la forme trigonométrique (la forme trigonométrique du nombre complexe $0$ n'étant pas unique) : 
\bigskip
\Propriete [] Pour chaque nombre complexe {$z\neq0$}, il existe un unique nombre $r>0$ 
et un nombre réel $\theta$, unique modulo $2\pi$, tels que 
$$
z=r\e^{i\theta}. \eqdef{formtrig}
$$
Cette expression est la forme trigonométrique de $z$ et l'on a 
$r=|z|$ et $\arg z\equiv\theta\ \,[2\pi]$. 
\bigskip


\Definition [] L'exponentielle d'un nombre complexe $z=x+iy$, avec $x$ et $y$ réels, est 
$$
\e^z:=\e^x\times\e^{iy}.
$$

\noindent
En particulier, on a 
$$
{\forall z\in\ob C}, \qquad {|\e^z|=\e^{\re(z)}}\qquad\mbox{et}
\qquad{\arg(\e^z)\equiv\im(z)\quad[2\pi]}.
$$


\bigskip
\noindent
La fonction exponentielle est un morphisme du groupe $(\ob C,+)$ dans le groupe $(\ob C^*,\times)$. Autrement dit, 
l'exponentielle satisfait la propriété fondamentale suivante : 
$$
{\forall (s,z)\in\ob C^2}, \qquad{\e^{s+z}=\e^s\times\e^z}.\eqdef{expfon}
$$


\bigskip
\noindent
L'exponentielle d'un nombre complexe $z$ n'est jamais nulle et son inverse est $\e^{-z}$.  
$$
{\forall z\in\ob C}, \qquad {\e^z\neq0}\qquad\mbox{et}\qquad {{1\F\e^z}=\e^{-z}}.
$$
A fortiori, la puissance positive ou non d'une exponentielle se calcule très simplement : 
$$
{\forall z\in\ob C},\qquad {\forall n\in\ob Z},\qquad {(\e^z)^n=\e^{nz}}. 
$$

Enfin, on résoud facilement les équations du type $\e^z=a$ pour $a\neq0$ à l'aide de \eqref{expfon} et de la propriété suivante. 
$$
{
\e^z=1\quad \Longleftrightarrow\quad \exists k\in\ob Z: z=2\pi i k\quad \Longleftrightarrow\quad z\in2\pi i\ob Z}.
$$


\Propriete [] Pour $z\in\ob C$, la fonction {$x\mapsto \e^{zx}$ est indéfiniment dérivable sur $\ob R$} et on a 
$$
\forall x\in\ob R, \qquad {{\d\F\d x}\e^{zx}=z\e^{zx}}.
$$
En particulier, on a $(\e^x)'=\e^x$ pour chaque nombre réel $x$. 

\Section Trigo, Trigonométrie. 

\Subsection Tan, Tangente.

Les solutions de l'équation trigonométrique $\cos(x)=0$ se caractérisent très simplement. 
Ainsi, pour chaque nombre réel $x$, on a 
$$
{\cos x=0\ \Longleftrightarrow\ x\equiv {\pi\F2}\quad[\pi]}.
$$
La fonction tangente $x\mapsto\tan(x)$ est alors définie 
comme le quotient de la fonction sinus par la fonction cosinus pour les nombres réels n'annulant pas le dénumérateur du quotient. 
$$
{
\forall x\not\equiv {\pi\F2}\quad[\pi]}, \qquad {\tan(x):={\sin(x)\F\cos(x)}}.
$$
La fonction tangente est $\pi$-périodique, autrement dit 
$$
{
\forall x\not\equiv {\pi\F2}\quad[\pi]}, \qquad {\tan(x+\pi)=\tan(x)},
$$ 
et impaire, autrement dit 
$$
{
\forall x\not\equiv {\pi\F2}\quad[\pi]}, \qquad {\tan(-x)=-\tan(x)}, 
$$
comme on peut le voir sur le graphe suivant : 

%% Mathematica : Table[{x,Tan[x]},{x,-Pi/2+0.32,Pi/2-0.32,0.02}]%
%%\readdata{\tangraph}{Graphes/GraphTan.txt}
%%\psset{xunit=.6cm,yunit=.6cm}
%\pspicture*[](-7,-3)(7,3)
%\psaxes*[labels=none,ticks=none]{<->}(0,0)(-7,-3)(7,3)
%\dataplot[plotstyle=curve,linewidth=.5pt,linecolor=red]{\tangraph}
%\rput{0}(6.283,0){\dataplot[plotstyle=curve,linewidth=.5pt,linecolor=red]{\tangraph}}
%\rput{0}(3.1415,0){\dataplot[plotstyle=curve,linewidth=.5pt,linecolor=red]{\tangraph}}
%\rput{0}(-3.1415,0){\dataplot[plotstyle=curve,linewidth=.5pt,linecolor=red]{\tangraph}}
%\rput{0}(-6.283,0){\dataplot[plotstyle=curve,linewidth=.5pt,linecolor=red]{\tangraph}}
%\psline[linestyle=dashed,linewidth=.3pt]{-}(1.5707,-3)(1.5707,3)
%\psline[linestyle=dashed,linewidth=.3pt]{-}(4.7122,-3)(4.7122,3)
%\psline[linestyle=dashed,linewidth=.3pt]{-}(-1.5707,-3)(-1.5707,3)
%\psline[linestyle=dashed,linewidth=.3pt]{-}(-4.7122,-3)(-4.7122,3)
%\rput{0}(1.35,-.3){\scalebox{.5}{${\pi\F2}$}}
%\rput{0}(4.4,-.3){\scalebox{0.5}{${3\pi\F2}$}}
%\rput{0}(-1.9,-.3){\scalebox{.5}{$-{\pi\F2}$}}
%\rput{0}(-5.1,-.3){\scalebox{0.5}{$-{3\pi\F2}$}}
%\rput{0}(0.2,-.2){\scalebox{.5}{$0$}}
%\rput{0}(3.34,-.2){\scalebox{.5}{$\pi$}}
%\rput{0}(6.5,-.2){\scalebox{.5}{$2\pi$}}
%\rput{0}(-2.9,-.2){\scalebox{.5}{$-\pi$}}
%\rput{0}(0,1){\scalebox{.5}{$+$}}
%\rput{0}(0,-1){\scalebox{.5}{$+$}}
%\rput{0}(-0.3,1){\scalebox{.5}{$1$}}
%\rput{0}(-.4,-1){\scalebox{.5}{$-1$}}
%\rput{0}(-.3,2.9){\scalebox{.5}{$y$}}
%\rput{0}(6.9,.3){\scalebox{.5}{$x$}}
%\endpspicture
%$$
\medskip

\centerline{%
	\tikzpicture[scale=0.5]
		\draw[clip] (-4.8,-3.1) rectangle (4.8,3.1);
		\draw[very thin,color=gray,clip,step={(1.570796327,1)}] (-4.8,-3.1) grid (4.8,3.1);
		\draw[->] (0,0) -- (4.2,0) node[below] {$x$};
		\draw[->] (0,0) -- (0,1.2) node[left] {$y$};
		\draw[domain=-4.55:-1.75,samples=66,color=red,smooth] plot (\x,{tan(\x r)}) ;
		\draw[domain=-1.4:1.4,samples=66,color=red,smooth] plot (\x,{tan(\x r)}) ;
		\draw[domain=1.75:4.55,samples=66,color=red,smooth] plot (\x,{tan(\x r)}) ;
	\endtikzpicture
}%
\Figure [Index=Courbes!Tangente] Graphe $y=\tan(x)$. 
\medskip

\noindent
La fonction tangente est indéfiniment dérivable sur son ensemble de définition et 
$$
\forall x\not\equiv{\pi\F2}\quad[\pi], \quad{\tan'(x)=1+\tan(x)^2={1\F\cos(x)^2}}.
$$

\Subsection Cotan, Cotangente. 

Les solutions de l'équation trigonométrique $\sin(x)=0$ se caractérisent très simplement. 
Ainsi, pour chaque nombre réel $x$, on a 
$$
{\sin x=0\ \Longleftrightarrow\ x\equiv 0\quad[\pi]}.
$$
La fonction cotangente $x\mapsto\cotan(x)$ est alors définie 
comme le quotient de la fonction cosinus par la fonction sinus pour les nombres réels n'annulant pas le dénumérateur. 
$$
{
\forall x\not\equiv 0\quad[\pi]}, \qquad {\cotan(x):={\cos(x)\F\sin(x)}}.
$$
La fonction cotangente est $\pi$-périodique, autrement dit 
$$
{
\forall x\not\equiv 0\quad[\pi]}, \qquad {\cot(x+\pi)=\cot(x)}
$$ 
et impaire, autrement dit 
$$
{
\forall x\not\equiv 0\quad[\pi]}, \qquad {\cot(-x)=-\cot(x)}, 
$$ 
comme on peut le voir sur le graphe suivant : 

% Mathematica : Table[{x,Cot[x]},{x,0.32,Pi-0.32,0.02}]
%%\readdata{\cotgraph}{Graphes/GraphCot.txt}
%%\psset{xunit=.6cm,yunit=.6cm}
%\pspicture*[](-7,-3)(7,3)
%\psaxes*[labels=none,ticks=none]{<->}(0,0)(-7,-3)(7,3)
%\dataplot[plotstyle=curve,linewidth=.5pt,linecolor=red]{\cotgraph}
%\rput{0}(6.283,0){\dataplot[plotstyle=curve,linewidth=.5pt,linecolor=red]{\cotgraph}}
%\rput{0}(3.1415,0){\dataplot[plotstyle=curve,linewidth=.5pt,linecolor=red]{\cotgraph}}
%\rput{0}(-3.1415,0){\dataplot[plotstyle=curve,linewidth=.5pt,linecolor=red]{\cotgraph}}
%\rput{0}(-6.283,0){\dataplot[plotstyle=curve,linewidth=.5pt,linecolor=red]{\cotgraph}}
%\psline[linestyle=dashed,linewidth=.3pt]{-}(3.1415,-3)(3.1415,3)
%\psline[linestyle=dashed,linewidth=.3pt]{-}(6.283,-3)(6.283,3)
%\psline[linestyle=dashed,linewidth=.3pt]{-}(-3.1415,-3)(-3.1415,3)
%\psline[linestyle=dashed,linewidth=.3pt]{-}(-6.283,-3)(-6.283,3)
%\rput{0}(1.35,-.3){\scalebox{.5}{${\pi\F2}$}}
%\rput{0}(4.4,-.3){\scalebox{0.5}{${3\pi\F2}$}}
%\rput{0}(-1.9,-.3){\scalebox{.5}{$-{\pi\F2}$}}
%\rput{0}(-5.1,-.3){\scalebox{0.5}{$-{3\pi\F2}$}}
%\rput{0}(-0.2,-.2){\scalebox{.5}{$0$}}
%\rput{0}(2.94,-.2){\scalebox{.5}{$\pi$}}
%\rput{0}(6,-.2){\scalebox{.5}{$2\pi$}}
%\rput{0}(-6.65,-.2){\scalebox{.5}{$-2\pi$}}
%\rput{0}(-3.4,-.2){\scalebox{.5}{$-\pi$}}
%\rput{0}(0,1){\scalebox{.5}{$+$}}
%\rput{0}(0,-1){\scalebox{.5}{$+$}}
%\rput{0}(-0.3,1){\scalebox{.5}{$1$}}
%\rput{0}(-.4,-1){\scalebox{.5}{$-1$}}
%\rput{0}(-.3,2.9){\scalebox{.5}{$y$}}
%\rput{0}(6.9,-.3){\scalebox{.5}{$x$}}
%\endpspicture
\medskip

\centerline{%
	\tikzpicture[scale=0.5]
		\draw[clip] (-3.2,-3.1) rectangle (3.2,3.1);
		\draw[very thin,color=gray,clip,step={(1.570796327,1)}] (-3.2,-3.1) grid (3.2,3.1);
		\draw[->] (0,0) -- (3.2,0) node[below] {$x$};
		\draw[->] (0,0) -- (0,3.2) node[left] {$y$};
		\draw[domain=-3:-0.1,samples=66,color=red,smooth] plot (\x,{cot(\x r)}) ;
		\draw[domain=0.1:3,samples=66,color=red,smooth] plot (\x,{cot(\x r)}) ;
	\endtikzpicture
}%
\Figure [Index=Courbes!Cotangente]  Graphe $ y=\cot(x)$. 
\medskip

\noindent
Les fonctions tangentes et cotangentes sont inverses l'une de l'autre : 
$$
{
\forall x\not\equiv 0\quad\Q[{\pi\F2}\W]}, \qquad {\cot(x)={1\F\tan(x)}}. 
$$
La fonction {cotangente est indéfiniment dérivable sur son ensemble de définition} et 
$$
\forall x\not\equiv0\quad[\pi], \quad\cot'(x)=-1-\cot(x)^2=-{1\F\sin(x)^2}
$$

\Subsection FormCos, Propriétes des fonctions trigonométriques.
\bigskip


%\Section Parité
%
%La fonction cosinus est paire et la fonction sinus est impaire, autrement dit 
%$$
%\forall x\in\ob R, \quad {\cos(-x)=\cos(x)}\quad\mbox{et}\quad {\sin(-x)=-\sin(x)}.
%$$


%\Section Périodicité
% 
%$$
%\forall x\in\ob R, \qquad{\cos(x+2\pi)=\cos(x)},\qquad{\sin(x+2\pi)=\sin(x),}
%$$

\Concept [] Anti-période $\pi$
 
$$
\forall x\in\ob R, \qquad{\cos(x+\pi)=-\cos(x)}\qquad\mbox{et}\qquad{\sin(x+\pi)=-\sin(x).}
$$
Symétrie centrale du cercle trigonométrique par rapport à l'origine du repère.
\bigskip

\Concept [] Angles supplémentaires

$$
\forall x\in\ob R, \qquad{\cos(\pi-x)=-\cos(x)}\qquad\mbox{et}\qquad{\sin(\pi-x)=\sin(x).}
$$
Symétrie du cercle trigonométrique par rapport à l'axe des ordonnées. 
\bigskip
\Concept [] Angles complémentaires

$$
\forall x\in\ob R, \qquad{\cos\Q({\pi\F2}-x\W)=\sin(x)}\qquad\mbox{et}
\qquad{\sin\Q({\pi\F2}-x\W)=\cos(x).}
$$
Symétrie du cercle trigonométrique par rapport à la première bissectrice du repère...
\bigskip
\Concept [] Relations d'Euler

$$
{\forall\theta\in\ob R}, \qquad {\ds\cos\theta=\re(\e^{i\theta})={\e^{i\theta}+\e^{-i\theta}\F2}}\quad\mbox{et}\quad
{\ds\sin\theta=\im(\e^{i\theta})={\e^{i\theta}-\e^{-i\theta}\F2i}}
$$
Ces relations fondamentales permettent de transformer un problème tri\-go\-no\-mé\-tri\-que réel 
en problème trigonométrique complexe et réciproquement.
\bigskip


\Concept [] Formules d'addition

\noindent
Soient {$a$ et $b$ des nombres réels}. Alors, on a 
$$
\eqalign{
&{\cos(a+b)=\cos a\cos b-\sin a\sin b},\qquad{\cos(a-b)=\cos a\cos b+\sin a \sin b},
\cr
&{\sin(a+b)=\sin a\cos b+\cos a\sin b},\qquad{\sin(a-b)=\sin a\cos b-\cos a\sin b},
\cr
&{\ds\tan(a+b)={\tan a+\tan b\F1-\tan a\tan b}}\ \qquad\mbox{et}\qquad \tan(a-b)={\tan a-\tan b\F1+\tan a\tan b},
}
$$
si les tangentes sont définies...
\bigskip

\Concept [] Duplication de l'angle

\noindent
Pour chaque {$x\in\ob R$}, on a 
$$
{\cos(2x)=\cos^2x-\sin^2x}, \qquad{\sin(2x)=2\sin x\cos x}\quad\mbox{et}\quad
\tan(2x)={2\tan x\F1-\tan^2x}.
$$
Ces formules permettent de déduire la valeur du cosinus, du sinus ou de la tangente d'un angle doublé de la valeur de ces fonctions 
pour l'angle. 
\bigskip
 
\Concept [] Formule de Moivre

$$
{\forall \theta\in\ob R}, \qquad{\forall n\in\ob Z}, \qquad {\cos(n\theta)+i\sin(n\theta)=(\cos\theta+i\sin\theta)^n}. 
$$
La formule précédente est à utiliser en conjonction avec le binôme de Newton. 
\bigskip

Exercice. Développer $\cos(6\theta)$. 
\bigskip
 

\Concept [] Relation entre cosinus et sinus
 
$$
\forall x\in\ob R,\qquad{\cos^2x+\sin^2x=1}.\eqdef{coscpsinc}
$$
Géométriquement, cette relation signifie que le point d'affixe $\cos x+i\sin x$ appartient 
au~cercle de centre $0$ et de rayon $1$ 
\bigskip

\Concept [] Linéarisation d'un produit 

Pour chaque couple $(a,b)$ de nombres réels, on a 
$$
\eqalign{
\sin a \sin b={\cos(a-b)-\cos(a+b)\F2},&\qquad\sin a\cos b={\sin(a+b)+\sin(a-b)\F2_{\strut}}\cr
\mbox{et }\quad\cos a\cos b^{\strut}=&{\cos(a+b)+\cos(a-b)\F2}.
}\eqdef{lin}
$$
Ces formules permettent de transformer un produit de fonctions trigonométriques en sommes de cosinus ou de sinus. 
C'est très utile pour les intégrer, par exemple. 
\bigskip


\Concept [] Factorisation d'une somme 

\noindent
Soient {$p$ et $q$ des nombres réels}. Alors, on a 
\medskip
\noindent
{$\ds\cos p+\cos q=2\cos\Q(\!{p+q\F2}\!\W)\cos\Q(\!{p-q\F2}\!\W)$}, \hfill
{$\ds\cos p-\cos q=-2\sin\Q(\!{p+q\F2}\!\W)\sin\Q(\!{p-q\F2}\!\W)$},
\medskip
\noindent
{$\ds\sin p+\sin q=2\sin\Q({p+q\F2}\W)\cos\Q({p-q\F2}\W)$},\hfill
{$\ds\sin p-\sin q=2\cos\Q({p+q\F2}\W)\sin\Q({p-q\F2}\W)$}.
\medskip
\noindent
Ces formules permettent de factoriser des expressions trigonométriques. 
C'est très utile pour résoudre des équations trigonométriques. 
\bigskip

\Concept [] Tangente de l'angle moitié 

\noindent
Soit {$x\not\equiv \pi\ \,[2\pi]$} un nombre réel et soit 
{$\ds t:=\tan{x\F 2}$}. Alors, on a 
$$
{\cos x={1-t^2\F1+t^2}}\qquad\mbox{et}\qquad
{\sin x={2t\F1+t^2}}.
$$
Si le nombre $x$ satisfait de plus {$\ds x\not\equiv {\pi\F2}\ \,[\pi]$}, on a 
$$
{\tan x={2t\F1-t^2}}.
$$
Ces formules, qui permettent d'exprimer le cosinus, le sinus et la tangente d'un angle en fonction 
de la tangente de l'angle moitié, servent essentiellement à effectuer des changements de variables dans les intégrales. 
\bigskip

\Subsection LineTrigo, Applications classiques.

\bigskip
\Concept Equation trigonométrique $f(x)=f(a)$
 
\noindent
Pour {$a\in\ob R$}, on a 
$$
\eqalign{
&{\cos(x)=\cos(a)\ \Longleftrightarrow\ x\equiv a\ [2\pi]\ \mbox{ ou }\ x\equiv -a\ [2\pi]}
\cr
&{\sin(x)=\sin(a)\ \Longleftrightarrow\ x\equiv a\ [2\pi]\ \mbox{ ou }\ x\equiv \pi-a\ [2\pi]}
}
$$
Si de plus {$\ds a\not\equiv{\pi\F2}\ \,[\pi]$}, on a 
$$
{\tan(x)=\tan(a)\ \Longleftrightarrow\ x\equiv a\ [\pi]}. 
$$ 

\Concept Equation trigonométrique $a\cos(x)+b\sin(x)=c$ 

Soient $a$ et $b$ des nombres réels tels que $(a,b)\neq(0,0)$. 
\bigskip


{{ \centerline{\bf Technique à retenir}
\medskip
\noindent\hfill$\ds
a\cos(x)+b\sin(x)=\underbrace{\sqrt{a^2+b^2}}_{\ss r}\Bigg(\underbrace{{a\F\sqrt{a^2+b^2}}}_{\cos(\theta)}\cos(x)
+\underbrace{{b\F\sqrt{a^2+b^2}}}_{\sin(\theta)}\sin(x)\Bigg)=r\cos(x-\theta).
$\hfill\null\par
}}
\bigskip

\noindent
Pour résoudre l'équation trigonométrique  
$$
a\cos(x)+b\sin(x)=c, \eqdef{eqtrig}
$$

on la transforme pour la mettre sous la forme $\ds\cos(\theta)\cos(x)+\sin(\theta)\sin(x)={c\F r}$, 
c'est-à-dire sous une forme étudiée au paragraphe précédent : 
$$
\cos(x-\theta)={c\F r}. 
$$ 
Pour cela, on divise \eqref{eqtrig} par le nombre strictement positif $r:=\sqrt{a^2+b^2}$ pour~obtenir~que 
$$
\eqref{eqtrig}\ \Longleftrightarrow\ {a\F\sqrt{a^2+b^2}}\cos(x)+{b\F\sqrt{a^2+b^2}}\sin(x)={c\F r}.
$$
Puis, on remarque alors que le point de coordonnées $\Q({a\F\sqrt{a^2+b^2}},{b\F\sqrt{a^2+b^2}}\W)$ 
appartient au cercle trigonométrique et donc qu'il existe un nombre réel $\theta$, unique modulo $2\pi$, tel que 
$$
\Q\{\eqalign{
\cos(\theta)={a\F\sqrt{a^2+b^2}},
\cr
\sin(\theta)={b\F\sqrt{a^2+b^2}}.}
\W.\eqdef{eqtrig2}
$$
On en déduit alors que 
$$
\eqref{eqtrig}\ \Longleftrightarrow\ \cos(\theta)\cos(x)+\sin(\theta)\sin(x)={c\F r}
$$
puis que 
$$
\eqref{eqtrig}\ \Longleftrightarrow\ \cos(x-\theta)={c\F r}.
$$
Deux cas peuvent se produire : 
\medskip

\qquad$\bullet$ Si $|c|\le r$, il existe un nombre réel $\tau$ tel que 
$$
\cos(\tau)={c\F r}
$$
et les solutions réelles de l'équation \eqref{eqtrig} sont les nombres $x$ vérifiant 
$$
x\equiv\theta+\tau\quad[2\pi] \quad\mbox{ou}\quad x\equiv \theta-\tau\quad[2\pi].
$$
\qquad\quad$\bullet$ Si $|c|>r$, l'équation \eqref{eqtrig} n'admet aucune solution réelle. 
\bigskip


\Concept [] Linéarisation

\noindent
Linéariser une expression trigonométrique, c'est l'écrire comme une somme de constantes, de cosinus et de sinus. Ainsi, 
une linéarisation des expressions $\cos^2(x)$ et $\sin^2(x)$ est 
$$
\forall x\in\ob R, \qquad {\cos^2(x)={1+\cos(2x)\F2}}\quad\mbox{et}\quad
{\sin^2(x)={1-\sin(2x)\F2}}.
$$
Pour linéariser une fonction trigonométrique, on utilise les formules \eqref{lin} ainsi que les~relations d'Euler 
conjointement au binôme de Newton. Ainsi, pour linéariser les expressions $\cos^n(x)$ et $\sin^n(x)$ pour des entiers $n\ge3$, 
on écrit 
$$
{
\eqalign{
\cos^n(x)&=\Q({\e^{ix}+\e^{-ix}\F 2}\W)^n={1\F 2^n}\sum_{0\le k\le n}\Q({n\atop k}\W)\e^{i(n-2k)x}, 
\cr
\sin^n(x)&=\Q({\e^{ix}-\e^{-ix}\F 2i}\W)^n={1\F(2i)^n}\sum_{0\le k\le n}\Q({n\atop k}\W)(-1)^k\e^{i(n-2k)x},
}}
$$
puis on transforme les exponentielles complexes en cosinus et en sinus via les relations d'Euler en remarquant que 
$$
\forall x\in\ob R, \qquad \e^{ix}+\e^{-ix}=2\cos(x)\quad\mbox{et}\quad\e^{ix}-\e^{-ix}=2i\sin(x). 
$$

Exercice. Pour $x\in\ob R$, montrer que $\ds \cos^4(x)={\cos(4x)\F8}+{\cos(2x)\F2}+{3\F8}$. 
\bigskip

\Concept [] Développement d'une expression trigonométrique 

\noindent
Développer une expression trigonométrique, c'est écrire une expression trigonométrique 
en fonction de $\cos(x)$ et de $\sin(x)$. En quelque sorte, c'est l'inverse de la linéarisation. 
somme de constantes, de cosinus et de sinus. Ainsi, les formules suivantes forment 
un~``développement'' des quantités $\cos(2x)$ et $\sin(2x)$. 
$$
\forall x\in\ob R, \qquad\cos(2x)=\cos^2(x)-\sin^2(x)=2\cos^2(x)-1\quad\mbox{et}\quad\sin(2x)=2\sin(x)\cos(x). 
$$
Pour développer une fonction trigonométrique, on utilise les formules d'addition ainsi que la~formule de Moivre 
conjointement au binôme de Newton. Ainsi, pour développer les expressions $\cos(nx)$ et $\sin(nx)$ pour des entiers $n\ge3$, 
on écrit 
$$
{
\cos(nx)+i\sin(nx)=(\cos x+i\sin x)^n=\sum_{0\le k\le n}\Q({n\atop k}\W)\cos^k(x)i^{n-k}\sin^{n-k}(x), 
}
$$
en simplifiant éventuellement les termes de puissance paire via la relation \eqref{coscpsinc}. 

Exercice. Pour chaque nombre réel $x$, démontrer que 
$$
\cos(3x)=4\cos^3(x)-3\cos(x)\qquad\mbox{ et }\qquad \sin(3x)=3\sin(x)-4\sin^3(x).
$$ 
Si $\tan(x)$ et $\tan(3x)$ sont définies, prouver de plus que
$$
\tan(3x)={3\tan(x)-\tan^3(x)\F1-3\tan^2(x)}.
$$

\Section Eq2emedg, Equations polynômiales du second degré. 

\Subsection Raccar, Racines carrées. 

\Definition [] Un nombre complexe {$z$ est une racine carrée d'un nombre complexe $a$} si, et~seulement si, 
$$
{z^2=a}.
$$
\medskip

\Propriete [] Le nombre complexe $0$ l'unique racine carrée de $0$. \par
\noindent Les racines carrées d'un nombre complexe {$a$ non nul, de module $r$ et d'argument $\theta$}, sont 
$$
z_1:={-\sqrt r\ \e^{i\theta/2}}\qquad\mbox{et}\qquad z_2:={\sqrt r\ \e^{i\theta/2}}. \eqdef{raccar}
$$
En particulier, chaque nombre complexe non nul admet exactement deux racines carrées. 
\bigskip
\Remarque : l'identité \eqref{raccar} donne les formes trigonométriques des racines carrées de $a\neq0$. 
On peut également chercher les racines carrées $z=x+iy$ du nombre complexe $a=u+iv$ non nul
en résolvant le système
$$
z^2=a\ \Longleftrightarrow\ \Q\{\eqalign{
x^2-y^2&=u
\cr
2xy&=v
}\W.
$$
Avant de se lancer dans la résolution du sytème, il est bon de savoir qu'il admet toujours deux solutions 
mais que ses solutions n'admettent pas forcément une expression simple. 

En effet, le système précédent étant équivalent au système 
$$
 \Q\{\eqalign{
x^2-y^2&=u,
\cr
2xy&=v,
\cr
x^2+y^2&=\sqrt{u^2+v^2},
}\W.
$$
ses deux solutions sont 
$$
z=\pm{\sqrt{u+\sqrt{u^2+v^2}}+i\sgn(v)\sqrt{\sqrt{u^2+v^2}-u}\F\sqrt2}.
$$


\Subsection Eqsecdeg, Equations polynomiales du second degré. 

\Definition [] Une équation polynomiale du second degré est une équation du type 
$$
az^2+bz+c=0,\eqdef{eqsd}
$$
pour des nombres $a\in\ob C^*$ et $(b,c)\in\ob C^2$. Le {discriminant} d'une telle équation est 
$$
{\Delta:=b^2-4ac}. 
$$
\bigskip

\Theoreme [$a\in\ob C^*$ et ${(a,b)}\in\ob C^2$]
\Bullet Si {$\Delta=0$}, alors l'équation \eqref{eqsd} admet une {unique racine} $\ds z={-b\F2a}$, qui est {double}.
\Bullet Si {$\Delta\neq0$}, alors l'équation \eqref{eqsd} admet {deux racines simples} 
$$
z_1:={-b+\omega\F2a}\qquad\mbox{et}\qquad z_2:={-b-\omega\F2a},
$$
où le symbole $\omega$ désigne une racine carrée du discriminant $\Delta$, c'est-à-dire 
 $$
\omega^2=\Delta.
$$

\Theoreme  [$a\in\ob R^*$ et ${(a,b)}\in\ob R^2$] {\bf\ {(cas réel)}.}
\Bullet Si {$\Delta<0$}, alors l'équation \eqref{eqsd} admet {deux racines simples non-réelles}  
$$
z_1:={-b+i\sqrt{-\Delta}\F2a}\qquad\mbox{et}\qquad z_2:={-b-i\sqrt{-\Delta}\F2a}.
$$
\Bullet Si {$\Delta=0$}, alors l'équation \eqref{eqsd} admet {une racine} $\ds z={-b\F2a}$, {double et réelle}.\smallskip
\Bullet Si {$\Delta>0$}, alors l'équation \eqref{eqsd} admet {deux racines réelles simples} 
$$
z_1:={-b+\sqrt\Delta\F2a}\qquad\mbox{et}\qquad z_2:={-b-\sqrt\Delta\F2a}.
$$

 En pratique on résoud les équations polynomiales du second degré en les factorisant.
\medskip
\centerline{Méthode conseillée.} 
\medskip
\noindent\qquad1) Factoriser le nombre $a$. 

2) Mettre sous la forme canonique en faisant apparaitre un carré
$$
\eqref{eqsd}\ \Longleftrightarrow\ a\Q(z^2+{b\F a}z+{c\F a}\W)=0\ \Longleftrightarrow\ a\Q(\underbrace{\Q(z+{b\F2a}\W)^2}_{\mbox{carré}}+\underbrace{{c\F a}-{b^2\F4 a^2}}_{-{\ss\Delta\F\ss 4a^2}}\W)=0.
$$
\vskip-2em
3) Déterminer une racine carrée de $\Delta$. 

4) Utiliser l'identité remarquable $A^2-B^2=(A-B)(A+B)$ pour factoriser 
$$
\eqref{eqsd}\ \Longleftrightarrow\ a(z-z_1)(z-z_2)=0.
$$
\qquad\quad 5) Conclure en utilisant que, dans $\ob C$, un produit de facteur est nul si, 

\qquad\qquad et seulement si, l'un au moins des facteurs est nul.



\Propriete [Title=Lien coefficients-racines] Soient $S$ et $P$ deux nombres complexes. Alors, 
$$
z_1\mbox{ et } z_2 \mbox{ sont solutions de l'équation }z^2-Sz+P=0\ \Longleftrightarrow\ \Q\{\eqalign{z_1+z_2&=S,\cr z_1\ z_2&=P.}\W.
$$ 

\Subsection Racunite, Racines n$^{\mbox{\sevenrm ièmes}}$ de l'unité. 

\Definition [] Soit $n\ge2$ un entier. {Une racine $n^{\mbox{\sevenrm ième}}$ 
d'un nombre complexe $a$} est un nombre complexe $z$ vérifiant 
$$
{z^n=a}.
$$
On appelle {racines $n^{\mbox{\sevenrm ièmes}}$ de l'unité} les racines $n^{\mbox{\sevenrm ièmes}}$ du nombre $1$. 
\bigskip

\Propriete [] Pour $n\ge2$, les racines $n^{\mbox{\sevenrm ième}}$ d'un nombre complexe $a$ non nul, 
de module $r$ et d'argument $\theta$, sont les $n$ nombres distincts 
$$
\root n\of{r}\ \e^{i\theta+{2\pi ik\F n}}\quad\mbox{ pour }\quad k\in\{0,\cdots,n-1\}.
$$
En particulier, {les racines $n^{\mbox{\sevenrm ième}}$ de l'unité sont les $n$ nombres}
$$
{\e^{{2\pi ik\F n}}\quad\mbox{ pour }\quad k\in\{0,\cdots,n-1\}}.
$$


\Definition [] On note $j$ et $\ol j$ les nombres complexes définis par 
$$
{j=\e^{2\pi i\F3}={-1+i\sqrt3\F2}}\qquad\mbox{et}\qquad{\ol j:=j^2=\e^{-2\pi i\F3}={-1-i\sqrt3\F2}}.
$$
Les nombres {$1$, $j$ et $\ol j$ sont les racines $3^{\mbox{\sevenrm ième}}$ de l'unité} et sont {solutions de} l'équation
$$
{z^2+z+1=0}.
$$
\bigskip

\Propriete [] {La somme des racines $n^{\mbox{\sevenrm ième}}$ de l'unité est égale à $0$}. 

\medskip