\startproduct Dis_fiches
\project project_Dis
\definepapersize[emc][width=29.7cm,height=21cm]
\setuppapersize[emc][emc]
\setuplayout[backspace=0.5cm,topspace=0.5cm,width=26.7cm,header=0.5cm,footer=0.5cm, headerdistance=0.5cm,footerdistance=0.5cm]

\defineitemgroup[List][levels=4]
\setupitemgroup[List][1][n,joinedup,nowhite]
\setupitemgroup[List][2][n,joineup,nowhite]
\setupitemgroup[List][3][3,joineup,nowhite]
\setupitemgroup[List][4][4,joineup,nowhite]
%\starttext
%\completecontent%[criterium=all]
%\stoptext

%\xmlprocessfile{maths}{xml/Limbo_Méthodes_de_calcul.xml}{}
%\xmlprocessfile{maths}{xml/Limbo_Dérivées_et_primitives.xml}{}
%\xmlprocessfile{maths}{xml/Limbo_Suites.xml}{}
%\xmlprocessfile{maths}{xml/Limbo_Systèmes.xml}{}
%\xmlprocessfile{maths}{xml/Limbo_Matrices.xml}{}
%\xmlprocessfile{maths}{xml/Limbo_Logique.xml}{}
%\xmlprocessfile{maths}{xml/Limbo_Nombres_complexes.xml}{}
%\xmlprocessfile{maths}{xml/Limbo_Intégration.xml}{}
%\xmlprocessfile{maths}{xml/Limbo_Ensembles_et_applications.xml}{}
%\xmlprocessfile{maths}{xml/Limbo_Limites.xml}{}
%\xmlprocessfile{maths}{xml/Limbo_Polynômes.xml}{}
%\xmlprocessfile{maths}{xml/Limbo_Probabilités.xml}{}
%\xmlprocessfile{maths}{xml/Limbo_Variables_aléatoires.xml}{}

\def\fiche#1{{\bf #1.} }
\def\rare#1{{\darkgray #1}}

%\starttext
%\chapter{Fiches}
%\completeFiches
%\stopcolumns
%\stoptext

%\starttext
%\completeindex
%\stoptext
\starttext
\begingroup\eightpoint\startcolumns[n=2,rule=on]
\iffalse\chapter{Nombres complexes}
\section{Forme algébrique (additive)}



\startList
\item \fiche{Nombre complexe} Un nombre complexe est un nombre du type $z=x+iy$ avec $x,y$ réels et $i^2=-1$.
\item \fiche{Opérations} \startformula\startmatrix
      \NC (x + iy) + (x' + iy')\NC=\NC(x + x') + i(y + y')\NC\Inframed{$+$}\NR
      \NC (x + iy) × (x' + iy')\NC=\NC(xx' - yy') + i(xy' + yx')\NC\Inframed{$×$}\NR
    \stopmatrix\stopformula
\stopList

\subsection{Parties réelles et imaginaires}

\item \fiche{Définition} Pour $x,y$ nombres réels, les parties réeeles et imaginaires du nombre complexe $z=x+iy$ sont les nombres réels $ ℜe(z):=x$ et $ℑm(z):=y$
\item\fiche{Linéarité} Pour $s, z$ nombres complexes et $λ, μ$ nombres réels,  
\startformula
\startmatrix
\NC ℜe(λs+μz)=λℜe(s)+μℜe(z)\NR
\NC ℑm(λs+μz)=λℑm(s)+μℑm(z)\NR
\stopmatrix
\stopformula
\item \fiche{Caractérisation} Pour $s, z$ nombres complexes, \startformula
s = z ⟺  \startcases
\NC ℜe(s)=ℜe(z)\NR
\NC ℑm(s)=ℑm(z)\NR
 \stopcases
\stopformula

\item \fiche{réel} Un nombre réel est un nombre complexe de partie imaginaire nulle
$$
z∈ℝ⟺ℜe(z)=z⟺ℑm(z)=0⟺ \overline z = z
$$

\item \fiche{imaginaire pur} Un nombre imaginaire pur est un nombre complexe de partie réelle nulle
$$
z∈iℝ⟺ℜe(z)=0⟺iℑm(z)=z⟺ \overline z = -z
$$

\subsection{Conjugué}

\item \fiche{Définition} Pour $x,y$ nombres réels, le conjugué de $z=x+iy$ est le nombre complexe $\overline z = x-iy$


\item\fiche{lien avec les parties réeles et imaginaires}
\startformula  \startcases[style=\displaystyle]
\NC z = ℜe(z) + i ℑm(z)\NR 
\NC\overline z = ℜe(z) - i ℑm(z)\NR
\stopcases
\qquad
\startcases
  \NC ℜe(z)={z+\overline z\F 2}\NR
  \NC ℑm(z)={z-\overline z\F2i}\NR
\stopcases\stopformula

\item\fiche{Linéarité} Pour $s, z$ nombres complexes et $λ, μ$ nombres réels,  
\startformula
\overline{λs+μz}=λ\overline{s}+μ\overline{z}
\stopformula

\item\fiche{Produit, quotient, puissances}  
\Align{
\NC \overline{sz}\NC =\overline{s}\overline{z}\NR
\NC \overline{s\F z}\NC={\overline{s}\F\overline{z}}\qquad(z≠0)\NR
\NC {\overline s^n}\NC= {\overline s}^n\qquad(n∈ℤ, s≠0 \text{ si }n<0)
}
\item\fiche{involution} Pour $z$ complexe $\overline{\overline z} = z$. 


\section{Forme trigonométrique (multiplicative)}

\section{module}

\startList
\item \fiche{Module} Le module d'un nombre complexe $z=x+iy$ est le nombre réel positif ou nul $|z|=\sqrt{x^2+y^2}$.
\item \fiche{}
\stopList
\fi

\chapter{Intégration sur un segment (version 7)}

\section{Définition}


\startList

\item \fiche{Primitive} $F$ est une primitive sur l'intervalle $I$ d'une application $f$ définie sur $I$ si, et seulement si, $F$ est dérivable sur $I$ et vérifie $\underbrace{\forall x∈I, F'(x)=f(x)}_{F'=f}$
\item\fiche{Caractérisation des primitives} Si $F$ est une primitive de $f$ sur un intervalle I, alors 
\startformula 
G:I → 𝕂\mbox{ est une primitive de f sur }I\  ⟺\   ∃ c ∈ 𝕂 :  ∀ x ∈ I,  G(x)=F(x)+c
\stopformula

\item \fiche{Primitives des fonctions continues} Toute fonction continue $f$ sur un intervalle $I$ admet une primitive sur $I$.
\item \fiche{Intégrale des fonctions continues} Si $f$ est continue sur $[a,b]$, l'intégrale de $f$ de $a$ à $b$ est le nombre 
$$
\int_a^bf(x)\d x=\Q[F(x)\W]_a^b=F(b)-F(a),
$$
où $F$ désigne n'importe quelle primitive $F$ de $f$ sur $[a,b]$.
\iffalse \crlf En particulier, pour toute fonction $f$ dérivable, de dérivée continue (de classe $\mc C^1$) sur $I$, on a 
\startformula 
\int_a^xf'(t)\d t=[f]_a^x=f(x)-f(a). 
\stopformula
\fi

\item \fiche{Théorème fondamental de l'analyse} Si $f$ est continue sur un intervalle $I$ contenant $a$, alors l'unique primitive de $f$ s'annulant en $a$ est l'application $F$ définie par 
\startformula 
\Align{
\NC F: I\NC  → ℝ\NR
\NC  x\NC  ↦  \int_a^xf(t)\d t}
\stopformula
Par ailleurs, $F$ est dérivable, de dérivée continue sur $I$ (de classe $\mc C^1$)

\item\fiche{Subdivision} Une subdivision de $[a,b]$ est une famille $(x_0,…x_n)$ de nombres réels vérifiant $a=x_0\Le x_1\Le …\Le x_n=b$.

\item \fiche{Intégrale des fonctions continues par morceaux (P.M.)} Si $a=x_0\Le x_1\Le…\Le x_n=b$ est une subdivision adaptée à la fonction continue par morceaux $f:[a,b]→ℝ$, alors l'intégrale de $f$ de $a$ à $b$ est le nombre
$$
\int_a^bf(x)\d x = \sum_{k=1}^n\int_{x_{k-1}}^{x_k} f_k(x)\d x,
$$
Pour $k∈⟦1,n⟧$, on rappelle que la restriction de $f$ à l'intervalle $]x_{k-1},x_k[$ est prolongeable par continuité en une fonction continue $f_k:[x_{k-1},x_k]→𝕂$

\item\fiche{Convention} On pose  $\int_a^af=0$ et $ \int_b^af:=-\int_a^bf$ lorsque l'intégrale de droite est définie.


%\stopList
\section{Propriétés}
%\startList


%\item \fiche{Parties réelles et imaginaires} $f:[a,b]→ℂ$ est continue (par morceaux) sur $[a,b]$ si, et seulement si, sa partie réelle et sa partie imaginaires le sont aussi et dans ce cas, $\int_a^bf=\int_a^bℜe(f) +i\int_a^bℑm(g)$.
\item \fiche{Relation de Chasles} Si $f$ est continue (par morceaux) sur un segment $S$ contenant $a$, $b$ et $c$ alors $\int_a^cf=\int_a^bf+\int_b^cf$ (les trois intégrales sont définies)
\item \fiche{Linéarité} Si $f$ et $g$ sont continues (p. m.) sur $[a,b]$ et si $ λ, μ ∈ ℝ$, alors $\int_a^b(λ f+ μ g)= λ\int_a^bf+ μ\int_a^bg$
\item \fiche{Positivité} Si $f$ est continue (par morceaux) sur $[a,b]$, à valeurs positives ou nulles, alors $\int_a^bf ⩾0$
\item \fiche{Croissance} Si $f$ et $g$ sont continues (p. m.) sur $[a,b]$ et si $\underbrace{∀ x ∈[a,b], f(x) ⩽ g(x)}_{f ⩽ g}$, alors $\int_a^bf ⩽ \int_a^bg$.
\item \fiche{Cas des fonctions continues, positives, d'intégrale nulle} Si $f$ est continue sur $[a,b]$, à valeurs positives ou nulles, alors 
$$
\int_a^bf(x)\d x =0 \quad ⟺ \quad\underbrace{\forall x∈[a,b], f(x)=0}_{f=0}
$$
\item \fiche{Valeur absolue} Si $f:[a,b]→ℝ$ est continue (par morceaux), alors $|f|$ l'est aussi et $\Q|\int_a^bf\W| ⩽ \int_a^b|f|$

\item \fiche{Sommes de Riemann} Si $f$ est continue sur $[0,1]$ alors $\displaystyle\lim_{n→∞}{1\F n}\sum_{k=1}^nf\Q({k\F n}\W) = \int_0^1f(x)\d x$. 

\section{Outils fondamentaux}

\item \fiche{Intégration par partie} Si $f$ et $g$ deux fonctions dérivables, de dérivées continues (de classe $\mc C^1$) sur $[a,b]$, alors 
\startformula 
\int_a^bf(x)g'(x)\d x=\big[f(x)g(x)\big]_a^b-\int_a^bf'(x)g(x)\d x. 
\stopformula


\item \fiche{changement de variable (non bijectif)} Si $f$ est continu sur un intervalle $I$ et si $φ:[a,b]→I$ est dérivable, de dérivée continue (de classe $\mc C^1$), alors 
\startformula 
\int_a^bf\big( φ(u)\big) φ'(u)\d u=\int_{ φ(a)}^{ φ(b)}f(x)\d x.
\stopformula

\item \fiche{Changement de variable} Si $f$ est continue (par morceaux) sur $[a,b]$ et si 
\startList \item $φ:[c,d] →[a,b]$ est une bijection
\item $φ$ est dérivable et de dérivée continue (de classe $\mc C^1$) sur $[c,d]$
\item $φ^{-1}$ est dérivable, de dérivée continue (de classe $\mc C^1$) sur $[a,b]$
\stopList
alors 
\startformula 
\int_a^bf(x)\d x=\int_{ φ^{-1}(a)}^{ φ^{-1}(b)}f\big( φ(u)\big) φ'(u)\d u 
\stopformula

\section{Dérivées et primitives}

\item Pour $x$ dans un intervalle sur lequel les fonctions sont dérivables pour les dérivées (et continues pour les primitives),   \crlf
\Align{
\NC (x^α)'\NC=\NC α x^{α-1}\qquad(α∈ℝ)\NC Monômes  \NC \int x^α\d x \NC=\NC \System{
  \NC {x^{α+1}\F α+1} +c \qquad(α≠-1)\NR
  \NC \ln|x| + c \qquad (α=-1)
}\NR
 \NC \big(\ln|x|\big)'\NC=\NC{1\F x}\qquad (x≠0)\NC Logarithme\NC \int{\d x \F x} \NC=\NC \ln|x] +c\NR
 \NC (\e^x)'\NC=\NC\e^x\NC Exponentielle\NC \int\e^x\d x \NC=\NC \e^x + c\NR
 \NC \cos'(x) \NC=\NC -\sin(x)\NC Cosinus\NC \int\sin(x)\d x \NC=\NC -\cos(x) + c\NR
\NC \sin'(x) \NC=\NC \cos(x)\NC Sinus  \NC \int\cos(x)\d x\NC=\NC\sin(x)+c\NR
 \NC \tan'(x)\NC=\NC{1\F \cos^2(x)}= 1+\tan^2(x)\NC Tangente\NC \int{\d x\F \cos^2(x)} \NC=\NC\int\big(1+\tan^2(x)\big)\d x = \tan(x) + c\NR 
 \NC \arctan'(x)\NC=\NC{1\F1+x^2}\NC Arctangente\NC \int{\d x\F1+x^2}\NC=\NC\arctan(x)+c
}


\chapter{Variables aléatoires réelles (discrètes, sur un univers fini)}

Dans tout ce chapitre, $(Ω, \sc P(Ω), P)$ désigne un espace probabilisé fini ($Ω=\{ω_1, …,ω_n\}$). \crlf
Lorsque $p∈[0,1]$, on pose $q=1-p$.

\section{Variable aléatoire réelle}

\item \fiche{VAR} Une variable aléatoire réelle est une application $X:Ω→ℝ$.
\item \fiche{VAR certaine} $X$ est une VAR certaine ssi il existe $c∈ℝ$ tel que $∀ω∈Ω, X(ω)=c$.
\item \fiche{VAR quasi-certaine} $X$ est une VAR quasi-certaine ssi il existe $c∈ℝ$ tel que $P(X=c)=1$.
\item \fiche{Univers image} L'univers image d'une VAR $X$ est $X(Ω)=\{X(ω):ω∈Ω\}$.
\item \fiche{Système complet} Le système complet associé à une VAR $X$ est $\{(X=x)\}_{x∈X(Ω)}$.
\item \fiche{loi d'une VAR} La loi d'une VAR $X$ est la probabilité $P_X$ définie sur $X(Ω)$ par 
$$
P_X(A) = P(X∈A)\qquad(A \text{ événement de } X(Ω))
$$
{\it Remarque : elle est complètement déterminée par la donnée de $P(X=x)$ pour $x∈X(Ω)$}
\item \fiche{fonction de répartition d'une VAR} La fonction de répartition d'une VAR $X$ est l'application $F_X:ℝ→[0,1]$ définie par 
$$
F_X(x)=P(X⩽x)\qquad(x∈ℝ)
$$

\item \fiche{Propriétés de $F_X$} La fonction de répartition $F_X$ d'une var $X$  est croissante sur $ℝ$, continue à droite et vérifie 
$$\lim_{x→-∞}F_X(x)=0  \text{ et } \lim_{x→∞}F_X(x)=1$$
{\it Pour une VAR discrète sur un ensemble fini, elle est aussi en escalier (constante par morceaux)}

\section {Espérance}

\item \fiche{Espérance} L'espérance d'une VAR discrète finie $X$ est le nombre réel 
$$
E(X) = \sum_{x∈ X(Ω)}xP(X=x)
$$
\item \fiche{VAR centrée} Une VAR $X$ est centrée ssi son espérance est nulle

\item \fiche{Théorème de transfert} Pour une VAR $X$ et pour $g:X(Ω)→ℝ$, on a 
$$
E(g(X)) = \sum_{x∈ X(Ω)}g(x)P(X=x)
$$

\item \fiche{Transformation affine} Pour une VAR  $X$, on a 
$$
E(aX+b)=aE(X)+b\qquad(a,b∈ℝ)
$$
\item \fiche{Positivité} Si $X$ est une VAR vérifiant $\underbrace{∀ ω∈Ω, X(ω) ⩾ 0}_{X ⩾ 0}$ (p.s.), alors $E(X)⩾0$.
\item \fiche{Croissance} Si $X$ et $Y$ vérifient $\underbrace{∀ ω∈Ω, X(ω) ⩽ Y(ω)}_{X ⩽ Y}$ (p.s.), alors $E(X)⩽E(Y)$.
\item \fiche{Cas des VAR positives, d'espérance nulle} Si $X$ vérifie $\underbrace{∀ ω∈Ω, X(ω) ⩾ 0}_{X ⩾ 0}$ (p.s.), alors 
$$
E(X) = 0 \quad ⟺ \quad P(X = 0) = 1 \quad ⟺ \quad X = 0\text{ p.s.}
$$

\section{Variance et écart type}

\item \fiche{Variance} La variance d'une VAR  $X$ est le nombre réel positif ou nul 
$$
V(X) = E\Big(\big(X-E(X)\big)^2\Big)
$$
\item \fiche{Formule de Koenig-Huygens} Pour une VAR  $X$, on a $V(X) = E(X^2)-E(X)^2$
\item \fiche{Variance nulle} Pour une VAR $X$, on a 
$$
V(X) = 0 ⟺ X = E(X) \text{ p.s.}
$$
\item \fiche{Transformation affine} Pour une VAR  $X$, on a 
$$
V(aX+b)=a^2V(X)\qquad(a,b∈ℝ)
$$
\item \fiche{Ecart type} L'ecart type d'une VAR  $X$ est le nombre réel positif ou nul $σ_(X) = \sqrt{V(X)}$
\item \fiche{VAR centrée réduite} Une VAR $X$ est centrée et réduite ssi son espérance est nulle et sa variance vaut $1$ (son écart type vaut $1$)

\section{Lois usuelles}
\item \fiche{loi certaine} $X$ suit la loi certaine ssi $X=c$ p.s. Dans ce cas, on a $E(X)= c $ et $V(X) = 0$. 
\item \fiche{loi de Bernouilli de paramètre $p∈[0,1]$}  
$$X\hookrightarrow B(p)\quad⟺\quad P(X=1)=p \text{ et  } P(X=0)=q
$$
Dans ce cas, on a $E(X)= p$ et $V(X) = pq$.
\item \fiche{loi binomiale de paramètre $n∈ℕ^*$ et $p∈[0,1]$} 
$$
X\hookrightarrow B(n,p)\quad⟺\quad
P(X=k)={n\choose k}p^kq^{n-k}\qquad(0⩽k⩽n)
$$
Dans ce cas, on a $E(X)=np$ et $V(X)=npq$.
\item \fiche{loi uniforme} $X$ suit la loi uniforme sur $⟦1,n⟧$ ssi $X\hookrightarrow \sc U(⟦1,n⟧)$ ssi 
$$
P(X=k)={1\F n}\qquad(1⩽k⩽n)
$$
Dans ce cas, on a $E(X)= {n+1\F2}$ et $V(X) = {n^2-1\F 12}$. 


\chapter{Polynômes}

\section{Forme additive}

\item\fiche{Polynôme} Un polynôme à coefficients dans $𝕂$ est une expression symbolique du type 
\startformula 
P = ∑_{k=0}^na_kX^k\NC\qquad (n⩾0\text{ et } (a_0,⋯,a_n)∈𝕂^{n+1})
\stopformula
L'ensemble de tous les polynômes à coefficients dans $𝕂$ est noté $𝕂[X]$.\blank[medium]

\item \fiche{Opérations algébriques} La somme, les multiples (plus généralement les combinaisons linéaires), les produits, les dérivées, les primitives et les composées (obtenus par substitution)   
de polynômes à coefficients dans $𝕂$ sont des polynômes à coefficients dans $𝕂$. \crlf Pour $n⩾0$, $P=∑_{k=0}^na_kX^k$ et $Q=∑_{k=0}^nb_kX^k$, on a  
\startList\item (somme) $\displaystyle P + Q = ∑_{k=0}^n(a_k+b_k)X^k$
\item (multiplication par un scalaire) $\displaystyle λ P = ∑_{k=0}^nλ a_kX^k$ pour $λ∈𝕂$
\item (produit) $\displaystyle P × Q = ∑_{k=0}^{2n}c_kX^k$ avec $\displaystyle c_k = ∑_{i=0}^ka_ib_{k-i} \qquad (0⩽ k⩽ 2n)$
\item (dérivation) $\displaystyle P' = \sum_{k=1}^n k a_kX^{k-1} = ∑_{k=0}^{n-1}(k+1)a_{k+1}X^k$
\item (primitives) Les primitives de $P$ sont les polynomes 
\startformula
c + \sum_{k=0}^n{a_k\F k+1}X^{k+1} = c + ∑_{k=1}^{n+1}{a_{k-1}\F k}X^k\qquad (c∈𝕂)
\stopformula
\item (substitution) le polynome $\displaystyle P(Q)=∑_{k=0}^na_kQ^k$ est obtenu en substituant $Q$ à l'indeterminée $X$.
\stopList
En particulier $P=P(X)$.\blank[medium]

\item \fiche{degré} Le degré d'un polynôme non nul $P= ∑_{k=0}^na_kX^k$ est le nombre entier naturel
\startformula 
\deg(P)=\max\{k⩾0:a_k≠0\}.
\stopformula
Par convention $\deg(0)=-∞$.\blank[medium]

\item L'ensemble de tous les polynômes à coefficients dans $𝕂$, de degré au plus  $n$, est noté $𝕂_n[X]$.
\startformula 
𝕂_n[X]=\Q\{P∈𝕂[X]:\deg(P)⩽n\W\}
\stopformula

\item \fiche{Opérations}\startList\item (multiplication par un scalaire) $\deg(λP) = \deg(P)$  pour $λ∈𝕂^*$.
\item (produit) $\deg(PQ) = \deg(P)+\deg(Q)$  pour $P$ et $Q$ polynômes non nuls.
\item (somme) $\deg(P+Q) ⩽\max(\deg(P), \deg(Q))$ avec égalité si $\deg(P)≠\deg(Q)$. 
\item (dérivée) $\deg(P') = \System{\NC \deg(P) -1 \NC \text{ si } \deg(P)⩾1\NR \NC -∞ \NC\text{ sinon } }$ 
\stopList

\section{Forme multiplicative}

\item \fiche{Division euclidienne} Pour $P∈𝕂[X]$ et $D∈𝕂[X]^*$, il existe un unique $(Q,R)∈𝕂[X]$ tel que 
\startformula
P = QD+R \text{ et } \deg(R)<\deg(D)
\stopformula

\item \fiche{Diviseur et multiple} Soient $P$ et $Q$ dans $𝕂[X]$. 
\startformula
\SystemR{
\NC D \text{ est un diviseur de } P \NR
\NC P \text{ est un multiple de } D \NR} ⟺ D | P ⟺ ∃ Q∈𝕂[X]\text{ tel que } P = QD
\stopformula

\item \fiche{Racine} $a∈𝕂$ est une racine de $P∈𝕂[X]$ $⟺ P(a)=0 ⟺ (X-a) | P$ 

\item \fiche{Multiplicité} Soit $a∈𝕂$ et $P∈𝕂[K]$.
\startformula
\text{$a$ est racine de $P$ de multiplicité $m$} ⟺ \System{
\NC P(a)=0\NR
\NC ⋮\NR
\NC P^{(m-1)}(a)=0\NR
\NC P^{(m)}(a)≠0\NR}⟺ \System{\NC (X-a)^m | P\NR
\NC  (X-a)^{m+1} \not| P\NR} 
\stopformula

\item \fiche{Théorème de D'Alembert-Gauss} Tout polynôme non constant $P∈ℂ[X]$ admet une racine~dans~$ℂ$.

\item \fiche{Décomposition sur $ℂ$} Pour chaque polynôme non constant $P∈ℂ[X]$, il existe une constante $α∈ℂ^*$ et des nombres complexes $(z_1,⋯,z_n)$ uniques à permutation près tels que 
\startformula
P =α∏_{k=1}^{\deg(P)}(X-z_k)
\stopformula

\item \fiche{Décomposition sur $ℂ$ avec multiplicité} Pour chaque polynôme non constant $P∈ℂ[X]$, il existe un entier $K⩾1$, une constante $α∈ℂ^*$ et des nombres complexes $(z_1,⋯,z_K)$ associés à des multiplicités $(n_1,⋯ ,n_K)$ uniques à permutation près tels que 
\startformula
P =α∏_{k=1}^K(X-z_k)^{n_k}
\stopformula
De plus, on a $\deg(P)=n_1+⋯+n_K$

\item \fiche{Décomposition sur $ℝ$ avec multiplicité} Pour chaque polynôme non constant $P∈ℝ[X]$, il existe deux entiers $K⩾0$ et $L⩾0$, une constante $α∈ℂ^*$, des nombres réels $(r_1,⋯,r_K)$ et des trinômes du second degré sans racines réelles $(Q_1, ⋯, Q_L)$ (de discriminant strictement négatif) associés à des multiplicités $(n_1,⋯ ,n_K)$ et $(m_1, ⋯, m_L)$ uniques à permutation près tels que 
$$
P =α∏_{k=1}^K(X-r_k)^{n_k}∏_{ℓ=1}^LQ_ℓ^{m_ℓ}
$$
De plus, on a $\deg(P)=n_1+⋯+n_K + 2m_1 + ⋯+ 2m_L$

\chapter{Espaces vectoriels}

\item \fiche{Espace vectoriel}
\startformula 
(E,+,⋅)\text{ $𝕂$- espace vectoriel} ⟺  \System{
\NC + \text{ loi interne, associative, commutative,}\NR
\NC \text{ admettant un élément neutre $0∈E$, }\NR
\NC \text{  tous les $x∈E$ sont  inversibles pour }+\NR
\NC ⋅ \text{ loi externe, associative, distributive sur $+$}\NR
\NC ∀x∈E, 1⋅x=x
}
\stopformula


\item \fiche{Combinaisons linéaires}Soit $E$ un espace vectoriel sur $𝕂$
\startformula 
\Align{\NC x \text{ combi. linéaire de } x_1,⋯,x_n∈E\NR
\NC \text{pour les coeffs } λ_1,⋯,λ_n∈𝕂}⟺ x= ∑_{k=1}^n λ_k.x_k
\stopformula

\item \fiche{Produit nul} Dans un $𝕂$-espace vectoriel $(E,+,⋅)$, on a  
\startformula 
 λ.x=0 ⟺  λ=0 \text{ ou } x=0
\stopformula

\item \fiche{sous-espace vectoriel} $F$ est un $𝕂$-sous-espace vectoriel de $E$ si, et seulement si, 
\startList
\item $F$ est un ensemble non vide {\it (en général, on montre que $0_E∈F$)}.
\item $F$ est inclus dans $E$, qui est un $𝕂$-espace vectoriel.
\item $F$ est stable par combinaison linéaire, c'est-à-dire
\startformula 
 ∀ ( λ, μ) ∈ 𝕂^2 , \qquad   ∀ (x,y) ∈ F^2 ,\qquad  λ. x+ μ. y ∈ F, 
\stopformula
\stopList


\item \fiche{intersection} Une intersection de sous-espaces vectoriels de $E$ est un sous-espace vectoriel de $E$


\item \fiche{Espace vectoriel engendré} L'espace vectoriel engendré par une partie~$A$ d'un espace vectoriel~$E$ est le plus petit sous-espace vectoriel de $E$ (pour l'inclusion) contenant $A$. 
\startformula 
\Vect(A) = ∩\limits_{A⊂F \text{ sev de }E}F
\stopformula

\item \fiche{caractérisation} L'espace vectoriel engendré par une partie~$A$ d'un espace vectoriel~$E$ est l'ensemble des combinaisons linéaires qu'il est possible de former avec les éléments de~$A$
\startformula 
\Vect(A)=\Q\{ ∑_{k=1}^n λ_kx_k: n⩾1,(x_1,⋯,x_n)∈A^n,(λ_1,⋯, λ_n) ∈ 𝕂^n\W\}.
\stopformula
Lorsque $A = \{x_1, \cdots, x_n\}$, on a plus simplement
\startformula 
\Vect(A)=\Q\{ ∑_{k=1}^n λ_kx_k: (λ_1,⋯, λ_n) ∈ 𝕂^n\W\}.
\stopformula 


\item \fiche{Famille génératrice} Soit  $\mc F = (e_1,\cdots,e_n)$ une famille finie de vecteurs de $E$. alors
\startformula 
\mc F \text{ engendre } E ⟺ E=\Vect(\mc F) ⟺  ∀x∈E, ∃(λ_1,⋯,λ_n)∈𝕂^n: x= λ_1e_1+⋯+λ_ne_n.
\stopformula

\item \fiche{Famille libre} Soit  $\mc F = (e_1,\cdots,e_n)$ une famille finie de vecteurs de $E$. alors

\startformula 
\mc F \text{ est libre si, et seulement si, }   ∑_{k=1}^nλ_kx_k=0 ⟺ λ_1=⋯ =λ_n=0
\stopformula
Une famille est libre \ssi il existe une seule combinaison linéaire nulle de ses vecteurs (celle
dont tous les coefficients sont nuls). 

\item \fiche{Famille liée} Une famille finie $\mc F = (e_1,\cdots,e_n)$ est liée si, et seulement si, elle n'est pas libre
\startformula 
\mc F \text{ est liée si, et seulement si, }   ∃(λ_1,⋯,λ_n)≠(0,⋯,0): ∑_{k=1}^nλ_kx_k=0
\stopformula
Une telle relation est appelée une relation de dépendance linéaire.

\item \fiche{Base} Une base d'un espace vectoriel $E$ est une famille finie de vecteurs de $E$ qui est libre et génératrice.

\item \fiche{Coordonnées} Si $\mc B=(e_1, ⋯,e_n)$ est une base d'un $𝕂$-espace vectoriel $E$, alors chaque vecteur $x$ de $E$ se décompose de manière unique sur $\mc B$
\startformula 
(e_1,⋯,e_n)\text{ base de }E ⟺  ∀ x ∈ E,   ∃!( x_1,⋯, x_n)∈𝕂^n: x= ∑_{k=1}^n x_ke_k
\stopformula
Les nombres $(x_1,⋯, x_n)$ sont appelés coordonnées (ou composantes) du vecteur $x$ dans la base $(e_1,⋯,e_n)$. La matrice des coordonnées de $x$ dans la base $\mc B$ est la matrice colonne
$$
\mc Mat_{\mc B}(x)=\Matrix{\NC x_1\NR\NC⋮\NR\NC x_n\NR}
$$

\item \fiche{Espaces vectoriels de référence}
\startList
\item $𝕂^n$ est un espace vectoriel, de base canonique $\Q((0, ⋯, 0, \underset {k}{1},0,⋯,0)\W)_{1⩽k⩽n}$
\item $𝕂_n[X]$ est un $𝕂$ espace vectoriel, de base canonique $\Q(X^k\W)_{0⩽k⩽n}$
\item $\{0\}$, $\mc M_{n,p}(𝕂)$, $𝕂^N$, $\mc F(𝕂,𝕂)$ sont des $𝕂$-espaces vectoriels
\item $ℂ$ est un $ℝ$-espace vectoriel de base canonique $(1, i)$
\stopList



\iffalse

La construction des polynômes formels n'est pas au programme, on pourra identier polynômes et fonctions polynomiales. Les démonstrations des résultats de ce paragraphe ne sont pas exigibles. Ensemble 𝕂[X]K[X] des polynômes à coefficients dans KK.
Opérations algébriques.

Ensembles 𝕂n[X]Kn[X] des polynômes à coefficients dans 𝕂K de degré au plus nn.
Division euclidienne. Multiples et diviseurs.
Racines, ordre de multiplicité d'une racine.Cas du trinôme.
Caractérisation de la multiplicité par factorisation d'une puissance de (X−a)(X−a).
Théorème de d'Alembert-Gauss.Résultat admis. Exemples simples de factorisation dans ℂ[X]C[X] et ℝ[X]R[X] de polynômes de ℝ[X]R[X]. Les méthodes devront être indiquées.
\fi

\chapter{Théorie de la dimension}

\item \fiche{Dimension finie} Un EV $E$ est de dimension finie $⟺$ il existe une famille 
génératrice finie de~$E$

\item \fiche{Cardinal} Si $\mc F$ est une famille libre et si $\mc G$ est une famille génératrice d'un EV $E$, alors 
\startformula
\card(\mc F)⩽\card(\mc G)
\stopformula

\item \fiche{Dimension infinie} Lorsque l'espace vectoriel $E$ n'est pas de dimension finie, on dit qu'il est de dimension infinie et l'on note $\dim_𝕂(E)=+∞$. \crlf

\item \fiche{Caractérisation de la dimension infinie} Si $E$ est un espace vectoriel, 
\startformula
E \text{ de dimension infinie } ⟺ E \text{ contient des familles libres de cardinal arbitrairement grand.} 
\stopformula


\item \fiche{Théorème de la base incomplète} Toute famille libre (ou vide) d'un EV $E$, engendré par une famille finie $\mc G$, peut être complétée avec des vecteurs de $\mc G$ pour en constituer une base.
\startformula
\System{\NC (e_1,⋯, e_m) \text{ libre}\NR
\NC (f_1,⋯,f_n) \text{ génératrice}}
⟹ \Align{
\NC ∃p∈⟦0,n⟧ \Et 1⩽i_1\Le ⋯\Le i_p⩽n:\NR
\NC( e_1,⋯,e_m,f_{i_1},⋯,f_{i_p})\text{ base}}
\stopformula

\item \fiche{Existence des bases} Tout espace vectoriel de dimension finie admet une base.

\item \fiche{Dimension} Par convention, $\dim_𝕂\{0\}=0$ et la dimension d'un $𝕂$-espace vectoriel $E≠\{0\}$ engendré par une famille finie est l'unique nombre entier positif vérifiant 
\startformula
\Align{
\NC \dim_𝕂(E)\NC = \max\{\card(\mc F):\mc F\text{ famille libre de }E\}\NR
\NC\NC = \min\{\card(\mc F):\mc F\text{ famille génératrice de }E\}\NR
\NC\NC= \card(\mc B)\qquad (\mc B\text{ base de }E)
}
\stopformula
{\it Les bases ont exactement $\dim(E)$ éléments, les familles libres ont au plus $\dim(E)$ éléments et les familles génératrices ont au moins $\dim(E)$ éléments.}
\blank[medium]

\item \fiche{Caractérisation des bases} Si $\mc F$ est une famille de $n$ vecteurs d'un EV de dimension $n$, alors
\startformula 
\NC \mc F \text{ libre } \NC ⟺ \mc F \text{ génératrice}\NR
 \NC\NC⟺ \mc F \text{ base}
\stopformula

\item \fiche{Sous-espace} Si $F$ est un sous espace d'un espace vectoriel $E$ de dimension finie, alors $F$ est de dimension finie et 
\startformula 
\dim_𝕂(F)⩽\dim_𝕂(E)
\stopformula

\item \fiche{Egalité} Si $F$ est un sous espace d'un espace vectoriel $E$ de dimension finie, alors 
\startformula 
F=E ⟺ \dim_ 𝕂(F)=\dim_𝕂(E)
\stopformula

\item \fiche{Rang d'une famille} Le rang d'une famille finie de vecteurs $\mc F$ est la dimension de l'espace qu'ils engendrent
\startformula 
\rg(\mc F)=\dim\Vect(\mc F)
\stopformula


\item \fiche{Rang et matrices} 
Si $X_1,⋯,X_n$ sont les matrices de $x_1,⋯,x_n$  dans $\mc B$ et si $M$ est la matrice rectangulaire dont les colonnes sont $X_1,⋯,X_n$ alors 
\startformula 
\rg(x_1,⋯,x_n)=\rg(X_1,⋯, X_n)=\rg(M)
\stopformula
{\it Le rang d'une famille de vecteurs est le rang de leurs matrices coordonnées (dans n'importe quelle base)}\blank[medium]


\item \fiche{Rang et familles} Si $\mc F$ est une famille finie de vecteurs d'un espace vectoriel $E$.
\startformula 
\Align{
\NC \mc F \text{ famille génératrice de } E \NC ⟺ \rg(\mc F)=\dim(E)\NR
\NC \mc F \text{ famille libre de } E \NC ⟺ \rg(\mc F)=\card(\mc F)}
\stopformula


\item \fiche{Dimensions de référence} pour $n⩾1$ et $p⩾1$, on a
\startformula 
\dim_𝕂(𝕂^n)=n\qquad \dim_𝕂\big(𝕂_n[X]\big)=n+1\qquad\dim_𝕂\mc M_{n,p}(𝕂)=np
\stopformula

\item \fiche{Espaces sur $ℂ$ et $ℝ$} Si $E$ est un $ℂ$-espace vectoriel, alors $E$ est également un $ℝ$-espace vectoriel. \crlf 
De plus, si $E$ est engendré par une famille finie, alors 
\startformula 
\dim_ℝ(E)=2 ×\dim_ℂ(E)
\stopformula
{\it Si $e_1, ⋯, e_n$ est une base du  $ℂ$-EV $E$, alors $e_1, ⋯, e_n, i.e_1, ⋯, i.e_n$ est une base du $ℝ$-EV $E$. }
\blank[medium]

\item \fiche{Produit cartésien} si $E$ et $F$ sont des $𝕂$-espaces vectoriels, alors $E×F$ l'est également pour les opérations définies par 
\startformula
\Align{
\NC (x,y)+(x', y') \NC = (x+x', y+y') \qquad \text{ pour $(x,y)$ et $(x',y')$ dans $E×F$}\NR
\NC λ.(x,y) \NC = (λ.x, λ.y) \qquad \text{ pour $λ∈𝕂$ et $(x,y)$ dans $E×F$}\NR
}
\stopformula
Si $E$ et $F$ sont de dimensions finies, on a 
\startformula 
\dim_𝕂(E×F) = \dim_𝕂(E) + \dim_𝕂(F)
\stopformula
{\it En particulier, si $e_1, ⋯, e_n$ est une base de $E$ et $f_1, ⋯, f_k$ est une base de $F$, alors  $(e_1, 0), ⋯, (e_n, 0), (0, f_1), ⋯, (0, f_k)$ est une base de $E×F$.}
\blank[medium]

\item \fiche{Sommes d'espaces vectoriels} La somme de $n$ sous-espaces vectoriels $E_1, ⋯E_n$ d'un $𝕂$-espace vectoriel $E$ est l'espace vectoriel
\startformula
E_1+⋯+E_n=\{x_1+⋯+x_n:(x_1,⋯,x_n)∈E_1×⋯×E_n\}
\stopformula 

\item \fiche{Somme directe} On dit que la somme $E_1+⋯+E_n$ est directe et l'on note $E_1⊕⋯⊕E_n$ à la place de $E_1+⋯+E_n$ si, et seulement si, pour $(x_1,⋯,x_n)∈E_1×⋯×E_n$, 
\startformula
 x_1+⋯+x_n=0\quad ⟹ \quad x_1=⋯=x_n=0
\stopformula

\item \fiche{Dimension d'une somme directe} Si $E_1, ⋯,E_n$ sont des sous-espaces de dimension finie d'un espace vectoriel $E$ en somme directe, alors
\startformula
\dim(E_1⊕⋯⊕E_n) = \dim(E_1)+⋯+\dim(E_n)
\stopformula


\item \fiche{Somme de deux espaces} Si $F$ et $G$ sont deux sous-espaces vectoriels de $E$, alors
\startformula
\Align{
\NC F+G\NC =\{x+y:x∈F,y∈G\}\NR
\NC F⊕G\NC ⟺  \quad F∩G = \{0\}\NR
}
\stopformula

\item \fiche{Dimension de la somme de deux espaces} Si $F$ et $G$ sont deux sous-espaces vectoriels de dimension finie de $E$, alors
\startformula
\Align{
\NC \dim(F+G)\NC =\dim(F)+\dim(G)-\dim(F∩G)\NR
\NC \dim(F⊕G)\NC =\dim(F)+\dim(G)\NR
}
\stopformula

\item \fiche{Supplémentaire} Si $F$ et $G$ sont des sous-espaces d'un espace vectoriel $E$, alors
\startformula
\text{$F$ et $G$ sont supplémentaires dans $E$} \quad ⟺\quad E=F⊕G \quad ⟺\quad \System{\NC E = F+G\NR \NC F⊕G\NR} 
\stopformula


\item \fiche{Supplémentaire en dimension finie} Tout sous-espace $F$ d'un espace vectoriel $E$ de dimension finie admet un supplémentaire $G$ dans $E$. De plus, on a 
\startformula
\dim_𝕂(G)=\dim_𝕂(E)-\dim_𝕂(F)
\stopformula

\item \fiche{Sommes directes et bases} Soient $E_1, ⋯, E_n$ des sous-espaces vectoriels de $E$ munis de bases (finies) $\mc B_1, ⋯, \mc B_n$. Alors
\startformula
E_1⊕⋯⊕E_n \quad ⟺\quad \big(\mc B_1, ⋯, \mc B_n\big) \text{ est une base de } E_1+⋯+E_n
\stopformula


\chapter{Etude asymptotique}

\section{suites}

\item \fiche{suite négligeable} Une suite $u$ est négligeable devant une suite $v$ si, et seulement s'il existe une suite $ε$ de limite nulle telle que $u_n=v_nε_n$ à partir d'un certain rang
\startformula 
u_n = o(v_n)\quad ⟺ \quad u_n = v_nε_n \text{ pour $n⩾N$ avec } \lim ε_n=0
\stopformula
{\it A la condition que la suite $v$ ne s'annule pas à partir d'un certain rang, on a }
\startformula 
u_n = o(v_n)\quad ⟺\quad  \lim {u_n\F v_n}=0
\stopformula


\item \fiche{opérations} Soient $u$, $v$, $u'$, $v'$ et $w$ des suites. Alors
\startformula
\Align{
\NC u=o(v)\text{ et } v = o(w) \NC ⟹ \NC u=o(w) \NC \text{(transitivité)}\NR
\NC u=o(w) \text{ et } v = o(w) \NC ⟹ \NC u+v=o(w) \NC \text{(addition)}\NR
\NC u=o(v) \NC ⟺ \NC λu=o(v) \NC  \text{(multiple, $λ∈ℝ^*$)}\NR
\NC \NC ⟺ \NC u=o(λw)\NC  \text{(multiple, $λ∈ℝ^*$)}\NR
\NC u=o(v) \text{ et } u' = o(v') \NC ⟹ \NC uv=o(vv') \NC \text{(multiplication 1)}\NR
\NC u=o(v) \NC ⟹ \NC uw=o(vw) \NC \text{(multiplication 2)}\NR
}
\stopformula

{\it Remarque, ce sont ces propriétés qui justifient que 
\startformula 
\Align{
\NC\ln(n)=o(n)=o(n²)\qquad  -5o(1)= o(1) = o(12)\NR 
\NC o(1)+o(1)= o(1)\qquad \ln(n).o(n) = o\big(n\ln(n)\big)=o(n²)\NR
}
\stopformula
}

\item \fiche{suite équivalente} La suite $u$ est équivalente à la suite $v$ si, et seulement s'il existe une suite $α$ de limite $1$ telle que $u_n=v_nα_n$ à partir d'un certain rang
\startformula 
u_n ∼ v_n\quad ⟺ \quad  u_n = v_n + o(v_n) \quad ⟺ \quad u_n = v_nα_n \text{ pour $n⩾N$ avec } \lim α_n=1
\stopformula
{\it A la condition que la suite $v$ ne s'annule pas à partir d'un certain rang, on a }
\startformula 
u_n ∼ v_n\quad ⟺\quad  \lim {u_n\F v_n}=1
\stopformula

\item \fiche{signe} si $u∼v$ alors $u$ et $v$ ont le même signe à partir d'un certain rang. De plus, si $u$ ne s'annule pas à partir d'un certain rang, alors, $v$ ne s'annule pas non plus à partir d'un certain rang

\item \fiche{opérations} Soient $u$, $v$, $U$, $V$ et $w$ des suites. Alors
\startformula
\Align{
\NC u∼u \NC  \NC  \NC \text{(reflexivité)}\NR	
\NC u∼v\NC ⟺ \NC v∼u \NC \text{(symétrie)}\NR
\NC u∼v\text{ et } v ∼w \NC ⟹ \NC u∼w \NC \text{(transitivité)}\NR
\NC u∼v \NC ⟺ \NC λu∼λv \NC  \text{(multiple, $λ∈ℝ^*$)}\NR
\NC u∼v \text{ et } U ∼V \NC ⟹ \NC uU∼vV \NC \text{(compatibilité avec la multiplication)}\NR
\NC u∼v \NC ⟹ \NC u^n∼v^n \NC \text{(compatibilité avec les puissances, $n∈ℕ$)}\NR
\NC u∼v \NC ⟹ \NC u^n∼v^n \NC \text{(compatibilité avec les puissances, $n∈ℤ^*$, $u_n≠0$ pour $n⩾N$)}\NR
\NC u∼v \text{ et } U ∼V \NC ⟹ \NC {u\F U}∼{v\F V}\NC \text{(compatibilité avec le quotient, $U_n≠0$ pour $n⩾N$)}\NR
}
\stopformula
{\it Remarque, ce sont ces propriétés qui justifient que 
	\startformula 
n∼n+1∼n+2\qquad  n+2∼n+1∼n\qquad  n^3= (n+1)^3 \qquad  {(n + 2)^2\F (n+1)^3} ∼ {n^2\F n^3}={1\F n}
\stopformula
}

\item \fiche{limites} Soient $u$ une suite et $ℓ∈ℝ^*$. 
\startformula 
\Align{
\NC \lim u = 0 \NC ⟺ u = o(1)\NR
\NC \lim u = ℓ \NC ⟺ u∼ℓ\NR
}\stopformula

\item \fiche{Théorème de croissance comparée} Pour $α>0$, on a  
\startformula
\ln(n) = o(n^α)\qquad n=o\big(\e^{αn}\big)\qquad\e^{αn}=o\big(n!\big)
\stopformula


\section{Fonctions}

Dans cette section, les fonctions sont définies sur un ensemble $D⊂ℝ$ quit contient ou est au contact de $a∈ℝ∪\{-∞, +∞\}$.
\blank[medium]

Pour simplifier les énoncés, on note $\mc V_a$ et l'on appelle voisinage de $a$ tout ensemble du type 
\startformula
\mc V_a = \System{
\NC D∩]a-δ, a+δ[ \NC \quad (δ>0) \NC \text{ si $a∈ℝ$}\NR  
\NC D∩[b, +∞[ \NC \quad (b∈ℝ) \NC \text{si $a=+∞$}\NR
\NC D∩]-∞, b[ \NC \quad(b∈ℝ) \NC \text{ si $a=-∞$}\NR
}
\stopformula

{\bf Pour une fonction, on doit toujours préciser en quel point $a$ l'estimation a lieu\crlf
\it Les propriétés suivantes sont les mêmes que pour les suites avec $x→a$ au lieu de~$n→+∞$. }
\blank[medium]


\item \fiche{fonctions négligeables} Une fonction $f:D→ℝ$ est négligeable devant une fonction $g:D→ℝ$ au voisinage de  $a$  si, et seulement s'il existe une fonction $ε:D→ℝ$ de limite nulle 
en $a$ telle que $f(x)=g(x)ε(x)$ au voisinage de $a$ (pour $x∈\mc V_a$)
\startformula 
f(x) = o_a\big(g(x)\big)⟺ f(x) = g(x)ε(x) \text{ pour $x∈\mc V_a$ avec } \lim_{x→a} ε(x)=0
\stopformula 
{\it A la condition que la fonction $f$ ne s'annule sur un visinage de $a$, on a }
\startformula 
f(x) = o_a\big(g(x)\big)\quad ⟺\quad  \lim_{x→a} {f(x)\F g(x)}=0
\stopformula

\item \fiche{opérations} Soient $f$, $g$, $h$, $F$, $G$ des fonctions de $D→ℝ$. Alors
\startformula
\Align{
\NC f(x)=o_a\big(g(x)\big)\text{ et } g(x) = o_a\big(h(x)\big) \NC ⟹ \NC f(x)=o_a\big(h(x)\big) \NC \text{(transitivité)}\NR
\NC f(x)=o_a\big(h(x)\big) \text{ et } g(x) = o_a\big(h(x)\big) \NC ⟹ \NC f(x)+g(x)=o_a\big(h(x)\big) \NC \text{(addition)}\NR
\NC f(x)=o_a\big(g(x)\big) \NC ⟺ \NC λf(x)=o_a\big(g(x)\big) \NC  \text{(multiple, $λ∈ℝ^*$)}\NR
\NC \NC ⟺ \NC f(x)=o_a\big(λg(x)\big)\NC  \text{(multiple, $λ∈ℝ^*$)}\NR
\NC f(x)=o_a\big(g(x)\big) \text{ et } F(x)= o_a\big(G(x)\big) \NC ⟹ \NC f(x)g(x)=o_a\big(F(x)G(x)\big) \NC \text{(multiplication 1)}\NR
\NC f(x)=o_a\big(g(x)\big) \NC ⟹ \NC f(x)h(x)=o_a\big(g(x)h(x)\big) \NC \text{(multiplication 2)}\NR
}
\stopformula

\item \fiche{fonction équivalente} La fonction $f$ est équivalente à la fonction $g$ en $a$ si, et seulement s'il existe une fonction $α$ de limite $1$ en $a$ telle que $f(x)=g(x)α(x)$ au voisinage de $a$
\startformula 
f(x) ∼\limits_a g(x)\quad ⟺ \quad f(x) = g(x)+o\limits_a\big(g(x)\big)\quad ⟺ \quad f(x) = g(x)α(x) \text{ pour $x∈\mc V_a$ avec } \lim_{x→a} α(x)=1
\stopformula
{\it A la condition que la suite $g(x)$ ne s'annule pas au voisinage de $a$, on a }
\startformula 
f(x) ∼\limits_a g(x)\quad ⟺\quad  \lim_{x→a} {f(x)\F g(x)}=1
\stopformula

\item \fiche{signe} si $f(x)∼\limits_ag(x)$ alors $f$ et $g$ ont le même signe au voisinage de $a$. De plus, si $f$ ne s'annule pas au voisinage de $a$, alors, $g$ ne s'annule pas non plus au voisinage de $a$.

\item \fiche{opérations} Soient $f$, $g$, $F$, $H$ et $g$ des fonctions de $D$ dans $ℝ$. Alors
\startformula
\Align{
\NC f∼\limits_af \NC  \NC  \NC \text{(reflexivité)}\NR	
\NC f∼\limits_ag\NC ⟺ \NC g∼\limits_af \NC \text{(symétrie)}\NR
\NC f\limits∼_ag\text{ et } g ∼\limits_ah \NC ⟹ \NC f∼_ah \NC \text{(transitivité)}\NR
\NC f∼\limits_ag \NC ⟺ \NC λf∼\limits_aλg \NC  \text{(multiple, $λ∈ℝ^*$)}\NR
\NC f∼\limits_ag \text{ et } F ∼\limits_aG \NC ⟹ \NC fF∼_agG \NC \text{(compatibilité avec la multiplication)}\NR
\NC f∼\limits_ag \NC ⟹ \NC f^n∼\limits_ag^n \NC \text{(compatibilité avec les puissances, $n∈ℕ$)}\NR
\NC f∼\limits_ag \NC ⟹ \NC f^n∼\limits_ag^n \NC \text{(compatibilité avec les puissances, $n∈ℤ^*$, $f(x)≠0$ pour $x∈\mc V_a$)}\NR
\NC f∼\limits_ag \text{ et } F ∼\limits_aG \NC ⟹ \NC {f\F F}∼\limits_a{g\F G}\NC \text{(compatibilité avec le quotient, $F(x)≠0$ pour $x∈\mc V_a$)}\NR
}
\stopformula


\item \fiche{limites} Soient $f:D→ℝ$ une fonction et $ℓ∈ℝ^*$. 
\startformula 
\Align{
\NC \lim_a f(x) = 0 \NC ⟺ f(x) = o_a(1)\NR
\NC \lim_a f(x) = ℓ \NC ⟺ f(x)∼\limits_aℓ\NR
}\stopformula

\item \fiche{Théorème de croissance comparée} Pour $α>0$, on a  
\startformula
\ln(x) = o_∞(x^α)\qquad x=o_∞\big(\e^{αx}\big)\qquad  x^α\ln(x) = o_{0^+}\big(1\big)\qquad  x\e^{αx} = o_{-∞}(1)
\stopformula


\item \fiche{Développements limités de référence} Pour $n⩾0$, on a 
\startformula 
\Align{
\NC {1\F 1-u} \NC =\displaystyle ∑_{k=0}^nu^k+o_0(u^n)\NR
\NC \e^u\NC =\displaystyle ∑_{k=0}^n{u^k\F k!}+o_0(u^n)\NR
\NC \ln(1+u)\NC =\displaystyle ∑_{k=1}^n(-1)^{k-1}{u^k\F k}+o_0(u^n)\NR
\NC \cos(u)\NC=\displaystyle ∑_{k=0}^n(-1)^k{u^{2k}\F (2k)!}+o_0(u^{2n})\NR
\NC \sin(u)\NC=\displaystyle ∑_{k=0}^n(-1)^k{u^{2k+1}\F (2k+1)!}+o_0(u^{2n+1})\NR
\NC (1+u)^α\NC =\displaystyle 1+∑_{k=1}^nα(α-1)⋯(α-k+1){u^k\F k!}+o_0(u^n)\NR
\NC \arctan(u)\NC = u +o_0(u)\NR
}
\stopformula
{\it Remarque, la formule $ (1+u)^α = 1+αu+{α(α-1)\F 2}u²+o_0(u²)$ suffit en général dans les exercices.}

\stopList

\chapter{Séries numériques}

Dans ce chapitre, $u$ désigne une suite de $ℝ^ℕ$ et $S$ désigne la suite des sommes partielles de $u$ définie par 
\startformula
S_n = ∑_{k=0}^nu_k\qquad(n⩾0)
\stopformula
{\it Remarque : une suite $v$ peut être considérée comme une série de terme général $u$ défini par 
\startformula
u_0= v_0 \text{ et }u_k=v_k-u_{v-1}\qquad (k⩾1)
\stopformula}

\section{séries quelconques}

\startList\item \fiche{Série} On dit que la série de terme général $u∈ℝ^ℕ$ converge ou qu'elle est de nature convergente si, et seulement si, la suite des sommes partielles de $u$ converge
\startformula
∑_{k=0}^∞u_k\text{ converge } ⟺ S \text{ converge}
\stopformula
En cas de divergence de $S$, on dit que la série diverge ou qu'elle est de nature divergente. \crlf

\item \fiche{Somme de la série} En cas de convergence de la suite $S$ des sommes partielles, sa limite $ℓ$ est appelée somme de la série et est notée
\startformula
∑_{k=0}^∞u_k = ℓ =\lim S =  \lim_{n→∞}∑_{k=0}^nu_k
\stopformula

\item \fiche{Suite des restes} La suite des reste $R$ d'une série convergente de terme général $u$ est définie par 
\startformula
R_n = ∑_{k=0}^∞u_k -S_n = \sum_{k=n+1}^∞u_k\qquad (n⩾0)
\stopformula
{\it En particulier, on a $\sum_{k=0}^∞u_k = S_n+R_n$ pour $n⩾0$}
\blank[medium]

\item \fiche{Condition nécéssaire de convergence} Si $u∈ℝ^ℕ$ est le terme général d'une série convergente, alors $u$ converge vers $0$.
\startformula
∑_{k=0}^∞u_k \text{ converge } ⟹\lim u= 0
\stopformula

\item \fiche{Relation de Chasles} Pour $u∈ℝ^ℕ$ et $m∈ℕ$, on a 
\startformula
∑_{k=0}^∞u_k\text{ converge } ⟺ ∑_{k=m}^∞u_k\text{ converge } 
\stopformula
Et en cas de convergence, on a 
\startformula
∑_{k=0}^∞u_k = ∑_{k=0}^{m-1}u_k + ∑_{k=m}^∞u_k
\stopformula
{\it Remarque : la convergence d'une série ne dépend que de ce qui se passe au voisinage de l'infini.}

\item \fiche{Linéarité} Soient $(λ,μ)∈ℝ^2$ et $u$ et $v$ les termes généraux de deux séries convergentes.Alors 
\startformula
∑_{k=0}^∞(λu_k+μv_k) \text{ converge et }∑_{k=0}^∞(λu_k+μv_k)=λ∑_{k=0}^∞u_k+μ∑_{k=0}^∞v_k
\stopformula

\item \fiche{Addition} Si $∑u_n$ et $∑v_n$ convergent, alors $∑(u_n+v_n)$ converge\crlf
 Si $∑u_n$ converge et si  $∑v_n$ diverge, alors $∑(u_n+v_n)$ diverge.\crlf
 
\item \fiche{Convergence absolue} On dit qu'une série de terme général $u$ converge absolument si et seulement si la série de terme général $|u|$ converge
\startformula
∑_{k=0}^∞u_k\text{ converge absolument} ⟺ ∑_{k=0}^∞|u_k|\text{ converge}
\stopformula 
\item \fiche{Convergence absolue et simple} Si la série $∑u_k$ converge absolument, alors la série $∑u_k$ converge et de plus
\startformula
\Q|∑_{k=0}^∞u_k\W|⩽ ∑_{k=0}^∞|u_k|
\stopformula 

\section{séries à termes positifs}

Dans cette section $u$ est à terme positifs ou nuls (mais les énoncés sont aussi valables pour les suites négatives ou nulles quite à tout multiplier par $-1$)
En pratique, il faut que $u$ reste de signe constant à partir d'un certain rang.
\blank[medium]

\item \fiche{équivalent} Soient $u$ et $v$ deux suites à termes positifs ou nuls. Si $u∼v$, alors les séries de termes généraux $u$ et $v$ ont la même nature
{\it Plus généralement, on a }
\startformula 
\SystemR{
	\NC u∼v\NR
	\NC u_n \text{ de signe constant pour $n⩾N$}\NR}
	⟹ ∑_{n=0}^∞u_k \text{ a même nature que }  ∑_{n=0}^∞v_k
	\stopformula

\item \fiche{inégalité} Soient $u$ et $v$ deux suites telles que $0⩽u_n⩽v_n$ pour $n⩾0$. \crlf
Si $∑u_k$ diverge, alors $∑v_k$ diverge  \crlf
Si $∑v_k$ converge, alors $∑u_k$ converge et on a 
\startformula 
0⩽∑_{n=0}^∞u_k⩽  ∑_{n=0}^∞v_k
\stopformula


\item \fiche{petit o} Soient $u$ et $v$ deux suites à termes positifs ou nuls telles que $u_n=o(v_n)$. \crlf
Si $∑u_k$ diverge, alors $∑v_k$ diverge  \crlf
Si $∑v_k$ converge, alors $∑u_k$ converge.

	
\section{séries de référence}

\item \fiche{Série de Riemann} Pour $α∈ℝ$, on a 
\startformula 
∑_{n=1}^∞{1\F n^α} \text{ converge } \quad ⟺\quad  α>1
\stopformula

\item \fiche{Série géométrique et dérivées} Pour $x∈ℝ$, les séries $∑x^n$, $∑nx^{n-1}$ et $∑n(n-1)x^{n-2}$ convergent si, et seulement si, $-1<x<1$. De plus, on a 
\startformula 
\Align{
\NC\displaystyle∑_{n=0}^∞ x^n \NC={1\F 1-x}\NR 
\NC\displaystyle∑_{n=1}^∞ nx^{n-1} \NC={1\F (1-x)²}\NR 
\NC\displaystyle∑_{n=2}^∞ n(n-1)a^{n-2} \NC={2\F (1-a)^3}\NR 
}\qquad -1<x<1
\stopformula

\item \fiche{Exponentielle} Pour $x∈ℝ$, la série $∑_{n=0}^∞ {x^n\F n!}$, converge et 
\startformula 
∑_{n=0}^∞ {x^n\F n!} =\e^x\qquad -1<x<1
\stopformula


\stopList




\chapter{Probabilités sur un espace quelconque}

\startList

Dans tout ce chapitre, $Ω$ désigne un ensemble quelconque. \crlf
Lorsque $p∈[0,1]$, on pose $q=1-p$.

\section{Espace probabilisé}

\item \fiche{Ensemble dénombrable} Soit $E$ un ensemble.
\startformula
\Align{
\NC E \text{est dénombrable} \NC⟺ \NC \text{il existe une surjection } f:ℕ→E\NR
\NC\NC⟺ \NC\text{il existe une injection } f:E→ℕ\NR}
\stopformula

\item\fiche{Ensembles dénombrables de référence} $ℕ$, $ℤ$ et $ℚ$ sont dénombrables alors que $ℝ$ ne l'est pas.\crlf
Un sous-ensemble d'un ensemble dénombrable est dénombrable (tout ensemble fini est dénombrable).

\item\fiche{Réunion et intersection} $I$ ensemble et $(A_i)_{i∈I}$ famille de parties de $Ω$ indicée par $I$
\startformula
\bigcup\limits_{i∈I}A_i=\{x∈Ω:∃i∈I, x∈A_i\}\qquad\text{et}\qquad \bigcap\limits_{i∈I}A_i=\{x∈Ω:∀i∈I, x∈A_i\}
\stopformula

\item \fiche{Tribu} Une tribu d'événements de $Ω$ est un ensemble $\sc A⊂\sc P(Ω)$ de parties de $Ω$ contenant $Ω$ 
et stable par passage au complémentaire et par réunion dénombrable
\startformula
\System{
\NC Ω∈\sc A\NR
\NC ∀A∈\sc A,\qquad \overline A∈\sc A\NR
\NC ∀ (A_i)_{i∈I}∈{\sc A}^I \text{ avec $I$ dénombrable}, \qquad \bigcup\limits_{i∈I}A_i∈\sc A\NR 
}
\stopformula
{\it On appelle aussi $σ$-algèbres les tribus et $σ$-additivité la stabilité par réunion dénombrable}


\item \fiche{Propriétés} Une tribu $\sc A$ contient l'ensemble vide $\varnothing$ et est stable par intersection dénombrable
\startformula
∀(A_i)_{i∈I}∈{\sc A}^I \text{ avec $I$ dénombrable}, \qquad \bigcap\limits_{i∈I}A_i∈\sc A
\stopformula

\item \fiche{Exemples} $\sc P(Ω)$ est la tribu discrète et $\{Ω, \varnothing\}$ est la tribu grossière de $Ω$.

\item \fiche{Espace probabilisable} $(Ω, \sc A)$ est un espace probabilisable $⟺ \sc A$ est une tribu de $Ω$.

\item \fiche{Système complet} Un système complet d'événements d'un espace probabilisable $(Ω, \sc A)$ est une famille dénombrable $(A_i)_{i∈I}\sc A^I$ d'événements dont la réunion vaut $Ω$ et incompatibles deux à deux 
\startformula
\bigcup_{i∈I}A_i=Ω\qquad\text{et}\qquad i≠j⟹ A_i∩A_j=\varnothing
\stopformula

\item \fiche{Tribu engendrée} La tribu $\sc T$ engendrée par un ensemble $E$ de parties de $Ω$ est la plus petite tribu (au sens de l'inclusion) de $Ω$ contenant $E$.
\startformula
{\sc T} = \bigcap\limits_{E⊂{\sc A}\text{ tribu de }Ω} {\sc A}
\stopformula

\item \fiche{Tribu engendrée par un système complet d'événements} La tribu $\sc T$ engendrée par un système complet d'événements $(A_i)_{i∈I}$ est la tribu définie par 
\startformula
{\sc T} = \Q\{\bigcup_{j∈J}A_j:J⊂I\W\}
\stopformula
{\it avec la convention que $\bigcup_{j∈\varnothing}A_j=\varnothing$}

\item \fiche{Probabilité} Une probabilité sur un espace probabilisable $(Ω,\sc A$ est une application $P:\sc A→[0,1]$ vérifiant $P(Ω)=1$ et 
\startformula
\stopformula

\item \fiche{Espace Probabilisé} $(Ω,\sc A, P)$ est un espace probabilisé si, et seulement si, $(Ω, \sc A)$ est un espace probabilisable dont $P$ est une probabilité.

\item\fiche{Additivité} Si $(A_i)_{1⩽i⩽n}$ sont des événements mutuellemens incompatibles alors
\startformula
 P\Q(\bigcup\limits_{1⩽i⩽n}A_i\W)=∑_{i=1}^nP(A_i)
\stopformula

\item\fiche{Complémentaire} Si $A$ est un événement, alors $P(\overline A)=1-P(A)$. 
\item\fiche{Réunion} Si $A$ et $B$ sont des événements, alors $P(A∪B)=P(A)+P(B)-P(A∩B)$

\item\fiche{Evénements} Soit $A$ un événement
\startformula
\Align{
\NC A \text{ est presque sûr } \NC⟺ \NC P(A)=1\NR
\NC A \text{ est négligeable } \NC⟺ \NC P(A)=0\NR
}
\stopformula
{\it Un événement presque sûr est aussi appelé événement quasi certain}


\item\fiche{Théorème de la limite monotone} Pour toute suite croissante (au sens de l'inclusion) d'événements $(A_n)_{n⩾0}$,on a 
\startformula
P\Q(\bigcup_{n⩾0}A_i\W)=\lim_{n→∞}P(A_i)
\stopformula
Pour toute suite cdéroissante (au sens de l'inclusion) d'événements $(A_n)_{n⩾0}$,on a 
\startformula
P\Q(\bigcap_{n⩾0}A_i\W)=\lim_{n→∞}P(A_i)
\stopformula

\item\fiche{Corrolaire} Pour toute suite d'événements $(A_n)_{n⩾0}$,on a 
\startformula
\Align{
\NC P\Q(\bigcup_{n⩾0}A_i\W)\NC =\NC \displaystyle\lim\limits_{n→∞}P\Q(\bigcup_{0⩽i⩽n}A_i\W)\NR
\NC P\Q(\bigcap_{n⩾0}A_i\W)\NC =\NC \displaystyle\lim_{n→∞}P\Q(\bigcap_{0⩽i⩽n}A_i\W)\NR
}
\stopformula


\item\fiche{Indépendance mutuelle} Des événements $(A_i)_{i∈I}$ sont mutuellements indépendants (avec $I$ dénombrable) si, et seulement si
$$
∀J⊂I,\qquad P\Q(\bigcap_{j∈J}A_j\W)=∏_{j∈J}P(A_j)
$$



\section{Variable aléatoire réelle}

\item \fiche{VAR} Une variable aléatoire réelle sur $(Ω,\sc A)$ est une application $X:Ω→ℝ$ vérifiant
\startformula
∀ x∈ℝ, \qquad \{ω∈Ω:X(ω)⩽x\}∈\sc A
\stopformula
{\it Autrement dit, les ensembles $[X⩽x]$ sont des événements pour $x∈ℝ$}
\blank[medium]


%\item \fiche{VAR certaine} $X$ est une VAR certaine ssi il existe $c∈ℝ$ tel que $∀ω∈Ω, X(ω)=c$.
%\item \fiche{VAR quasi-certaine} $X$ est une VAR quasi-certaine ssi il existe $c∈ℝ$ tel que $P(X=c)=1$.
%\item \fiche{Univers image} L'univers image d'une VAR $X$ est $X(Ω)=\{X(ω):ω∈Ω\}$.
%\item \fiche{Système complet} Le système complet associé à une VAR $X$ est $\{(X=x)\}_{x∈X(Ω)}$.

{\it Remarque : la loi d'une VAR $X$ est la probabilité $P_X$ définie sur $ℝ$ par 
$$
P_X(A) = P([X∈A])\qquad(A \text{ événement de } ℝ),
$$
l'ensemble $ℝ$ étant muni de la tribu engendrée par les intervalles du type $]-∞, x]$ pour $x∈ℝ$. }
\blank[medium]

\item \fiche{fonction de répartition d'une VAR} La fonction de répartition d'une VAR $X$ est l'application $F_X:ℝ→[0,1]$ définie par 
$$
F_X(x)=P(X⩽x)\qquad(x∈ℝ)
$$
{\it Remarque : la loi de $X$ est complètement déterminée par la donnée de $F_X$}


\item \fiche{Propriétés de $F_X$} La fonction de répartition $F_X$ d'une var $X$  est croissante sur $ℝ$, continue à droite et vérifie 
\startformula
\lim_{x→-∞}F_X(x)=0  \text{ et } \lim_{x→∞}F_X(x)=1
\stopformula

\section {Variable aléatoires réelles discrètes}


\item \fiche{VAR discrète} Une VAR $X:(Ω, \sc A)→ℝ$ est discrète $⟺$ $X(Ω)$ est dénombrable.

{\it Remarque : La loi d'une VAR discrète est complétement déterminée par la donnée des valeurs $P(X=x)$ pour $x∈X(Ω)$.}
\blank[medium]


\item \fiche{Tribu engendrée par une VAR discrète} La tribu $\sc A_X$ des événements liés à $X$ est la tribu engendrée par le système complet $([X=x])_{x∈X(Ω)}$.
\blank[small]
{\it Cette tribu est appelée tribu engendrée par la VAR $X$ et constitue l'information apportée par $X$.}

\blank[medium]
{\it Remarque : Si $X$ est une VAR discrète et si $g:X(Ω)→ℝ$ est une application, alors $Y=g(X)$ est une VAR discrète. }



\item \fiche{Espérance} $X$ VAR discrète admet une espérance $⟺$ $\sum\limits_{x∈ X(Ω)}xP(X=x)$ converge absolument.
Auquel cas, l'espérance de $X$ est la somme de cette série
\startformula
E(X) = \sum_{x∈ X(Ω)}xP(X=x)
\stopformula


\item \fiche{Théorème de transfert} Soit $X$ une VAR discrète et $g:X(Ω)→ℝ$. La VAR discrète $g(X)$ admet une espèrance si, et seulement si, la série $\sum_{x∈ X(Ω)}g(x)P(X=x)$ converge absolument.
Dans ce cas, on a 
\startformula
E(g(X)) = \sum_{x∈ X(Ω)}g(x)P(X=x)
\stopformula

\item \fiche{Transformation affine} Soient $a∈ℝ^*$ et $b∈ℝ$. Une VAR  $X$ admet une espérance si, et seulement si la VAR $aX+b$ admet une espérance. 
Dans ce cas, on a 
\startformula
E(aX+b)=aE(X)+b\qquad(a,b∈ℝ)
\stopformula


\item \fiche{Moments} Le moment d'ordre $r∈ℕ$ d'une VAR discrète $X$ est, s'il existe, le nombre 
\startformula
m_r(X)=E(X^r)
\stopformula
{\it Remarque si $X$ admet un moment d'ordre $r∈ℕ^*$ alors $X$ admet des moments d'ordre $k∈⟦0,r⟧$.}

\item \fiche{Positivité} Si $X$ est une VAR discrète vérifiant $\underbrace{∀ ω∈Ω, X(ω) ⩾ 0}_{X ⩾ 0}$ (p.s.), alors $E(X)⩾0$.
\item \fiche{Croissance} Si $X$ et $Y$ vérifient $\underbrace{∀ ω∈Ω, X(ω) ⩽ Y(ω)}_{X ⩽ Y}$ (p.s.), alors $E(X)⩽E(Y)$.
\item \fiche{Cas des VAR positives, d'espérance nulle} Si $X$ vérifie $\underbrace{∀ ω∈Ω, X(ω) ⩾ 0}_{X ⩾ 0}$ (p.s.), alors 
\startformula
E(X) = 0 \quad ⟺ \quad P(X = 0) = 1 \quad ⟺ \quad X = 0\text{ p.s.}
\stopformula

\section{Variance et écart type}

\item \fiche{Variance} La variance d'une VAR discrète $X$ est, lorsqu'elle existe, le nombre réel positif ou nul 
\startformula
V(X) = E\Big(\big(X-E(X)\big)^2\Big)
\stopformula
{\it Remarque : Une VAR discrète admet un moment d'ordre $2$ $⟺$ elle admet une esperance et une variance.}

\item \fiche{Formule de Koenig-Huygens} Si une VAR discrète $X$ admet un moment d'ordre $2$, alors 
\startformula
V(X) = E(X^2)-E(X)^2
\stopformula

\item \fiche{Variance nulle} Pour une VAR discrète $X$, on a 
\startformula
V(X) = 0 ⟺ X = E(X) \text{ p.s.}
\stopformula

\item \fiche{Transformation affine}  Soient $a∈ℝ^*$ et $b∈ℝ$. Une VAR discrète $X$ admet une variance si, et seulement si la VAR discrète $aX+b$ admet une variance. 
Dans ce cas, on a  
\startformula
V(aX+b)=a^2V(X)\qquad(a,b∈ℝ)
\stopformula
{\it La formule est trivialement vraie si $a=0$}

\item \fiche{Ecart type} L'ecart type d'une VAR discrète $X$ admettant une variance est le nombre réel positif ou nul $σ_(X) = \sqrt{V(X)}$
\item \fiche{VAR centrée réduite} Une VAR discrète $X$ est centrée et réduite ssi son espérance est nulle et sa variance vaut $1$ (son écart type vaut $1$)

\section{Lois usuelles}
\item \fiche{loi certaine} $X$ suit la loi certaine ssi $X=c$ p.s. Dans ce cas, on a $E(X)= c $ et $V(X) = 0$. 
\item \fiche{loi de Bernouilli de paramètre $p∈[0,1]$}  
$$X\hookrightarrow B(p)\quad⟺\quad P(X=1)=p \text{ et  } P(X=0)=q
$$
Dans ce cas, on a $E(X)= p$ et $V(X) = pq$.
\item \fiche{loi binomiale de paramètre $n∈ℕ^*$ et $p∈[0,1]$} 
$$
X\hookrightarrow B(n,p)\quad⟺\quad
P(X=k)={n\choose k}p^kq^{n-k}\qquad(0⩽k⩽n)
$$
Dans ce cas, on a $E(X)=np$ et $V(X)=npq$.
\item \fiche{loi uniforme} $X$ suit la loi uniforme sur $⟦1,n⟧$ ssi $X\hookrightarrow \sc U(⟦1,n⟧)$ ssi 
$$
P(X=k)={1\F n}\qquad(1⩽k⩽n)
$$
Dans ce cas, on a $E(X)= {n+1\F2}$ et $V(X) = {n^2-1\F 12}$. 
\item \fiche{loi géométrique} $X$ suit la loi géométrique de paramètre $p∈]0,1[$ ssi $X\hookrightarrow \sc G(p)$ ssi 
$$
P(X=k)=q^{k-1}p\qquad(1⩽k)
$$
Dans ce cas, on a $E(X)= {1\F p}$ et $V(X) = {q\F p^2}$.
\item \fiche{loi de poisson} $X$ suit la loi de poisson de paramètre $λ>0[$ ssi $X\hookrightarrow \sc P(λ)$ ssi 
$$
P(X=k)={λ^k\F k!}\e^{-λ}\qquad(0⩽k)
$$
Dans ce cas, on a $E(X)= λ$ et $V(X) = λ$.


\chapter{Exemples de suites réelles}
\startList

\section{Informations}

Les outils de ce chapitre font partis des bases importantes, ils sont utilisés en concours pour :
\startList\item Reconnaître le type d'une suite et identifier ses éléments caractéristiques
\item Trouver une formule pour $u_n$
\item Utiliser la somme des termes d'une suite au cours d'un calcul complexe.
\stopList
Utilisation en ds/concours : fréquente.


\section{Suites arithmétiques}

\item \fiche{définition}Une suite réelle $u$ est arithmétique de raison $r∈ℝ$ si, et seulement si, \startformula
u_{n+1} = u_n + r\qquad(n⩾0)
\stopformula
\item \fiche{formule}Soit $u$ une suite arithmétique de raison $r$. alors
\startformula
u_n = u_0 + nr\qquad(n⩾0).
\stopformula
\item \fiche{somme}Soit $u$ une suite arithmétique de raison $r$. Alors
\startformula
\sum_{k=m}^nu_k = {\darkgray (m-n+1)  {u_m+u_n\F 2}}=\text{nombre de termes}×\text{moyenne aux extrémités}
\stopformula

\section{Suites géométriques}

\item \fiche{définition}Une suite réelle $u$ est géométrique de raison $q∈ℝ$ si, et seulement si,  \startformula
u_{n+1} = qu_n\qquad(n⩾0)
\stopformula
\item \fiche{formule}Soit $u$ une suite géométrique de raison $q$. alors
\startformula
u_n = u_0 q^n\qquad(n⩾0).
\stopformula
\item \fiche{somme}Soit $u$ une suite géométrique de raison $q≠1$. Alors
\startformula
\sum_{k=m}^nu_k = {\darkgray u_m{1-q^{n-m+1}\F1-q}}={\text{premier terme}-\text{terme suivant le dernier}\F 1 - \text{raison}}
\stopformula
{\it Remarque :} si $q=1$, $u$ est une suite constante (arithmétique de raison nulle)
\startformula
\sum_{k=m}^nu_k = {\darkgray(m-n+1)u_m}=\text{nombre de termes}×\text{terme constant}
\stopformula

\section{Suites arithmético-géométriques}

\item \fiche{définition}Une suite réelle $u$ est arithmético-géométrique si, et seulement s'il existe des nombres réels $a$ et $b$ tels que 
\placeformula[suiteArithméticoGéométriqueDef]
\startformula 
u_{n+1} = au_n+b\qquad(n⩾0)
\stopformula

\item \fiche{formule} Soit $u$ une suite suite arithmético-géométrique pour $a≠1$ et $b≠0$. Alors, 
\startformula
u_n = c + a^n(u_0-c)\qquad\Q(c = {b\F 1-a}, n⩾0\W).
\stopformula
{\it Méthode recommandée : }
\startList
\item Chercher la suite constante $c$ vérifiant (\in[suiteArithméticoGéométriqueDef]) en résolvant l'équation $x=ax+b$
\item En faisant une soustraction, remarquer que la suite $v_n=u_n-c$ est géométrique de raison $a$ 
\item en déduire la formule.
\stopList

\section{Récurrences linéaires}

\item \fiche{définition}Une suite réelle $u$ satisfait une récurrence linéaire du second ordre si, et seulement s'il existe des nombres réels $a≠0$, $b$ et $c$ tels que 
\placeformula[récurrenceLinéaireDef]
\startformula 
au_{n+2}+bu_{n+1}+cu_n = 0\qquad(n⩾0)
\stopformula

\item \fiche{éléments caractéristiques}Le polynôme caractéristique associé à (\in[récurrenceLinéaireDef]) est le trinôme
\placeformula[récurrenceLinéairePolynomeCaractéristiqueDef]
\startformula
P=aX^2+bX+c
\stopformula
L'équation caractéristique associée à la relation (\in[récurrenceLinéaireDef]) est l'équation $aX^2+bX+c = 0$
\startformula
\underbrace{ax^2+bx+c=0}_{P(x)=0}
\stopformula

\item \fiche{formule} Si le polynôme caractéristique (\in[récurrenceLinéairePolynomeCaractéristiqueDef]) admet une racine double $s$, une suite réelle $u$ est solution de (\in[récurrenceLinéaireDef]) si, et seulement s'il existe des nombres réels $λ$ et $μ$ tels que 
\startformula
u_n=λs^n+μ{\darkred n}z^n\qquad(n⩾0)
\stopformula
Si  le polynôme caractéristique (\in[récurrenceLinéairePolynomeCaractéristiqueDef])  admet deux racines distinctes $s$ et $z$, une suite réelle $u$ est solution de (\in[récurrenceLinéaireDef]) si, et seulement s'il existe des nombres $λ$ et $μ$ tels que
\startformula
u_n=λs^n+μz^n\qquad(n⩾0)
\stopformula
{\it Remarque : }  Dans le cas où les racines de $P$ sont réelles, les constante $λ$ et $μ$ sont réelles. 
Lorsque les racines de $P$ ne sont pas réelles, elles sont conjuguées. En posant $s=r\e^{iϑ}$ et $z=r\e^{-iϑ}$, 
on peut alors trouver deux constantes réelles $α$ et $β$ telles que  
\startformula
u_n=αr^n\cos(nϑ)+βr^n\sin(nϑ)\qquad(n⩾0)
\stopformula

\stopList

\chapter{Exemples de suites réelles}
\startList
\stopList

\stopcolumns\endgroup
\stoptext
\stopproduct
