

 
Remarques : 
\startList
\item utiliser correctement $>$ et $⩾$
\item ne pas réecrire le sujet sur votre copie
\item Pour traiter la question n d'un exercice : 
\startitemize[1]
\item il est possible d'admettre le resultats des questions 1 à $n-1$. (si vous les utilisez sans les avoir préalablement démontrés, il est conseillé d'écrire sur votre copie aque vous les admettez)
\item on n'a pas le droit d'utiliser les résultats des questions $k>n$ pour prouver la question $n$
\stopitemize
\item utiliser un discriminant pour résoudre $x^2-2x+1=0$
\item manque de soin/mauvause ecriture = erreurs plus faciles à faire
\item eviter les récurrences inutiles (C.1)
\item nested copies quand on les rend
\item erreurs de lecture/innatention
\item suivre les consignes (mq $u$ croissante \importan{strictement})
\item propriétés de $\ln$ ,$\exp$
\item Optimiser les démonstration par récurrence (ecrire 3 lignes de moins * 5 demos par recurrence sur un devoir = 15 lignes de gagnées)

\item pour dériver un quotient, éviter d'écrire $f'={u'v-uv'\F v^2}$ avec $u=$, $v=$, $u'=$, v'=$ car 
\startitemize[1]
\item Ca fait \quote{neuneu}, c'est basique et au niveau prépa, tout le monde est sensé savoir dériver un quotient
\item Ca vous ralentit en vous faisant écrire des choses inutiles
\item Si vous faites ça systèmatiquement, ça vous rend faible et incapable de dériver un quotient rapidement
\item Ca n'empeche pas les gens de faire des érreurs de dérivation, au contraire
\stopitemize

\item Presenter correctement pour qu'il n'y ait pas de confusion possible entre \quote{une tentative de Bluff} et \quote{ecrire sur le papier ce que vous pensez que l'on devrait trouver, votre démarche mathématique, ...} 
\item $u$ croissante ⟺ $u_n+1-u_n⩾0$ ou ${u_{n+1}\F u_n}⩾1$ lorsque $u_n$ est une suite strictement positive

\item démarche moisie des étudiants de faire de la cuisine avec les hypothèses dans tous les sens au lieu de se concentrer sur l'objectif, et d'utiliser les hypothèses en cours e route
$u_{n+1}>0$ via $u_n>0$ donc $u_n²>0$ donc ...
\stopList