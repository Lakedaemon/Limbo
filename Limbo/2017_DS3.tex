\startcomponent component_DS1
\project project_Res_Mathematica
\environment environment_Maths
\environment environment_Inferno
\xmlprocessfile{exo}{xml/Limbo_Exercices.xml}{}
\iffalse
\setupitemgroup[List][1][n,inmargin][after=,before=,left={\bf Exo },symstyle=bold,inbetween={\blank[big]}]
\setupitemgroup[List][2][a,joineup][after=,before=,inbetween={\blank[small]}]
\setupitemgroup[List][3][a,joineup][after=,before=,inbetween={\blank[small]}]
\setupitemgroup[List][4][1,joineup,nowhite]
\fi

\setupitemgroup[List][1][A,inmargin][after=,before=,left={\bf Exo },symstyle=bold,inbetween={\blank[big]}]
%\setupitemgroup[List][1][R,joineup][after=,before=,inbetween={\blank[small]}]
%\setupitemgroup[List][1][n,inmargin][after=,before=,left={\bf Exo },symstyle=bold,inbetween={\blank[big]}]
%\setupitemgroup[List][1][n,joineup][after=,before=,inbetween={\blank[small]}]
\setupitemgroup[List][2][n,joineup][after=,before=,inbetween={\blank[small]}]
\setupitemgroup[List][3][a,joineup,nowhite]
\setupitemgroup[List][4][a,joineup,nowhite]
\definecolor[myGreen][r=0.55, g=0.76, b=0.29]%
\setuppapersize[A4]
\setuppagenumbering[location=]
\setuplayout[header=0pt,footer=0pt]
\def\conseil#1{{\myGreen\it #1}}%


\starttext
\setupheads[alternative=middle]
%\showlayout
\def\gah#1{\margintext{Exercice #1}}

\iftrue
\page
\centerline{\bfb DEVOIR SURVEILLE 3}
\blank[big]

\startList


\setupitemgroup[List][1][A,inmargin][after=,before=,left={\bf Exo },symstyle=bold,inbetween={\blank[big]}]
\setupitemgroup[List][2][n,joineup][after=,before=,inbetween={\blank[small]}]
\setupitemgroup[List][3][a,joineup,nowhite]
\setupitemgroup[List][4][a,joineup,nowhite]


\item%Exo
{\it Les flèches (optionnelles) sont utilisées pour aider les étudiants à distinguer les vecteurs, dans cet exercice mélangeant suites et espaces vectoriels }\crlf
On considère l’ensemble de suites suivant :
\startformula
E = \{\vec u=(u_n)_{n∈ℕ} : ∀ n ∈ ℕ , 8u_{n+3} −12u_{n+2} + 6u_{n+1} − u_n = 0\} 
\stopformula
Soient $\vec i=(i_n)_{n∈ℕ}$, $\vec j=(j_n)_{n∈ℕ}$, $\vec k=(k_n)_{n∈ℕ}$ les suites définies par 
\startformula
\Align{
\NC i_n=\Q({1\F 2}\W)^n\qquad (n∈ℕ)\NR
\NC j_n=n\Q({1\F 2}\W)^n\qquad (n∈ℕ)\NR
\NC k_n=n^2\Q({1\F 2}\W)^n\qquad (n∈ℕ)
}
\stopformula
\startList%1
\item Prouver que $E$ forme un sous-espace vectoriel de l’espace $ℝ^ℕ$ des suites réelles
\item\startList%2
\item Prouver que $\vec i∈E$, que $\vec j∈E$ et que $\vec k∈E$ (sur 6 points)
\item Prouver que $(\vec i,\vec j,\vec k)$ est une famille libre de $E$
\stopList%2
\item Soit $\vec u=(u_n)_{n∈ℕ}$ une suite de $E$. On définit une suite $\vec w=(w_n)_{n∈ℕ}$ en posant
\startformula
w_k=2^{k+1}u_{k+1}-2^ku_k\qquad(k∈ℕ)
\stopformula
\startList%2
\item On admet qu'il existe $(λ,μ)∈ℝ^2$ tel que 
\startformula
w_k=λ+μk\qquad(k∈ℕ)
\stopformula
Pour $n∈ℕ^*$, calculer de deux manière la somme $\D S_{n-1}=∑_{k=0}^{n-1}w_k$.
\item En déduire que $\vec u$ est une combinaison linéaire des suites $\vec i$, $\vec j$, $\vec k$.  
\stopList%2
\item A l'aide de ce qui précède, déterminer une base de $E$.
\item {\bf On va démontrer ici le résultat admis au 3a).}\crlf
On considère une suite $\vec u=(u_n)_{n∈ℕ}$ de $E$ et on pose $\vec w=(w_n)_{n∈ℕ}$ avec 
\startformula
w_k = 2^{k+1}u_{k+1}-2^ku_k\qquad(k∈ℕ)
\stopformula
\startList%2
\item Prouver que $∀k∈ℕ, w_{k+2}-2w_{k+1}+w_k=0$
\item En déduire l'expression de $w_k$ en fonction de $k$ admise au 3a)
\stopList%2
\item On pose $F=\{\vec u=(u_n)_{n∈ℕ}∈E:u_0-2u_1=0\}$
\startList%2
\item Montrer que $F$ est un sous-espace vectoriel de $E$
\item Montrer que $\vec i∈F$.
\item Soit $\vec u=(u_n)_{n∈ℕ}∈F$. Justifier que $∃!(α, β, γ)∈ℝ^3$ tel que 
\startformula
u_n=αi_n+βj_n+γk_n\qquad(n∈ℕ)
\stopformula
puis montrer que $γ=-β$.
\item En déduire une base de $F$
\stopList%2
\stopList%1

\goodbreak

\item%Exo(dénombrement)
Un site internet demande de choisir un mot de passe de $8$ caractères 
parmi les $26$ lettres minuscules de l’alphabet, les $10$ chiffres et $6$ signes de ponctuation (,;!.:?).
\startList 
\item Combien y a-t-il de mots de passe possibles ?
\item Combien y a-t-il de mots de passe possibles commençant par $5$ minuscules et se terminant par un nombre avec $3$ chiffres distincts
\item Combien y a-t-il de mots de passe possibles avec au moins un chiffre ou un signe de ponctuation ? 
\item Combien y a t-il de mots de passe avec au moins deux signes de ponctuation ? 
\item Combien y a t-il de mots de passe avec exactement trois chiffres et un signe de ponctuation
\item Combien y a-t'il de mots de passe constitués uniquement de $8$ chiffres distincts placés dans l'ordre croissant
\stopList


\setupitemgroup[List][1][A,inmargin][after=,before=,left={\bf Exo },symstyle=bold,inbetween={\blank[big]}]
\setupitemgroup[List][2][A,joineup][after=,before=,inbetween={\blank[small]}]
\setupitemgroup[List][3][n,joineup,nowhite]
\setupitemgroup[List][4][a,joineup,nowhite]

\item%Exo(proba)
Soit $n ∈ ℕ^∗$. On considère trois urnes : l’urne n° $1$ contient deux boules rouges et trois boules bleues, 
l’urne n° $2$ contient une boule rouge et aucune boule bleue et l’urne n° $3$ contient une 
boule bleue et aucune boule rouge.
On choisit d’abord une de ces trois urnes au hasard avec équiprobabilité. Une fois cette urne choisie, 
on effectue dans cette urne et sans jamais en changer un nombre fini de n de tirages successifs 
d’une boule, avec remise dans cette urne. Lorsque l’urne a été choisie, les tirages sont 
considérés comme indépendants.
Pour $i∈\{1, 2, 3\}$ on note $U_i$ l’événement \quote{l’urne choisie pour les tirages est l’urne n° $i$}.
Pour tout entier naturel non nul $k$, on note $R_k$ l’événement \quote{le $k\high{ième}$ tirage a amené une boule
rouge}.
\startList%Partie
\item%A
\startList%1
\item Donner les probabilités conditionnelles $P_{U_1}(R_k)$, $P_{U_2}(R_k)$, $P_{U_3}(R_k)$ et en déduire que $P(R_k)={7\F 15}$.
\item\startList%22
\item Justifier que $P_{U_1}(R_1∩R_2∩⋯∩R_n)=\Q({2\F 5}\W)^n$.
\item Préciser les valeurs de $P_{U_2}(R_1∩R_2∩⋯∩R_n)$ et $P_{U_3}(R_1∩R_2∩⋯∩R_n)$.
\item En déduire que $P(R_1∩R_2∩⋯∩R_n)={1\F 3}\Q({2\F 5}\W)^n+{1\F 3}$
\stopList%22
\item Montrer que les événements $R_1$ et $R_2$ ne sont pas indépendants pour la probabilité $P$
\stopList%1
\item%Partie B
\startList%1
\item
Pour $2⩽k⩽n$, montrer que 
\startformula
P_{R_1∩R_2∩⋯∩R_{k-1}}(R_k)={1+\Q({2\F 5}\W)^k\F 1+\Q({2\F 5}\W)^{k-1}}
\stopformula
\item Pour $1⩽k⩽n$, on note $A_k$ l’événement \quote{une boule bleue apparaît pour la première fois 
au tirage n° $k$} et $A$ l’événement \quote{aucune boule bleue n’apparaît jamais lors des $n$ tirages}.
\startList%22
\item Calculer $P(A_1)$
\item Soit $k⩾2$. Exprimer $A_k$ en fonction des événements $R_k$.\crlf
En déduire, avec les questions précédentes, pour $k⩾2$, la valeur de $P(A k)$ en fonction de $k$.
\item Calculer la probabilité $P(A)$. 
\stopList%22
\stopList%1
\stopList%Partie
\setupitemgroup[List][1][A,inmargin][after=,before=,left={\bf Exo },symstyle=bold,inbetween={\blank[big]}]
\setupitemgroup[List][2][n,joineup][after=,before=,inbetween={\blank[small]}]
\setupitemgroup[List][3][a,joineup,nowhite]
\setupitemgroup[List][4][a,joineup,nowhite]

\goodbreak
\item%Exo
On pose $∀x∈ℝ,\quad g(x)=\e^x-x$. 
\startList%1
\item\startList%11
\item Etudier les variations de $g$
\item Pour $n⩾2$, prouver que l'équation $g(x)=n$ admet une unique solution strictement positive, que l'on notera $β_n$, et une unique solution strictement négative, que l'on notera $α_n$. 
\stopList%11
\item%2
\startList%22
\item Pour $n⩾2$, montrer que $1⩽g(\ln n)⩽n$.
\item Pour $n⩾2$, montrer que $\D g(\ln(2n))=n+g(\ln(n))-\ln 2$ puis que 
\startformula g(\ln(2n))⩾n
\stopformula
\item Pour $n⩾2$, en déduire que $\D\ln(n)⩽β_n⩽\ln(2n)$ puis que $\D\lim_{n→+∞}β_n$.
\item Déterminer $\D\lim_{n→+∞}{β_n\F\ln(n)}$
\stopList%22
\item {\bf Dans cette question, on cherche une valeur approchée de $α_2$}.\crlf
On considère la suite $u$ définie par $u_0=-1$ et 
\startformula
u_{n+1}=\e^{u_n}-2\qquad(n∈ℕ)
\stopformula
\startList
\item Calculer $g(-2)$ et $g(-1)$. En déduire que $α_2∈]-2,-1[$
\item Montrer que $α_2⩽u_k⩽-1$ pour $k∈ℕ$. 
\item On pose $∀t∈[α_2, -1]$, $h(t)=\e^t-2$. Montrer que 
\startformula
h(x)-α_2⩽{1\F \e}(x-α_2)\qquad(α_2⩽x⩽-1)
\stopformula
\item En déduire que $\D 0⩽u_{k+1}-α_2⩽{1\F\e}(u_k-α_2)$ pour $k∈ℕ$.
\item Montrer que $\D 0⩽u_k-α_2⩽\Q({1\F\e}\W)^k$ pour $k∈ℕ$
\item En déduire que la suite $u$ converge et préciser sa limite.
\stopList
\stopList





\stopList%Exos

\stoptext
\stopcomponent
\endinput
