\startcomponent component_DS1
\project project_Res_Mathematica
\environment environment_Maths
\environment environment_Inferno
\xmlprocessfile{exo}{xml/Limbo_Exercices.xml}{}
\iffalse
\setupitemgroup[List][1][R,inmargin][after=,before=,left={\bf Exo },symstyle=bold,inbetween={\blank[big]}]
\setupitemgroup[List][2][n,joineup][after=,before=,inbetween={\blank[small]}]
\setupitemgroup[List][3][a,joineup][after=,before=,inbetween={\blank[small]}]
\setupitemgroup[List][4][1,joineup,nowhite]
\fi

%\setupitemgroup[List][1][A,inmargin][after=,before=,left={\bf Exo },symstyle=bold,inbetween={\blank[big]}]
%\setupitemgroup[List][1][R,joineup][after=,before=,inbetween={\blank[small]}]
\setupitemgroup[List][1][n,inmargin][after=,before=,left={\bf Exo },symstyle=bold,inbetween={\blank[big]}]
\setupitemgroup[List][2][n,joineup][after=,before=,inbetween={\blank[small]}]
\setupitemgroup[List][3][a,joineup][after=,before=,inbetween={\blank[small]}]
\setupitemgroup[List][4][1,joineup,nowhite]
%\setupitemgroup[List][4][a,joineup,nowhite]
\definecolor[myGreen][r=0.55, g=0.76, b=0.29]%
\setuppapersize[A4]
\setuppagenumbering[location=]
\setuplayout[header=0pt,footer=0pt]
\def\conseil#1{{\myGreen\it #1}}%


\starttext
\setupheads[alternative=middle]
%\showlayout
\def\gah#1{\margintext{Exercice #1}}

\iffalse
\page
\centerline{\bfb DEVOIR MAISON 4}
\blank[big]

\startList\item%Exos
Résoudre, en discutant suivant la valeur du paramètre $m∈ℝ$, le système
\startformula
\System{
\NC x\NC-my\NC+m^2z\NC =2m\NR
\NC mx\NC-m^2y\NC+mz\NC = 2m\NR
\NC mx\NC + y\NC-m^2z\NC = 1-m
}
\stopformula

\item\startList
\item Montrer que, pour $0⩽p⩽n$, on a
\startformula
∑_{k=p}^n{k\choose p}={n+1\choose p+1}
\stopformula
\item En déduire l’expression de la somme $E_1 (n) = 1 + 2 +⋯ + n$ où $n∈ℕ$.\crlf
{\it Une démonstration n’utilisant pas le résultat précédent n'est pas recevable !}
\item\startList\item Déterminer des nombres réels $a$ et $b$ tels que 
\startformula
k^2= a{k\choose 2}+b{k\choose 1}\qquad(k⩾2)
\stopformula
\item Retrouver l’expression de la somme $E_2 (n) = 1^2 + 2^2 +⋯+ n^2$  où $n∈ℕ$.\crlf
{\it La encore, vous ne devez utiliser que les résultats précédents}
\stopList
\stopList


\item%Exos
Jojo le glacier propose $g$ variétés de glaces, $s$ variétés de sorbets et 4 variétés d’accompagnements : chantilly, nappage chocolat, nappage caramel, éclats d’amandes.
Le client choisit un certain nombre de boules de glace ou de sorbet 
(avec éventuellement plusieurs fois le même parfum) et, s’il le désire, un ou plusieurs accompagnement(s). 
Les éventuels accompagnements sont toujours disposés dans l’ordre suivant : chantilly, nappage, éclats d’amandes
(mais on n’est pas obligé de mettre les trois). 
On ne peut pas choisir deux fois le même accompagnement ni choisir simultanément les deux nappages. La commande est servie dans un cornet 
ou une coupelle, en gauffre.

\startList\item Dans toute cette question, le client choisit d’être servi dans un cornet. Par conséquent, 
on tient compte de l’ordre dans lequel les boules sont disposées dans le cornet. Ainsi, le cornet 
fraise-fraise-chocolat n’est pas le même que le cornet fraise-chocolat-fraise.
\startList\item Combien Jojo peut-il servir de cornets différents avec $b$ boules, de parfums distincts 
ou non (glace ou sorbet), sans accompagnement ?
\item Même question dans le cas où le cornet est surmonté d’un (unique) accompagnement.
\stopList
\item\startList\item Combien Jojo peut-il servir de cornets différents avec $b$ boules, de parfums distincts 
(glace ou sorbet), sans accompagnement ?
\item Même question dans le cas où le cornet est surmonté de deux accompagnements.
\stopList
\item\startList\item
Combien Jojo peut-il servir de cornets avec 4 boules (de parfums distincts ou non, sans accompagnement) dont au moins une est un sorbet ?
\item Combien Jojo peut-il servir de cornets avec 4 boules (de parfums distincts ou non,
sans accompagnement) dont au moins 3 sont des glaces ?
\stopList
\item A partir de maintenant, le client choisit d’être servi dans une coupelle. On ne tient donc plus
compte de l’ordre dans lequel les boules sont disposées dans la coupelle. Ainsi, la coupelle
vanille-passion est identique à la coupelle passion-vanille.
\startList
\item Combien Jojo peut-il servir de coupelles différentes avec b boules de parfums distincts,
sans accompagnement ?
\item Combien Jojo peut-il servir de coupelles différentes avec 2 boules de glace de parfums
distincts, 1 boule de sorbet et 3 accompagnements.
\item Chaque boule vaut 2 € et chaque accompagnement vaut 1 €. Évéanne veut acheter une
coupelle à 5 € dont tous les parfums sont distincts. Combien de coupelles différentes
peut lui proposer Jojo ?
\item Les glaces et sorbets de Jojo sont répartis en $c$ catégories contenant chacune 4 parfums. Il
y a les  classiques  : vanille, chocolat, café et pistache ; les  fruitées  : fraise, framboise,
citron et banane ; les  exotiques  : rhum-raison, noix de coco, passion et mangue ; les
 américaines  : noix de pécan, cookies, noix de macadamia et caramel-brownie ; les
 italiennes  : amaretto, amarena, straciatella et tiramisu ; etc
Valentine veut une coupelle  méga-five  : 5 boules, de parfums distincts, dont deux
choisies dans une première catégorie, deux d’une autre catégorie et une dernière d’une
troisième catégorie. Le tout agrémenté de trois accompagnements, cela va de soi ! Combien existe-t-il de coupelles  méga-five  différentes ?
\item Combien Jojo peut-il servir de coupelles différentes avec b boules, de parfums distincts
ou non, sans accompagnement ? Indication : on pourra se servir de gauffrettes pour
séparer les parfums...
\stopList
\stopList


\else
\page
\centerline{\bfb CORRECTION DU DEVOIR MAISON 4}
\blank[big]

\startList\item%Exos
En procédant aux opérations $L_2\leftarrow L_2-mL_1$ et $L_3\leftarrow L_3-mL_1$, nous obtenons que 
\startformula(S)\quad
\System{
\NC x\NC-my\NC+m^2z\NC =2m\NR
\NC mx\NC-m^2y\NC+mz\NC = 2m\NR
\NC mx\NC + y\NC-m^2z\NC = 1-m
}⟺
\System{[align={center, center, center, left}]
\NC x\NC-my\NC+m^2z\NC =2m\NR
\NC \NC \NC(m-m^3)z \NC= 2m-2m²\NR
\NC \NC (1+m²)y\NC-(m^2+m^3)z\NC = 1-m-2m²
}
\stopformula
Comme $m-m^3=m(1-m^2)=m(1-m)(1+m)$ et comme $2m-2m²=2m(1-m)$,  
il résulte de la deuxième ligne que 
\startitemize[1]\item lorsque $m\not\in\{-1,0,1\}$, le système admet une unique solution donnée par 
\startformula
\System{
\NC z={2m-2m²\F m-m^3}={2\F 1+m}\NR
\NC y={1-m-2m²+(m^2+m^3)z\F 1+m^2}={1-m-2m²+2m^2\F 1+m^2}={1-m\F 1+m^2}\NR
\NC x=2m+my-m^2z=2m+m{1-m\F 1+m^2}-{2m^2\F 1+m}
}
\stopformula
\item Le système $S$ n'a pas de solution lorsque $m=-1$ (sur la ligne 2, on obient $0=-4$. 
\item Lorsque $m=0$
\startformula(S)\quad
\System{
\NC x\NC-my\NC+m^2z\NC =2m\NR
\NC mx\NC-m^2y\NC+mz\NC = 2m\NR
\NC mx\NC + y\NC-m^2z\NC = 1-m
}⟺
\System{
\NC x\NC\NC\NC =0\NR
\NC \NC y\NC\NC = 1
}
\stopformula
A fortiori, $(S)$ possède une infinité de solution de la forme $(0,1,z)\quad (z∈ℝ)$
\item Lorsque $m=1$, 
\startformula(S)\quad
\System{
\NC x\NC-my\NC+m^2z\NC =2m\NR
\NC mx\NC-m^2y\NC+mz\NC = 2m\NR
\NC mx\NC + y\NC-m^2z\NC = 1-m
}⟺
\System{
\NC x\NC-y\NC+z\NC =2\NR
\NC \NC 2y\NC-2z\NC =-2
}⟺\System{
\NC 2x\NC\NC\NC =2\NR
\NC \NC 2y\NC-2z\NC =-2
}
\stopformula
En particulier, $(S)$ possède une infinité de solution de la forme $(1,z-1,z)\ (z∈ℝ)$
\stopitemize

\item\startList\item Fixons $p⩾0$. \crlf Pour $n⩾p$, prouvons par récurrence la proposition $\D\sc P_n:∑_{k=p}^n{k\choose p}={n+1\choose p+1}$ 
\item\startitemize[1]\item La proposition $\mc P_p$ est vraie car 
\startformula
∑_{k=p}^p{k\choose p}= {p\choose p} = 1 = {p+1\choose p+1}
\stopformula
\item Supposons $\mc P_n$ pour un entier $n⩾p$ et prouvons $\mc P_{n+1}$.\crlf
Il résulte de $\mc P_n$ et de la formule de Pascal que 
\startformula
\Align{
\NC \D ∑_{k=p}^{n+1}{k\choose p}\NC \D=∑_{k=p}^n{k\choose p} + {n+1\choose p}\qquad \text{\it D'abord, on se ramène au rang précédent}\NR
\NC\NC \D={n+1\choose p}+{n+1\choose p+1}\qquad\text{ d'après $\mc P_n$}\NR
 \NC\NC={n+2\choose p+1}\quad\text{d'après la formule de Pascal}}
\stopformula
En particulier, la proposition $\mc P_{n+1}$ est vraie
\stopitemize
En conclusion, la proposition $\mc P_n$ est vraie pour $n⩾p$.

\item Comme ${k\choose 1}={k!\F (k-1)!1!}=k\quad(k⩾1$, Le résultat précédent pour $p=1$ induit que 
\startformula
E_1(n)=∑_{k=1}^nk= ∑_{k=1}^n{k\choose 1 }= {n+1\choose 2}={(n+1)!\F (n-1)!2!}={n(n+1)\F 2}
\stopformula
En particulier, on retrouve le résultat que l'on aurait trouvé en utilisant la somme des termes d'une suite arithmétique.

\item\startList\item Pour $k⩾2$, on remarque que ${k\choose 2}= {k(k-1)\F 2}$ et ${k\choose 1}= k$. A fortiori, pour $a=2$ et $b=1$, nous obtenons que 
\startformula
a{k\choose 2}+b{k\choose 1}=2×{k(k-1)\F 2}+1×k=k^2\qquad(k⩾2)
\stopformula
\item En procédant comme précédemment, nous remarquons que 
\startformula
\Align{
\NC E_2 (n) \NC = \D ∑_{k=1}^nk^2=1+∑_{k=2}^nk^2\NR 
\NC\NC \D ={\red 1}+∑_{k=2}^n\Q(2×{k\choose 2}+{\red 1×{k\choose 1}}\W)\NR
\NC\NC \D =2 ∑_{k=2}^n{k\choose 2}  + \red{1+∑_{k=2}^n{k\choose 1}}\NR
\NC\NC \D =2{n+1\choose 3} + \red{∑_{k=1}^n{k\choose 1}}\NR
 \NC\NC \D =2{n+1\choose 3} + {\red {n+1\choose 2}}\NR
\NC\NC \D = 2×{(n-1)n(n+1)\F 6} + {\red {n(n+1)\F 2}}\NR
\NC\NC \D = {n(n+1)\F 6}×\big(2(n-1) + 3\big)={n(n+1)(2n+1)\F 6}
}
\stopformula
\stopList

\item%Exo
\startList\item 
\startList\item Constituer un cornet à $b$ boules, sans accompagnement, revient à constituer une $b$-liste (avec répétitions possibles) dont les éléments sont choisis parmi les $s+g$ parfums.
Il y en a ${s+g\choose 1}^b=(s+g)^b$. 
\item Si l'on ajoute un unique accompagnement, il faut choisir un accompagnement parmi les quatre accompagnements possibles, il y en a 
$(s+g)^b×{4\choose 1}=4(s+g)^b$
\stopList
\item\startList\item Constituer un cornet à $b$ boules de parfums distincts revient à constituer une $b$-liste sans répétition  dont les éléments sont choisis parmi les $s+g$ parfums. 
Il y en a $A_{s+g}^b$. 
\item Pour choisir deux accompagnements, on en choisit $2$ parmi les $4$ possibles et on enlève le cas particulier napage chocolat et napage café (cela fait ${4\choose 2}-1={4×3\F 2}-1=5$ possibilités. 
	La réponse est donc $5A_{s+g}^b$
\stopList
\item\startList\item Pour compter ces cornets, on prend tous les cornets et on enlève ceux qui ne contiennent pas de sorbet. Il y en a donc 
$(s+g)^4-g^4$
\item Pour compter ces cornets, on prend les cornets avec $4$ boules de glaces (il y en a $g^4$) et on ajoute les cornets avec $3$ boules de glace et $1$ boule de sorbet (il y en a ${4\choose 1}g^3s$ car il faut choisir la position d'apparition du sorbet).
En tout, il y en a $g^4+4sg^3$.
\stopList
\item \startList
\item Il peut en servir ${s+g\choose b}$ : On choisit $b$ parfums parmi les $s+g$ possibles pour constituer une coupelle
\item Il peut en servir ${g\choose 2}×{s\choose 1}×2$. On choisit $2$ parfums de glace parmi $g$ possibles, puis $1$ parfum de sorbet parmi $s$ possibles. Enfin, il n'y a que 2 façons possibles de choisir $3$ accompagnements (tout sauf napage café ou tout sauf nappage chocolat)
\item Il peut lui servir une coupelle avec $2$ boules et un accompagnement (il y en a ${s+g\choose 2}×{4\choose 1}$) ou une coupelle avec une boule et 3 accompagnements (il y en a ${s+g\choose 1}×2$). En tout, il y en a
${s+g\choose 2}×{4\choose 1}+{s+g\choose 1}×2$
\item Il y en a $A_c^3×{4\choose 2}×{4\choose 2}×{4\choose 1}×2$. On commence par choisir la première catégorie, la seconde catégorie et la troisième catégorie ($A_c^3$ choix possibles) puis on choisit les parfums dans chaque catégorie dans l'ordre, puis les accompagnements 
\item Il y a autant de coupelles possibles que de façon d'écrire un mot de $g+s+b-1$ lettres avec $b$ points et $g+s-1$ barres : les points entre les barres representent le nombre de boules choisies
pour le parfum correspondant. Il suffit de choisir la position des $b$ boules parmi ces $g+s+b-1$ cases. 
En tout, il y en a ${b+g+s-1\choose b}$
\stopList

\stopList




\fi
\stoptext
\stopcomponent
\endinput
