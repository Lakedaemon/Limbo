\startcomponent component_DS1
\project project_Res_Mathematica
\environment environment_Maths
\environment environment_Inferno
\xmlprocessfile{exo}{xml/Limbo_Exercices.xml}{}
\iffalse
\setupitemgroup[List][1][R,inmargin][after=,before=,left={\bf Exo },symstyle=bold,inbetween={\blank[big]}]
\setupitemgroup[List][2][n,joineup][after=,before=,inbetween={\blank[small]}]
\setupitemgroup[List][3][a,joineup][after=,before=,inbetween={\blank[small]}]
\setupitemgroup[List][4][1,joineup,nowhite]
\fi

%\setupitemgroup[List][1][A,inmargin][after=,before=,left={\bf Exo },symstyle=bold,inbetween={\blank[big]}]
%\setupitemgroup[List][1][R,joineup][after=,before=,inbetween={\blank[small]}]
%\setupitemgroup[List][1][n,inmargin][after=,before=,left={\bf Exo },symstyle=bold,inbetween={\%blank[big]}]
%\setupitemgroup[List][2][n,joineup][after=,before=,inbetween={\blank[small]}]
%\setupitemgroup[List][3][a,joineup][after=,before=,inbetween={\blank[small]}]
%\setupitemgroup[List][4][1,joineup,nowhite]
%\setupitemgroup[List][4][a,joineup,nowhite]
\definecolor[myGreen][r=0.55, g=0.76, b=0.29]%
\setuppapersize[A4]
\setuppagenumbering[location=]
\setuplayout[header=0pt,footer=0pt]
\def\conseil#1{{\myGreen\it #1}}%


\starttext
\setupheads[alternative=middle]
%\showlayout
\def\gah#1{\margintext{Exercice #1}}

\iftrue
\page
\centerline{\bfb DEVOIR MAISON 7}
\blank[big]

\setupitemgroup[List][1][A,joineup][after=,before=,inbetween={\blank[small]}]
\setupitemgroup[List][2][n,joineup][after=,before=,inbetween={\blank[small]}]
\setupitemgroup[List][3][a,joineup][after=,before=,inbetween={\blank[small]}]
\setupitemgroup[List][4][1,joineup,nowhite]

\centerline{\bf CORRECTION DU PROBLÈME 1 - UNE ÉLECTION EN DEUX TOURS ... OU TROIS}
\blank[big]
\startList%Partie
\item%A
 \startList%B
\item D'après la formule des probabilités totales, appliquée au système complet d'événements (non-négligeables) $\{A_1, B_1, R_1\}$, on a 
\startformula
\Align{
\NC P(A_2)\NC =P(A_1)×P_{A_1}(A_2)+P(B_1)×P_{B_1}(A_2)+P(R_1)×P_{A_1}(A_2)\NR
\NC\NC {40\F 100}×{7\F 10}+{30\F 100}×{5\F 100}+{30\F 100}×{3 \F 10}\NR
\NC P(B_2)\NC =P(A_1)×P_{A_1}(B_2)+P(B_1)×P_{B_1}(B_2)+P(R_1)×P_{A_1}(B_2)\NR
\NC\NC {40\F 100}×{2\F 10}+{30\F 100}×{85\F 100}+{30\F 100}×{4 \F 10}={800+2550+1200\F 10000}={4550\F 10000}
}
\stopformula
\item Il résulte de la formule des probabilité conditionnelle, de la formule de conditionnement et du résultat de la aquestion précédente que 
\startformula
P_{B_2}(B_1)={P_{B_1∩B_2}\F P(B_2)}={P(B_1)×P_{B_1}(B_2)\F 4550/10000}={30\F 100}×{85\F 100}×{10000\F 4550}={2550\F 4550}
\stopformula
\item Comme les événements $A_1∩A_2$, $B_1∩B_2$ et $C_1∩C_2$ sont mutuellement indépendants, il résulte de la formule de conditionnement que 
\startformula
\Align{
\NC \NC P(\text{\quote{un électeur vote de la même façon aux deux tours}})\NR
\NC =\NC P\big((A_1∩A_2)∪(B_1∩B_2)∪(C_1∩C_2)\big)\NR
\NC = \NC P(A_1∩A_2) + P(B_1∩B_2) + P(C_1∩C_2)\NR
\NC = \NC P(A_1)×P_{A_1}(A_2) + P(B_1)×P_{B_1}(B_2) + P(R_1)×P_{R_1}(R_2)\NR
\NC = \NC{40\F 100}×{7\F 10} + {30\F 100}×{85\F 100} + {30\F 100}×{3\F 10}\NR
\NC =\NC {2800+2550+900\F 10000}= {6250\F 10000}
}
\stopformula
\stopList
\item {\bf La campagne sur internet}
\startList
\item Les probabilités données par l'énoncés sont 
\startformula
P_{V_k}(V_{k+1})={6\F 10},\quad P_{V_k}(\overline{V_{k+1}})={4\F 10},\quad P_{\overline{V_k}}(V_{k+1})={8\F 10}\quad\Et\quad P_{\overline{V_k}}(\overline{V_{k+1}})={2\F 10},
\stopformula
\item Soit $k∈ℕ$. D'après la formule des probabilités totales, appliqué au système complet d'événements (non-négligeables) $\{V_k, \overline{V_k}\}$, il vient 
\startformula
\Align{
\NC P(V_{k+1}) \NC =P(V_k)×P_{V_k}(V_{k+1})+P(\overline{V_k})×P_{\overline{V_k}}(V_{k+1})\NR
\NC \NC = P(V_k)× {6\F 10} + \big(1-P(V_k)\big)×{8\F 10}\NR
\NC \NC = -{2\F 10}P(V_k)  + {8\F 10}
}
\stopformula
\item D'après l'expression obtenue précédemment, la suite définie par $u_k=P(V_k)$ pour $k⩾0$ est arithmético-géométrique. 
En posant $c={8\F 12}={2\F 3}$ l'unique solution de l'équation $c=-{2\F 10}c+{8\F 10}$, nous remarquons que la suite définie par $v_n=u_n-c$ satisfait la relation
\startformula
v_0=u_0-c=P(V_0)-{2\F 3}\quad \Et\quad v_{k+1}=-{2\F 10}v_k\quad(k⩾0)
\stopformula
La suite $v$ étant géométrique de raison $-{2\F 10}$, nous en déduisons d'une part que 
\startformula
v_k = v_0\Q(-{2\F 10}\W)^k=\Q(P(V_0)-{2\F 3}\W)×\Q(-{2\F 10}\W)^k\qquad(k⩾0)
\stopformula
et d'autre part que 
\startformula
P(V_k)=u_k=v_k+c={2\F 3}+\Q(P(V_0)-{2\F 3}\W)×\Q(-{2\F 10}\W)^k\qquad(k⩾0)
\stopformula
\item Comme la raison de la suite $v$ est dans $]-1,1[$, on a $\lim v = 0$. 
A fortiori, $\lim_{k→+∞}P(V_k)=\lim u = \lim\Q(v+{2\F 3}\W)={2\F 3}$.
\stopList
\item 
\startList
\item Il résulte de l'indépendance entre les tirs et de la formule de conditionnement que 
\startformula
\Align{[align=left]
\NC P(C_1)=P(A_1)=p_1\NR
\NC P(C_2)=P(\overline{A_1}∩B_1)=P(\overline{A_1})×P(B_1)=(1-p_1)p_2\NR
\NC P(C_3)=P(\overline{A_1}∩\overline{B_1}∩A_2)=P(\overline{A_1})×P(\overline{B_1})×P(A_2)=(1-p_1)(1-p_2)p_1
}
\stopformula
\item Soit $n ∈ℕ^*$, Selon la parité de $n$, on va avoir deux formules différentes :
\startitemize[1]
\item Lorsque $n$ est pair, posons $n=2k$ et remarquons que 
\startformula
\Align{
\NC P(C_n)\NC =P\Q(\overline{A_k}B_n∩⋂_{i=1}^{k-1}(\overline{A_i}∩\overline{B_i})\W)\NR
\NC \NC =P(\overline{A_k})×P(B_n)×∏_{i=1}^{k-1}\Q(P(\overline{A_i})×P(\overline{B_i})\W)\NR
\NC \NC = (1-p_1)^k(1-p_2)^{k-1}p_2=(1-p_1)^{n/2}(1-p_2)^{n/2-1}p_2
}
\stopformula
\item Lorsque $n$ est impair, posons $n=2k+1$ et remarquons que 
\startformula
\Align{
\NC P(C_n)\NC =P\Q({A_n}∩⋂_{i=1}^k(\overline{A_i}∩\overline{B_i})\W)\NR
\NC \NC =P(A_n)×∏_{i=1}^{k}\Q(P(\overline{A_i})×P(\overline{B_i})\W)\NR
\NC \NC = p_1(1-p_1)^k(1-p_2)^k=(1-p_1)^{n/2}(1-p_2)^{n/2}p_1
}
\stopformula
\stopitemize
\item La somme dont on doit calculer la limite est pénible à exprimer, à cause des deux cas pairs et impairs. 
Rusons pour ne faire qu'un seul cas et essayer de limiter notre souffrance, en commençant par calculer la valeur de la somme suivante
\startformula
\Align{
\NC S_N\NC \D=∑_{n=1}^{2N}P(C_n)=∑_{1⩽n⩽2N\atop n=2k}P(C_n) + ∑_{1⩽n⩽2N\atop n=2k+1}P(C_n)\NR
\NC \NC \D=∑_{k=1}^NP(C_{2k})+ ∑_{k=0}^{N-1}P(C_{2k+1})\NR
\NC \NC \D=∑_{k=1}^N(1-p_1)^k(1-p_2)^{k-1}p_2+ ∑_{k=0}^{N-1}(1-p_1)^k(1-p_2)^kp_1
}
\stopformula
Il résulte alors des formules bien connues pour ces deux sommes des termes de suites géométriques de raison $(1-p_1)(1-p_2)≠1$ que 
\startformula
S_N = {(1-p_1)p2-(1-p_1)^{N+1}(1-p_2)^Np_2\F 1-(1-p_1)(1-p_2)} + {p_1-(1-p_1)^{N+1}(1-p_2)^{N+1}p_1\F 1-(1-p_1)(1-p_2)} 
\stopformula
comme $0⩽1-p_1<1 $et $0⩽1-p_2<1$, on a  
\startformula
\lim_{n→+∞}(1-p_1)^n=0=\lim_{n+∞}(1-p_2)^n
\stopformula
de sorte que 
\startformula
\Align{
\NC \lim_{N→+∞}S_N\NC ={(1-p_1)p2\F 1-(1-p_1)(1-p_2)} + {p_1\F 1-(1-p_1)(1-p_2)}\NR
\NC\NC = {(1-p_1)p2+p_1\F 1-(1-p_1)(1-p_2)}= {p_2-p_1p2+p_1\F 1-(1-p_1-p_2+p_1p_2)}=1
}
\stopformula
Maintenant, remarquons que la suite définie par $s_N=\D∑_{n=1}^NP(C_n)$ pour $N⩾0$ est croissante (car on ajoute des nombres positifs ou nuls) et nous déduisons de la relation
$ 2\lfloor {n\F 2}\rfloor⩽n⩽2\lfloor {n\F 2}\rfloor + 2$ (en bref, on encadre $n$ entre deux nombres pairs en dessous et au dessus que l'on exprime via les parties entières) que 
\startformula
S_{\lfloor {n\F 2}\rfloor}=s_{2\lfloor {n\F 2}\rfloor}⩽s_n⩽s_{2\lfloor {n\F 2}\rfloor + 2}=S_{\lfloor {n\F 2}\rfloor+1}\qquad(n⩾0)
\stopformula
En remarquant que la suite tout à gauche et la suite tout à droite convergent vers $1$ lorsque $n$ tend vers $+∞$, il résulte alors du principe des gendarmes que
\startformula
\lim_{n→+∞}s_n=1
\stopformula
{\it Si l'un d'entre vous a une preuve {\bf juste, rigoureuse, plus simple ou plus élégante} utilisant les suites et les valeurs trouvées pour $P(C_n)$, je suis preneur.\crlf
(dire que $Ω=∪_{n=0}^∞C_n$, et que $(C_n)_{n⩾0}$ est un système complet d'événements, etc...  ne compte pas)} 
\stopList
\stopList
\blank[big]
\centerline{\bf CORRECTION DU PROBLÈME 2 - TIRAGES SANS REMISE PUIS AVEC REMISE}
\blank[big]

\startList
\item {\bf Étude du cas particulier où $b$ et $n$ valent $2$}\crlf
\startList
\item On a $P(X_0)=P(B_1)={2\F 4}={1\F 2}$ (probabilité uniforme pour le tirage d'une des deux blanches parmi les 4 boules).
De même, il résulte de la formule de conditionnement (et de la formule des probabilités composées pour $X_2$) que 
\startformula
\Align{
\NC P(X_1)=P(N_1∩B_2)=P(N_1)×P_{N_1}(B_2)={2\F 4}×{2\F 3}={1\F 3}\NR
\NC P(X_2)=P(N_1∩N_2∩B_3)=P(N_1)×P_{N_1}(N_2)×P_{N_1∩N_2}(B_3)={2\F 4}×{1\F 3}×{2\F 2}={1\F 6}
}
\stopformula
\item D'après la formule des probabilités totales appliquée au système complet d'événements (non négligeables) $\{X_0, X_1, X_2\}$, on a 
\startformula
\Align{
\NC P(Y_0)\NC =P(X_0)×P_{X_0}(Y_0)+P(X_1)×P_{X_1}(Y_0)+P(X_2)×P_{X_2}(Y_0)\NR
\NC \NC = {1\F 2}×P_{B_1}(B_2')+{1\F 3}×P_{N_1∩B_2}(B_3')+{1\F 6}×P_{N_1∩N_2∩B_3}(B_4')
\NC \NC = {1\F 2}×{1\F 3}+{1\F 3}×{1\F 2}+{1\F 6}×{1\F 1}={3\F 6}={1\F 2}
}
\stopformula
\item Soit $i∈ℕ$. D'après la formule des probabilités composées, on a
\startformula
\Align{[align={left,left}]
\NC P(X_0∩Y_i)\NC = P(X_0)×P_{X_0}(Y_i)={1\F 2}×P_{B_1}(N_2'∩⋯∩N_{1+i}'∩B_{2+i})\NR
\NC \NC = {1\F 2}×P_{B_1}(N_2')×⋯×P_{B_1∩N_2'∩⋯∩N_{1+i}'}(B_{2+i})\NR
\NC \NC = {1\F 2}×\Q({2\F 3}\W)^i×{1\F 3}\NR
\NC P(X_1∩Y_i)\NC = P(X_1)×P_{X_1}(Y_i)={1\F 3}×P_{N_1∩B_2}(N_3'∩⋯∩N_{2+i}'∩B_{3+i})\NR
\NC \NC = {1\F 3}×P_{N_1∩B_2}(N_3')×⋯×P_{N_1∩B_2∩N_3'∩⋯∩N_{2+i}'}(B_{3+i})\NR
\NC \NC = {1\F 3}×\Q({1\F 2}\W)^i×{1\F 2}\NR
\NC  P(X_2∩Y_i) \NC= P(X_2)×P_{X_2}(Y_i)={1\F 6}×P_{N_1∩N_2∩B_3}(N_4'∩⋯∩N_{3+i}'∩B_{4+i}')\NR
\NC \NC ={1\F 6}×P_{N_1∩N_2∩B_3}(B_4')={1\F 6}×{1\F 1}={1\F 6} \text{ si $i=0$}\NR
\NC \NC ={1\F 6}×0=0 \text{ si $i⩾1$}}
\stopformula
\item Soit $i⩾1$. D'après la formule des probabilités totales, appliquée au système complet d'événements $\{X_0, X_1, X_2$, 
nous obtenons que 
\startformula
\Align{
\NC P(Y_i)\NC =P(X_0∩Y_i)+P(X_1∩Y_i)+P(X_2∩Y_i)\NR
\NC \NC = {1\F 2}×\Q({2\F 3}\W)^i×{1\F 3} + {1\F 3}×\Q({1\F 2}\W)^i×{1\F 2} + 0
}
\stopformula
En particulier, nous utilisons l'égalité $P(Y_0)={1\ F2}$ et nous obtenons deux sommes de termes de suite géométriques de raison différente de $1$ en calculant  
\startformula
\Align{
\NC ∑_{i=0}^NP(Y_i)\NC = P(Y_0) + ∑_{i=1}^N\Q({1\F 2}×\Q({2\F 3}\W)^i×{1\F 3} + {1\F 3}×\Q({1\F 2}\W)^i×{1\F 2}\W)\NR
\NC\NC = {1\F 2} + {1\F 2}×{1\F 3}×∑_{i=1}^N\Q({2\F 3}\W)^i + {1\F 3}×{1\F 2}×∑_{i=1}^N\Q({1\F 2}\W)^i\NR
\NC \NC = {1\F 2} + {1\F 6}{{2\F 3}-\Q({2\F 3}\W)^{N+1}\F 1-{2\F 3}} + {1\F 6}×{{1\F 2}-\Q({1\F 2}\W)^{N+1}\F 1-{1\F 2}}
}
\stopformula
En particulier, lorsque $N$ tends vers $+∞$, nous obtenons que 
\startformula
\lim_{N→+∞}∑_{i=0}^NP(Y_i) = {1\F 2}+{1\F 6} ×{{2\F 3}\F 1-{2\F 3}} + {1\F 6}×{{1\F 2}\F 1-{1\F 2}}={1\F 2}+{2\F 6}+{1\F 6}=1
\stopformula
\stopList
\item {\bf Retour au cas général}
\startList
\item Pour $k∈ ⟦1, n⟧$, nous remarquons que $P(X_k)=P(N_1∩⋯∩N_k∩B_{k+1})$ et nous déduisons de la formule des probabilités composées que 
\startformula
\Align{
\NC P(X_k)\NC =P(N_1)×P_{N_1}(N_2)×⋯×P_{N_1∩⋯∩N_k}(B_{k+1})\NR
\NC\NC = ∏_{i=1}^k {n+1-i\F n+b +1- i}×{b\F n+b - k}\NR
\NC\NC ={n!\F (n-k)!}× {(n+b-k)!\F (n+b)!}×{b\F n+b - k}\NR 
\NC\NC ={(n+b-k-1)!\F (n-k)!}× {n!\F (n+b)!}×b\NR 
\NC\NC ={(n+b-k-1)!\F (n-k)!\red (b-1)!}× {n!{\red (b-1)!}\F (n+b)!}×b\NR 
\NC\NC ={n+b-k-1\choose b-1}× {n!{\red b!}\F (n+b)!}\NR
\NC\NC ={n+b-k-1\choose b-1}× {1\F {n+b\choose b}}\NR
\NC\NC = {{n-k+b-1\choose b-1}\F {n+b\choose b}}
}
\stopformula
\item Comme $\{X_0,⋯,X_n\}$ est un système complet d'événements de $Ω$ (puisque ces événements sont incompatibles deux à deux 
	et puisque le joueur $A$ tire forcément un nombre $k$ de noires dans $⟦0, n⟧$ avant de tirer une blanche...), nous avons

\startformula
1=P(Ω)=P(⋃_{k=1}^nX_k)=∑_{k=0}^nP(X_k)
\stopformula
Il résulte alors de la formule trouvée précedemment que 
\startformula
1= ∑_{k=0}^n{{n-k+b-1\choose b-1}\F {n+b\choose b}}
\stopformula
En plutipliant de chaque coté par ${n+b\choose b}$, il vient alors que 
\startformula
{n+b\choose b}=∑_{k=0}^n{n-k+b-1\choose b-1}
\stopformula
Par conséquent on vient de démontrer la formule suivante :
\placeformula[gah]
\startformula
∀N∈ℕ, ∀a∈ℕ,∑_{k=0}^N{k+a\choose a}={N+a+1\choose a+1}
\stopformula
\item Soit $k⩾1$, $N⩾1$ et $a∈ℕ$. Nous remarquons que 
\startformula
\Align{
\NC k{k+a\choose a}\NC =k{(k+a)!\F a!k!}={(k+a)!\F a!(k-1)!}=(a+1){(k+a)!\F (a+1)!(k-1)!}\NR
\NC\NC=(a+1){(k+a)!\F (a+1)!(k-1)!}=(a+1){k+a\choose a+1}
}
\stopformula
En particulier, nous désuisons de la relation de Chasles que  
\startformula
∑_{k=0}^N k{k+a\choose a}=0+∑_{k=1}^N k{k+a\choose a}=∑_{k=1}^N (a+1){k+a\choose a+1}=(a+1)∑_{k=1}^N{k+a\choose a+1}
\stopformula
En procédant au changement d'indice $ℓ=k-1$, il suit
\startformula
∑_{k=0}^N k{k+a\choose a}=(a+1)∑_{ℓ=0}^{N-1}{ℓ+1+a\choose a+1}
\stopformula

Et il résulte de la formule \in[gah] obtenue plus haute (pour $N'=N-1$, $k'=ℓ$ et $a'=a+1$) que 
\startformula
∑_{k=0}^N k{k+a\choose a}=(a+1){N-1+a+1+1\choose a+1+1}=(a+1){N+a+1\choose a+2}
\stopformula
\item D'après la formule précédemment obtenue pour $P(X_k)$, nous avons 
\startformula
\sum_{k=0}^n(n-k)P(X_k) = \sum_{k=0}^n(n-k){{n-k+b-1\choose b-1}\F {n+b\choose b}}= {1\F {n+b\choose b}}\sum_{k=0}^n(n-k){n-k+b-1\choose b-1}
\stopformula
En procédant au changement d'indice $k'=n-k$, il vient 
\startformula
\sum_{k=0}^n(n-k)P(X_k) = {1\F {n+b\choose b}}\sum_{k'=0}^nk'{k'+b-1\choose b-1}
\stopformula
On reconnait alors la formule obtenue à la question précédente (pour $N=n$, $k=k'$ et $a=b-1$), de sorte que 
\startformula
\sum_{k=0}^n(n-k)P(X_k) = {1\F {n+b\choose b}} (b-1+1)∑_{ℓ=0}^{n-1}{ℓ+1+b-1\choose b-1+1}
={1\F {n+b\choose b}} b∑_{ℓ=0}^{n-1}{ℓ+b\choose b}
\stopformula
Il résulte alors de la formule \in[gah] (pour $N=n-1$, $a=b$ et $k=ℓ$) que 
\startformula
∑_{k=0}^n(n-k)P(X_k) = {1\F {n+b\choose b}} b {n-1+b+1\choose b+1}={b!n!\F (n+b)!}b{(n+b)!\F (b+1)!(n-1)!}={bn\F b+1}
\stopformula
Comme $\D\sum_{k=0}^n(n-k)P(X_k)+∑_{k=0}^nkP(X_k)=∑_{k=0}^nnP(X_k)=n∑_{k=0}^nP(X_k)=n$, nous concluons alors que 
\startformula
∑_{k=0}^nkP(X_k)=n-{bn\F b+1}={n\F b+1}
\stopformula
\item Pour tout $k∈⟦0,n⟧$ et tout $i∈ℕ$, nous avons 
\startformula
\Align{
\NC P(X_k∩Y_i)\NC =P(X_k)×P_{X_k}(Y_i)=P(X_k)×P_{N_1∩⋯∩N_k∩B_{k+1}}(N_{k+2}'∩⋯∩N_{k+1+i}∩B_{k+2+i}')\NR
\NC\NC = P(X_k)×\Q({n-k\F n+b -k -1}\W)^i{b-1\F n+b-k-1}={{n-k+b-1\choose b-1}\F {n+b\F b}}×\Q({n-k\F n+b -k -1}\W)^i{b-1\F n+b-k-1}
}
\stopformula 
Pour savoir si les événements $X_k$ et $Y_i$ sont indépendants, il est nécessaire de connaître la probabilité $P(Y_i)$, que nous n'avons pas.
Son calcul a l'air très difficile (voire infaisable). Par exemple, on a 
\startformula
P(Y_i)=∑_{k=0}^nP(X_k∩Y_i)=∑_{k=0}^n{{n-k+b-1\choose b-1}\F {n+b\F b}}×\Q({n-k\F n+b -k -1}\W)^i{b-1\F n+b-k-1}
\stopformula
La somme précédente étant assez horrible, il n'est pas attendu des étudiants qu'ils sachent la calculer.
De sorte que la question semble ambigue et mal posée.\crlf 
L'auteur du sujet souhaitait certainement poser la question \quote{Les événements $X_k$ et $Y_i$ sont ils indépendants pour tout $i$ et pour tout $k$}, à laquelle il est beaucoup plus facile de répondre.
Par exemple, en remarquant que la probabilité $P_{X_k}(Y_i)=\Q({n-k\F n+b -k -1}\W)^i{b-1\F n+b-k-1}$ varie selon les valeurs de $k$ (ce n'est apparemment pas une constante indépendante de $k$) et ne peut donc pas être égale à la probabilité $P(Y_i)$.\crlf 
A fortiori, les événements $X_k$ et $Y_i$ ne peuvent pas être indépendants pour tout $i$ et tout $k$.

\stopList
\stopList








\stoptext
\stopcomponent
\endinput
