\startcomponent component_DS1
\project project_Res_Mathematica
\environment environment_Maths
\environment environment_Inferno
\xmlprocessfile{exo}{xml/Limbo_Exercices.xml}{}
\iffalse
\setupitemgroup[List][1][R,inmargin][after=,before=,left={\bf Exo },symstyle=bold,inbetween={\blank[big]}]
\setupitemgroup[List][2][n,joineup][after=,before=,inbetween={\blank[small]}]
\setupitemgroup[List][3][a,joineup][after=,before=,inbetween={\blank[small]}]
\setupitemgroup[List][4][1,joineup,nowhite]
\fi

%\setupitemgroup[List][1][A,inmargin][after=,before=,left={\bf Exo },symstyle=bold,inbetween={\blank[big]}]
%\setupitemgroup[List][1][R,joineup][after=,before=,inbetween={\blank[small]}]
%\setupitemgroup[List][1][n,inmargin][after=,before=,left={\bf Exo },symstyle=bold,inbetween={\blank[big]}]
\setupitemgroup[List][1][n,joineup][after=,before=,inbetween={\blank[small]}]
\setupitemgroup[List][2][a,joineup][after=,before=,inbetween={\blank[small]}]
\setupitemgroup[List][3][1,joineup,nowhite]
%\setupitemgroup[List][4][a,joineup,nowhite]
\definecolor[myGreen][r=0.55, g=0.76, b=0.29]%
\setuppapersize[A4]
\setuppagenumbering[location=]
\setuplayout[header=0pt,footer=0pt]
\def\conseil#1{{\myGreen\it #1}}%


\starttext
\setupheads[alternative=middle]
%\showlayout
\def\gah#1{\margintext{Exercice #1}}


\centerline{TD EV et exercices théoriques}
\blank[big]
\startList\item{\bf Démos de cours}\startList\item Soit $F$ un sous-espace vectoriel d'un espace vectoriel $E$. Prouver que $0_E=0_F$.
\item Soient $F$ et $G$ deux sous-espaces vectoriels d'un $𝕂$-espace vectoriel $E$. Prouver que $F∩G$ est un sous-espace vectoriel de $E$.
\stopList
\item Les ensembles suivants sont-ils des espaces vectoriels sur $𝕂$ (pour les opérations usuelles sur les vecteurs de $𝕂^n$) ?
\startList
\item $𝕂×𝕂$
\item $𝕂*×𝕂$
\item $\{0\}×𝕂$
\item $𝕂\ssm\{(0,1)\}$
\item $𝕂^+×𝕂$
\item $E=\{(x,y,z,t)∈𝕂^4,2x+3y-4z=0\}$
\item $E=\{(x,y,z)∈𝕂^3,x^2+y^2+z^2⩽4\}$
\item $E=\{(a,b,a+b):a∈𝕂,b∈𝕂\}$
\item $E=ℂ$
\item $E=\{(x,y,z,t)∈𝕂^4:x=z,y=2t\}$
\item $E=\{(a,b,c)∈𝕂^3:ab=0\}$
\item $E=\{(x,y,z)∈𝕂^3:x⩽y⩽z\}$
\item $E=\{(x+y,z+2t,x+z+t):(x,y,z,t)∈𝕂^4\}$
\item $E=\{(a,b,1):a∈𝕂,b∈𝕂\}$
\item $E=\{(a,b,c)∈𝕂^3:a+b-c+3=0\}$
\stopList

\item Les ensembles suivants sont-ils des sous-espaces vectoriels de l'ensemble des applications de $ℝ$ dans $ℝ$ ?
\startList
\item L'ensemble des fonctions positives sur $ℝ$
\item L'ensemble des fonctions définies sur $ℝ$ qui s'annulent en $1$
\item L'ensemble des fonctions dérivables vérifiant $f'(x)+xf(x)=0\qquad(x∈ℝ)$
\item l'ensemble des fonctions croissantes sur $ℝ$
\item l'ensemble des fonctions réelles qui tendent vers $+∞$ en $+∞$.
\item l'ensemble des fonctions réelles qui tendant vers $0$ en $+∞$
\item l'ensemble des fonctions $π$-périodiques
\stopList

\item Soient $F$ et $G$ deux sous-espaces vectoriels d'un même $𝕂$-espace vectoriel $E$. Prouver que 
\startformula
F∪G\text{ espace vectoriel}⟺(F⊂E\Ou G⊂E)
\stopformula

\stopList
\stoptext
\stopcomponent
\endinput