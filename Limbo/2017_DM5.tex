\startcomponent component_DS1
\project project_Res_Mathematica
\environment environment_Maths
\environment environment_Inferno
\xmlprocessfile{exo}{xml/Limbo_Exercices.xml}{}
\iffalse
\setupitemgroup[List][1][R,inmargin][after=,before=,left={\bf Exo },symstyle=bold,inbetween={\blank[big]}]
\setupitemgroup[List][2][n,joineup][after=,before=,inbetween={\blank[small]}]
\setupitemgroup[List][3][a,joineup][after=,before=,inbetween={\blank[small]}]
\setupitemgroup[List][4][1,joineup,nowhite]
\fi

%\setupitemgroup[List][1][A,inmargin][after=,before=,left={\bf Exo },symstyle=bold,inbetween={\blank[big]}]
%\setupitemgroup[List][1][R,joineup][after=,before=,inbetween={\blank[small]}]
\setupitemgroup[List][1][n,inmargin][after=,before=,left={\bf Exo },symstyle=bold,inbetween={\blank[big]}]
\setupitemgroup[List][2][n,joineup][after=,before=,inbetween={\blank[small]}]
\setupitemgroup[List][3][a,joineup][after=,before=,inbetween={\blank[small]}]
\setupitemgroup[List][4][1,joineup,nowhite]
%\setupitemgroup[List][4][a,joineup,nowhite]
\definecolor[myGreen][r=0.55, g=0.76, b=0.29]%
\setuppapersize[A4]
\setuppagenumbering[location=]
\setuplayout[header=0pt,footer=0pt]
\def\conseil#1{{\myGreen\it #1}}%


\starttext
\setupheads[alternative=middle]
%\showlayout
\def\gah#1{\margintext{Exercice #1}}

\iffalse
\page
\centerline{\bfb DEVOIR MAISON 5}
\blank[big]

\startList

\item%Exo1
Dans cet exercice, on montre d'abord que $∀x∈[0,{π\F2}[,\sin(x)⩽x⩽\tan(x)$\crlf
On améliore ensuite cet encadrement en prouvant que 
\startformula
∀x∈[0,{π\F 2}[, {3\sin(x)\F 2+\cos(x)}⩽x⩽{2\sin(x)+\tan(x)\F 3}
\stopformula
\startList\item%1.
On pose $∀x∈[0,{π\F 2}[, φ(x)=\sin(x)-x$ et $w(x)=\tan(x)-x$
\startList\item Montrer que $φ$ est décroissante sur $[0,{π\F 2}[$. En déduire que $∀x∈[0,{π\F 2}[, φ(x)⩽0$
\item Montrer que $w$ est monotone sur $[0,{π\F 2}[$.
\item A l'aide de ce qui précède, montrer que $\D ∀x∈[0,{π\F 2}[, \sin(x)⩽x⩽\tan(x)$
\stopList
\item%2
A l'aide des formules d'Euler, montrer que $\D ∀x∈ℝ, \Q(\sin{x\F 2}\W)^2={1-\cos(x)\F 2}$\crlf
On admet qu'on montrerait de même que $\D ∀x∈ℝ, \sin(x)=2\sin{x\F 2}\cos{x\F 2}$
\item%3
On pose $∀x∈[0,{π\F 2}[, f(x)=2x+x\cos(x)-3\sin(x)$
\startList\item Calculer $f'(x)$ pour tout $x∈[0,{π\F 2}[$
\item Montrer à l'aide de 2) que $∀x∈[0,{π\F 2}[, f'(x)=2\sin(x)\big(\tan{x\F 2}-{x\F 2}\big)$
\item A l'aide de 1.c, déterminer le signe de $f'(x)$ pour tout $x∈[0,{π\F 2}[$
\item Montrer que $\D ∀x∈[0,{π\F 2}[, {3\sin x\F 2+\cos x}⩽x$. 
\stopList
\item%4
On pose $∀x∈[0,{π\F 2}[, g(x)=2\sin(x)+\tan(x)-3x$
\startList\item On pose $∀t∈ℝ, P(t)=2t^2-t-1$. Déterminer, selon les valeurs du réel $t$, le signe de $P(t)$
\item Montrer que $\D ∀x∈[0,{π\F 2}[, g'(x)={(\cos(x)-1)P(\cos x)\F (\cos x)^2}$
\item A l'aide de ce qui précède, montrer que $∀x∈[0,{π\F 2}[, x⩽{2\sin x+\tan x\F 3}$
\stopList
\item%5
Montrer que $\D ∀x∈[0,{π\F 2}[, \tan(x)⩾{2\sin(x)+\tan(x)\F 3}$ et $\D \sin(x)⩽{3\sin(x)\F 2+\cos(x)}$\crlf
L'encadrement obtenu à l'aide de 4.c) et 3.d) est alors meilleur que celui obtenu en 1.c)
\stopList

\item%Exo2
Une puce évolue sur trois cases $A$, $B$ et $C$. A l’instant $t = 0$, la puce se situe sur la case $A$ 
puis elle se déplace de façon aléatoire selon la règle suivante :
\startitemize[1]
\item Si la puce se trouve en $A$ ou en $B$ à l’instant $k ∈ ℕ$, 
alors elle ira sur l’une des deux autres cases avec équiprobabilité à l’instant $k + 1$.
\item Si la puce se trouve en $C$ à l’instant $k∈ℕ$, alors elle y restera à l’instant $k + 1$.
\stopitemize
Pour tout $n ∈ ℕ$, on définit les événement suivants :
\startitemize[1]\item $A_n$ : \quote{La puce se trouve en $A$ à l’instant $n$},
\item $B_n$ : \quote{La puce se trouve en $B$ à l’instant $n$},
\item $C_n$ : \quote{La puce se trouve en $C$ à l’instant $n$}
\stopitemize
et on pose $a_n = P (A_n)$, $b_n = P (B_n)$ et $c_n=P(C_n)$.
\startList\item Exprimer chaque quantité $a_{n+1}$ , $b_{n+1}$ et $c_{n+1}$ en fonction de $a_n$ , $b_n$ et $c_n$.
\item Pour tout $n∈ℕ$, on pose $u_n = a_n + b_n$ et $v_n = a_n − b_n$.
\startList\item Montrer que $(u_n)$ est une suite géométrique et en déduire la valeur de $c_n$ en fonction de $n$.
\item A l’aide de la suite $(v_n)_{n∈ℕ^∗}$ déterminer les valeurs de $a_n$ et $b_n$ en fonction de $n$.
\stopList
\item Soit $n ∈ ℕ^∗$. Sachant que la puce est en $C$ à l’instant $n + 1$, 
calculer la probabilité qu’elle y ait été pour la première fois à l’instant $n$.
\stopList

\item%Exos\exo(C116}
\centerline{Calculs dans l'espace $𝕂^n$}
On considère l’espace vectoriel $E:= ℝ^4$ et les vecteurs 
$a:= (2,-1, 1, 0)$,  $b:= (-1, 2,-2, 1)$,  $c:= (-2, 2,-3, 1)$, 
$d:= (1, 4,-2, 2)$ et $e:= (8, 2, 5, 1)$.
\startList
\item Étudier si la famille $(c, d, e)$ est libre. 
Si ce n’est pas le cas, donner une relation de dépendance linéaire entre ses vecteurs.
\item Montrer que la famille $(a, b,c, d)$ est une base de $E$.
\item Déterminer les coordonnées des vecteurs $c$, $d$ et $e$ dans la base $(a,b,c,d)$.
\item Démontrer que $G:= \big\{(x, y, z, t)∈ℝ^4:x-z-t=0 \Et y-2t = 0\big\}$ est un sous-espace vectoriel de $E$. 
En donner une base.
\item Démontrer que $c∈G$. Donner une base de $G$ comportant $c$.
\item Déterminer, suivant les valeurs du réel $m$, 
si les vecteurs $u_m := (-2m, 2m, 0, 0)$ et $v_m := (m+6,-4, m+2, 0)$ forment une famille libre ou liée.
\item Les vecteurs $u_m$ et $v_m$ forment-ils une base de $G$ ?
\item Donner les équations cartésiennes vérifiées par les vecteurs $(x, y, z, t)$ qui sont dans $F := \Vect(c, d)$. 
\item Déterminer une base de $F∩G$. 
\stopList
	
\stopList%Exo


\else
\page
\centerline{\bfb CORRECTION DU DEVOIR MAISON 5}
\blank[big]


\startList

\item%Exo1
\startList
\item%1
\startList
\item La fonction $φ$ est dérivable sur $[0,{π\F 2}]$ et 
\startformula
φ'(x)=\cos(x)-1⩽0\qquad(0⩽x⩽{π\F 2})
\stopformula
En particulier, l'application $φ$ est décroissante sur $[0,{π\F 2}]$. Comme $φ(0)=\sin(0)-0=0$, nous en déduisons que 
\startformula
φ(x)⩽φ(0)=0\qquad(0⩽x⩽{π\F 2})
\stopformula
\item De même, l'application $w$ est dérivable sur $[0,{π\F 2}[$ et 
\startformula
w'(x)=\tan'(x)-1=1+\tan(x)^2-1=\tan(x)^2⩾0\qquad(0⩽x\Le {π\F 2})
\stopformula
En particulier, l'application $w$ est croissante sur $[0,{π\F 2}[$.
\item Comme $w(0)=\tan(0)-0=0$ et comme $w$ est croissante sur $[0,{π\F 2}]$, nous obtenons que 
\startformula
w(x)⩾w(0)=0\qquad(0⩽x\Le {π\F 2})
\stopformula
Comme $φ(x)⩽0⟺ \sin(x)⩽x$ et $w(x)⩾0⟺\tan(x)⩾x$, il résulte des deux inégalités obtenues précédemment que  
\startformula
\sin(x)⩽x⩽\tan(x)\qquad(0⩽x\Le {π\F 2})
\stopformula
\stopList
\item%2
Rappelons que $\sin(x)={\e^{ix}-\e^{-ix}\F 2i}$ et que $\cos(x)={\e^{ix}+\e^{-ix}\F 2}$ pour $x∈ℝ$. 
De sorte que, pour $x∈ℝ$, il résulte de la première  formule appliquée pour $x'={x\F2}$ que  
\startformula
\sin^2 {x\F 2}=\Q({\e^{ix/2}-\e^{-ix/2}\F 2i}\W)^2={\e^{ix}-2+\e^{-ix}\F -4}
\stopformula
De même, nous avons 
\startformula
{1-\cos(x)\F 2}={1\F 2}\Q(1-{\e^{ix}+\e^{-ix}\F 2}\W)={2-\e^{ix}-\e^{-ix}\F 4}=\sin^2 {x\F 2}
\stopformula
De même, nous remarquons que 
\startformula
\Align{
\NC 2\sin{x\F 2}\cos{x\F 2}\NC =2{\e^{ix/2}-\e^{-ix/2}\F 2i}{\e^{ix/2}+\e^{-ix/2}\F 2}={(\e^{ix/2})^2-(\e^{-ix/2})^2\F2i}\NR
\NC \NC ={\e^{ix}-\e^{-ix}\F 2i}=\sin(x)
}
\stopformula
\item%3
\startList
\item L'application $f$ est dérivable sur $[0,{π\F 2}[$ et nous remarquons que 
\startformula
f'(x)=2+\cos(x)-x\sin(x)-3\cos(x)=2-x\sin(x)-2\cos(x)\qquad(0⩽x\Le {π\F 2})
\stopformula
\item En particulier, nous remarquons que 
\startformula
f'(x)={\red 2-2\cos(x)}-x\sin(x)={\red 4{1-\cos(x)\F 2}} - 2{x\F 2}\sin(x)\qquad(0⩽x\Le {π\F 2})
\stopformula
En particulier, il résulte des formules établies en 2) que 
\startformula
\Align{
\NC f'(x)\NC ={\red 4\sin^2{x\F 2}} - 2{x\F 2}\sin(x)\NR
\NC\NC ={\red 2×\underbrace{2\sin{x\F 2}\cos{x\F 2}}_{\black \sin(x)}\tan{x\F 2}} - 2{x\F 2}\sin(x)\NR
\NC\NC=2\sin(x)\Q(\tan{x\F 2}-{x\F 2}\W)\qquad(0⩽x\Le {π\F 2})}
\stopformula
\item Comme la fonction $\sin$ est positive sur $[0,{π\F 2}[$, il résulte de 1.c que 
\startformula
f'(x)=2\underbrace{\sin(x)}_{⩾0}\underbrace{\Q(\tan{x\F 2}-{x\F 2}\W)}_{⩾0\text{ d'après 1.c}}⩾0
\stopformula
\item Comme la dérivée de $f$ est positive sur $[0,{π\F 2}[$, la fonction $f$ est croissante sur cet intervalle. Or $f(0)=0$ de sorte que la fonction$f$ est positive.
A fortiori, nous obtenons que 
\startformula
\Align{
\NC\NC 2x+x\cos(x)-3\sin(x)⩾0\NR 
\NC ⟺\NC 3\sin(x)⩽x(\underbrace{2+\cos(x)}_{⩾1})\NR
\NC ⟺\NC {3\sin(x)\F 2+\cos(x)}⩽x\qquad(0⩽x\Le {π\F 2})
}
\stopformula
\stopList

\item En procédant à une mise sous forme canonique, nous obtenons que 
\startformula
\Align{
\NC P(t)\NC =2t^2-t-1=2\Q(t^2-{t\F 2}-{1\F 2}\W)=2\Q((t-{1\F 4})^2-{1\F 16}-{8\F 16}\W)\NR
\NC\NC=2\Q((t-{1\F 4})^2-{9\F 16}\W)=2\Q((t-{1\F 4}-{3\F 4})(t-{1\F 4}+{3\F 4}\W)\NR
\NC\NC=2\Q((t-1)(t+{1\F 2}\W)
}
\stopformula
En particulier, comme les racines du trinôme $P$ sont $1$ et $-{1\F 2}$, on a 
\startformula
P(t)\Le 0⟺ -{1\F 2}\Le t\Le 1\Et P(t)=0 ⟺ t∈\{-{1\F 2}, 1\}
\stopformula
\item Pour $x∈[0,{π\F 2}[$, nous remarquons que 
\startformula
\Align{
\NC g'(x)\NC=2\cos(x)+\tan'(x)-3=2\cos(x)+{1\F\cos^2(x)}-3={2\cos^3(x)+1-3\cos^2(x)\F \cos^2(x)}\NR
\NC\NC={(\cos(x)-1)\Q(2\cos^2(x)-\cos(x)-1\W)\F \cos^2(x)}
={(\cos(x)-1)P\big(\cos(x)\big)\F \cos^2(x)}
}
\stopformula
\item Pour $x∈[0,{π\F 2}[$, nous remarquons que $\cos(x)∈]0,1]$, de sorte que $\cos(x)-1⩽0$ et $P(\cos x)⩽0$ et par conséquent $g'(x)⩾0$. 
Comme $g(0)=0$ et comme la fonction $g$ est croissante sur $[0,{π\F2}[$, nous obtenons que 
\startformula
g(x)⩾g(0)=0\qquad (0⩽x\Le {π\F 2})
\stopformula
Et nous concluons en remarquant que 
\startformula
g(x)⩾0⟺2\sin(x)+\tan(x)⩾3x⟺ {2\sin(x)+\tan(x)\F 3}⩾x
\stopformula
\item \startitemize[1]\item Pour établir la première inégalité, il suffit de montrer que la fonction 
\startformula
h(x)=\tan(x)- {2\sin(x)+\tan(x)\F 3}=2×{\tan(x)-\sin(x)\F 3}
\stopformula
est positive sur $[0,{π\F 2}[$, 
ce qui est résulte immédiatement de l'inégalité $\sin(x)⩽\tan(x)\qquad(0⩽x\Le {π\F 2})$ démontrée au 1.c
\item Pour établir la seconde inégalité, il suffit de montrer que l'application 
\startformula
k(x)=\sin(x)-{3\sin(x)\F 2+\cos(x)}={2\sin(x)+\sin(x)\cos(x)-3\sin(x)\F 2+\cos(x)}=\sin(x){\cos(x)-1\F 2+\cos(x)}
\stopformula
est négative sur $[0, {π\F 2}[$, 
ce qui est évident car $\sin(x)⩾0$, $\cos(x)-1⩽0$ et $2+\cos(x)⩾1$ sur cet intervalle
\stopitemize
{\it Pour établir ce genre d'inégalité, on utilise en général un tableau de variation. Ici, on a de la chance, 
il est possible d'exploiter une question précédemment traitée et une factorisation providentielle}

\stopList



\item%Exo2
\startList\item Il résulte de la formule de conditionnement que 
\startformula
\Align{
\NC a_{n+1}=P(A_{n+1})=P(B_n∩A_{n+1})=P(B_n)×P_{B_n}(A_{n+1})=b_n×{1\F 2}\NR 
\NC b_{n+1}=P(B_{n+1})=P(A_n∩B_{n+1})=P(A_n)×P_{A_n}(B_{n+1})=a_n×{1\F 2}
}
\stopformula
De même, il résulte de la formule des probabilités totales appliquée au système complet d'événement $\{A_n, B_n, C_n\}$ que 
\startformula
\Align{
\NC c_{n+1}\NC =P(C_{n+1} =P(A_n∩C_{n+1})+P(B_n∩C_{n+1})+P(C_n∩C_{n+1})\NR
\NC\NC =P(A_n)×P_{A_n}(C_{n+1})+P(B_n)×P_{B_n}(C_{n+1})+P(C_n)×P_{C_n}(C_{n+1})\NR
\NC\NC =a_n×{1\F 2}+b_n×{1\F 2}+c_n×1
}
\stopformula
\item Pour tout $n∈ℕ$, on pose $u_n = a_n + b_n$ et $v_n = a_n − b_n$.
\startList\item D'après les relations précédemment établies, nous avons 
\startformula
u_{n+1}=a_{n+1}+b_{n+1}=b_n×{1\F 2}+a_n×{1\F 2}={1\F 2}(a_n+b_n)={1\F 2}u_n
\stopformula
En particulier, la suite $u$ est géométrique de raison ${1\F 2}$. 
Comme $u_0=a_0+b_0=P(A_0)+P(B_0)=1+0=1$, nous remarquons que 
\startformula
u_n=u_0\Q({1\F 2}\W)^n={1\F 2^n}\qquad (n∈ℕ)
\stopformula
Comme $A_n∪B_n$ est le complémentaire de $C_n$ et comme les événements $A_n$ et $B_n$ sont incompatibles, il vient
\startformula
c_n=P(C_n)=1-P(A_n∪B_n)=1-(P(A_n)+P(B_n))=1-(a_n+b_n)=1-u_n=1-{1\F 2^n}\qquad(n∈ℕ)
\stopformula
\item De même, la suite $v$ est géométrique de raison $-{1\F 2}$ car
\startformula
v_{n+1}=a_{n+1}-b_{n+1}=b_n×{1\F 2}-a_n×{1\F 2}=-{1\F 2}(a_n-b_n)=-{1\F 2}v_n
\stopformula
Comme $v_0=a_0-b_0=P(A_0)-P(B_0)=1-0=1$, il vient
\startformula
v_n=v_0\Q(-{1\F 2}\W)^n=\Q(-{1\F 2}\W)^n\qquad(n∈ℕ)
\stopformula
En remarquant que $a_n=(u_n+v_n)/2$ et que $b_n=(u_n-v_n)/2$ nous en déduisons alors que 
\startformula
\Align{
\NC a_n={1\F 2}\Q({1\F 2}\W)^n+{1\F 2}\Q(-{1\F 2}\W)^n=\Q({1\F 2}\W)^{n+1}-\Q(-{1\F 2}\W)^{n+1}\NR
\NC b_n={1\F 2}\Q({1\F 2}\W)^n-{1\F 2}\Q(-{1\F 2}\W)^n=\Q({1\F 2}\W)^{n+1}\Q(-{1\F 2}\W)^{n+1}\NR
}\qquad(n∈ℕ)
\stopformula
\stopList
\item Soit $n ∈ ℕ^∗$. Nous devons calculer la probabilité conditionnelle
\startformula
\Align{
\NC P_{C_{n+1}}((A_{n-1}∪B_{n-1})∩C_n)\NC =P_{C_{n+1}}(A_{n-1}∩C_n)+P_{C_{n+1}}(B_{n-1}∩C_n)\NR
\NC\NC = {P(A_{n-1}∩C_n∩C_{n+1})\F P(C_{n+1})}+{P(B_{n-1}∩C_n∩C_{n+1})\F P(C_{n+1})}\NR
\NC\NC = {P(A_{n-1}∩C_n)\F P(C_{n+1})}+{P(B_{n-1}∩C_n)\F P(C_{n+1})}\NR
\NC\NC = {P(A_{n-1})×P_{A_{n-1}}(C_n)\F P(C_{n+1})}+{P(B_{n-1})×P_{B_{n-1}}(C_n)\F P(C_{n+1})}\NR
\NC\NC = {a_{n-1}{1\F 2}\F c_{n+1}}+{b_{n-1}×{1\F 2}\F c_{n+1}}={{1\F 2}(a_{n-1}+b_{n-1})\F c_{n+1}}\NR
\NC\NC ={{1\F 2}u_{n-1}\F c_{n+1}}={u_n\F c_{n+1}}={{1\F 2^n}\F 1-{1\F 2^{n+1}}}={2\F 2^{n+1}-1}}
\stopformula

\stopList




\item%Exos\exo(C116}
\centerline{Calculs dans l'espace $𝕂^n$}
On considère l’espace vectoriel $E:= ℝ^4$ et les vecteurs 
$a:= (2,-1, 1, 0)$,  $b:= (-1, 2,-2, 1)$,  $c:= (-2, 2,-3, 1)$, 
$d:= (1, 4,-2, 2)$ et $e:= (8, 2, 5, 1)$.
\startList
\item  La famille $(c,d,e)$ est liée car 
\startformula
-3c+2d-e=-3 (-2, 2,-3, 1)+2(1, 4,-2, 2)-(8, 2, 5, 1)=(0,0,0,0)
\stopformula
J'avais mal lu la consigne et j'ai bêtement prouvé que la famille $(a,b,c)$ est libre ci dessous 
\crlf{\it conclusion : bien lire les questions posées}\crlf
Montrons que la famille $(c, d, e)$ est libre. Soient $(x,y,z)∈ℝ^3$ 
tels que $0=xa+yb+zc=(2x-y-2z,-x+2y+2z,x-2y-3z,y+z)$. 
En remarquant que 
\startformula
\System{
\NC 2x\NC -y\NC +2z\NC =0\NR
\NC -x\NC+2y\NC+2z\NC=0\NR
\NC x\NC-2y\NC-3z\NC=0\NR
\NC \NC \boxed{y}\NC +z\NC =0
} ⟹ \System{
\NC 2x\NC \NC +3z\NC =0\NR
\NC -x\NC\NC\NC=0\NR
\NC x\NC\NC-z\NC=0\NR
\NC \NC \boxed{y}\NC +z\NC =0
}⟹\System{
\NC x=0\NR
\NC y=0\NR
\NC z=0
}
\stopformula
A fortiori, la famille $(a,b,c)$ est libre \crlf
{\it Plus tard, avec les matrices et le rang, nous pourrons traiter cette question plus simplement et rapidement}
\item Pour commencer, montrons que la famille $(a,b,c,d)$ est génératrice.
Soient $(X,Y,Z,T)∈ℝ^4$ et $(x,y,z,t)∈ℝ^4$. Alors, nous remarquons que 
\startformula
\Align{[align={left, left}]
\NC \NC xa+yb+zc+td =(X,Y,Z,T)\NR 
\NC ⟺ \NC (2x-y-2z+t,-x+2y+2z+4t,x-2y-3z-2t,y+z+2t)=(X,Y,Z,T)\NR
\NC ⟺ \NC \System{
\NC 2x\NC -y\NC -2z\NC+t\NC =X\NR
\NC -x\NC+2y\NC+2z\NC+4t\NC =Y\NR
\NC x\NC-2y\NC-3z\NC-2t\NC=Z\NR
\NC \NC \boxed{y}\NC +z\NC+2t\NC =T
}\NR%\NR
\NC ⟺\NC\System{[align={left, left, left, left, left}]
\NC 2x\NC \NC -z\NC+3t\NC =X+T\NR
\NC \boxed{-x}\NC\NC\NC\NC =Y-2T\NR
\NC x\NC\NC-z\NC+2t\NC=Z+2T\NR
\NC \NC \boxed{y}\NC +z\NC+2t\NC =T
}\NR%\NR
\NC ⟺\NC\System{[align={left, left, left, left, left}]
\NC \NC \NC -z\NC+3t\NC =X+2Y-3T\NR
\NC \boxed{-x}\NC\NC\NC\NC =Y-2T\NR
\NC \NC\NC\boxed{-z}\NC+2t\NC=Y+Z\NR
\NC \NC \boxed{y}\NC +z\NC+2t\NC =T
}\NR\NR
\NC ⟺\NC\System{[align={left, left, left, left, left}]
\NC \NC \NC \NC\boxed{t}\NC =X+Y-Z-3T\NR
\NC \boxed{-x}\NC\NC\NC\NC =Y-2T\NR
\NC \NC\NC\boxed{-z}\NC+2t\NC=Y+Z\NR
\NC \NC \boxed{y}\NC \NC+4t\NC =Y+Z+T
}\NR\NR
\NC ⟺\NC\System{[align={left, left, left, left, left}]
\NC \NC \NC \NC\boxed{t}\NC =X+Y-Z-3T\NR
\NC \boxed{-x}\NC\NC\NC\NC =Y-2T\NR
\NC \NC\NC\boxed{-z}\NC\NC=-2X-Y+3Z+6T\NR
\NC \NC \boxed{y}\NC \NC\NC =-4X-3Y+5Z+13T
}
}
\stopformula
En particulier, nous remarquons que ce système possède pour chaque valeur de $(X,Y,Z,T)$ une unique solution, 
ce qui prouve que la famille $(a,b,c,d)$ est génératrice. En remarquant que ce système n'admet qu'une seule solution (la solution $x=y=z=t=0$) lorsque $X=Y=Z=T=0$, 
nous en déduisons également que la famille $(a,b,c,d)$ est libre. En particulier, la famille $(a,b,c,d)$ est une base.\crlf
{\it Et dans quelques semaines, un simple calcul de rang pour traiter cette question...}

\item Nous avons évidemment $c=0⋅a+0⋅b+1⋅c+0⋅d$, De sorte que les coordonnées de $c$ dans la base $(a,b,c,d)$ sont $\Matrix{\NC 0\NR\NC 0\NR\NC 1\NR\NC 0}$. 
De même, les coordonnées de $d$ sont $\Matrix{\NC 0\NR\NC 0\NR\NC 0\NR\NC 1}$.
Enfin, en prenant $(X,Y,Z,T)=e$ dans la résolution de système précédent, nous obtenons $x=0$, $y=0$; $z=-3$ et $t=2$ et nous remarquons effectivement que 
\startformula
0⋅(2,-1, 1, 0)+0⋅(-1, 2,-2, 1)-3⋅(-2, 2,-3, 1)+2⋅(1, 4,-2, 2)=(8, 2, 5, 1).
\stopformula
En particulier, les coordonnées de $e$ dans la base $(a,b,c,d)$ sont $\Matrix{\NC 0\NR\NC 0\NR\NC -3\NR\NC 2}$.
\item Nous remarquons que 
\startformula
\Align{
\NC G \NC=\{(x,y,z,t)∈ℝ^4:x=z+T\Et y=2t\}=\{(z+t,2t,z,t):(z,t)∈ℝ^2\}\NR
\NC\NC =\{z(1,0,1,0)+t(1,2,0,1):(z,t)∈ℝ^2\}=\Vect\big((1,0,1,0),(1,2,0,1)\big)
}
\stopformula
En particulier, $G$ est un sous-espace vectoriel de $ℝ^4$ engendré par la famille $\{(1,0,1,0),(1,2,0,1)\}$ 
qui est libre car elle comporte deux vecteurs non-colinéaires
En particulier, c'est une base de $G$.
{\it D'une pierre trois coups...}

\item Pour $(x,y,z,t)= c=(-2,2,-3,1)∈ℝ^4$, nous remarquons que $x-z-t=-2+3-1=0$ et $y-2t=2-2=0$. 
En particulier $c∈G$. La famille $\mc F=\{c, (1,0,1,0)\}$ constitue une famille libre de vecteurs de $G$ (deux vecteurs non colinéaires) 
Et comme $(1,2,0,1)=c+3⋅(1,0,1,0)$, nous observons que $(1,2,0,1)∈\Vect(\mc F)$ et a fortiori que 
\startformula
G=\Vect\big((1,0,1,0),(1,2,0,1)\big)⊂\Vect(\mc F)⊂G
\stopformula
En particulier, nous avons $\Vect(\mc F)=G$ de sorte que la famille libre $\mc F$ engendre $G$. C'est donc une base de $G$.
{\it Plus tard, cela sera plus facile avec la dimension et le théorème de la base incomplète...}

\item Soit $m∈ℝ$. Les vecteurs $u_m$ et $v_m$ forment une famille libre \ssi le système suivant possède une unique solution (la solution nulle)
\startformula
\Align{
\NC xu_m+yv_m=0\NC ⟺ \NC x(-2m, 2m, 0, 0)+y(m+6,-4, m+2, 0)=0
\NR \NC\NC⟺\NC\System{
\NC -2mx\NC+(m+6)y\NC=0\NR
\NC 2mx\NC -4y\NC = 0\NR
\NC \NC y(m+2)\NC = 0
}}
\stopformula
\startitemize[1]
\item Si $m≠-2$, $y=0$ de sorte que si $m≠0$, $x=0$ et la famile est libre.
\item Si $m=-2$, le système est équivalent à l'equation $4x+4y=0$ qui a une infinité de solution $(x,y)$, la famille est liée
\item Si $m=0$, le système est équivalent à l'équation $y=0$ qui a une infinité de solution $(x,y)$, la famille est liée
\stopitemize
En conclusion, la famille est libre \ssi $m\not∈\{-2,0\}$. {\it Question facile via les matrices et un calcul de rang}
\item Les vecteurs $u_m$ et $v_m$ ne peuvent pas constituer une base de $G$ parce que $v_m$ n'appartient pas à $G$ (sauf si $m=0$ mais dans ce cas, la famille est liée)
\item $F=\Vect(c,d)$ de sorte que 
\startformula
\startAlign
\NC(x,y,z,t)∈F\NC ⟺ ∃ α,β∈ℝ:(x,y,z,t)=αc+βd⟺ ∃ α,β∈ℝ:\startSystem
\NC x = 2α+β\NR
\NC y= 2α+4β\NR
\NC z=3α+2β\NR
\NC t=α+2β\stopSystem\NR
\NC\NC⟺ ∃ α,β∈ℝ:\startSystem
\NC x-2t = -3β\NR
\NC y -2t= 0\NR
\NC z-3t=-4β\NR
\NC t=α+2β\stopSystem⟺ ∃ α,β∈ℝ:\startSystem
\NC 4x-8t = -12β\NR
\NC y -2t= 0\NR
\NC z-3t=-4β\NR
\NC 2t=2α+4β\stopSystem\NR
\NC\NC⟺ ∃ α,β∈ℝ:\startSystem
\NC 4x-2t-3z = 0\NR
\NC y -2t= 0\NR
\NC z-3t=-4β\NR
\NC z-t=2α\stopSystem⟺\startSystem
\NC 4x-2t-3z = 0\NR
\NC y -2t= 0\stopSystem
\stopAlign
\stopformula
Les équations cartésiennes que vérifient les vecteurs $(x,y,z,t)$ de $F$ sont donc $y-2t=0$ et $4x-3z-2t=0$. 
{\it C'est une question (très) difficile. Le procédé qui a été utilisé ici est une élimination : 
on a utilisé deux des quatres équations pour éliminer les variables $α$ et $β$ et obtenir 2 équations ne faisant intervenir que $x$,$y$,$z$ et $t$.} 
\item Les vecteurs de $F∩G$ sont les vecteurs $(x,y,z,t)$ vérifiant le système d'équations 
\startformula 
\startSystem
\NC x-z-t=0\NR 
\NC y-2t = 0\NR
\NR 4x-3z-2t=0
\stopSystem⟺\startSystem
\NC x-z-t=0\NR 
\NC y-2t = 0\NR
\NR z+2t=0
\stopSystem⟺\startSystem
\NC x=-t\NR 
\NC y= 2t\NR
\NR z=-2t
\stopSystem
\stopformula
En particulier, nous remarquons que $F∩G=\Vect((-1,2,-2,1))$.
La famille $\mc F=\{(-1,2,-2,1)\}$ est une famille génératrice de $F∩G$ d'après la question précédente. 
Comme c'est une famille comportant un unique vecteur non nul, elle est aussi libre de sorte que $\mc F$ est une base de $F∩G$

\stopList%Exo




\fi

\iffalse


\item%Exo2
A chaque journée de cours, Mademoiselle J. mange soit à la cantine de son lycée soit 
n’a pas le temps de manger en raison d’un temps d’attente trop long devant cette cantine. Précisément, 
quand le professeur lui permet de sortir de classe avant la sonnerie, 
elle parvient à manger avec probabilité $2/5$. Lorsque le professeur lui impose de sortir à la sonnerie, 
elle n’a alors qu’une chance sur cinq de manger. Malgré son programme fort chargé, 
le professeur compatit au pauvre sort de Mademoiselle J. et la laisse donc sortir en avance avec probabilité $2/3$. 
On suppose que, d’un jour à l’autre, 
les décisions du professeur de laisser ou non sortir Mademoiselle J. en avance sont indépendantes. On introduit les événements suivants :
\startitemize[1]\item Pour $n∈ℕ^*$, on note $M_n$ : \quote{Mademoiselle J. parvient à manger au $n\high{ème}$ jour de cours}
\item Pour $n∈ℕ^*$, on note $E_n$ : \quote{Le professeur laisse sortir Mademoiselle J. en avance au $n\high{ème}$  jour de cours}
\item Pour $n∈ℕ^*$, on note $A_n$ : \quote{Mademoiselle elle n’a pas mangé deux fois de suite pour la première fois 
aux $(n − 1)\high{ème}$ et $n\high{ème}$ jours de cours}
\stopitemize
\startList\item%1
Dans cette question, on considère un jour de cours $n∈ℕ^∗$ quelconque.
\startList\item Montrer que $P(M_n)={1\F 3}$
\item On constate que Mademoiselle J. n’a pas mangé, 
quelle est la probabilité que l’enseignant ne l’ait pas laissé sortir à l’avance.
\stopList
\item%2
Pour tout $n∈ℕ^*$, on pose $u_n=P(A_n)$. 
\startList
\item Calculer $u_1$ et $u_2$.
\item A l’aide de la formule des probabilités totales établir 
\startformula ∀n∈ℕ^*, u_{n+2}={1\F 3}u_{n+1}+{2\F 9}u_n\stopformula
{\it Pour le choix du système complet d’événement, on pourra dessiner un arbre représentant ce qui se passe aux $n\high{ème}$ et $n\high{ième}$ jours }
\item En déduire l'expression de $u_n$ en fonction de $n$.
\stopList
\item%3.
Pour tout entier naturel $n$ non nul, on note $S_n=∑_{k=1}^nu_k$. 
\startList
\item Montrer que $S_n$ représente la probabilité d'un certain événement, dont on donnera un libellé explicite.
\item Montrer que la suite $(S_n)_{n∈ℕ^*}$ converge, déterminer sa limite et interpréter le résultat.
\stopList

\stopList

\item%Exo2
A chaque journée de cours, Mademoiselle J. mange soit à la cantine de son lycée soit 
n’a pas le temps de manger en raison d’un temps d’attente trop long devant cette cantine. Précisément, 
quand le professeur lui permet de sortir de classe avant la sonnerie, 
elle parvient à manger avec probabilité $2/5$. Lorsque le professeur lui impose de sortir à la sonnerie, 
elle n’a alors qu’une chance sur cinq de manger. Malgré son programme fort chargé, 
le professeur compatit au pauvre sort de Mademoiselle J. et la laisse donc sortir en avance avec probabilité $2/3$. 
On suppose que, d’un jour à l’autre, 
les décisions du professeur de laisser ou non sortir Mademoiselle J. en avance sont indépendantes. On introduit les événements suivants :
\startitemize[1]\item Pour $n∈ℕ^*$, on note $M_n$ : \quote{Mademoiselle J. parvient à manger au $n\high{ème}$ jour de cours}
\item Pour $n∈ℕ^*$, on note $E_n$ : \quote{Le professeur laisse sortir Mademoiselle J. en avance au $n\high{ème}$  jour de cours}
\item Pour $n∈ℕ^*$, on note $A_n$ : \quote{Mademoiselle elle n’a pas mangé deux fois de suite pour la première fois 
aux $(n − 1)\high{ème}$ et $n\high{ème}$ jours de cours}
\stopitemize
\startList\item%1
Dans cette question, on considère un jour de cours $n∈ℕ^∗$ quelconque.
\startList\item 
Il résulte de la formule des probabilités totales appliquée au système complet d'événement $\{E_n,\overline{E_n}\}$ que 
\startformula
P(M_n)=P(E_n)×P_{E_n}(M_n)+P(\overline{E_n})×P_{\overline{E_n}}(M_n)={2\F 3}×{2\F 5}+{1\F 3}×{1\F 5}={4+1\F 3×5}={1\F 3}
\stopformula
\item D'après la formule de Bayes (appliqué au système  complet d'événement $\{E_n,\overline{E_n}\}$), on a 
\startformula
P_{\overline{M_n}}(\overline{E_n})={P(\overline{E_n}∩\overline{M_n})\F P(\overline{M_n}}={P(\overline{E_n})×P_{\overline{E_n}}(\overline{M_n})\F P(\overline{M_n})}={{1\F 3}×{4\F 5}\F {2\F 3}}={2\F 5}
\stopformula
\stopList
\item%2
Pour tout $n∈ℕ^*$, on pose $u_n=P(A_n)$. 
\startList
\item {il est bizarre de calculer $u_1$ (erreur d'énoncé ?) Cela a plus de sens de calculer $u_2$ et $u_3$}. Il résulte de l'indépendance mutuelle des événements $M_1$, $M_2$ et $M_3$ que 
\startformula
\Align{
\NC u_2=P(\overline{M_1}∩\overline{M_2})=P(\overline{M_1})×P(\overline{M_2})=\Q({2\F 3}\W)^2\NR
\NC u_3=P(M_1∩\overline{M_2}∩\overline{M_3})=P(M_1)×P(\overline{M_2})×P(\overline{M_3})={1\F 3}×\Q({2\F 3}\W)^2
}
\stopformula
\item {\it conseil : faire apparaitre un arbre représentnat ce qui ce passe au $n\high{ième}$, au $(n+1)\high{ième}$ et au $(n+2)\high{ième}$ jour}\crlf
D'après la formule des probabilités totales appliquée au système complet d'événements $\{A_n, \overline{A_n}\}$, on a 
\startformula
P(A_{n+2})=P(\overline{M_{n+2}}∩\overline{M_{n+1}}∩\overline{A_n}})=
\stopformula
\startformula ∀n∈ℕ^*, u_{n+2}={1\F 3}u_{n+1}+{2\F 9}u_n\stopformula
{\it Pour le choix du système complet d’événement, on pourra dessiner un arbre représentant ce qui se passe aux $n\high{ème}$ et $n\high{ième}$ jours }
\item En déduire l'expression de $u_n$ en fonction de $n$.
\stopList
\item%3.
Pour tout entier naturel $n$ non nul, on note $S_n=∑_{k=1}^nu_k$. 
\startList
\item Montrer que $S_n$ représente la probabilité d'un certain événement, dont on donnera un libellé explicite.
\item Montrer que la suite $(S_n)_{n∈ℕ^*}$ converge, déterminer sa limite et interpréter le résultat.
\stopList

\stopList
\fi
\stoptext
\stopcomponent
\endinput
