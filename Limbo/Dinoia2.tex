\startcomponent Lettre
\environment environment_Lettre

\starttext
\defineparagraphs[rightpar][n=2] 
\setupparagraphs [rightpar][1][width=.5\textwidth] 
\setupparagraphs [rightpar][2][width=.5\textwidth,align=flushright]

\defineparagraphs[leftpar][n=2] 
\setupparagraphs [leftpar][1][width=.5\textwidth] 
\setupparagraphs [leftpar][2][width=.5\textwidth,align=flushright]
\setuppagenumbering[location=]

\startleftpar\leftpar
Longeville-lès-Metz, le 11/07/2015
\stopleftpar

\startleftpar % Two columns, please
Olivier Binda\crlf
16 rue des mésanges,\crlf 
57050 Longeville-lès-Metz\crlf
tel : 07 70 00 27 39\crlf
mail : olivier.binda@wanadoo.fr

%\blank[big]
%N° SIRET : 49376272800027\crlf
%N° TVA intracommunautaire : FR32493762728
\leftpar
%\hfil {\tfb\bf FACTURE n°29}\hfil \blank[big]
Mme Collignon Dinoia Thérese\crlf
3 rue St Gorgon\crlf
57160 Lessy\crlf
\stopleftpar
\vfil\vfil

\hfil Madame, \hfil
\blank[big]
Nous avons bien reçu votre lettre recommandée avec accusé réception,  ci-jointe en copie, à laquelle nous allons répondre point par point.
\blank[medium]
I) Nous ne vous avons pas demandé de venir chez nous, nous vous avons fait savoir par SMS (joint) que :\crlf
a) Le robinet de l'évier fuyait\crlf
b) Nous souhaitions une copie des notices d'emploi du four et du lave-vaisselle, qui ne nous ont pas été remises à notre entrée dans l'appartement.\crlf
\blank[medium]

II) En tant que locataires, nous respectons les conditions générales du contrat car : \blank[small]
7g) nous payons par virement bancaire permanent en temps et en heure notre loyer et les avances de charges.\crlf
7h) nous usons des locaux loués paisiblement, en respectant leur destination (habitation).\crlf
7i) nous répondons et comptons répondre des dégradations survenues pendant le cours du bail à hauteur de nos obligations.\crlf
7k) nous n'avons pas fait de changement de distribution ou de transformation dans l'appartement.\crlf
7l) nous ne sous-louons ou ne cédons aucune partie de l'appartement.\crlf
7m) nous vous avons informée de la naissance de notre fille.\crlf
7n) nous avons laissé votre homme de confiance effectuer les travaux que vous avez demandés.\crlf
7o) nous comptons bien respecter le droit de visite en cas de vente ou de nouvelle location de 2h/j les jours ouvrables (à noter que cette clause n'implique pas pour le bailleur le droit d'entrer chez les locataires).\crlf
7q) nous avons souscrit auprès de la MAAF une assurance multi-risques habitation contre les risques locatifs, l'incendie, les explosions, les dégats des eaux, incluant une responsabilité civile, 
dont vous avez reçu une attestation à notre entrée dans l'appartement.
\blank[big]

III) Le désordre et l'encombrement sont des notions subjectives et transitoires et ne constituent pas des causes de rupture de bail. 
\blank[big]

IV) Une modification du papier peint est un aménagement, qui n'entraine pas de transformation du local. 
De ce fait, ce n'est pas non plus une cause possible de rupture de bail.
\blank[big]

V) Vous n'êtes pas sans savoir que le bailleur a également des obligations que vous n'avez pas respectées en venant dans l'appartement loué sans y être invitée, pour hurler, critiquer, insulter, aggresser verbalement
et menacer d'expulsion, notamment: 
\blank[medium]
6c) assurer au locataire une jouissance paisible et la garantie des vices ou défauts de nature à y faire obstacle. \crlf
6d) maintenir les locaux en état de servir à l'usage prévu par le contrat en effectuant les réparations autres que locatives (nous attendons toujours les clefs du garage).\crlf
6e) ne pas s'opposer aux aménagements réalisés par le locataire dès lors qu'ils n'entraînent pas une transformation du local.
\blank[medium]

Notez également qu'un propriétaire qui ne respecterait pas les procédures et expulserait de force l'occupant des lieux commet le délit d'expulsion illégale. 
Créé par la loi Alur de 2014, ce délit est prévu par l'article 226-4-2 du Code pénal 
qui le définit comme le fait de forcer un tiers à quitter le lieu qu'il habite sans avoir obtenu le concours de l'Etat 
dans les conditions prévues à l'article L. 153-1 du code des procédures civiles d'exécution, à l'aide de manoeuvres, menaces, 
voies de fait ou contraintes. L'auteur d'une expulsion illégale encourt une peine de 3 ans de prison et 30 000 euros d'amende.
\blank[big]


En conclusion,  comme nous satisfaisons les conditions générales du bail, nous contestons ce congé non-recevable. 
\blank[medium]
Notez qu'en cas de contestation, d'après la loi du 6/7/89, un juge peut, même d'office, vérifier la réalité du motif du congé et le respect des obligations prévues. 
Il peut notamment déclarer non valide le congé si la non-reconduction du bail n'apparaît pas justifiée par des éléments sérieux et légitimes.
\blank[medium]
Ainsi, votre lettre recommandée ne constituant pas une demande valable de congé, nous entendons continuer à jouir paisiblement de notre location, 
en respectant chacun nos obligations légales. Par ailleurs, au regard de la clause 6f du bail conjointement signé, nous vous saurons gré de bien vouloir nous faire parvenir gratuitement et dans les plus brefs délais 
une quittance de loyer, pour chaque mois de location réglé depuis la signature du bail, jusqu'à notre départ.
\blank[big]
Recevez nos salutations, 


\stoptext
\stopcomponent
