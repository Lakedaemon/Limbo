\startcomponent Lettre
\environment environment_Lettre

\starttext
\defineparagraphs[rightpar][n=2] 
\setupparagraphs [rightpar][1][width=.5\textwidth] 
\setupparagraphs [rightpar][2][width=.5\textwidth,align=flushright]

\defineparagraphs[leftpar][n=2] 
\setupparagraphs [leftpar][1][width=.5\textwidth] 
\setupparagraphs [leftpar][2][width=.5\textwidth,align=flushright]
\setuppagenumbering[location=]

\startleftpar\leftpar
Longeville-lès-Metz, le 27/02/2015
\stopleftpar

\startleftpar % Two columns, please
Olivier Binda\crlf
16 rue des mésanges,\crlf 
57050 Longeville-lès-Metz\crlf
tel : 07 70 00 27 39\crlf
mail : olivier.binda@wanadoo.fr

%\blank[big]
%N° SIRET : 49376272800027\crlf
%N° TVA intracommunautaire : FR32493762728
\leftpar
%\hfil {\tfb\bf FACTURE n°29}\hfil \blank[big]
M le gérant\crlf
Pic location ADA\crlf
1-3 avenue Miribel\crlf
55430 Belleville
\stopleftpar




\vfil\vfil





Madame, Monsieur,
\blank[big]
En date du 20/02/2015, vous m'avez fait parvenir 3 documents concernant le sinistre du 25/08/2014 sur le véhicule immatriculé BB-870-BS (Mercedes Sprinter).
Au vu de ces documents m'apparaissent 2 raisons de contester la somme de 15900€ TTC réclamée dans votre courrier du 06/02/2015.
\blank[big]
D'une part, l'un de ces feuillets, qui semble être un rapport ou une conclusion d'expertise par différence de valeurs, 
indique en définitive que le véhicule a été reconnu économiquement non réparable et a été cédé à la société \quote{Poids lourds 59} 
pour la somme de 6000€ HT le 15/12/2014. 

J'en déduis qu'aucune réparation n'a été réalisée et n'est donc facturable.
Par ailleurs, la V.R.A.D.E. est estimée à 14500€ HT, la valeur résiduelle à 6000€ HT, ce qui conduit à une différence de valeur de 8500€ HT.
Enfin, il n'est pas indiqué ce qui a été cédé réellement : caisse seule ou caisse et chassis et poste de conduite ? 
\blank[big]
D'autre part, les deux autres feuillets concernent des chiffrages sur différents postes de réparation du véhicule :
\startitemize
\item {\bf Chiffrage 1 : Caisse déménageur. }\crlf
Ce poste comprend en particulier un \quote{hayon d'hollandia} pour une valeur de 3500€ HT, 
situé à l'arrière de la caisse et qui, 
par conséquent ne me semble pas concerné par le choc caisse/pont. 
Une photographie prise après le choc en particulier ne montre pas que le hayon aurait souffert.
\item {\bf Chiffrage 2 : Chassis-dessous. }\crlf
Le choc étant survenu à une vitesse inférieure à 30 km/h entre le haut de la caisse et le pont, 
je me questionne sur ses conséquences réelles au niveau du chassis d'un véhicule de ce gabarit. 
Ce poste s'élève à 439.55€ HT.
\item {\bf Chiffrage 3 : Choc arrière avec voiture (peugeot 207). }\crlf 
Je ne me considère pas responsable du choc arrière dont les dégats sont estimés à 5251.20€ HT
\stopitemize
En outre, il n'est pas indiqué qui a établi ces chiffrages 
(qui par ailleurs ne correspondent pas à l'\quote{estimation des dommages apparents} reprise sur le 1\high{er} feuillet).
\blank[big]
En conclusion, la somme de 15900€ TTC (13250€ HT) réclamée dans votre courrier du 06/02/2015 ne me semble pas adéquate et je me permets de la contester.
\blank[big]	
Par ailleurs, je vous prie de bien vouloir me faire parvenir également, comme demandé précédemment une copie du ou des rapports d'assurance établis concernant d'une part le choc pont/caisse et d'autre part le choc camion/véhicule, et établissant les responsabilités retenues.
\blank[big]	
Je me permets de rappeler ici que le jour du sinistre, le gabarit de sécurité normalement placé en amont du pont (par le centre commercial Auchan Sémécourt) n'était pas en état susceptible de lui faire jouer son rôle préventif et je m'étonne que l'assurance n'ait pas tenu compte de cet élément dans l'établissement des responsabilités respectives.
\blank[big]	
Dans l'attente de votre réponse, je vous prie d'agréer Madame, Monsieur, l'expression de ma considération distinguée.
	
	



\stoptext
\stopcomponent
