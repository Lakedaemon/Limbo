\startcomponent component_DS1
\project project_Res_Mathematica
\environment environment_Maths
\environment environment_Inferno
\xmlprocessfile{exo}{xml/Limbo_Exercices.xml}{}
\iffalse
\setupitemgroup[List][1][R,inmargin][after=,before=,left={\bf Exo },symstyle=bold,inbetween={\blank[big]}]
\setupitemgroup[List][2][n,joineup][after=,before=,inbetween={\blank[small]}]
\setupitemgroup[List][3][a,joineup][after=,before=,inbetween={\blank[small]}]
\setupitemgroup[List][4][1,joineup,nowhite]
\fi

%\setupitemgroup[List][1][A,inmargin][after=,before=,left={\bf Exo },symstyle=bold,inbetween={\blank[big]}]
%\setupitemgroup[List][1][R,joineup][after=,before=,inbetween={\blank[small]}]
%\setupitemgroup[List][1][n,inmargin][after=,before=,left={\bf Exo },symstyle=bold,inbetween={\%blank[big]}]
%\setupitemgroup[List][2][n,joineup][after=,before=,inbetween={\blank[small]}]
%\setupitemgroup[List][3][a,joineup][after=,before=,inbetween={\blank[small]}]
%\setupitemgroup[List][4][1,joineup,nowhite]
%\setupitemgroup[List][4][a,joineup,nowhite]
\definecolor[myGreen][r=0.55, g=0.76, b=0.29]%
\setuppapersize[A4]
\setuppagenumbering[location=]
\setuplayout[header=0pt,footer=0pt]
\def\conseil#1{{\myGreen\it #1}}%


\starttext
\setupheads[alternative=middle]
%\showlayout
\def\gah#1{\margintext{Exercice #1}}

\iftrue

% Proposer à Nathan et Damien de faire le DS 6 2016-2017

\page
\centerline{\bfb CORRECTION DU DEVOIR SURVEILLE 7}
\blank[big]

\setupitemgroup[List][1][n][after=,before=,inbetween={\blank[small]}]
\setupitemgroup[List][2][a,joineup][after=,before=,inbetween={\blank[small]}]
\setupitemgroup[List][3][i,joineup][after=,before=,inbetween={\blank[small]}]
\setupitemgroup[List][4][1,joineup,nowhite]

\centerline{\bf EXO 1}
\centerline{\bf Partie A}
\startList
\item Pour $k∈ℕ^*$, $X_k$ suit une loi uniforme sur $⟦1, n⟧$ (équiprobabilité du tirage des $n$ boules). De sorte que $E(X_k)={n+1\F 2}$
\item\startList
\item Il faut au plus $n$ tirages ($n$ un) pour obtenir une somme supérieure ou égale à $n$ et au moins un tirage (un $n$), les nombres intermédiaires sont également possibles donc $T_n(Ω)=⟦1,n⟧$
\item $P(T_n=1) = {1\F n}$ (probabilité de tirer la boule $n$ au premier tour)

$P(T_n=n)=\Q({1\F n}\W)^{n-1}$ (probabilité de tirer $1$ au $n-1$ premiers tirages + indépendance mutuelle des tirages avec remise)
\stopList
\item Soit $n=2$. La VAR $T_2$ prend la valeur $1$ si la première boule tirée n'est pas la boule $1$ et prend la valeur $2$ sinon. D'après 2b, on a $P(T_2=2)=P(X_1=1)={1\F 2}$ et $P(T_1=1)=P(X_2=2)={1\F 2}$
 De sorte que $T_2\hookrightarrow \mc U⟦1,2⟧$.
 \item Soit $n=3$. D'après 2b, on a $P(T_3=1)={1\F 3}$, $P(T_3=3)={1\F3^2}$ et par complémentarité 
 \startformula
 P(T_3=2)=1-P(T_3=1)-P(T_3=3)=1-{1\F 3}-{1\F 9}={5\F 9}
 \stopformula
 {\it cela ne ressemble pas à une loi connue. }
 En particulier, nous obtenons que 
 \startformula
 E(T_3)=1×P(T_3=1)+2×P(T_3=2)+3×P(T_3=3)={1\F 3}+2×{5\F 9}+3×{1\F 9}={16\F 9}
\stopformula
\stopList
\centerline{\bf Partie B}
\startList
\item Soit $k∈ℕ^*$. Comme $S_k(Ω)$ est l'ensemble des valeurs que l'on peut obtenir en sommant $k$ entiers de l'ensemble $⟦1,n⟧$, on a
\startformula
S_k(Ω)=⟦k, kn⟧
\stopformula
\item Soit $k∈⟦1,n-1⟧$.
\startList
\item $S_{k+1}=∑_{i=1}^{k+1}X_i=∑_{i=1}^kX_i+X_{k+1}=S_k+X_{k+1}$
\item Soit $i∈⟦k+1,n⟧$. La formule des probabilités totales, appliquée au système complet d'événements $\big(S_k=j)\big)_{k⩽j⩽kj}$ induit que 
\startformula
P(S_{k+1}=i)=∑_{j=k}^{kn}P(S_k=j∩S_{k+1}=i)
\stopformula
Comme l'événement $(S_k=j)∩(S_{k+1}=i)$ n'est possible que si $1⩽i-j$ et $k⩽j$ (d'ou $k⩽j⩽i-1$) et qu'il vaut dans ce cas 
\startformula
(S_k=j)∩(S_{k+1}=i)=(S_k=j)∩(X_{k+1}= i-j)
\stopformula
Nous déduisons alors de l'indépendance de $X_k=i-j$ avec $S_k=j$ que  
\startformula
\Align{
\NC P(S_{k+1}=i)\NC \displaystyle=∑_{j=k}^{i-1}P(S_k=j∩X_{k+1}=i-j)\NR
\NC\NC \displaystyle=∑_{j=k}^{i-1}P(S_k=j)\underbrace{P(X_k=i-j)}_{1/n}={1\F n}∑_{j=k}^{i-1}P(S_k=j)
}
\stopformula
\stopList
\item\startList
\item Pour $k∈ℕ^*$ et $j∈ℕ^*$, la formule de Pascal donne 
\startformula
{j-1\choose k-1} + {j-1\choose k} = {j\choose k}.
\stopformula
\item A fortiori, pour $1⩽k⩽i-1$, en reconnaissant une somme telescopique, il vient   
\startformula
\Align{
\NC \displaystyle∑_{j=k}^{i-1}{j-1\choose k-1}\NC\displaystyle 
={k-1\choose k-1} + ∑_{j=k+1}^{i-1}{j-1\choose k-1}\NR
\NC\NC\displaystyle ={k-1\choose k-1} + ∑_{j=k+1}^{i-1}\Q({j\choose k}-{j-1\choose k}\W)\NR
\NC\NC\displaystyle =1 + ∑_{j=k+1}^{i-1}{j\choose k}-∑_{j'=k}^{i-2}{j'\choose k}\qquad\text{CDI $j'=j-1$}\NR
\NC\NC\displaystyle =1+{i-1\choose k}-{k\choose k}={i-1\choose k}
}
\stopformula
\item Pour $k∈⟦1,n⟧$, démontrons par récurrence la proposition $\mc H_k$
\startitemize[1]
\item $\mc H_1$ est vraie car $P(S_1=i)=P(B_1=i)={1\F n}={1\F n^1}{i-1\choose 0}$ pour $1⩽i⩽n$
\item Supposons $\mc H_k$ pour un entier $k∈⟦1,n-1⟧$ {\it (récurrence finie)} et prouvons $\mc H_{k+1}$.
D'après le résultat de la question 2a et $\mc H_k$, nous avons 
\startformula
\Align{
\NC P(S_{k+1}=i)\NC =\displaystyle {1\F n}∑_{j=k}^{i-1}P(S_k=j)\NR
\NC \NC =\displaystyle {1\F n}∑_{j=k}^{i-1}{1\F n^k}{j-1\choose k-1}\NR
\NC \NC =\displaystyle {1\F n^{k+1}}∑_{j=k}^{i-1}{j-1\choose k-1} ={1\F n^{k+1}}{i-1\choose k}\NR
\NC\NC =\displaystyle {1\F n^{k+1}}{i-1\choose (k+1)-1},\NR
}
\stopformula
d'après le résultat de la question 3b (et car $k⩽i⩽n$ et $k⩽j⩽n-1$).
En particulier $\mc H_{k+1}$ est vraie
\stopitemize
En conclusion, la proposition $\mc H_k$ est vraie pour $1⩽k⩽n$
\stopList
\item%4
\startList
\item Soit $k∈⟦1, n-1⟧$. On a trivialement $(T_n>k)=(S_k⩽n-1)$ : Si le premier rang $T_n$ pour lequel la somme des tirages est supérieur ou égal à $n$
est au moins égale à k+1, alors nécéssairement la somme des tirages obtenues au cour des $k$ premiers tirages est strictement inférieure à $n$ (et réciproquement).
\item  Pour $0⩽k⩽n$, Il résulte du résultat des trois questions précédentes que 
\startformula
\Align{
\NC P(T_n>k)\NC\displaystyle  =P(S_k⩽n-1)=∑_{i=1}^{n-1}P(S_k=i)\NR
\NC \NC\displaystyle  =0+∑_{i=k}^{n-1}P(S_k=i)\qquad(\text{$P(S_k=i)=0$ si $i<k$})\NR
\NC\NC\displaystyle  = ∑_{i=k}^{n-1}{1\F n^k}{i-1\choose k-1}\qquad\text{résultat de 3c}\NR
\NC\NC\displaystyle  ={1\F n^k} ∑_{i=k}^{n-1}{i-1\choose k-1}\NR
\NC\NC\displaystyle  ={1\F n^k} {n-1\choose k}\qquad\text{résultat de 3b}
}
\stopformula
\stopList
\item Xomme $T_n(Ω)=⟦1,n⟧$, nous remarquons que 
\startformula
\Align{
\NC E(T_n)\NC\displaystyle  =∑_{k=1}^nk×P(T_n=k)=∑_{k=1}^nk×\big(P(T_n>k-1)-P(T_n>k)\big)\NR
\NC\NC\displaystyle  =∑_{k=1}^nkP(T_n>k-1)-∑_{k=1}^nkP(T_n>k)\qquad\text{CDI $k'=k-1$}\NR
\NC\NC\displaystyle  =∑_{k'=0}^{n-1}(k'+1)P(T_n>k')-∑_{k=1}^nkP(T_n>k)\NR
\NC\NC\displaystyle  =P(T_n>0)+∑_{k'=1}^{n-1}(k'+1)P(T_n>k')-∑_{k=1}^{n-1}kP(T_n>k)-\underbrace{nP(T_n>n)}_{0}\NR
\NC\NC\displaystyle  =P(T_n>0)+∑_{k=1}^{n-1}P(T_n>k)\NR
\NC\NC\displaystyle  =∑_{k=0}^{n-1}P(T_n>k)\NR
\NC\NC\displaystyle  =∑_{k=0}^{n-1}{1\F n^k}{n-1\choose k}\qquad\text{Bdn $(a+b)^{n-1}$ pour $a=1$ et $b={1\F n}$}\NR
\NC\NC\displaystyle  =\Q(1+{1\F n}\W)^{n-1}
}
\stopformula
\item Via un développement limité, nous obtenons que 
\startformula
\Align{
\NC E(T_n)\NC =\Q(1+{1\F n}\W)^{n-1}=\e^{(n-1)\ln\Q(1+{1\F n}\W)}=\e^{(n-1)\Q({1\F n}+o\Q({1\F n}\W)\W)}\NR
\NC \NC = \e^{1+o(1)}
}
\stopformula
De sorte que $\displaystyle\lim_{n→+∞}E(T_n)=\e$.
\stopList
\centerline{\bf Partie C}
Dans cette partie, on fait varier l’entier $n$ et on étudie la suite de variable $(T_n)_{n≥1}$.
\startList
\item Soit $Y$ la variable aléatoire définie par $\displaystyle P(Y=k)={k-1\F k!}$ pour $k∈ℕ^*$.
\startList
\item%a
Nous déduisons de la définition de $Y$ que  
\startformula
\Align{
\NC ∑_{k=1}^{+∞}P(Y=k)\NC\displaystyle  =∑_{k=1}^{+∞}{k-1\F k!} \NR
\NC \NC =∑_{k=1}^{+∞}{k-1\F k!} \NR
\NC \NC\displaystyle  =∑_{k=1}^{+∞}{k\F k!} -∑_{k=1}^{+∞}{1\F k!} \NR
\NC \NC\displaystyle  =∑_{k=1}^{+∞}{1\F (k-1)!} -∑_{k=0}^{+∞}{1\F k!} +1\qquad\text{ CDV $k'=k-1$"}\NR
\NC \NC\displaystyle  =∑_{k'=0}^{+∞}{1\F k'!} -∑_{k=0}^{+∞}{1\F k!} +1=1\NR
}
\stopformula
\item {\it nous allons justifier la convergence absolue de la série et donc l'existence de l'esperance en fin de calcul (nos égalités deront justifiées). }
En cas de convergence, nous avons 
\startformula
\Align{
\NC E(Y)\NC\displaystyle  =∑_{k=1}^{+∞}kP(Y=k)=∑_{k=1}^{+∞}k{k-1\F k!}\NR
\NC\NC\displaystyle  =∑_{k=1}^{+∞}k{k-1\F k!}=∑_{k=2}^{+∞}k{k-1\F k!}\NR
\NC\NC\displaystyle  =∑_{k=2}^{+∞}{1\F (k-2)!}\qquad{\text{CDV $k'=k-2$}}\NR
\NC\NC\displaystyle  =∑_{k'=0}^{+∞}{1\F k'!}=\e^0=1
}
\stopformula
\stopList
\item Soit $k∈ℕ^*$. D'après la formule obtenue en 4b et d'après la formule 
\startformula
n^k∼(n-1)(n-2)⋯(n-k)={(n-1)!\F (n-1-k)!}
\stopformula
(il y a bien $k$ facteurs dans le produit de droite et $n∼n-1$, $n∼n-2$, ⋯, $n∼(n-k)$), 
nous avons 
\startformula
P(T_n>k)={1\F n^k}{n-1\choose k}={1\F n^k}{(n-1)!\F (n-1-k)!k!}∼{1\F n^k}
\stopformula
En particulier, $\displaystyle \lim_{n→+∞}P(T_n>k)={1\F k!}$
\item Soit $k∈ℕ^*$. Nous déduisons du résultat de la question précédente que 
\startformula
\Align{
\NC\displaystyle  \lim_{n→+∞}P(T_n=k)\NC\displaystyle  =\lim_{n→+∞}\Q(P(T_n>k-1)-P(T_n>k)\W)\NR
\NC \NC \displaystyle =\lim_{n→+∞}P(T_n>k-1)-\lim_{n→+∞}P(T_n>k)\NR
\NC\NC \displaystyle ={1\F (k-1)!}-{1\F k!}={k-1\F k!}\NR
\NC\NC =P(Y=k)
}
\stopformula
\stopList

\blank[big]
\centerline{\bf EXO 2}
On rappelle que $2<\e<3$ et que si les suites $(u_{2n})_{n⩾1}$ et $(u_{2n+1})_{n⩾1}$ convergent vers la même limite $ℓ$, alors la suite $(u_n)_{n⩾1}$ converge vers $ℓ$. 
\blank[medium]
On s'intéresse à la série de terme général $\displaystyle u_n=(-1)^n{\ln n\F n}$ pour $n⩾1$.
\startList
\item On note $\displaystyle w_n=∑_{k=1}^n{1\F k}-\ln(n)$ pour $n⩾1$. 
\startList
\item D'après le cours 
\startformula
\ln(1+x)=x-{x^2\F 2}+o_0(x^2)\qquad\Et\qquad {1\F 1-u}=1+u+u^2+o_0(u^2)
\stopformula
De sorte que pour $u=-x$,il vient
\startformula
{1\F 1+x}=1-x+x^2+o_0(x^2)
\stopformula
\item En particulier, nous obtenons que 
\startformula
\Align{
\NC w_{n+1}-w_n \NC \displaystyle =∑_{k=1}^{n+1}{1\F k}-\ln(n+1)-\Q(∑_{k=1}^n{1\F k}-\ln(n)\W)\NR
\NC\NC\displaystyle  {1\F n+1} + ∑_{k=1}^n{1\F k}-∑_{k=1}^n{1\F k}-\big(\ln(n+1)-\ln b\big)\NR
\NC\NC\displaystyle  {1\F n+1} -\big(\ln(n+1)-\ln b\big)\NR
\NC\NC\displaystyle  {1\F n}×{1\F 1+1/n} -\ln\Q(1+{1\F n}\W)\qquad\text{Factoriser le terme dominant}\NR
\NC\NC\displaystyle  = {1\F n}\Q(1-{1\F n}+o\Q({1\F n}\W)\W)-\Q({1\F n}-{1\F 2n^2}+o\Q({1\F n^2}\W)\W)\NR
\NC\NC\displaystyle  = -{1\F 2n^2}+o\Q({1\F n^2}\W)∼-{1\F 2n^2}
}
\stopformula
\item La série de terme général $(w_{n+1} − w_{n})$ converge car elle a la même nature que la série de terme général ${1\F 2n^2}$ qui est une série de Riemann convergeante ($α=2>1$).
En effet l'équivalent obtenu est de signe constant (strictement négatif au voisinage de l'infini).
A fortioti sa suite des sommes partielles converge vers un nombre réel. Or, on a 
\startformula
w_1+∑_{k=1^n-1}(w_{k+1} − w_{k})=w_n\qquad(n⩾1)
\stopformula
A fortiori, la suite $(w_n)$, qui est égale à cette suite des sommes partielles (à une constante près), est convergeante également.
On a donc
\startformula
w_n=γ+o(1)
\stopformula
\stopList
\item $f$ est de classe $\mc C^1$ sur $]0,+∞[$ et de plus 
\startformula
f'(x)={1\F x}{1\F x}-{\ln x\F x^2}={1-\ln(x)\F x^2}\qquad(x>0).
\stopformula
En particulier, comme $1-\ln(x)=0⟺x=\e$ et $1-\ln(x)>0⟺x<\e$, l'application $f$ est strictement croissante sur $]0,\e]$ et strictement décroissante sur $[\e,+∞[$. 
De plus, d'après le théorème de croissance comparée, on a $f(\e)={1\F\e}$,  
\startformula
\lim_{x→0^+}f(x)=-∞\qquad\Et\qquad\lim_{x→+∞}f(x)=0
\stopformula
\item On note $\displaystyle S_n=∑_{k=1}^nu_k$ pour $n⩾1$. 
\startList
\item \startitemize[1]
\item On a $\displaystyle \lim_{n→+∞}\Q(S_{2n+1}-S_{2n}\W)=\lim_{n→+∞}u_{2n+1}=\lim_{n→+∞}-{\ln(2n+1)\F 2n+1}=0$
\item On a 
\startformula S_{2n+2}-S_{2n}=u_{2n+2}+u_{2n+1}={\ln(2n+2)\F 2n+2}-{\ln(2n+1)\F 2n+1}=f(2n+2)-f(2n+1)⩽0
\stopformula car $f$ est décroissante sur $[\e,+∞[$. 
En particulier, la suite $(S_{2n})_{n⩾2}$ est décroissante.
\item On a 
\startformula 
S_{2n+3}-S_{2n+1}=u_{2n+3}+u_{2n+2}=-{\ln(2n+3)\F 2n+3}+{\ln(2n+2)\F 2n+2}=f(2n+2)-f(2n+3)⩾0
\stopformula car $f$ est décroissante sur $[\e,+∞[$. 
En particulier, la suite $(S_{2n+1})_{n⩾2}$ est croissante.
\stopitemize
En conclusion les suites $(S_{2n})_{n ⩾2}$ et $(S_{2n+1})_{n⩾2}$ sont adjacentes et convergent doncvers une même limite finie $ℓ$.
\item Il résulte alors de la propriété rappelée dans le sujet que la suite $(S_n)_{n⩾4}$ converge.Comme il s'agit de la suite des sommes partielles de la série de terme général $u_n$, celle si converge.
Par contre, elle ne converge pas absolument car 
\startformula
|u_n|={\ln n\F n}⩾{1\F n}
\stopformula
Et, comme la série de Riemann de temre général ${1\F n}$ diverge, la série de terme général $|u_n|$ diverge.
\stopList
\item%4
 On note $\displaystyle v_n=∑_{k=1}^n{\ln(k)\F k}-{\ln(n)^2\F 2}$ pour $n⩾1$. 
\startList
 \item Soit  $n⩾3$. Comme la fonction $f$ est décroissante sur $[\e,+∞[$ et donc sur $[3,+∞[$ (puisque le sujet a eu la bonté de nous rappeler que $\e<3$), nous avons 
\startformula
{\ln(n+1)\F n+1}=f(n+1)⩽f(t)={\ln t\F t}⩽f(n)={\ln n\F n}\qquad (n⩽t⩽n+1)
\stopformula
En intégrant sur $[n,n+1$, il résulte alors de la croissance de l'intégale que 
\startformula
{\ln(n+1)\F n+1}=\int_n^{n+1}{\ln(n+1)\F n+1}\d t
⩽\int_n^{n+1}{\ln(t)\F t}\d t⩽\int_n^{n+1}{\ln n\F n}\d t = {\ln(n)\F n}\qquad(n⩾3)
\stopformula
\item Pour $a>0$ et $b>0$, nous avons (la fonction $f$ étant continue sur $]0,+∞[$ et donc sur $[a,b]$)
\startformula
\int_a^b{\ln(t)\F t}\d t=\Q[{\ln(t)^2\F 2}\W]_a^b={\ln(b)^2-\ln(a)^2\F 2}
\stopformula
\item D'après le résultat des deux  questions précédentes, nous avons 
\startformula
\Align{
\NC v_{n+1}-v_n\NC\displaystyle  =∑_{k=1}^{n+1}{\ln(k)\F k}-{\ln(n+1)^2\F 2}-\Q(∑_{k=1}^n{\ln(k)\F k}-{\ln(n)^2\F 2}\W)\NR
\NC\NC\displaystyle  = {\ln(n+1)\F n+1}+{\ln(n)^2-\ln(n+1)^2\F 2}\NR
\NC\NC\displaystyle  ={\ln(n+1)\F n+1}-\int_n^{n+1}{\ln t\F t}\d t⩽0
}
\stopformula
En particulier, la suite $v$ est décroissante. Par ailleurs, nous remarquons que 
\startformula
\Align{
\NC v_n=\NC\displaystyle  ∑_{k=1}^n{\ln(k)\F k}-{\ln(n)^2\F 2}\NR 
\NC \NC\displaystyle  =∑_{k=1}^n{\ln(k)\F k}{\ln(n)^2-\ln(1)^2\F 2}\NR
\NC \NC\displaystyle  =∑_{k=1}^n{\ln(k)\F k}-\int_1^n{\ln(t)\F t}\d t\NR
\NC \NC\displaystyle  =∑_{k=1}^n{\ln(k)\F k}-∑_{k=1}^{n-1}\int_k^{k+1}{\ln(t)\F t}\d t\NR
\NC \NC\displaystyle  =\underbrace{{\ln(n)\F n}}_{⩾0}+ ∑_{k=1}^{n-1}\Q(\underbrace{{\ln(k)\F k}-\int_k^{k+1}{\ln(t)\F t}\d t}_{⩾0\text{ pour $k⩾3$}}\W)\NR
\NC \NC\displaystyle  ⩾ ∑_{k=1}^2\Q({\ln(k)\F k}-\int_k^{k+1}{\ln(t)\F t}\d t\W)=c\NR 
}
\stopformula
En conclusion, la suite $(v_n)_{n⩾3}$ est décroissante et minorée. Elle est doncconvergeante ({\it trop facile, z'ont même pas cherché à noyer le poisson})
\stopList
\item%5
Soit $n⩾1$. Alors, nous remarquons que 
\startformula
\Align{
\NC\displaystyle  S_{2n}=∑_{k=1}^{2n}u_k\NC =∑_{1⩽k⩽2n\atop k=2ℓ}u_k+∑_{1⩽k⩽2n\atop k=2ℓ+1}u_k\NR 
\NC \NC\displaystyle  =∑_{ℓ=1}^n u_{2ℓ}+∑_{ℓ=1}^{n-1}u_{2ℓ+1}\NR 
\NC \NC\displaystyle  =∑_{ℓ=1}^n {\ln (2ℓ)\F 2ℓ}-∑_{ℓ=1}^{n-1}{\ln(2ℓ+1)\F 2ℓ+1} 
}
\stopformula
A fortiori, nous obtenons que
\startformula
\Align{
\NC\displaystyle  S_{2n}+∑_{k=1}^{2n}{\ln(k)\F k}\NC\displaystyle  =
S_{2n}+∑_{k=1\atop k=2ℓ}^{2n}{\ln(k)\F k}+∑_{k=1\atop k=2ℓ+1}^{2n}{\ln(k)\F k}\NR
\NC \NC\displaystyle  =
S_{2n}+∑_{ℓ=1}^n{\ln(2ℓ)\F 2ℓ}+∑_{ℓ=1}^{n-1}{\ln(2ℓ+1)\F 2ℓ+1}\NR
\NC\NC\displaystyle  =2∑_{ℓ=1}^n{\ln(2ℓ)\F 2ℓ}\NR
\NC\NC\displaystyle  =2∑_{ℓ=1}^n{\ln(2)+\ln(ℓ)\F 2ℓ}\NR
\NC\NC\displaystyle  =2∑_{ℓ=1}^n{\ln(2)\F 2ℓ}+2∑_{ℓ=1}^n{\ln(ℓ)\F 2ℓ}\NR
\NC\NC\displaystyle  =\ln(2)∑_{ℓ=1}^n{1\F ℓ}+∑_{ℓ=1}^n{\ln(ℓ)\F ℓ}\NR
}
\stopformula
En soustrayant $\displaystyle  ∑_{k=1}^{2n}{\ln(k)\F k}$ de chaque coté, il vient alors 
\startformula
\Align{
\NC S_{2n}\NC\displaystyle  =\ln(2)∑_{ℓ=1}^n{1\F ℓ}+∑_{ℓ=1}^n{\ln(ℓ)\F ℓ}-∑_{k=1}^{2n}{\ln(k)\F k}\NR
\NC \NC\displaystyle  = \ln(2)∑_{k=1}^n{1\F k} + \Q(v(n)+{\ln(n)^2\F 2}\W)-\Q(v(2n)+{\ln(2n)^2\F 2}\W)\NR 
\NC \NC\displaystyle  = \ln(2)∑_{k=1}^n{1\F k} + v(n)-v(2n)+{\ln(n)^2\F 2}-{(\ln(2)+ln(n)^)2\F 2}\NR 
\NC \NC\displaystyle  =\ln(2)∑_{k=1}^n{1\F k}+v_n-v_{2n}-{\ln(2)^2\F 2}-\ln(2)\ln(n)
}
\stopformula
\item Il résulte de la définition de la suite $w$ et de  la relation précédente que 
\startformula
S_{2n}=\ln(2) w_n+v_n-v_{2n}-{\ln(2)^2\F 2}\qquad(n⩾1)
\stopformula
On $w_n=γ+o(1)$ et comme la suite $v$ converge vers une limite finie $ℓ$ d'après la question 4c, il en est de même pour la suite $(v_{2n})_{n⩾1}$. 
De sorte, qu'en faisant tendre $n$ vers $+∞$ dans la relation précédente, obtenons que 
\startformula
∑_{n=1}^{+∞}(-1)^n{\ln(n)\F n}=\lim_{n→+∞}S_{2n}=γ\ln(2)+ℓ-ℓ-{\ln(2)^2\F 2}=γ\ln(2)-{\ln(2)^2\F 2}.
\stopformula

\stopList
\blank[medium]
\centerline{\bf EXO 3 extrait de EML lyon 2018}
Soit $n⩾2$. Pour tout polynôme $P$ de $R[X]$, on pose 
\startformula
φ(P)={1\F n}X(1-X)P'+XP
\stopformula
\startList
\item \startList
\item \startitemize[1]
\item $ℝ_n[X]$ et $ℝ[X]$ sont des $ℝ$-espaces vectoriels de référence
\item Pour $P∈ℝ_n[R]$, ${1\F n}X(X-1)P'$ est un polynôme de $ℝ[X]$ (de degré $⩽n+1$), de même que $XP$. 
De sorte que $φ(P)∈ℝ_{n+1}[X]⊂ℝ[X]$. On a donc bien une application de $ℝ_n[X]$ dans $ℝ[X]$. 
\item Soient $(λ,μ)∈ℝ^2$ et $(P,Q)∈ℝ_n[X]^2$. On remarque que 
\startformula
\Align{
\NC φ(λP+μQ)\NC\displaystyle  ={1\F n}X(X-1)(λP+μQ)'+X(λP+μQ)\NR
\NC\NC\displaystyle  = {1\F n}X(X-1)(λP'+μQ')+X(λP+μQ)\NR
\NC\NC\displaystyle  = λ\Q({1\F n}X(X-1)P'+XP\W)+μ\Q({1\F n}X(X-1)Q'+XQ\W)\NR
\NC \NC = λφ(P)+μφ(Q)
}
\stopformula
\stopitemize
En conclusion, on a bien une application linéaire de $ℝ_n[X]$ dans $ℝ[X]$
\item On a $φ(X^n)={1\F n}X(1-X)nX^{n-1}+X^{n+1}=-X^{n+1}+X^n+X^{n+1}=X^n$. 
\item Pour cela, il suffit de prouver que $φ(ℝ_n[X])⊂ℝ_n[X]$, ce qui nous permet de restreindre l'application linéaire du a) à l'arrivée à $ℝ_n[X]$.
Soit $P∈ℝ_n[X]$. Alors il eixte un (unique) nombe réel $λ$ et un polynôme $Q∈ℝ_{n-1}[X]$ tel que 
\startformula
P=λX^n+Q
\stopformula
Et alors $φ(P)=λφ(X^n)+φ(Q)$
Comme $\deg Q⩽n-1$ et $\deg(Q')⩽\deg(Q)-1⩽n-2$, nous remarquons que $\deg (φ(Q))⩽n$. 
A fortiori, $φ(P)∈ℝ_n[X]$. On a bien un endomorphisme de $ℝ_n[X]$. 
\stopList
\item Pour déterminer la matrice $A$ de $φ$ dans la base canonique $ℬ$ de $ℝ_n[X]$, on calcule 
\startformula
\Align{
\NC φ(X^k)\NC\displaystyle  ={1\F n}X(1-X)kX^{k-1}+X^{k+1}\NR
\NC\NC\displaystyle  ={k\F n}X^k-{k\F n}X^{k+1}+X^{k+1}\NR
\NC \NC\displaystyle  ={k\F n}X^k+{n-k\F n}X^{k+1}\qquad(0⩽k⩽n)
	}
\stopformula
A fortiori, cela donne la (grosse) matrice
\startformula
\Matrix{
\NC 0 \NC 0      \NC   \NC ⋯\NC ⋯\NC 0\NR
\NC 1 \NC {1\F n}\NC ⋱\NC   \NC \NC ⋮\NR
\NC 0\NC {n-1\F n}\NC {2\F n}\NC ⋱\NC\NC⋮\NR
\NC ⋮\NC ⋱\NC⋱    \NC ⋱\NC ⋱ \NC\NR
\NC ⋮\NC \NC ⋱\NC {2\F n}\NC {n-1\F n}\NC 0\NR
\NC 0\NC ⋯\NC ⋯\NC 0\NC {1\F n}\NC 1
}
\stopformula
Cette matrice est de rang $n$ (on fait $n$ pivots successifs sur les colonnes en utilisant $C_1$ puis $C_2$ jusqu'à $C_n$)
\item \startList
\item L'endomorphisme $φ$ ne peut pas être injecti d'après le théorème du rang $\dim_ker(φ)=\dim(ℝ_n[X]-\rg(φ)=n+1-n=1$.
Et donc $\ker φ≠\{0\}$. 
\item Soit $P≠0$ un polynôme de $\ker(φ)$. Alors, on a 
\startformula
\Align{
\NC P∈\ker φ\NC ⟺φ(P)=0\NR
\NC\NC ⟺ {1\F n}X(1-X)P'+XP=0\NR
\NC\NC ⟺P=-{1\F n}(X-1)P'
}
\stopformula
Si $z$ est une racine de multiplicité $m$ de $P$, nous savons que $z$ est une racine de multiplicité $m-1$ de $P'$. 
Il découle alors de l'identité précédente que $z$ est nécessairement égale à $1$ (sinon, $z$ ne peut pas être racine de $P$ de multiplicité $m$).
A fortiori, la seule racine possible pour $P$ est $1$ De sorte que $P=λ(X-1)^k$. Et alors, comme 
$0=φ(P)={1\F n}X(1-X)P'+XP={1\F n}X(X-1)λk(X-1)^{k-1}+λX(X-1)^k$, nous obtenons que 
\startformula
0=λX\Q(-{k\F n}+1\W)(X-1)^k
\stopformula
En particulier, comme $λ≠0$, nous avons ncéssairement $k=n$ de sorte que $P=λ(X-1)^n$. Réciproquement, nous remarquons que $φ\big((X-1)^n\big)=0$. 
En conclusion $ \ker(φ)=\Vect\big((X-1)^n\big)$
Montrer que $P$ admet $1$ comme unique racine (dans $ℂ$) et que $P$ est de degré $n$. 
\item $(X-1)^n$ est une base de $\ker(φ)$, d'après le résultat de la proposition précédente
\item Pour $k∈⟦0, n⟧$, on pose $P_k=X^k(1-X)^{n-k}$ et on remarque que 
\startformula
\Align{
\NC φ(P_k)\NC \displaystyle ={1\F n}X(1-X)P_k'+XP_k\NR
\NC \NC \displaystyle = {1\F n}X(1-X)(X^k(1-X)^{n-k})'+XX^k(1-X)^{n-k}\NR
\NC \NC \displaystyle = {1\F n}X(1-X)(kX^{k-1}(1-X)^{n-k}-(n-k)X^k(1-X)^{n-k-1})+X^{k+1}(1-X)^{n-k}\NR
\NC \NC \displaystyle = {k\F n}X^k(1-X)^{n-k+1}-{n-k\F n}X^{k+1}(1-X)^{n-k}+X^{k+1}(1-X)^{n-k}\NR
\NC \NC \displaystyle = {k\F n}X^k(1-X)^{n-k+1}+{k\F n}X^{k+1}(1-X)^{n-k}\NR
\NC \NC \displaystyle = {k\F n}(1-X + X )X^k(1-X)^{n-k}={k\F n}X^k(1-X)^{n-k}\NR
\NC\NC \displaystyle = {k\F n}P_k
}
\stopformula
{\it Le sujet nous a fait trouver des vecteurs propres (le polynôme $P_k$) associés à des valeurs propres (le nombre réel ${k\F n}$). C'est un grand classique ECS2}
\item Montrons que la famille $(P_0, P_1, …, P_n)$ forme une base de $ℝ_n[X]$ {\it En ECS2, on dirait simplement que c'est une famille de vecteurs propres associés à des valeurs propres distinctes $2$ à $2$. En ECS1, il faut travailler un peu plus.}
Soient $(λ_0, ⋯,λ_n)∈ℝ^{n+1}$ tel que 
\startformula
0=∑_{k=0}^nλ_kP_k=∑_{k=0}^nλ_kX^k(1-X)^{n-k}
\stopformula
En substituant $0$ à $X$, nous obtenons alors que $0=∑_{k=0}^nλ_k0^k=λ_0$. En reportant dans l'identité initiale, il vient 
\startformula
0=∑_{k=1}^nλ_kP_k=∑_1{k=0}^nλ_kX^k(1-X)^{n-k}=X∑_{k=1}^nλ_kX^{k-1}(1-X)^{n-k}
\stopformula
De sorte que 
\startformula
0=∑_{k=1}^nλ_kX^{k-1}(1-X)^{n-k}
\stopformula
Et on peut recommencer à substituer $0$ à $X$ pour montrer que $λ_1=0$, puis $λ_2=0$, etc...
A la fin, on prouve (par récurrence finie) que $λ_0=⋯=λ_n=0$ de sorte que la famille $\mc C=(P_0, ⋯,P_n)$ est libre.
Comme c'est une famille de $n+1$ vecteurs de polynômes de $ℝ_n[X]$, qui est de dimension $n+1$, c'en est une base.
Enfin, du fait de la relation $φ(P_k)={k\F n}P_k$, la matrice de $φ$ dans cette noluvelle base est diagonale {\it c'est l'interet de la diagonalisation ECS2 que nous venons de faire à notre insu}
\startformula
\mc Mat_{\mc C}(φ)=\Matrix{
\NC 0\NC 0\NC ⋯\NC 0\NR
\NC 0\NC {1\F n}\NC ⋱\NC ⋮\NR
\NC ⋮\NC⋱\NC ⋱\NC 0\NR
\NC 0\NC ⋯\NC 0 \NC {n\F n}
}
\stopformula
\stopList
\stopList
{\it Un sujet pas trop dur (comparativement aux autres qu'on a fait en DS... une difficulté due à la dimension $n+1$ en algèbre linéaire.
	Pour les probas, il faut connaitre son binôme de Newton, ses coeffs du binôme, les sommes telescopiques, pour l'analyse les sommes, les primitives et les dls, ça aide}
\stoptext
\stopcomponent
\endinput
