\startcomponent component_DS1
\project project_Res_Mathematica
\environment environment_Maths
\environment environment_Inferno
%\xmlprocessfile{exo}{xml/Limbo_Exercices.xml}{}
%\setupitemgroup[List][1][R,inmargin][after=,before=,left={\bf Exo },symstyle=bold,inbetween={\blank[big]}]
%\setupitemgroup[List][2][n,joineup][after=,before=,inbetween={\blank[small]}]
%\setupitemgroup[List][3][a,joineup][after=,before=,inbetween={\blank[small]}]
%\setupitemgroup[List][4][1,joineup,nowhite]


\setuppapersize[A4]
\setuppagenumbering[location=]
\setuplayout[header=0pt,footer=0pt]




\starttext


% Faire une simulation de prélécture et évaluer le score des stratégies I et II pour comparer

\setupheads[alternative=middle]

\centerline{Tuto : maths de prépa}

Le but d'un ds est d'obtenir le plus de points possible en un temps limité.

\section{Prélecture complète du sujet (5 à 10 minutes)}

On lit toutes les questions d'un exercice, suffisamment lentement pour comprendre de quoi il s'agit, ce qui est demandé.
On repère les questions faciles (qu'on sait faire sans avoir à réfléchir ) d'une croix 
et les questions à priori très difficiles avec un autre symbôle (zéro ?)

{\bf Avec : }
\startList
\item prédeterminer les questions faciles/moyennes/difficiles
\item Vision et compréhension globale du sujet, des exercices (on sait ou on va, ce qu'on cherche à faire)
\item Idées gratuites (parfois les idées nécéssaires à la résolution d'une question sont données par la suite)
\stopList

\section{Déroulement du ds}

Stratégie I (du renard) : recommandée en ECS
\startList
\item prélecture du sujet (nécéssaire)
\item Commencer par traiter toutes les questions faciles, que l'on sait bien traiter (rapidement)
\item Quand ces questions sont épuisées, passer aux questions moyennes (qui vont demander plus de reflexion et de travail) 
\item S'il reste du temps, s'attaquer au reste 
\stopList

{\it Intérêt : } 
\startList
\item On ammasse rapidement des points (efficacité, bon taux points/heure)
\item La réussite met en confiance (super moral)
\item Grande plage de temps restant en fin de ds pour traiter les questions qui demandent plus de reflexion (bonus).
\stopList

Stratégie II (du bourrin) : adaptée seulement aux apprentis-prof de maths
\startList
\item Pas de prélecture du sujet
\item Choisir l'ordre dans lequel on fait les exercices selon ce qu'on aime
\item Faire les questions dans l'ordre.
\stopList

{\it Contre}
\startList
\item On laisse passer des questions très faciles dans les exercices qu'on n'a pas le temps de traiter (grosse chute dans le classement)
\item On traite globalement moins de questions (20 points de moins en moyenne)
\item Gros risque de panique et méga-plantage (en commençant tôt à s'attaquer à un exo ou une question trop difficile)
\item Probabilité de plantage accrue pour ceux qui ont du mal à lâcher une question qu'ils n'arrivent pas à faire.
\stopList


\stoptext
\stopcomponent
\endinput
