\startcomponent component_DS1
\project project_Res_Mathematica
\environment environment_Maths
\environment environment_Inferno
\xmlprocessfile{exo}{xml/Limbo_Exercices.xml}{}
\iffalse
\setupitemgroup[List][1][n,inmargin][after=,before=,left={\bf Exo },symstyle=bold,inbetween={\blank[big]}]
\setupitemgroup[List][2][a,joineup][after=,before=,inbetween={\blank[small]}]
\setupitemgroup[List][3][a,joineup][after=,before=,inbetween={\blank[small]}]
\setupitemgroup[List][4][1,joineup,nowhite]
\fi

\setupitemgroup[List][1][n,inmargin][after=,before=,left=,symstyle=bold,inbetween={\blank[big]}]
%\setupitemgroup[List][1][R,joineup][after=,before=,inbetween={\blank[small]}]
%\setupitemgroup[List][1][n,inmargin][after=,before=,left={\bf Exo },symstyle=bold,inbetween={\blank[big]}]
%\setupitemgroup[List][1][n,joineup][after=,before=,inbetween={\blank[small]}]
\setupitemgroup[List][2][a,joineup][after=,before=,inbetween={\blank[small]}]
\setupitemgroup[List][3][a,joineup,nowhite]
\setupitemgroup[List][4][a,joineup,nowhite]
\definecolor[myGreen][r=0.55, g=0.76, b=0.29]%
\setuppapersize[A4]
\setuppagenumbering[location=]
\setuplayout[header=0pt,footer=0pt]
\def\conseil#1{{\myGreen\it #1}}%


\starttext
\setupheads[alternative=middle]
%\showlayout
\def\gah#1{\margintext{Exercice #1}}


\defineframedtext
  [framedcode]
  [strut=yes,
   offset=2mm,
   width=7cm,
   align=right]

\definetyping[code][numbering=line,
                    bodyfont=small,
                    before={\startframedcode},
                    after={\stopframedcode}]
\iftrue
\page
\centerline{\bfb DEVOIR SCILAB 1 (1 heure)}
\blank[big]

\startList
% implémenter une dichotomie
% ecrire un programme pour trouver le 7 eme term d'une suite
% input
% if
% test if even

% Analyze
\item\startList\item Que va afficher le programme suivant : 
\startcode
u=0
v=1
for k=2:5
   w = 2*u+v
   u = v
   v = w
   disp(w)
end
\stopcode
\item Que se passe t'il, sur le plan mathématique, dans ce programme ?
\item Modifier le programme pour qu'il compte et affiche combien de fois $v=2u-1$. 
\item Transformer le programme initial en fonction \quote{calcul} prennant un entier $n$ et retournant le nombre $v$ que l'on obtiendrait si la boucle allait jusqu'à $k=n$. 
% completer un programme
% deviner l'affichage
\item Mathématiquement, exprimez en fonction de $n$ le nombre $calcul(n)$. 

\stopList

\item Ecrire un programme demandant à l'utilisateur le nombre de questions $n$ qu'il pense avoir fausses à ce devoir et affichant :
\startitemize[1]
\item \quote{pessimiste !} pour $n⩾9$
\item \quote{optimiste !} pour $n⩽1$
\item \quote{attentif ?} pour $1\Le n⩽4$
\item \quote{réaliste ?} sinon
\stopitemize

% input, for 
\item Ecrire un programme demandant des entiers $n$ et $p$ à l'utilisateur, puis affichant la valeur de la somme $∑_{k=1}^np^k$. 

% while
\item La suite récurrente définie par $u_0=0$ et $u_{n+1}={u_n^2+1\F 2}$ pour $n⩾0$ est croissante 
et tends vers $1$. Ecrire un programme affichant le premier indice pour lequel $u_n⩾ {9\F 10}$ et la valeur $u_n$ correspondante.

% function ou function & if
\item Ecrire une fonction divise(a,b) permettant de déterminer si la division d'un entier $a$
par un entier $b$ est un nombre entier ou non. La fonction devra renvoyer \%t (pour vrai) ou \%f (pour faux).

% function & for
\item Ecrire une fonction  \quote{sommeDesDiviseurs} calculant la somme des diviseurs d'un nombre entier 
(le programme pourra utiliser la fonction \quote{divise} de la question précédente)

% for 
\item Deux nombres distincts $m$ et $n$ sont dits amis si la somme des diviseurs de $m$ (autres que $m$) est égale
à $n$ et si la somme de tous les diviseurs de $n$ (autres que $n$) est égale à $m$. Par exemple 220 et 284 sont des nombres amis car 
\startformula
\Align{[align={left, left}]
\NC 220 = 220×1 = 110×2 = 55×4= 20 × 11  \NC = 142 + 71 + 4 + 2 + 1\NR 
\NC 284 = 284×1 = 142×2 = 71×4 \NC=  110 + 55 + 20 + 11 + 4 + 2 + 1\NR}
\stopformula
Ecrire un programme qui affiche tous les couples de nombres amis entre 1 et 1000.\crlf
(le programme pourra utiliser la fonction \quote{sommeDesDiviseurs} de la question précédente)
% for
% while
% function
% analyze
% modify
% big programm

\stopList

\stoptext
\stopcomponent
\endinput
