\startcomponent component_DS1
\project project_Res_Mathematica
\environment environment_Maths
\environment environment_Inferno
\xmlprocessfile{exo}{xml/Limbo_Exercices.xml}{}
\iffalse
\setupitemgroup[List][1][R,inmargin][after=,before=,left={\bf Exo },symstyle=bold,inbetween={\blank[big]}]
\setupitemgroup[List][2][n,joineup][after=,before=,inbetween={\blank[small]}]
\setupitemgroup[List][3][a,joineup][after=,before=,inbetween={\blank[small]}]
\setupitemgroup[List][4][1,joineup,nowhite]
\fi

%\setupitemgroup[List][1][A,inmargin][after=,before=,left={\bf Exo },symstyle=bold,inbetween={\blank[big]}]
%\setupitemgroup[List][1][R,joineup][after=,before=,inbetween={\blank[small]}]
%\setupitemgroup[List][1][n,inmargin][after=,before=,left={\bf Exo },symstyle=bold,inbetween={\%blank[big]}]
%\setupitemgroup[List][2][n,joineup][after=,before=,inbetween={\blank[small]}]
%\setupitemgroup[List][3][a,joineup][after=,before=,inbetween={\blank[small]}]
%\setupitemgroup[List][4][1,joineup,nowhite]
%\setupitemgroup[List][4][a,joineup,nowhite]
\definecolor[myGreen][r=0.55, g=0.76, b=0.29]%
\setuppapersize[A4]
\setuppagenumbering[location=]
\setuplayout[header=0pt,footer=0pt]
\def\conseil#1{{\myGreen\it #1}}%


\starttext
\setupheads[alternative=middle]
%\showlayout
\def\gah#1{\margintext{Exercice #1}}

\iftrue
\page
\centerline{\bfb DEVOIR MAISON 14}
\blank[big]

\setupitemgroup[List][1][A,joineup][after=,before=,inbetween={\blank[small]}]
\setupitemgroup[List][2][n,joineup][after=,before=,inbetween={\blank[small]}]
\setupitemgroup[List][3][a,joineup][after=,before=,inbetween={\blank[small]}]
\setupitemgroup[List][4][1,joineup,nowhite]


\centerline{\bf CORRECTION DU PROBLÈME 1}
\blank[big]
\startList%
\item%1 Généralités
\startList
\item%a

\startitemize[1]
\item $\mc C_A$ est inclus, par définition, dans $\mc M_n(ℝ)$, qui est un $ℝ$-espace vectoriel de référence 
\item $\mc C_A$ contient la matrice nulle car $A×0=0=0×A$ de sorte que $\mc C_A≠\emptyset$
\item Soient $(λ,μ)∈ℝ^2$ et $(M,N)∈\mc C_A^2$. Comme $AM=MA$ et $AN=NA$, il résulte de la distributivité du produit matriciel et de la loi du scalaire mobile que  
\startformula
(λM+μN)×A=λMA+μNA=λAM+μAN=(λM+μN)×A
\stopformula
En particulier, $λM+μN∈\mc C_A$. L'ensemble $\mc C_A$ est donc stable par combinaison linéaire.
\stopitemize
En conclusion, $\mc C_A$ est un sous-espace vectoriel de $\mc M_n(ℝ)$. 
\item%b
Pour $M∈\mc M_n(ℝ)$, et $λ∈ℝ$, il résulte de l'associativité du produit, de la loi du scalaire mobile et du fait que $I_n$ soit un élément neutrte pour la multiplication, que 
\startformula
M×(λI_n)=λ(M×I_n)=λM=λ(I_n×M)=(λI_n)×M
\stopformula
En particulier $M∈\mc C_{λI_n}$, de sorte que 
\startformula
\mc C_{λI_n}=\mc M_n(ℝ)\qquad(λ∈ℝ).
\stopformula
\item Soit $A∈\mc M_n(ℝ)$. Si $A∈\Vect(I_n)$, il résulte des calculs effectués dans la question précédente que 
$\dim(\mc C_A)=\dim\big(\mc M_n(ℝ)\big)=n^2⩾2$.

Si $A\∉\Vect(I_n)$, alors $\mc C_A$ est un sous-espace vectoriel de $\mc M_n(ℝ)$, contenant les matrices $I_n$ (car $A×I_n=A=I_n×A$) et $A$ (car $A×A=A^2=A×A$). 
Comme la famille $\{I_n,A\}$ est libre (puisque $A∉\Vect(I_n)$), $\mc C_A$ est de dimension au moins $2$ car il contient une famille libre de deux vecteurs.
\crlf
Dans tous les cas, $\mc C_A$ est au moins de dimension $2$.
\item {\it Houla, c'est une question technique ça... qui exploite le fait que les bases permettent de transformer (via isomorphisme) vecteurs en coordonnées, applications linéaires en matrices, etc...}
Soit $\Fun{ψ:\mc L(ℝ^n)→M_n(ℝ)|f↦\mc Mat_ℬ(f)}$ l'isomorphisme (d'après le cours) associant à un endomorphisme $f$ de $ℝ^n$, sa matrice dans la base $ℬ$.
Alors, je prétends que $ψ$ transforme $\mc C_f$ en $\mc C_A$, c'est à dire que $ψ(\mc C_f)=\mc C_A$, ce qui implique (les deux espaces étant isomorphes) que 
\startformula
\dim(\mc C_f)=\dim(\mc C_A)
\stopformula
En effet, comme $A= \mc Mat_ℬ(f)=ψ(f)$, pour $f∈ℒ(ℝ^n)$, nous remarquons que 
\startformula
\Align{
\NC g∈\mc C_f\NC ⟺ f∘g=g∘f⟺\mc Mat_ℬ(f∘g)=\mc Mat_ℬ(g∘f)\NR
\NC \NC ⟺\mc Mat_ℬ(f)×\mc Mat_ℬ(g)=\mc Mat_ℬ(g)×\mc Mat_ℬ(f)\NR
\NC \NC ⟺\mc A×\mc Mat_ℬ(g)=\mc Mat_ℬ(g)×A\NR
\NC\NC ⟺\mc Mat_ℬ(g)∈\mc C_A\NR
\NC\NC ⟺ ψ(g)∈\mc C_A
}
\stopformula
A fortiori, on a bien $ψ(\mc C_f)=ψ(\mc C_A)$.
\item {\it du Scilab !}\crlf
\starttyping
function [r]=commutables(A,B)
   r = A*B == B*A
endfunction
\stoptyping
\stopList
\item%2
\startList
\item%a
{\it On peut utiliser la méthode MARRE mais on va faire ici plus court/élégant via des équivalences}\crlf
Pour $M∈\mc M_n(ℝ)$, nous remarquons que 
\startformula
\Align{
\NC M∈\mc C_B\NC ⟺M×B=B×M⟺M\Q(A-{\text{tr}(A)\F 2}I_2\W)=\Q(A-{\text{tr}(A)\F 2}I_2\W)M\NR
\NC\NC MA-{\text{tr}(A)\F 2}M=AM-{\text{tr}(A)\F 2}⟺MA=AM\NR
\NC\NC ⟺M∈\mc C_A
}
\stopformula
En particulier, nous avons $\mc C_B=\mc C_A$. 
\item Nous remarquon que 
\startformula
B=\Matrix{
\NC {a-d\F 2}\NC b\NR
\NC c\NC {d-a\F 2}}\qquad B^2=\Matrix{
\NC \Q({a-d\F 2}\W)^2+bc\NC {a-d\F 2}b+b{d-a\F 2}\NR
\NC c{a-d\F 2}+{d-a\F 2}c\NC bc+\Q({d-a\F 2}\W)^2
}=\Q(bc+\Q({d-a\F 2}\W)^2\W)I_2
\stopformula
En particulier, pour $r=bc+\Q({d-a\F 2}\W)^2$ et $P=x^2-rI_2$, nous avons 
\startformula
P(B)=B^2-rI_2=0
\stopformula
De sorte que $P$ est un polynôme annulateur de $B$.
\item Soient $r∈ℝ$ et $N=\Matrix{\NC a\NC b\NR \NC c\NC d}∈\mc M_2(ℝ)$. alors, on a 
\startformula
\Align{[align={right, left,left}]
\NC N∈\mc C_M\NC ⟺\NC NM=MN⟺\Matrix{\NC a\NC b\NR\NC c\NC d}×\Matrix{\NC 0\NC r\NR\NC 1\NC 0}=\Matrix{\NC 0\NC r\NR\NC 1\NC 0}×\Matrix{\NC a\NC b\NR\NC c\NC d}\NR
\NC \NC ⟺\NC \Matrix{
\NC b\NC ar\NR\NC d\NC cr}=\Matrix{
\NC rc\NC rd\NR \NC a\NC b}\NR
\NC \NC  ⟺\NC \System{
\NC b=rc\NR
\NC ar=rd\NR
\NC d=a\NR
\NC cr=b
}\NR
\NC\NC ⟺\NC \System{
\NC b=rc\NR
\NC d=a
}
}
\stopformula
A fortiori, on a
\startformula
\mc C_M=\Q\{aI_2+c\Matrix{\NC 0\NC r\NR\NC 1\NC 0}:(a,c)∈ℝ^2\W\}=\Vect\Q(I_2, \Matrix{\NC 0\NC r\NR\NC 1\NC 0}\W)
\stopformula
De sorte que $\mc C_M$ est de dimension $2$ quel que soit $r∈ℝ$
\item {\it Question bizarrement posée. Il faut lire la question suivante pour voir où ils veulent en venir... }\crlf
Supposons que les trois familles de deux vecteurs $(e_1,f(e_1)$, $(e_2, f(e_2))$ et $(e_1+e_2, f(e_1+e_2))$ soient liées. 
Comme $e_1$, $e_2$ et $e_1+e_2$ sont des vecteurs non nuls, il existe $(a,b,c)∈ℝ^3\ssm\{(0,0,0)\}$ tels que 
\startformula
\System{
\NC f(e_1)=ae_1\NR
\NC f(e_2) = be_2\NR
\NC f(e_1+e_2)=c(e_1+e_2)
}
⟹ 0=f(e_1+e_2)-f(e_1)-f(e_2)=(c-a)e_1+(c-b)e_2=0
\stopformula
Comme $(e_1,e_2$ est libre (puisque c'est une base), nous avons forcément $c-a=0=c-b$, de sorte que $a=c=b$.
Nous en déduisons alors que $f(e_1)=ae_1=a\text{Id}(e_1)$ et $f(e_2)=be_2=ae_2=a\text{Id}(e_2)$.
Comme les applications linéaires $f$ et $a\text{Id}$ coincident sur la base $(e_1,e_2)$, elles sont égales, de sorte que 
\startformula
f=a\text{Id}
\stopformula
Ceci contredit la définition de $f$, car la matrice canoniquement associée à $f$ est $B=A-{\text{tr}(A)\F 2}I_2≠aI_2$, en effet $(A, I_2)$ est libre
\item Il résulte du raisonnement par l'absurde  mené à la question précédente qu'il existe un vecteur $v∈ℝ^2$ tel que $(v, f(v))$ soit libre car les trois familles 
$(e_1, f(e_1)$, $(e_2, f(e_2))$ et $(e_1+e_2, f(e_1+e_2))$ ne peuvent pas être toutes liées. 
A fortiori, la famille libre $𝓓=(v, f(v))$ de deux vecteurs de $ℝ^2$, en constitue une base. \crlf
{\it Accrochez vous !}\crlf
Comme $X^2-r$ est un polynôme annulateur de $B$, c'est également un polynôme annulateur de $f$, de sorte que
$f^2-r\text{Id}=0$. 

En particulier, $f^2(v)=r\text{Id}(v)=rv$. De sorte que 
\startformula
\mc Mat_𝓓(f)=\Matrix{\NC 0\NC r\NR
\NC 1\NC 0}=M
\stopformula
\item {\it Faisons la synthèse des questions précédentes}
Il résulte de la question 2a que $\dim(\mc C_A)=\dim(\mc B)$. Or d'après la question 1d, il vient 
$\dim(\mc C_B)=\dim(\mc C_f)$. Or dans la base $𝓓$, $\mc Mat_𝓓(f)=M$, de sorte que l'on peut montrer que 
$\dim(\mc C_f)=\dim(\mc C_M)$ {\it (un peu comme à la question 1d, sauf qu'au lieu d'utiliser la base canonique, on utilise la base $𝓓$)}
Et enfin, il résulte de la question 2c que $\dim(\mc C_M)=2$ de sorte que $\dim(\mc C_A)=2$. Comme $(A, I_2)$ est une famille libre de $\mc C_A$, c'en est une base.
{\it Une manière bien torturée d'arrivr à ce résultat, dans un cas relativement simple...la dimension $2$}
\stopList
\item%3
\startList
\item%a
{\it On remarque au brouillon que $C_1-C_2=0$ et que $C_1-C_3=0$...}
Les vecteurs $(1,-1,0)$ et $(1,0,-1)$ appartiennent au noyau de $f$ car
\startformula
A×\Matrix{
\NC 1\NR
\NC -1\NR
\NC 0} = \Matrix{
\NC 1-1+0\NR
\NC -1+1+0\NR
\NC 1-1+0
}=0\Et A×\Matrix{
\NC 1\NR
\NC 0\NR
\NC -1} = \Matrix{
\NC 1+0-1\NR
\NC -1+0+1\NR
\NC 1+0-1
}=0
\stopformula
Comme $A$ est clairement de rang $1$, il résulte du thèorème du rang que $\dim(\ker f)=\dim(ℝ^3)-\rg(f)=3-1-2$ de sorte que 
\startformula
\ker(f)=\Vect\big((1,-1,0),(1,0,-1)\big)
\stopformula
Ensuite, l'image de $f$ qui est de dimension $1$, est $\IM(f)=\Vect(f(1,0,0))=\Vect\big((1,-1,1)\big)$.
{\it le sujet demande l'image et le noyau de $a$, un léger abus, il derait demander le noyau et l'image de l'endomorphisme $f$ associé à $A$ dans la base canonique}
\item On remarque que $A^2=\Matrix{
\NC 1-1+1\NC 1-1+1\NC 1-1+1\NR
\NC -1+1-1\NC -1+1-1\NC -1+1-1\NR
\NC 1-1+1\NC 1-1+1\NC 1-1+1}=A$. 

A fortiori, $A$ est la matrice d'un projecteur, donc $f$ est un projecteur.
L'endomorphisme $f$ est donc une projection sur son image $\IM (f)$ déterminée précédemment, parallélement à son noyau, déterminée précédemment également.
\item $P$ est inversible car c'est une matrice de $\mc M_3(ℝ)$ de rang $3$. En effet
\startformula
\rg(P)=\rg\Matrix{
\NC 2\NC 1\NC 0\NR
\NC -1\NC -1\NC 0\NR
\NC 1\NC 0\NC -1}=\rg\Matrix{
\NC 2\NC 1\NC 0\NR
\NC -1\NC -1\NC 0\NR
\NC 0\NC 0\NC -1}=\rg\Matrix{
\NC 1\NC 0\NC 0\NR
\NC -1\NC -1\NC 0\NR
\NC 0\NC 0\NC -1}=3
\stopformula
{\it J'aurais mieux fait d'inverser directement $P$, vu qu'on a aussi besoin de l'inverse...}
En déduisons que la matrice $P$ est inversible, d'inverse 
\startformula
P^{-1}=Q=\Matrix{
\NC 1\NC 1\NC 1\NR
\NC -1\NC -2\NC -1\NR
\NC 1\NC 1\NC 0
},
\stopformula
du calcul
\startformula
Q×P=
\Matrix{
\NC 1\NC 1\NC 1\NR
\NC -1\NC -2\NC -1\NR
\NC 1\NC 1\NC 0
}×\Matrix{
\NC 1\NC 1\NC 1\NR
\NC -1\NC -1\NC 0\NR
\NC 1\NC 0\NC -1
}=\Matrix{
\NC 1-1+1\NC 1-1\NC 1-1\NR
\NC -1+2-1\NC -1+2\NC -1+1\NR
\NC 1-1\NC 1-1\NC 1
}=I_3.
\stopformula
 {\it J'ai fait la recherche de matrice inverse au brouillon bien sûr, avec l'algorithme standard}

 \item $Φ$ est un automorphisme de $\mc M_n(ℝ)$ car c'est un endomorphisme de $\mc M_n(ℝ)$
 \startitemize[1]
\item  $\mc M_n(ℝ)$ est un $ℝ$-espace vectoriel (au départ et à l'arrivée)
\item Pour $M∈ \mc M_n(ℝ)$, $P×M×P^{-1}$ est définit et appartient à $\mc M_n(ℝ)$ en tant que produit de matrices réelles carrées de taille $n$.
De sorte que $Φ:\mc M_n(ℝ)→\mc M_n(ℝ)$ est une application
\item Soient $(λ,μ)∈ℝ²$ et $(M,N)∈\mc M_n(ℝ)^2$, il résulte de la distributivité et de la loi du scalaire mobile que  
\startformula
\Align{
\NC Φ(λM+μN)\NC =P(λM+μN)P^{-1}=P(λM)P^{-1}+P(μN)P^{-1}\NR
\NC\NC =λPMP^{-1}+μPNP^{-1}\NR
\NC\NC=λΦ(M)=+μΦ(N)
}
\stopformula
\stopitemize
De plus, le noyau de $Φ$ est réduit à la matrice nulle car (en multipliant par $P$ à droite et $P^{-1}$ à gauche)
\startformula
M∈\ker(Φ)⟺Φ(M)=0⟺PMP^{-1}=0⟺PM=0×P=0⟺M=P^{-1}×0=0
\stopformula
A fortiori, $Φ$ est injective et même (espace vectoriel de même dimension finie au départ et à l'arrivée) bijective
C'est doncun automorphisme de $\mc M_n(ℝ)$. 
Enfin, pour une matrice $M$ de $\mc M_n(ℝ)$, nous remarquons que $Φ^{-1}(M)=P^{-1}MP$ et aussi que (multiplication par $P^{-1}$ à gauche et par $P$ à droite)
\startformula
\Align{
\NC M∈\mc C_{PAP^{-1}}\NC ⟺M×PAP^{-1}=PAP^{-1}×M\NR
\NC\NC ⟺P^{-1}M×PAP^{-1}=AP^{-1}×M\NR
\NC\NC ⟺P^{-1}M×PA=AP^{-1}×M×P\NR
\NC\NC ⟺P^{-1}MP×A=A×P^{-1}MP\NR
\NC\NC ⟺Φ^{-1}(M)×A=A×Φ^{-1}(M)\NR
\NC\NC ⟺ Φ^{-1}(M)∈\mc C_A\NR
\NC\NC ⟺ M∈Φ(\mc C_A)\NR
}
\stopformula
A fortiori, nous obtenons que $Φ(\mc C_A)=C_{PAP^{-1}}=C_{Φ(A}$
\item Nous remarquons que 
\startformula
\Align{
\NC PAP^{-1}\NC =PAQ=\Matrix{
\NC 1\NC 1\NC 1\NR
\NC -1\NC -2\NC -1\NR
\NC 1\NC 1\NC 0
}×\Matrix{
\NC 1\NC 1\NC 1\NR
\NC -1\NC -1\NC -1\NR
\NC 1\NC 1\NC 1
}×\Matrix{
\NC 1\NC 1\NC 1\NR
\NC -1\NC -2\NC -1\NR
\NC 1\NC 1\NC 0
}\NR
\NC\NC =\Matrix{
\NC 1\NC 1\NC 1\NR
\NC 0\NC 0\NC 0\NR
\NC 0\NC 0\NC 0
}×\Matrix{
\NC 1\NC 1\NC 1\NR
\NC -1\NC -2\NC -1\NR
\NC 1\NC 1\NC 0
}=\Matrix{
\NC 1\NC 0\NC 0\NR
\NC 0\NC 0\NC 0\NR
\NC 0\NC 0\NC 0
}=D }
\stopformula

{\it Pour info, nous venons de diagonaliser la matrice $A$ (algorithme ECS2), c'est sensé simplifier le probleme car $D$ est plus simple que $A$}
Etant donnée une matrice $M=\Matrix{
\NC a\NC b\NC c\NR
\NC d\NC e\NC f\NR
\NC g\NC g\NC i
}$, nous avons
\startformula
\Align{
\NC M∈\mc C_D\NC ⟺MD=DM⟺\Matrix{
\NC a \NC 0\NC 0\NR
\NC d\NC 0\NC 0\NR
\NC g\NC 0\NC 0}=\Matrix{
\NC a\NC b\NC c\NR
\NC 0\NC 0\NC 0\NR
\NC 0\NC 0\NC 0\NR
}\NR
\NC \NC ⟺ M = \Matrix{
\NC a\NC 0\NC 0\NR
\NC 0\NC e\NC f\NR
\NC 0\NC g\NC i
}
}
\stopformula
De sorte que $\mc C_D=\mc C_{PAP^{-1}}=\Vect\Q(D, \Matrix{
\NC 0\NC 0\NC 0\NR
\NC 0\NC 1\NC 0\NR
\NC 0\NC 0\NC 0
},  \Matrix{
\NC 0\NC 0\NC 0\NR
\NC 0\NC 0\NC 1\NR
\NC 0\NC 0\NC 0
},  \Matrix{
\NC 0\NC 0\NC 0\NR
\NC 0\NC 0\NC 0\NR
\NC 0\NC 1\NC 0
},  \Matrix{
\NC 0\NC 0\NC 0\NR
\NC 0\NC 0\NC 0\NR
\NC 0\NC 0\NC 1
}\W)$
et que $\dim(\mc C_A)=\dim(Φ(\mc C_A))\dim(\mc C_D)=5$
{\it Wahh, c'est bourrin quand même...}
\stopList
\item \startList
\item Comme $p$ est un projecteur de $ℝ^n$, on sait que
\startformula
ℝ^n=\ker p ⊕ \IM p
\stopformula
Comme $p$ est de rang $p$, la dimension de son image est $p$.
Soit $e_1,…,e_p$ une base de son image et $e_{p+1}, …,e_n$ une base de son noyau (qui est de dimension $n-p$, d'après l'identité ci-dessus).
Alors, d'après la propriété de concaténation des bases, $ℬ=(e_1, …, e_n)$ est une base de $ℝ^n$ et comme les vecteurs de l'image de $p$ sont  invariants par $p$ 
alors que les vecteurs du noyau de $p$ sont transformés en $0$, on a 
\startformula
\mc Mat_{ℬ}(p)=\Matrix{
\NC 1\NC 0\NC ⋯\NC ⋯\NC ⋯\NC 0\NR
\NC 0\NC ⋱\NC⋱ \NC\NC \NC⋮\NR
\NC⋮\NC ⋱ \NC 1\NC ⋱\NC \NC⋮\NR
\NC⋮\NC  \NC ⋱\NC 0\NC ⋱ \NC⋮\NR
\NC⋮\NC  \NC \NC ⋱\NC ⋱ \NC⋮\NR
\NC 0\NC ⋯ \NC ⋯\NC ⋯\NC ⋯ \NC 0
}=\Matrix{\NC I_r\NC 0\NR\NC 0\NC 0}=J_r,
\stopformula
La matrice ci dessus contenant $r$ $1$ sur la diagonale principale
\item $f$ appartient à $\mc C_p$ si et seulement si $f$ commute avec $p$. Procédons {\it avec la méthode MARRE} par double implication.
Supposons que $p$ commute avec $f$. Alors, pour $x∈\ker(p)$, on a
\startformula
p\big(f(x)\big)=p∘f(x)=f∘p(x)=f\big(p(x)\big)=f(0)=0
\stopformula
De sorte que $p(x)∈\ker(p)$. Le noyau de $p$ est donc stable par $f$.
Par ailleurs, si $y∈\IM(p)$, alors il existe $x∈ℝ^n$ tel que $y=p(x)$ et alors
\startformula
f(y)=f\big(p(x)\big)=f∘p(x)=p∘f(x)=p\big(f(x)\big)∈\IM(p)
\stopformula
De sorte que l'image de $p$ est également stable par $f$. \crlf
Réciproquement, supposons que le noyau et l'image de $p$ soit stable par $f$.
Soit $x∈ℝ^n$. Comme $ℝ^n=\ker p ⊕ \IM p$, il existe un vecteur $y∈\IM(p)$ (et donc $y=p(y)$) et un vecteur $z∈\ker(p)$ (et donc $p(z)=0$) tels que $x=y+z$.
A fortiori, commeles vecteurs de l'image de $p$ sont invariants par $p$, il vient 
\startformula
\Align{
\NC f∘p(x)\NC =f\big(p(x)\big)=f\big(p(y+z)\big)=f\big(p(y)+p(z)\big)=f\big(y+0\big)\NR
\NC \NC =\underbrace{f(y)}_{∈\IM(p)}=p\big(f(y)\big) + 0=p\big(f(y)\big)+p\big(\underbrace{f(z)}_{∈\ker(p)}\big)\NR
\NC\NC =p\big(f(y+z)\big)=p\big(f(x)\big)\NR
\NC\NC=p∘f(x)
}
\stopformula
En particulier, les applications $f∘p$ et $p∘f$ coincident sur $ℝ^n$, donc elles sont égales. De sorte que $p∘f=f∘p$, c'est à dire qu'elles commutent.
{\it Ouch, dur...}
\item La matrice de $f∈\mc C_p$ dans la base de la question a doit avoir la forme $\Matrix{\NC M_r\NC 0\NR\NC 0\NC N_{n-r}}$, 
avec $M_r$ matrice carrée quelconque de taille $r$ et $N_{n-r}$ matrice carrée quelconque de taille $n-r$, 
Car le noyau et l'images de $p$ doivent être stables par $f$, 
{\it $M_r$ est la matrice de la restriction de $f$ à $\IM(p)$ et $N_{n-r}$ est la matrice de la restriction de $f$ à $\ker(p)$}
\item En particulier, comme $\mc C_p$ est l'ensemble des applications $f$ dont la matrice (dans la base de a) vérifie les contraintes énoncées dans la question précédentes, on a 
\startformula
\dim(\mc C_p) = r^2+(n-r)^2
\stopformula
\item $M$ est la matrice d'un projecteur $p$ de rang $r=\rg(M)$. A fortiori, on a 
\startformula
\dim(\mc C_M)= \dim(\mc C_p) = r^2+(n-r)^2=\rg(M)^2+\big(n-\rg(M)\big)^2
\stopformula
\stopList


\centerline{\bf Correction du Problème 2}

{\bf 0.} Comme $\displaystyle\lim_{n→+∞}\arctan(n)=+{π\F 2}$, comme $\displaystyle\lim_{n→∞}{\sin n\F n}=0$ et comme 
\startformula
n\sin{1\F n}∼n×{1\F n}=1→1,
\stopformula
il vient
\startformula
\lim_{n→+∞}M_n=\Matrix{\NC {π\F 2}\NC 0\NR\NC 1\NC 3}
\stopformula
\startList
\item Soit $n∈ℕ$. Slors, nous déduisons du changement d'indice $i'=i-r$ et de la $r$-périodicité de la suite $x$ que
\startformula
\Align{
\NC \displaystyle{1\F r}∑_{i=n}^{n+r-1}x_i-y\NC\displaystyle={1\F r}∑_{i=n}^{n+r-1}x_i-{1\F r}∑_{i=0}^{r-1}x_i\NR
\NC \NC \displaystyle= {1\F r}\Q(∑_{i=r}^{n+r-1}x_i-∑_{i=0}^{n-1}x_i\W)\NR
\NC\NC \displaystyle= {1\F r}\Q(∑_{i'=0}^{n-1}x_{i'+r}-∑_{i=0}^{n-1}x_i\W)\NR
\NC\NC \displaystyle= {1\F r}\Q(∑_{i'=0}^{n-1}x_{i'}-∑_{i=0}^{n-1}x_i\W)=0\NR
}
\stopformula
En particulier, nous obtenons que $\displaystyle y={1\F r}∑_{i=n}^{n+r-1}x_i$ pour $n∈ℕ$.
\item Pour $n⩾1$, nous remarquons que 
\startformula
\Align{
\NC w_{n+r}\NC =\displaystyle (n+r)y_{n+r}-(n+r)y=(n+r){1\F n+r}∑_{i=0}^{n+r-1}x_i-(n+r)y\NR
\NC \NC \displaystyle= ∑_{i=0}^{n+r-1}x_i-(n+r)y\NR
\NC \NC \displaystyle= ∑_{i=0}^{n-1}x_i + ∑_{i=n}^{n+r-1}x_i-(n+r)y\NR
\NC \NC \displaystyle= ny_n + \underbrace{∑_{i=n}^{n+r-1}x_i}_{ry}-(n+r)y\NR
\NC \NC = ny_n -ny=w_n
}
\stopformula
En particulier, la suite $(w_n)_{n⩾1}$ est $r$-périodique.
\item Une suite $r$-périodique $(x_n)_{n⩾0}$ est bornée car 
\startformula
|x_k|⩽M=\max\big\{|x_0|, ⋯, |x_{r-1}|\big\}\qquad(k⩾0)
\stopformula
En effet, pour $k∈ℕ$, il résulte de la division euclidienne de $k$ par $r$ qu'il existe deux entiers $q⩾0$ et $s∈⟦0,r-1⟧$ tel que $k=qr+s$ et alors, on a 
\startformula
|x_k|=|x_{qr+s}|=|w_s|⩽M
\stopformula
\item Comme la suite $w$ est $r$-périodique, elle est bornée, de sorte qu'il existe un nombre réel $M$ vérifiant $|w_n|⩽M$ pour $n⩾1$.
Alors, nous obtenons que 
\startformula
|y_n-y|=\Q|{w_n\F n}\W|={|w_n|\F n}⩽{M\F n}\qquad (n⩾1).
\stopformula
Comme $\displaystyle\lim_{n→+∞}{M\F n}=0$, il résulte de la propriété des gendarmes que 
\startformula
\displaystyle\lim_{n→+∞}(y_n-y)=0
\stopformula
et donc que $\displaystyle\lim_{n→+∞}y_n=y$.
\stopList
\item \startList
\item Nous remarquons que 
\startformula
\System{
\NC B-{1\F 6}C=A\NR
\NC B+C=I_2}⟺\System{
\NC B-{1\F 6}C=A\NR
\NC {7\F 6}C=I_2-A}⟺
\System{
\NC B=A+{1\F 7}(I_2-A)={1\F 7}(I_2+6A)={1\F 7}\Matrix{\NC 3\NC 4\NR \NC 3\NC 4}\NR
\NC C={6\F 7}(I_2-A)={6\F 7}\Matrix{\NC {2\F 3}\NC {-2\F 3}\NR \NC{-1/2}\NC {1\F 2}}
={1\F 7}\Matrix{\NC 4\NC -4\NR \NC -3\NC 3}
}
\stopformula
\item De simples calculs matriciels donnent
\startformula
\Align{
\NC B^2\NC ={1\F 7^2}\Matrix{\NC 21\NC 28\NR \NC 21\NC 28}={1\F 7}\Matrix{\NC 3\NC 4\NR \NC 3\NC 4}=B\NR
\NC C^2\NC ={1\F 7^2}\Matrix{\NC 28\NC -28\NR \NC -21\NC 21}={1\F 7}\Matrix{\NC 4\NC -4\NR \NC -3\NC 3}=C\NR
\NC BC\NC ={1\F 7^2}\Matrix{\NC 12-12\NC -12+12\NR \NC 12-12\NC -12+12}=0\NR
\NC CB\NC ={1\F 7^2}\Matrix{\NC 12-12\NC 16-16\NR \NC -9+9\NC -12+12}=0\NR
}
\stopformula
En particulier, $B$ et $C$ sont des matrices de projecteurs qui commutent.
\item Pour $n⩾1$, il résulte alors du binôme de Newton ($B$ et $C$ commutent) que
\startformula
\Align{
\NC A^n\NC \displaystyle=\Q(B-{1\F 6}C\W)^n=∑_{k=0}^n{n\choose k}\Q(-{1\F 6}C\W)^kB^{n-k}\NR
\NC\NC \displaystyle=B^n+\Q(-{1\F 6}C\W)^n+∑_{k=1}^{n-1}{n\choose k}\Q(-{1\F 6}C\W)^kB^{n-k}\NR
\NC\NC \displaystyle=B+\Q(-{1\F 6}\W)^nC+∑_{k=1}^{n-1}{n\choose k}\Q(-{1\F 6}\W)^k\underbrace{CB}_{0}\NR
\NC\NC \displaystyle=B+\Q(-{1\F 6}\W)^nC
}
\stopformula
En particulier, nous obtenons que $A^n=B+\Q(-{1\F 6}\W)^nC∈\Vect(B,C)$
\item Il résulte du calcul précédent que $\lim_{n→+∞}A^n=B$. Par ailleurs, nous déduisons de la formule des sommes géométriques que 
\startformula
\Align{
\NC P_n(A)\NC \displaystyle={1\F n}∑_{k=0}^{n-1}A^k={1\F n}\Q(I_2+∑_{k=1}^{n-1}\Q(B+\Q(-{1\F 6}\W)^kC\W)\W)\NR
\NC\NC \displaystyle={1\F n}\Q(I_2+(n-1)B+ \Q(∑_{k=1}^{n-1}\Q(-{1\F 6}\W)^k\W)C\W)\NR
\NC\NC \displaystyle={1\F n}\Q(I_2+(n-1)B+ {-{1\F 6}-\Q(-{1\F 6}\W)^n\F 1+{1\F 6}}C\W)
}
\stopformula
En particulier, nous obtenons que $\displaystyle\lim_{n→+∞}P_n(A)=B$
\stopList
\item\startList
\item Nous remarquons que 
\startformula
M^2=\Matrix{
\NC 0\NC 0\NC 1\NR
\NC 1\NC -3\NC 3\NR
\NC 3\NC -8\NC 6}\qquad\Et\qquad 
M^3=\Matrix{
\NC 1\NC -3\NC 3\NR
\NC 3\NC -8\NC 6\NR
\NC 6\NC -15\NC 10}
\stopformula
Alors, nous remarquons que 
\startformula
M^3-3M^2+3M-I_3=\Matrix{
\NC 1-1\NC -3+3\NC 3-3\NR
\NC 3-3\NC -8+9-1\NC 6-9+3\NR
\NC 6-9+3\NC -15+24-9\NC 10-18+9-1}=0
\stopformula
de sorte que le polynôme unitaire $P=X^3-3X^2+3X-1$ est un polynôme annulateur de $M$. 
\item Procédons à la division euclidienne (théorique) de $X^n$ par $P$. Alors, il existe deux polynômes $Q∈ℝ[X]$ et $P∈R_2[X]$ uniques tels que $X^n=PQ+R$.
{\it On veut surtout le reste de la division en fait...}

Comme $P=X^3-3X^2+3X-1=(X-1)^3$, en substituant $1$ à $X$ puis en dérivant et en substituant $1$ à $X$, puis en dérivant $2$ fois et en substituant $1$ à $X$ dans 
\startformula
X^n=PQ+R=(X-1)^3Q+a+bX+cX^2,
\stopformula
il vient
\startformula
\System{
\NC a+b+c = 1\NR
\NC b+2c = n\NR
\NC 2c = n(n-1)
}⟺ \System{
\NC c={n(n-1)\F 2}\NR
\NC b=n-n(n-1)=n(2-n)\NR
\NC a=1-b-c=1-{n(n-1)\F 2}-n(2-n)={n^2-3n+2\F 2}
}
\stopformula

En notant $R=a+bX+cX^2$ et en substituant $M$ à $X$ il vient
\startformula
\Align{
\NC X^n\NC =P(M)Q(M)+R(M)=0+aI_3+bM+cM^2\NR
\NC\NC ={n^2-3n+2\F 2}I_3+n(2-n)M+{n(n-1)\F 2}M^2\qquad(n⩾1)
}
\stopformula
\item La suite $\big(P_n(M)\big)_{n⩾1}$ n'est pas convergente. {\it Prouvons le par l'absurde en essayant d'éviter de souffrir}.
En effet, si elle était convergente alors 
\startformula
(I_3-M)×P_n(M)={1\F n}(1-M)∑_{k=0}^{n-1}M^k={1\F n}\Q(I_3-M^n\W)
\stopformula
le serait aussi et en particulier, la suite ${1\F n}M^n$ serait convergente, ce qui n'est clairement pas le cas, vu l'expression trouvée à la question précédente.
\stopList
\item\startList
\item {\it Wow, des permutations maintenant et le symbôle de Kronecker...qui n'est pas expliqué, c'est mal...
\startformula
δ_{i,j}=\System{
\NC 1 \NC \Si i=j\NR
\NC 0 \NC \Si i≠j}
\stopformula
Autrement dit, la matrice $A_σ$ a un $1$ par ligne. Sur la ligne $i$, le $1$ est situé sur la colonne de même rang que $σ(i)$, l'image par $i$ de la permutation
}
\item Notons $A_σ=(a_{i,j})_{1⩽i⩽n\atop 1⩽j⩽n}$, $A_τ=(b_{i,j})_{1⩽i⩽n\atop 1⩽j⩽n}$ et $A_σA_τ=(c_{i,j})_{1⩽i⩽n\atop 1⩽j⩽n}$. Soit $i∈⟦1,n⟧$ et $j∈⟦1,n⟧$. D'après le cours, on a 
\startformula
\Align{
\NC c_{i,j}\NC \displaystyle=∑_{k=1}^na_{i,k}b_{k,j}=∑_{k=1}^n\underbrace{δ_{i, σ(k)}δ_{k,τ(j)}}_{\rlap{=0 \text{ si $i≠σ(k)$ ou $k≠τ(j)$ et $=1$ sinon}}}\NR
\NC\NC \displaystyle= δ_{i, σ(σ^{-1}(i))}δ_{σ^{-1}(i),τ(j)}=δ_{σ^{-1}(i),τ(j)}=δ_{i,σ(τ(j))}
}
\stopformula
(la seule valeur de $k$ pouvant donner autre chose que $0$ est $k=σ^{-1}(i)$ et on a également $σ^{-1}(i)=τ(j) ⟺i=σ(τ(j))$ ). En particulier, on obtient que la matrice $A_σA_τ$ a les mêmes coefficients que la matrice $A_{σ∘τ}$, de sorte que 
\startformula
A_σA_τ=A_σ∘τ
\stopformula
{\it Pour info, les MP* font ce genre de question relativement souvent et ils en ont plusieurs comme cela à tous leurs ds...
La, il valait mieux l'admettre peut être ^^}
\item On a $A_{Id}=I_p$ (les $1$ sont tous sur la diagonale principale) et comme $A_σA_{σ^{-1}}=A_{σ∘σ^{-1}}=A_{Id}=I_p$, nous remarquons que la matrice $A_σ$ est inversible, d'inverse 
$A_{σ^{-1}}$. 
\item Il y a $p!$ permutations de $⟦1,p⟧$ et il y a donc autant de matrices de permutation
\item Comme il n'y a qu'un nombre fini de matrice de permutations et comme les matrices $(A_σ)^n=A_{σ^n}$ sont toutes des matrices de permutation pour $n⩾1$, il y en a au moins eux qui sont égales. autrement dit, 
il existe deux entiers distincts $p$ et $q$ tels que $(A_σ)^p=(A_σ)^q$. 
{\it Wow, on fait ce genre de raisonnement quand on fait de la théorie des groupes... ça a l'air fun fun la prépa ECS à st louis...}
\item Comme $(A_σ)^p=(A_σ)^q$ et comme ces matrice sont inversibles, on remarque que 
\startformula
(A_σ)^p\Q((A_σ)^q\W)^{-1}=I_r=(A_σ)^p(A_σ)^{-q}=(A_σ)^{p-q}
\stopformula
En particulier, il existe un entier $r=p-q≠0$ tel que $(A_σ)^r=I_p$
{\it en prenant pour $p$ le plus grand des deux entiers et pour $q$ le plus petit des entiers, on obtient également que $r=q-p>0$ de sorte que $r⩾1$}
\item La suite $(A_σ)^n$ est $r$-périodique car 
\startformula
(A_σ)^{n+r}=(A_σ)^n(A_σ)^r=(A_σ)^nI-r=(A_σ)^n\qquad(n⩾1)
\stopformula
\item {\it Bon, la je crois que l'auteur souhaite nous voir faire un raisonnement avec les matrices identique à celui utilisé pour obtenir le lemme au 1 (que l'on a jamais utilisé).
Si on fait le parallèle, la suite $(A_σ)^n$ est $r$-périodique et donc $P_n(A_σ)$ est sensé converger d'après le lemme vers $y={1\F n}∑_{k=0}^{r-1}(A_σ)^k$, etc...\crlf
bref, passons à l'exercice
}
\stopList
\stopList


\centerline{Correction de l'exercice}

{\it Préparons nous à un festival de méthode MARRE}
\startList
\item Soit $k∈ℕ$.  
\startitemize[1]
\item Montrons que $N_k⊂N_{k+1}$. Soit $x∈N_k=\ker(f^k)$. Alors $f^k(x)=0$ de sorte que $f^{k+1}(x)=f\Q(f^k(x)\W)=f(0)=0$ et par conséquent $x∈\ker(f^{k+1})=N_{k+1}$. CQFD
\item Montrons que $I_{k+1}⊂I_k$. Soit $y∈I_{k+1}=\IM(f^{k+1})$. Alors il existe $x∈ℝ^n$ tel que $y=f^{k+1}(x)$ et alors, on remarque que $y=f^k(f(x))∈\IM(f^k)=I_k$. CQFD
\stopitemize 
\item Supposons qu'il n'existe pas d'entiers $p⩽n$ tel que $N_p=N_{p+1}$. Alors la suite croissante de noyau $N_1⊂N_2⊂N_3⊂⋯⊂N_n⊂N_{n+1}$ est strictement croissante (aucun noyau n'est égale au suivant), de sorte qu'en pmrenant leur dimension, on obtient que 
\startformula
\dim(N_1)<\dim(N_2)<\dim(N_3)<⋯<\dim(N_{n+1})
\stopformula
Mais alors, on obtient une suite strictement croissante de $n+1$ entiers de $⟦0,n⟧$. De sorte que $\dim(N_1)=0$, d'où $f$ est injective et donc bijective mais dans ce cas tous les noyaux $N_k$ valent $\{0\}$ et on a notre contradiction...
CQFD, il existe au moins un entier $p⩽n$ tel que $N_p=N_{p+1}$.
{\it Il y a une idée importante à retenir ici : quand une application $f$ a un noyau de dimension non nulle, les noyaux itérés de $f$ ont une dimension qui augmente strictement puis qui devient constante}
\item $I_{p+1}=I_p$ d'après a) Et il résulte du théorème du rang que 
\startformula
\dim(I_{p+1})=\dim(ℝ^n)-\dim(N_{p+1})=\dim(ℝ^n)-\dim(N_p)=\dim(I_p)
\stopformula
Comme les noyaux ont même dimension finie et l'un est inclus dans l'autre, on a $I_{p+1}=I_p$. 
\item Pour $k∈ℕ$, prouvons par récurrence la proposition $\mc P_k:N_{p+k}=N_p$. 
\startitemize[1]
\item Les propositions $\mc P_0$ et $\mc P_1$ sont vraies d'après le résultat de la question b 
\item Supposons la proposition $\mc P_k$ pour un entier $k⩾0$. 

Rappelons que $N_k⊂N_{p+k+1}$ (suite croissante de noyaux) et montrons maintenant que $N_{p+k+1}⊂N_p$.

Soit $x∈N_{p+k+1}=\ker(f^{p+k+1})$. alors, on a $0=f^{p+k+1}(x)=f^{p+k}\big(f(x)\big)$. A fortiori, $f(x)∈\ker(f^{p+k})=N_{p+k}=N_p=\ker(f^p)$ de sorte que 
\startformula
0=f^p\big(f(x)\big)=f^{p+1}(x).
\stopformula
En particulier $x∈\ker(f^{p+1})=N_{p+1}=N_p$. CQFD

En conclusion, $N_{p+k+1}⊂N_p$ et $N_p=N_{p+k+1}$ de sorte que la proposition $\mc P_{k+1}$ est vraie
\stopitemize
En conclusion, la proposition $\mc P_k$ est vraie pour $k∈ℕ$. 
\item On remarque que $N_p∩I_p=\{0\}$. En effet, si $y∈N_p∩I_p$, alors $f^p(y)=0$ et il existe un $x∈ℝ^n$ tel que $y=f^p(x)$ mais alors $0=f^p(y)=f^p(f^p(x))=f^{2p}(x)$
et donc $x∈N_{2p}=N_p$ de sorte que $y=f^p(x)=0$. .
En particulier, nous obtenons que $I_p⊕N_p$. Par ailleurs, il résulte du thèorème du rang appliqué à la fonction $f^p$ que 
$\dim(ℝ^n)=\dim\ker (f^p)+\dim(\IM(f^p)=\dim N_p+\dim I_p=\dim(N_p⊕I_p)$.
Comme $I_p⊕N_p⊂ℝ^n$, il vient
\startformula
ℝ^n=I_p⊕N_p
\stopformula
{\it magnifique...}

\stopList







\stoptext
\stopcomponent
\endinput
