\startcomponent component_DS1
\project project_Res_Mathematica
\environment environment_Maths
\environment environment_Inferno
\xmlprocessfile{exo}{xml/Limbo_Exercices.xml}{}
\iffalse
\setupitemgroup[List][1][R,inmargin][after=,before=,left={\bf Exo },symstyle=bold,inbetween={\blank[big]}]
\setupitemgroup[List][2][n,joineup][after=,before=,inbetween={\blank[small]}]
\setupitemgroup[List][3][a,joineup][after=,before=,inbetween={\blank[small]}]
\setupitemgroup[List][4][1,joineup,nowhite]
\fi

\setupitemgroup[List][1][R,inmargin][after=,before=,left={\bf Exo },symstyle=bold,inbetween={\blank[big]}]
\setupitemgroup[List][2][A,joineup][after=,before=,inbetween={\blank[small]}]
\setupitemgroup[List][3][n,joineup][after=,before=,inbetween={\blank[small]}]
\setupitemgroup[List][4][a,joineup,nowhite]

\setuppapersize[A4]
\setuppagenumbering[location=]
\setuplayout[header=0pt,footer=0pt]

\starttext
\setupheads[alternative=middle]
%\showlayout
\def\gah#1{\margintext{Exercice #1}}
\centerline{\bfb DEVOIR SURVEILLE 1}
\blank[big]

%S1 suites de référence, S2 logique,S3 récurrence, produits, sommes, S4 R, suites | S5 ensembles, applications, S6 limites et C°

%1617-DS-02-cor S1 + S4 bof
%/home/lakedaemon/Dropbox/Belial_Maths/ECS_2017_2018/Devoirs/DS2__suites_ensembles_applications_continuité/ds01.pdf 
% Exo 6 (S1 arithmético-géométrique)
% EXO 7 suites presque homeographique (+ fonctions)
% exo 8 (recurrence, factorielles)
%/home/lakedaemon/Dropbox/Belial_Maths/ECS_2017_2018/Devoirs/DS2__suites_ensembles_applications_continuité/DS1_osezene.pdf
% EXO 3 suites et tableaux de variation
% /home/lakedaemon/Dropbox/Belial_Maths/ECS_2017_2018/Devoirs/DS2__suites_ensembles_applications_continuité/DS_20152016.pdf
%EXO 1 page 18 (suites et récurrence)
% /home/lakedaemon/Dropbox/Belial_Maths/ECS_2017_2018/Devoirs/DS2__suites_ensembles_applications_continuité/DS_20142015.pdf
% exo 4 p 2 fonctions et suites récurrentes
\startList\item%Exo
Les parties A, B et C de cet exercice peuvent être traitées de manière indépendante.
Soit $f$ la fonction définie par $\D f(x)={x^2\F 2x-1}$
\startList\item\startList\item Donner le domaine de définition $\𝓓f$ de la fonction $f$
\item Etudier les variations de la fonction $f$
\item Montrer qu’il existe des réels $a$, $b$ et $c$ tels que $f(x)=ax+b+{c\F 2x-1}$ pour $x∈\𝓓f$ 
et donner la valeur de ces trois réels
\item Donner une équation de la tangente à la courbe de $f$ au point d'abscisse $2$.
\stopList
\item On définit une suite $u$ en posant $u_0=2$ et $u_{n+1}=f(u_n)$ pour $n⩾0$
\startList\item Démontrer que $f(x)⩾1$ pour $x⩾1$.
\item Montrer que $u_n⩾1$ pour $n∈ℕ$
\item Résoudre l'équation $f(x)=x$
\item Etudier la monotonie de la suite $u$
\stopList
\item On se propose  d’exprimer $u_n$ en fonction de $n$. Pour cela, 
on pose $v_n={u_n-1\F u_n}$ et $w_n=\ln(v_n)$ pour $n∈ℕ$. 
\startList\item Vérifier que $v_n$ et $w_n$ sont définis pour tout entier naturel $n$.
\item Démontrer que $w$ est une suite géométrique
\item Pour $n∈ℕ$, exprimer $w_n$ puis $v_n$ en fonction de $n$ et en déduire que
\startformula
u_n={1\F 1-\Q({1\F 2}\W)^{2n}}
\stopformula
\stopList
%\item\filterpages[2015/DS1/ds2.pdf][2][width=18cm]
%\item\filterpages[2016/DS1/DS1_2013.pdf][2][width=18cm]%Exo 3
%\item\filterpages[2016/DS1/DS1_2015.pdf][2][width=18cm]%Exo 2
%\item\filterpages[2016/DS1/DS1(2009-2010).pdf][2][width=18cm]%Sommes
\stopList



\item%Exos
Pour $n∈ℕ$, on pose $u_n={2n\choose n}={(2n)!\F n!^2}$
\startList\item Pour $n∈ℕ$, simplifier le quotient $u_{n+1}\F u_n$. En déduire que $u_{n+1}⩾2u_n$. 
\item Démontrer par récurrence que $u_{n+1}⩾2^n$ pour $n∈ℕ$.
\item En déduire $\D\lim_{n→+∞}u_n$.
\stopList

\item%Exos
Soit $α$ un nombre réel strictement positif. On définit la suite $u$ par $\System{
[align=left]
\NC u_0=α\NR
2u_{n+1}=3u_n^2\qquad(n∈ℕ)}$
\startList\item Montrer que $u_n>0$ pour $n∈ℕ$. alors, on pose $v_n=\ln u_n$ pour $n∈ℕ$.
\item Montrer que la suite $v$ est arithmético-géométrique
\item Déterminer une expression de $v_n$ en fonction de $n$
\item En déduire une expression de $u_n$ en fonction de $n$
\item A quelle condition portant sur $α$, la suite $u$ est-elle convergente ?
\stopList

\item \exo{C85}

\stopList%exos


\iffalse
\page
\centerline{\bfb CORRECTION DU DEVOIR SURVEILLE 1}
\blank[big]
\startList\item%\item\filterpages[2015/DS1/ds2-correction.pdf][1,2][width=18cm]
\%item\filterpages[2016/DS1/corrige-DS1_2013.pdf][2][width=18cm]
%\item\filterpages[2016/DS1/corrige-DS1_2015.pdf][1,2][width=18cm]
%\item\filterpages[2016/DS1/DS1(2009-2010).pdf][6,7][width=18cm]


\item \solution{C85}
\stopList
\fi
\stoptext
\stopcomponent
\endinput
