\startproduct Limbo
\project project_Poly
\setupCourse{e}
%\definereferenceformat[eqref][left=(,right=)]
%\definehighlight[avoid][color=darkgray]
\iffalse

\olspept
\DefineRGBcolor F0F9E3=VLGreen.
\DefineRGBcolor E5F9D1=LGreen.
\DefineRGBcolor DAF9BE=TGreen.
\DefineRGBcolor 5DA93B=Green.
\DefineRGBcolor F6DCCA=VLRed.
\DefineRGBcolor F6D4BD=LRed.
\DefineRGBcolor DAF9BE=TRed.
\DefineRGBcolor B5F9A1=TTRed.
\DefineRGBcolor F6B080=Red.
\DefineRGBcolor F9F5E3=VLOrange.
\DefineRGBcolor F9F5D0=LOrange.
\DefineRGBcolor DAF9BE=TOrange.
\DefineRGBcolor B5F9A1=TTOrange.
\DefineRGBcolor D7A93B=Orange.
\DefineRGBcolor EEEEEE=VLBlack.
\DefineRGBcolor DDDDDD=LBlack.
\DefineRGBcolor CCCCCC=TBlack.
\DefineRGBcolor B5F9A1=TTBlack.
\DefineRGBcolor 000000=Black.

\def\transparent{%
	\CS long\EC\def\Demonstration##1\CQFD{}%
}%
%\transparent
\def\Students{%
	\DefineRGBcolor FFFFFF=VLGreen.
	\DefineRGBcolor FFFFFF=LGreen.
	\DefineRGBcolor FFFFFF=TGreen.
	\DefineRGBcolor 000000=Green.
	\DefineRGBcolor FFFFFF=VLRed.
	\DefineRGBcolor FFFFFF=LRed.
	\DefineRGBcolor FFFFFF=TRed.
	\DefineRGBcolor FFFFFF=TTRed.
	\DefineRGBcolor 000000=Red.
	\DefineRGBcolor FFFFFF=VLOrange.
	\DefineRGBcolor FFFFFF=LOrange.
	\DefineRGBcolor FFFFFF=TOrange.
	\DefineRGBcolor FFFFFF=TTOrange.
	\DefineRGBcolor 000000=Orange.
	\DefineRGBcolor FFFFFFF=VLBlack.
	\DefineRGBcolor FFFFFF=LBlack.
	\DefineRGBcolor FFFFFF=TBlack.
	\DefineRGBcolor FFFFFF=TTBlack.
	\DefineRGBcolor 000000=Black.
}
\fi




\definecolor[myGreen][r=0.55, g=0.76, b=0.29]%
\definecolor[lighterYellow][r=0.98, g=0.75, b=0.18]% 
\definecolor[myYellow][r=0.98, g=0.66, b=0.15]%
\definecolor[darkerYellow][r=0.96, g=0.50, b=0.09]% 
\definecolor[myBlue][r=0, g=0.51, b=0.56]%
\definecolor[myMagenta][r=0.61, g=0.15, b=0.69]% 
%before={\startaxiomframe},after={\stopaxiomframe}
\def\colorA#1{{\myBlue #1}}
\def\colorB#1{{\myMagenta #1}}
\def\later#1{{\myMagenta\it #1}}
\def\interesting#1{{\myBlue\it #1}}
\def\identityOne#1{#1}
\def\shouldTest#1{$\underline{\text{#1}}$}
\def\displayCommand#1{#1}
\defineenumeration[definition][command=\myGreen,left={\somenamedheadnumber{chapter}{current}.},location=serried,width=fit,text={\displayCommand{Définition}},location=serried,counter=property,alternative=left,title=yes,style=normal,list=all,listtext={Définition }]
\defineenumeration[theorem][command=\darkerYellow,left={\somenamedheadnumber{chapter}{current}.},location=serried,width=fit,text={\displayCommand{Théorème}},counter=property,title=yes,style=normal,list=all,listtext={Théorème }]
\defineenumeration[property][command=\myYellow,left={\somenamedheadnumber{chapter}{current}.},location=serried,width=fit,text={\displayCommand{Propriété}},title=yes,way=bychapter,style=normal,list=all,listtext={Propriété }]
\defineenumeration[corollary][counter=property,command=\lighterYellow,left={\somenamedheadnumber{chapter}{current}.},location=serried,width=fit,text={\displayCommand{Corollaire}},title=yes,style=normal,list=all,listtext={Corollaire }]
\defineenumeration[notation][command=\myBlue,left={\somenamedheadnumber{chapter}{current}.},location=serried,width=fit,text={\displayCommand{Notation}},counter=property,title=yes,style=normal,list=all,listtext={Notation }]
\defineenumeration[convention][command=\myBlue,left={\somenamedheadnumber{chapter}{current}.},location=serried,width=fit,text={\displayCommand{Convention}},counter=property,title=yes,style=normal,list=all,listtext={convention }]

\defineenumeration[craft][command=\myMagenta,left={\somenamedheadnumber{chapter}{current}.},location=serried,width=fit,text={\displayCommand{Méthode}},counter=property,title=yes,style=normal,list=all,listtext={Méthode }]

`\catcode`𝕂=\active
\def𝕂{ℝ}%

\iffalse
\definetextbackground
[axiomframe]
[
mp=background:random,
location=paragraph,
rulethickness=1pt,
width=broad,
leftoffset=1em,
rightoffset=1em,
before={\testpage[3]\blank[3*big]},
after={\blank[3*big]}
]

\startuseMPgraphic{background:random}
path p;
for i = 1 upto nofmultipars :
p = (multipars[i]
topenlarged 10pt
bottomenlarged 10pt) randomized 4pt ;
fill p withcolor lightgray ;
draw p withcolor \MPvar{linecolor}
withpen pencircle scaled \MPvar{linewidth};
endfor;
\stopuseMPgraphic
\fi

\starttext
\centerline{\bfd Mathématiques}
\centerline{\tf (poly illustré de cours)}
\blank[big]
\centerline{\bfb ECS 1ère année}
\blank[big]
\centerline{Olivier Binda}
\blank[medium]
\centerline{\currentdate}
\stoptext

%\ctxlua{L.test()}%


\def\pn{\medskip}

% defines unnumbered domain
% see http://wiki.contextgarden.net/Titles#Unnumbered_titles_in_table_of_contents
% see http://wiki.contextgarden.net/Table_of_Contents#Including_unnumbered_heads_in_the_ToC
\definehead[domain] [chapter]
\setuphead[domain] [incrementnumber=list]
\setuplist[domain][width=0em,style=bold]  
\setuplist[chapter][width=1.5em]  
\setuplist[section][width=2.5em]                                              
\setuplist[subsection][width=3.5em, margin=2.5em]   
\setuplist[subject][margin=1.5em]
\setupcombinedlist[content][list={domain, chapter,section,subsection}]


%\defineitemgroup[List][levels=4]
\setupitemgroup[List][1][n,joinedup,nowhite]
\setupitemgroup[List][2][a,joineup,nowhite]
\setupitemgroup[List][3][5,joineup,nowhite]
\setupitemgroup[List][4][6,joineup,nowhite]


%defineitemgroup[Set][levels=4]
\setupitemgroup[Set][1][1,joinedup,nowhite]
\setupitemgroup[Set][2][2,joineup,nowhite]
\setupitemgroup[Set][3][3,joineup,nowhite]
\setupitemgroup[Set][4][4,joineup,nowhite]

\applySettups{introduction}
\applySettups{suites}
\applySettups{séries}
\applySettups{sommes}
\applySettups{logique}
\applySettups{ensembles}
\applySettups{C}
\applySettups{fonction}
\applySettups{limf}
\applySettups{continuité}
\applySettups{dérivation}
\applySettups{fn}
\applySettups{dl}
\applySettups{intégration}
\applySettups{intégrales}
\applySettups{systemes}
\applySettups{matrices}
\applySettups{ev}
\applySettups{polynomes}
\applySettups{probabilités}
\applySettups{VAR}
\applySettups{poly}
%\applySettups{maths}
% Fondements théoriques
% alphabet grec
%\enabletrackers[lxml.loading,lxml.setups]
\xmlprocessfile{poly}{xml/ECS_poly.xml}{}
%\startdomain[title=Bases théoriques
%\xmlprocessfile{maths}{xml/ECS_Avant_propos.xml}{}
% A1.1 (propositions)
%\xmlprocessfile{maths}{xml/ECS_Logique.xml}{}
% A1.2 (suites)
%\xmlprocessfile{maths}{xml/ECS_sommes_produits_récurrences.xml}{}
\iffalse% A1.3 (fonctions)
\xmlprocessfile{maths}{xml/ECS_Ensembles_et_applications.xml}{}
% A6.4 combinatoire


% A4.1 ℝ 
\stopdomain

\startdomain [title=Analyse]
% Analyse (suites) 
%A4.2 exemples suites réelles
\xmlprocessfile{maths}{xml/ECS_Suites_fondamentales.xml}{}
%A4.3 suites réelles 
\xmlprocessfile{maths}{xml/ECS_Suites.xml}{}
%B2.1 etude asymptotique
%B2.3 Séries

% Analyse (fonctions)
% A5.1 limites et continuité fonctions réelles
% A5.2 etude globale sur un intervalle
\xmlprocessfile{maths}{xml/ECS_Continuité_limites.xml}{}
\xmlprocessfile{maths}{xml/Limbo_Continuité.xml}{}
% A5.3 dérivation
% B2.8 extrema
% B2.9 fonctions convexes
\xmlprocessfile{maths}{xml/Limbo_Dérivation.xml}{}
% B2.5 dérivées successives
% A5.4 intégration sur un segment + sommes de Riemann
\xmlprocessfile{maths}{xml/ECS_Dérivées_et_primitives.xml}{}
\xmlprocessfile{maths}{xml/Limbo_Intégration.xml}{}
%B2.6 Formule de Taylor
%B2.2 comparaison des fonctions
%B2.7 développements limités
%B2.4 integrales generalisées

\xmlprocessfile{maths}{xml/Limbo_Développements_limités.xml}{}


\stopdomain 
\startdomain[title=Algèbre] 

% Algèbre (non-linéaire)
%A2.1
\xmlprocessfile{maths}{xml/Limbo_Nombres_complexes.xml}{}
%A2.2
\xmlprocessfile{maths}{xml/ECS_Polynômes.xml}{}

% Algèbre (linéaire)
%A3.2
\xmlprocessfile{maths}{xml/ECS_Systèmes.xml}{}
%A3.1
\xmlprocessfile{maths}{xml/ECS_Matrices.xml}{}
%A3.3
\xmlprocessfile{maths}{xml/ECS_Espaces_vectoriels.xml}{}
%B1.1 Espaces de dim finie
%B1.2 sommes directes supplémentaires
%B1.3 Applications linéaires 



\stopdomain
\startdomain[title=Probabilités]

% Probabilités
% A6.1 Univers fini : Expériences aléatoires, événements, probabilité, probabilité conditionnelles, indépendance
\xmlprocessfile{maths}{xml/ECS_Probabilités.xml}{}
% A6.2 VAR réelle (finie)
\xmlprocessfile{maths}{xml/ECS_Variables_aléatoires.xml}{}
% A6.3 loi usuelles


%B3.1 espace probabilisé quelconque
%B3.2 VAR réelles
%B3.3 VAR discretes
%B3.4 loi discretes usuelles
%B3.5 VAR à densité
%B3.6 loi à densité usuelles
%B3.7 convergence et approximation

\fi

\starttext
\completecontent[criterium=all]
\stoptext

\stopproduct
