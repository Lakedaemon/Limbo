\startcomponent component_DS1
\project project_Res_Mathematica
\environment environment_Maths
\environment environment_Inferno
\xmlprocessfile{exo}{xml/Limbo_Exercices.xml}{}
\iffalse
\setupitemgroup[List][1][R,inmargin][after=,before=,left={\bf Exo },symstyle=bold,inbetween={\blank[big]}]
\setupitemgroup[List][2][n,joineup][after=,before=,inbetween={\blank[small]}]
\setupitemgroup[List][3][a,joineup][after=,before=,inbetween={\blank[small]}]
\setupitemgroup[List][4][1,joineup,nowhite]
\fi

%\setupitemgroup[List][1][A,inmargin][after=,before=,left={\bf Exo },symstyle=bold,inbetween={\blank[big]}]
%\setupitemgroup[List][1][R,joineup][after=,before=,inbetween={\blank[small]}]
%\setupitemgroup[List][1][n,inmargin][after=,before=,left={\bf Exo },symstyle=bold,inbetween={\blank[big]}]
\setupitemgroup[List][1][n,joineup][after=,before=,inbetween={\blank[small]}]
\setupitemgroup[List][2][a,joineup][after=,before=,inbetween={\blank[small]}]
\setupitemgroup[List][3][1,joineup,nowhite]
%\setupitemgroup[List][4][a,joineup,nowhite]
\definecolor[myGreen][r=0.55, g=0.76, b=0.29]%
\setuppapersize[A4]
\setuppagenumbering[location=]
\setuplayout[header=0pt,footer=0pt]
\def\conseil#1{{\myGreen\it #1}}%


\starttext
\setupheads[alternative=middle]
%\showlayout
\def\gah#1{\margintext{Exercice #1}}


\centerline{TD SCILAB : Programmation}
\blank[big]
\startList\item {\bf Echange du contenu de deux variables.} \crlf Supposons que a et b sont deux variables déjà définies dans Scilab. \crlf
Donner les instructions pour échanger les valeurs des variables a et b, puis les faire afficher. \crlf
Tester votre programme (prendre par exemple a=3 et b=5)

\item {\bf Demande, affichage et transformation de données.} \crlf Demandez à l'utilisateur son année de naissance (avec {\bf input}), sauvegardez la dans une variable {\bf annee}, 
puis écrire les instructions nécessaires à l’affichage de son âge, en années, puis en secondes.

\item {\bf Utilisation simple de tests (if, then, elif, else)}.\crlf
Le niveau de pollution à l’ozone dans l'atmosphère est calculé à partir d’une concentration en ozone de la façon suivante :
\startitemize[1]
\item Le niveau 1 est atteint à partir de 140 unités ($μg/m^3$), 
\item le niveau 2 à partir de 180 unités
\item le niveau 3 à partir de 220 unités. 
\stopitemize
Ecrire un programme qui demande la concentration d’ozone dans l’air et qui donne le niveau de pollution. 
{\it l’utilisateur doit voir l’affichage \quote{Aujourd’hui, niveau de pollution $0$ (resp. $1$, $2$, ou $3$)}.}

\item {\bf Analyse de programme\high{1}.} Deviner ce que fait le script suivant :
\starttyping 
n=input( "entrer un entier n"); 
s=0;
for i=1:n
  s=s+i
end
disp(s)\stoptyping
Vérifier alors votre intuition en affichant le résultat correspondant à la formule mathématique ...

\item {\bf Analyse de programme\high{2} : suites et boucle for.}
\startList\item Deviner l’affichage du programme suivant avant de l’exécuter :
\starttyping
u=0
for i=1:6
  u=2*u+3
  disp(u)
end
 \stoptyping
\item Quelle est la suite mathématiques en jeu ?
\item Modifier le programme précédent afin qu’il affiche uniquement $u_100$ .
\item bonus : trouver la forme explicite de la suite pour en déduire un programme sans la boucle for. 
\stopList

\item {\bf Calcul de termes de suites}\crlf
Dans les cas suivants, écrire un programme demandant $n$ et affichant $u_n$ :
\startList\item $u_0 = 0.1$ et $∀n ∈ ℕ, u_{ n+1} = 1 − \e^{ −2∗u_n}$ (pour vérifier $u_5 = 0.6976037$).
\item  $u_1 = 1$ et $∀n ∈ ℕ^∗$, $u_{n+1} = 2nu n + 3$ (pour vérifier $u_7 = 135759$).
\item Modifier les programmes précédents pour en faire des fonctions.
\stopList

\item {\bf Analyse de programme\high{3}.}\crlf
Soit le script suivant :
\starttyping
n=input('n?')
u=0 // u=u0
v=1 // v=u1
for i=1:n
  w=2*u-3*v // w=u(i+1)
  u=v // u= .......
  v=w // v= .........
end
disp(u,'u=') 
disp(v,'v=')
\stoptyping
{\bf Avant d’exécuter le script,}
\startList\item Déterminer les valeurs successivement prises par $u$, $v$, $w$ et $i$ lorsque $n=3$.
\item Comprendre le commentaire définissant $w$ à ligne 5 et compléter les deux commentaires suivants.
\item Quelle est la suite récurrente définie par ce programme ?
\item Quels sont les termes de cette suite affichés à la fin ?\crlf
Exécuter alors le script pour les valeurs de $n=1$, $n=3$, $n=10$. Que se passe-t-il pour $n=-10$ ?
\stopList

\item {\bf Sommes et boucle for}
\startList\item On pose $S n =∑_{k=1}^n{1\F k}$. 
Ecrire une fonction qui à un entier $n$ renvoie la valeur de $S_n$.
Conjecturer $\lim\L_{n→+∞} S_n$ , $\lim\L_{n→+∞} {S_n\F n}$ ainsi que $\lim\L_{n→+∞} {S_n\F \ln n}$
\item Ecrire une fonction qui à deux entiers $n$ et $p$ renvoie la valeur de $∑_{k=1}^nk^p$. \crlf
Vérifier votre script pour les valeurs $p = 1$ et $p = 2$ à l’aide des formules mathématiques.
\stopList
\item {\bf Suites.} Soit la suite u définie par $u_1 = 0$, $u_2 = −9$ et la relation de récurrence : 
\startsdformula 
u_{n+2} = 6u_{n+1} − 9u_n\qquad(n⩾1)
\stopsdformula
On pose pour tout $n ∈ ℕ$, $S_n$ la somme des $n$ premiers termes de la suite.
\startList\item Ecrire un script qui demande $n$ et affiche la valeur de $u_n$.
\item Compléter le script précédent, afin qu’il renvoie également la valeur de $S_n$.
\stopList

\item {\bf Suites.}  Ecrire une fonction qui à un entier $n$ associe la valeur de $u_n$ où la suite $u$ est définie par récurrence par 
$u_1 = 0$ et $u_2 = 1$ et pour tout entier 
\startsdformula u_{n+2} = (n + 1)u_{n+1} − u_n\qquad(n⩾1)\stopsdformula 
(pour vérifier $u_6 = 85$)


\goodbreak
\item {\bf Algorithme de Syracuse.}\crlf
On choisit un entier naturel quelconque. S’il est pair, on le divise par 2, sinon on le multiplie par 3 et on
ajoute 1. En répétant ce processus, on constate qu’au bout d’un certain nombre d’itérations, on aboutit
à 1 (actuellement personne n’a su le démontrer).
Exemple : 7 ; 22 ; 11 ; 34 ; 17 ; 52 ; 26 ; 13. ...
\startList
\item Terminer \quote{à la main} la liste précédente.
\item Écrire une fonction syracuse(n) qui affiche le nombre suivant le nombre n dans la liste de syracuse.
\item Compléter ce programme de façon à lui faire effectuer la boucle complète jusqu’à arriver à 1.
\item Convertir votre programme en fonction qui comptabiliser le nombre d'opérations nécessaires pour arriver jusqu'à $1$.
\item Ecrire un programme qui trouve le nombre compris entre $1$ et $1000$, pour lequel il y a le plus d'opérations ?

\stopList


\item {\bf Analyse (while).}\crlf
Deviner l’affichage du programme suivant avant de l’exécuter :
\starttyping
x=0
while x<=5
x=x+1
disp(x)
end
\stoptyping

\item {\bf On s’intéresse au jeu suivant :} la machine choisit un entier au hasard entre $1$ et $100$ et le joueur doit le deviner.
\startList\item Compléter le programme suivant afin de pouvoir y jouer !
\starttyping
cible=grand(1,’uin’,1,100)// l’ordinateur choisit un entier au hasard entre 1 et 100
essai=input(’Entrez votre proposition’)
while ...................................
if ................. then
disp("Trop grand")
else disp("Trop petit")
end
essai=input(’Entrez votre proposition’)
end
disp("Gagné")
\stoptyping
\item Ajouter une variable qui compte le nombre d’essais du joueur et l’afficher.
\item Comment optimiser la stratégie du joueur pour faire le moins d’essais possibles ?
En combien d’essais au maximum peut-on alors y arriver ?
\item Modifier le programme pour que le joueur n’ait droit qu’à 7 essais maximum. On affichera ”Perdu” si le joueur
n’a toujours pas trouvé au septième essai.
\stopList


\item {\bf termes de suites.} On considère la suite définie par $u_0 = 4$ et 
\startformula
∀n ∈ ℕ, \qquad u_{n+1} = u_n^2+ 1
\stopformula
\startList\item À l’aide de l'instruction while, écrire un programme demandant un réel $x$ et renvoyant 
le premier terme de la suite qui soit supérieur ou égal à~$x$.\crlf
{\it (Pour vérifier : $x = 17$ affiche $17$ et $x = 1000$ affiche $84101$).}
\item Modifier votre programme pour qu’il renvoie également l’indice de ce terme.
\crlf{\it (Pour vérifier : pour $x=1000$ on trouve $n=3$). } \crlf A quelle vitesse diverge cette suite ?
\item Réécrire ce programme pour en faire une fonction. Attention, il y aura deux valeurs de sortie !
\stopList

\item {\bf suites et limites. }Soit la suite $u$ définie par $u_0 = 3/4$ et la relation : 
\startformula
∀n ∈ ℕ, u_{n+1} = u_n − u_n^2.
\stopformula
On admet que la suite $u$ est décroissante et converge vers $0$.
Ecrire un programme qui déterminer le premier entier $n$ pour lequel $u_n < 10^{−3}$ .

\item Soit $u$ la suite définie par $u_0=1$ et $\D u_{n+1}={2u_n^2\F 1+5u_n}\qquad(n∈ℕ)$.\crlf
Une  étude mathématiques montre que la suite $u$ décroît et converge vers $0$ et que $∀n ∈ ℕ, |u n | ≤ \Q({2\F 5}\W)^n$.
\startList\item Utiliser l’étude mathématique pour déterminer un entier $n$ tel que $|u_n | ≤ 10^{−3}$ (choisir le plus petit possible).
\item Ecrire un programme scilab qui calcule le premier entier $n$ tel que $|u_n | ≤ 10^{−3}$.
Le comparer avec l’entier trouvé dans la question précédente.
\stopList

\item {\bf Bonus.} On veut déterminer la limite de la suite $u$ définie par $u_0 = 0$ et 
\startformula 
∀n ∈ ℕ, \qquad u_{n+1} = a + u_n,\qquad \text{ avec }a > 0
\stopformula
\startList\item On fait l’hypothèse suivante : si $|u_{n+1} − u_n | < 0.001$, alors $u_n$ est une approximation de la limite 
de la suite à $0.001$ près. \crlf{\it Ce n’est pas vrai dans tous les cas, mais si la suite est plutôt \quote{sympa}, l’approximation est bonne.}
\crlf Ecrire maintenant un programme qui demande à l’utilisateur $a$ et qui renvoie une valeur approchée de la limite 
de la suite $u$ à $0.001$ près.\crlf
{\it indication : introduire les variables $u$ pour $u_n$ et $v$ pour $u_n+1$.}
\item En fait, on peut montrer mathématiquement que pour tout a > 0, la suite u converge vers $α={1+\sqrt{1+4a}\F 2}$.
Compléter le programme précédent afin qu’il affiche la valeur théorique de la limite de la suite u à savoir $α$, puis
vérifier l’hypothèse d’approximation faite ci-dessus.
\stopList


\stoptext
\stopcomponent
\endinput