\startcomponent component_DS1
\project project_Res_Mathematica
\environment environment_Maths
\environment environment_Inferno
\xmlprocessfile{exo}{xml/Limbo_Exercices.xml}{}
\iffalse
\setupitemgroup[List][1][n,inmargin][after=,before=,left={\bf Exo },symstyle=bold,inbetween={\blank[big]}]
\setupitemgroup[List][2][a,joineup][after=,before=,inbetween={\blank[small]}]
\setupitemgroup[List][3][a,joineup][after=,before=,inbetween={\blank[small]}]
\setupitemgroup[List][4][1,joineup,nowhite]
\fi

\setupitemgroup[List][1][A,inmargin][after=,before=,left={\bf Exo },symstyle=bold,inbetween={\blank[big]}]
%\setupitemgroup[List][1][R,joineup][after=,before=,inbetween={\blank[small]}]
%\setupitemgroup[List][1][n,inmargin][after=,before=,left={\bf Exo },symstyle=bold,inbetween={\blank[big]}]
%\setupitemgroup[List][1][n,joineup][after=,before=,inbetween={\blank[small]}]
\setupitemgroup[List][2][n,joineup][after=,before=,inbetween={\blank[small]}]
\setupitemgroup[List][3][a,joineup,nowhite]
\setupitemgroup[List][4][a,joineup,nowhite]
\definecolor[myGreen][r=0.55, g=0.76, b=0.29]%
\setuppapersize[A4]
\setuppagenumbering[location=]
\setuplayout[header=0pt,footer=0pt]
\def\conseil#1{{\myGreen\it #1}}%


\starttext
\setupheads[alternative=middle]
%\showlayout
\def\gah#1{\margintext{Exercice #1}}

\iftrue
\page
\centerline{\bfb DEVOIR SCILAB 1 (1 heure)}
\blank[big]

\startList
% implémenter une dichotomie
% ecrire un programme pour trouver le 7 eme term d'une suite
% input
% if
% test if even

% for
\item Ecrire un programme affichant le terme $u_{666}$ de la suite définie par 
\startformula
u_0=13\qquad u_n=\sqrt{1 + u_{n-1}}\qquad(n⩾1)
\stopformula

% function ou function & if
\item Ecrire une fonction divise(a,b) permettant de déterminer si la division d'un entier $a$
par un entier $b$ est un nombre entier ou non. On pourra renvoyer c=%t (pour vrai) dans l'affirmative et c=%f (pour faux) sinon

% function & for
\item Ecrire une fonction  \quote{sommeDesDiviseurs} calculant la somme des diviseurs d'un nombre entier 
(le programme pourra utiliser la fonction de l'exercice précédent)

\item (Mega bonus) Deux nombres distincts $m$ et $n$ sont dits amis si la somme des diviseurs de $m$ (autres que $m$) est égale
à $n$ et la somme de tous les diviseurs de $n$ (autres que $n$) est égale à $m$ ; par exemple 220 et 284 sont
des nombres amis.\crlf
Ecrire un programme qui affiche tous les couples de nombres amis entre 1 et 1000
% for
% while
% function
% analyze
% modify
% big programm

\stopList

\stoptext
\stopcomponent
\endinput
