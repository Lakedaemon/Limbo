%2multibyte Version: 5.50.0.2953 CodePage: 65001

\documentclass{article}
%%%%%%%%%%%%%%%%%%%%%%%%%%%%%%%%%%%%%%%%%%%%%%%%%%%%%%%%%%%%%%%%%%%%%%%%%%%%%%%%%%%%%%%%%%%%%%%%%%%%%%%%%%%%%%%%%%%%%%%%%%%%%%%%%%%%%%%%%%%%%%%%%%%%%%%%%%%%%%%%%%%%%%%%%%%%%%%%%%%%%%%%%%%%%%%%%%%%%%%%%%%%%%%%%%%%%%%%%%%%%%%%%%%%%%%%%%%%%%%%%%%%%%%%%%%%
\usepackage{amssymb}
\usepackage{amsfonts}
\usepackage{amsmath}

\setcounter{MaxMatrixCols}{10}
%TCIDATA{OutputFilter=LATEX.DLL}
%TCIDATA{Version=5.50.0.2953}
%TCIDATA{Codepage=65001}
%TCIDATA{<META NAME="SaveForMode" CONTENT="1">}
%TCIDATA{BibliographyScheme=Manual}
%TCIDATA{Created=Thursday, September 06, 2007 15:11:45}
%TCIDATA{LastRevised=Monday, August 29, 2016 15:33:12}
%TCIDATA{<META NAME="GraphicsSave" CONTENT="32">}
%TCIDATA{<META NAME="DocumentShell" CONTENT="Standard LaTeX\Blank - Standard LaTeX Article">}
%TCIDATA{CSTFile=40 LaTeX article.cst}
%TCIDATA{PageSetup=28,28,28,28,0}
%TCIDATA{AllPages=
%H=36
%F=36,\PARA{038<p type="texpara" tag="Body Text" > \ \ \ \ \ \ \ \ \ \ \ \ \ \ \ \ \ \ \ \ \ \ \ \ \ correction-fiche-1-\thepage }
%}


\newtheorem{theorem}{Theorem}
\newtheorem{acknowledgement}[theorem]{Acknowledgement}
\newtheorem{algorithm}[theorem]{Algorithm}
\newtheorem{axiom}[theorem]{Axiom}
\newtheorem{case}[theorem]{Case}
\newtheorem{claim}[theorem]{Claim}
\newtheorem{conclusion}[theorem]{Conclusion}
\newtheorem{condition}[theorem]{Condition}
\newtheorem{conjecture}[theorem]{Conjecture}
\newtheorem{corollary}[theorem]{Corollary}
\newtheorem{criterion}[theorem]{Criterion}
\newtheorem{definition}[theorem]{Definition}
\newtheorem{example}[theorem]{Example}
\newtheorem{exercise}[theorem]{Exercise}
\newtheorem{lemma}[theorem]{Lemma}
\newtheorem{notation}[theorem]{Notation}
\newtheorem{problem}[theorem]{Problem}
\newtheorem{proposition}[theorem]{Proposition}
\newtheorem{remark}[theorem]{Remark}
\newtheorem{solution}[theorem]{Solution}
\newtheorem{summary}[theorem]{Summary}
\newenvironment{proof}[1][Proof]{\noindent\textbf{#1.} }{\ \rule{0.5em}{0.5em}}
\input{tcilatex}
\begin{document}


\subsection{ \ \ \ \ \ \ \ \ \ \ \ \protect\underline{Correction de la fiche
n$%
%TCIMACRO{\U{b0}}%
%BeginExpansion
{{}^\circ}%
%EndExpansion
$1 sur les fonctions de r\'{e}f\'{e}rences.}}

\subsubsection{\ \ \ \ \ \ \ \ }

\paragraph{\protect\underline{Exercice 1 : ( Fractions )}}

1. $A_{1}=\dfrac{\frac{a}{b}}{c}=\dfrac{a}{bc}$ \ \ 

2. $A_{1}=\dfrac{\frac{1}{b}}{\frac{c}{a}}=$ $\dfrac{a}{bc}$\ \ \ \ \ \ \ \
\ \ \ \ \ \ \ \ \ \ \ 

3. $A_{1}=\dfrac{\frac{a^{2}}{b}-b}{\frac{a}{b}+1}=\dfrac{\frac{a^{2}-b^{2}}{%
b}}{\frac{a+b}{b}}=\dfrac{a^{2}-b^{2}}{a+b}=a-b$\ \ \ \ \ \ \ \ \ \ \ \ \ \
\ \ 

4. \ $A_{1}=\dfrac{\frac{b}{a+b}-1}{\frac{b}{a-b}-1}=\dfrac{\frac{b-a-b}{a+b}%
}{\frac{b-a+b}{a-b}}=\dfrac{\frac{-a}{a+b}}{\frac{2b-a}{a-b}}=-\dfrac{a(a-b)%
}{(a+b)(2b-a)}.$

\paragraph{\protect\underline{Exercice 2 : ( puissances et racine )}}

1. D\'{e}velopper et r\'{e}duire les expressions suivantes :

a. $A=(3+\sqrt{2})^{2}=3^{2}+2\times 3\times \sqrt{2}+\left( \sqrt{2}\right)
^{2}=11+6\sqrt{2}$

b. $B=\left( \sqrt{6}-\sqrt{3}\right) ^{2}$ $=6-2\sqrt{6}\sqrt{3}+3=9-6\sqrt{%
2}$

c. \ $C=\left( a+b+c\right) ^{2}$ $=a^{2}+b^{2}+c^{2}+2ab+2ac+2bc$\ \ 

d. $D=\left( x-2y\right) ^{3}=x^{3}-6x^{2}y+12xy^{2}-8y^{3}$

2. Ecrire les expressions suivantes sous la forme $(x)^{n}$ :

a. $A=\dfrac{(6)^{3}(3)^{3}}{8}=$ $\dfrac{(6)^{3}(3)^{3}}{2^{3}}=\QDOVERD( )
{6\times 3}{2}^{3}=9^{3}=3^{6}$\ \ 

b. $B=(5)^{5}\left( 5\right) ^{-2}$ $=5^{5-2}=5^{3}$\ 

c. \ $C=\sqrt{3}\times \left( 9\right) ^{4}=3^{\frac{1}{2}}\times
(3^{2})^{4}=3^{\frac{1}{2}}\times (3)^{8}=3^{\frac{17}{2}}$

\paragraph{\protect\underline{Exercice 3 : ( r\'{e}solutions d'in\'{e}%
quations )}}

1. Les fonctions $x\mapsto 2x-1$ et $x\mapsto 6x-5$ \'{e}tant d\'{e}finies
sur $%
%TCIMACRO{\U{211d} }%
%BeginExpansion
\mathbb{R}
%EndExpansion
,$ le domaine d'\'{e}tude de cette in\'{e}galit\'{e} est $%
%TCIMACRO{\U{211d} }%
%BeginExpansion
\mathbb{R}
%EndExpansion
.$

Soit $x\in 
%TCIMACRO{\U{211d} }%
%BeginExpansion
\mathbb{R}
%EndExpansion
.$

$2x-1\leq 6x-5\Leftrightarrow 4\leq 4x\Leftrightarrow x\geq 1$

Conclusion : $S=\left[ 1,+\infty \right[ .$

2. La fonction $x\mapsto 9x^{2}-36$ \'{e}tant d\'{e}finie sur $%
%TCIMACRO{\U{211d} }%
%BeginExpansion
\mathbb{R}
%EndExpansion
,$ le domaine d'\'{e}tude de cette in\'{e}galit\'{e} est $%
%TCIMACRO{\U{211d} }%
%BeginExpansion
\mathbb{R}
%EndExpansion
.$

Soit $x\in 
%TCIMACRO{\U{211d} }%
%BeginExpansion
\mathbb{R}
%EndExpansion
.$

$9x^{2}-36\geq 0\Leftrightarrow x^{2}\geq 4\Leftrightarrow (x\leq -2)$ ou $%
\left( x\geq 2\right) $

Conclusion \ : $S=\left] -\infty ,-2\right] \cup \left[ 2,+\infty \right[ $

3. La fonction $x\mapsto \dfrac{x-3}{x+1}$ \'{e}tant d\'{e}finie sur $%
%TCIMACRO{\U{211d} }%
%BeginExpansion
\mathbb{R}
%EndExpansion
\diagdown \left\{ -1\right\} ,$ le domaine d'\'{e}tude de cette in\'{e}galit%
\'{e} est $%
%TCIMACRO{\U{211d} }%
%BeginExpansion
\mathbb{R}
%EndExpansion
\diagdown \left\{ -1\right\} .$

Soit $x\in 
%TCIMACRO{\U{211d} }%
%BeginExpansion
\mathbb{R}
%EndExpansion
\diagdown \left\{ -1\right\} .$

Utilisons un tableau de signe : $%
\begin{array}{cccccccc}
x & -\infty &  & -1 &  & 3 &  & +\infty \\ 
x+1 &  & - &  & + &  & + &  \\ 
x-3 &  & - &  & - &  & + &  \\ 
\dfrac{x-3}{x+1} &  & + & \parallel & - & 0 & + & 
\end{array}%
$

Conclusion \ : $S=\left] -1,3\right] $

4. Soit $x\in 
%TCIMACRO{\U{211d} }%
%BeginExpansion
\mathbb{R}
%EndExpansion
\diagdown \left\{ -1,2\right\} .$

On utilise un tableau de signes avec $(3x+1)^{2}>0$ : $%
\begin{array}{cccccccccc}
x & -\infty &  & -1 &  & \frac{1}{2} &  & 2 &  & +\infty \\ 
x+1 &  & - &  & + &  & + &  & + &  \\ 
2x-1 &  & - &  & - &  & + &  & + &  \\ 
x-2 &  & - &  & - &  & - &  & + &  \\ 
\frac{(2x-1)(3x+1)^{2}}{(x-2)^{5}(x+1)} &  & - & \parallel & + & 0 & - & 
\parallel & + & 
\end{array}%
.$

Conclusion : $S=\left] -\infty ,-1\right[ \cup \left] \frac{1}{2},2\right[ .$

5. $\dfrac{x+2}{x+1}\leq 1$ \ 

La fonction $x\mapsto \dfrac{x+2}{x+1}$ \'{e}tant d\'{e}finie sur $%
%TCIMACRO{\U{211d} }%
%BeginExpansion
\mathbb{R}
%EndExpansion
\diagdown \left\{ -1\right\} ,$ le domaine d'\'{e}tude de cette in\'{e}galit%
\'{e} est $%
%TCIMACRO{\U{211d} }%
%BeginExpansion
\mathbb{R}
%EndExpansion
\diagdown \left\{ -1\right\} .$

Soit $x\in 
%TCIMACRO{\U{211d} }%
%BeginExpansion
\mathbb{R}
%EndExpansion
\diagdown \left\{ -1\right\} .$

$\dfrac{x+2}{x+1}\leq 1\Leftrightarrow \dfrac{x+2}{x+1}-1\leq
0\Leftrightarrow \dfrac{1}{x+1}\leq 0$

Conclusion : $S=\left] -\infty ;-1\right[ $

6. Soit $x\in 
%TCIMACRO{\U{211d} }%
%BeginExpansion
\mathbb{R}
%EndExpansion
^{\ast }.$

$\dfrac{x}{2}+\dfrac{8}{x}\geqslant 4\Leftrightarrow \dfrac{x^{2}-8x+16}{2x}%
\geq 0\Leftrightarrow \dfrac{\left( x-4\right) ^{2}}{2x}\geq
0\Leftrightarrow x>0$

Conclusion : $S=%
%TCIMACRO{\U{211d} }%
%BeginExpansion
\mathbb{R}
%EndExpansion
^{+\ast }.$

7 .

Soit $x\in 
%TCIMACRO{\U{211d} }%
%BeginExpansion
\mathbb{R}
%EndExpansion
\diagdown \left\{ 1\right\} .$

\ \ \ $\dfrac{x}{x^{2}+1}\leqslant \dfrac{1}{(x-1)}\Leftrightarrow \dfrac{x+1%
}{\left( x-1\right) \left( x^{2}+1\right) }\geq 0$

On utilise un tableau de signes :$%
\begin{array}{cccccccc}
x & -\infty &  & -1 &  & 1 &  & +\infty \\ 
x+1 &  & - &  & + &  & + &  \\ 
x-1 &  & - &  & - &  & + &  \\ 
x^{2}+1 &  & + &  & + &  & + &  \\ 
\dfrac{x+1}{\left( x-1\right) \left( x^{2}+1\right) } &  & + & 0 & - & 
\parallel & + & 
\end{array}%
.$

Conclusion : $S=\left] -\infty ;-1\right] \cup \left] 1;+\infty \right[ .$\
\ 

8. \ Soit $x\in 
%TCIMACRO{\U{211d} }%
%BeginExpansion
\mathbb{R}
%EndExpansion
^{+\ast }.$

$\dfrac{(2x-1)\ln x}{2-x}>\ln x\Leftrightarrow \dfrac{(2x-1)\ln x}{2-x}-\ln
x>0\Leftrightarrow 3\QDOVERD( ) {x-1}{2-x}\ln x>0$

On utilise un tableau de signes :$%
\begin{array}{cccccccc}
x & 0 &  & 1 &  & 2 &  & +\infty \\ 
\ln x &  & - & 0 & + &  & + &  \\ 
x-1 &  & - & 0 & + &  & + &  \\ 
2-x &  & + &  & + & 0 & - &  \\ 
\ln x\QDOVERD( ) {x-1}{2-x} & \parallel & + & 0 & + & \parallel & - & 
\end{array}%
.$

Conclusion : $S=\left] 0;1\right[ \cup \left] 1;2\right[ .$

\paragraph{\protect\underline{Exercice 4 : }}

1.a.Les fonctions $x\mapsto \exp (-x)$ et $x\mapsto 1-x$ \'{e}tant d\'{e}%
finies sur \ $%
%TCIMACRO{\U{211d} }%
%BeginExpansion
\mathbb{R}
%EndExpansion
,$ $f$ est d\'{e}finie sur $%
%TCIMACRO{\U{211d} }%
%BeginExpansion
\mathbb{R}
%EndExpansion
.$

Par soustraction de fonctions d\'{e}rivables sur $%
%TCIMACRO{\U{211d} }%
%BeginExpansion
\mathbb{R}
%EndExpansion
,$ $f$ est d\'{e}rivable sur $%
%TCIMACRO{\U{211d} }%
%BeginExpansion
\mathbb{R}
%EndExpansion
.$

$\forall $ $x\in 
%TCIMACRO{\U{211d} }%
%BeginExpansion
\mathbb{R}
%EndExpansion
,$ $\ f^{\prime }(x)=-\exp (-x)+1.$

Etudions le signe de $f^{\prime }(x).$

Soit $x\in 
%TCIMACRO{\U{211d} }%
%BeginExpansion
\mathbb{R}
%EndExpansion
.$

$-\exp (-x)+1\geq 0\Leftrightarrow 1\geq \exp (-x)\Leftrightarrow x\geq 0$

Donc $\forall x\in 
%TCIMACRO{\U{211d} }%
%BeginExpansion
\mathbb{R}
%EndExpansion
^{+},$ $f^{\prime }(x)\geq 0$ et \ $\forall x\in 
%TCIMACRO{\U{211d} }%
%BeginExpansion
\mathbb{R}
%EndExpansion
^{-\ast },$ $f^{\prime }(x)\leq 0$

En conclusion, $f$ est d\'{e}croissante sur $%
%TCIMACRO{\U{211d} }%
%BeginExpansion
\mathbb{R}
%EndExpansion
^{-\ast }$ et $f$ est croissante sur $%
%TCIMACRO{\U{211d} }%
%BeginExpansion
\mathbb{R}
%EndExpansion
^{+}.$

$%
\begin{array}{cccc}
x & -\infty & 0 & +\infty \\ 
f(x) & \text{ \ \ }\searrow \text{\ \ \ \ \ \ } & 0 & \text{ \ \ \ \ }%
\nearrow \text{\ \ \ \ \ \ \ }%
\end{array}%
$

\ \ b. D'apr\`{e}s le tableau de variation de $f,$ $f$ admet en $0$ un
minimum absolu ( voir 1.2.1 )

donc $\forall x\in 
%TCIMACRO{\U{211d} }%
%BeginExpansion
\mathbb{R}
%EndExpansion
,f(x)\geq 0$

Conclusion : $\forall x\in 
%TCIMACRO{\U{211d} }%
%BeginExpansion
\mathbb{R}
%EndExpansion
,$ $\exp (-x)\geq 1-x$

2. Montrons que : $\forall x\in 
%TCIMACRO{\U{211d} }%
%BeginExpansion
\mathbb{R}
%EndExpansion
^{+},$ $x\geq \ln (1+x).$

Une premi\`{e}re m\'{e}thode serait d'\'{e}tudier les variations de la
fonction $x\mapsto \ln (1+x)-x$ et d'utiliser son tableau de variation pour
conclure sur son signe.

Une deuxi\`{e}me m\'{e}thode plus simple est d'utiliser l'in\'{e}galit\'{e}
prouv\'{e}e en 1.b.

$\forall x\in 
%TCIMACRO{\U{211d} }%
%BeginExpansion
\mathbb{R}
%EndExpansion
,$ $-x\in 
%TCIMACRO{\U{211d} }%
%BeginExpansion
\mathbb{R}
%EndExpansion
\Rightarrow \exp (-(-x))\geq 1-(-x)\Rightarrow \exp (x)\geq 1+x$

La fonction $\ln $ \'{e}tant croissante sur $%
%TCIMACRO{\U{211d} }%
%BeginExpansion
\mathbb{R}
%EndExpansion
^{+\ast }$ et $\forall x\in 
%TCIMACRO{\U{211d} }%
%BeginExpansion
\mathbb{R}
%EndExpansion
^{+},\exp (x)>0$ et $1+x>0,$ on obtient l'in\'{e}galit\'{e} :

$\forall x\in 
%TCIMACRO{\U{211d} }%
%BeginExpansion
\mathbb{R}
%EndExpansion
^{+},\exp (x)\geq 1+x\Rightarrow \ln (\exp (x))\geq \ln (1+x)\Rightarrow
x\geq \ln (1+x)$

\paragraph{\protect\underline{Exercice 5 :}}

1. $f(x)=\sqrt{\ln x-1}$ \ \ 

$Df=\left[ e;+\infty \right[ $

2. \ $h(x)=\sqrt{x^{2}-4}$ \ \ 

$Dh=$\ \ $\left] -\infty ,-2\right] \cup \left[ 2,+\infty \right[ $ \ ( voir
exo 3 question 2)

3. $g(x)=\sqrt{x-2}\sqrt{x+2}$ \ \ 

$Dg=\left[ 2,+\infty \right[ $

\paragraph{\protect\underline{Exercice 6 :}}

Soit $f$ la fonction d\'{e}finie par $f(x)=\ln (\frac{x-1}{x+1}).$

1) D\'{e}terminons l'ensemble de d\'{e}finition de $f$.

La fonction $f$ est d\'{e}finie pour $x+1\neq 0$ et $\QDOVERD( )
{x-1}{x+1}>0 $ \ donc $D_{f}=\left] -\infty ,-1\right[ \cup \left] 1,+\infty %
\right[ .$

2) Etudions les variations de $f$ .

Par composition de fonctions d\'{e}rivables, $f$ est d\'{e}rivable sur $%
D_{f} $ et $\forall x\in D_{f},$ $f^{\prime }(x)=\dfrac{2}{(x+1)(x-1)}.$

$\forall x\in D_{f},f^{\prime }(x)>0$

Donc $f$ est croissante sur $D_{f}.$

\paragraph{\protect\underline{Exercice 7 :}\ \ \ \ \ \ \ }

1. Soit $x\in 
%TCIMACRO{\U{211d} }%
%BeginExpansion
\mathbb{R}
%EndExpansion
.$ ( la fonction exp \'{e}tant d\'{e}finie sur $%
%TCIMACRO{\U{211d} }%
%BeginExpansion
\mathbb{R}
%EndExpansion
$ )

$e^{2x-3}=1\Leftrightarrow 2x-3=\ln (1)$ \ ( on utilise $\forall x\in 
%TCIMACRO{\U{211d} }%
%BeginExpansion
\mathbb{R}
%EndExpansion
,\ln (\exp x)=x$ voir 5.2 )

\ \ \ \ \ \ \ \ \ \ \ \ \ $\Leftrightarrow x=\frac{3}{2}$

Conclusion : $S=\left\{ \frac{3}{2}\right\} $

2. Cette \'{e}galit\'{e} est d\'{e}finie ssi $4x-9>0$

\ \ \ \ \ \ \ \ \ \ \ \ \ \ \ \ \ \ \ \ \ \ \ \ \ \ \ \ \ \ \ \ \ \ \ ssi $%
x\in \left] \frac{9}{4},+\infty \right[ $

Soit $x\in \left] \frac{9}{4},+\infty \right[ .$

\ \ $\ln (4x-9)=0\Leftrightarrow 4x-9=\exp (0)$ \ ( on utilise $\forall x\in 
%TCIMACRO{\U{211d} }%
%BeginExpansion
\mathbb{R}
%EndExpansion
^{+\ast },\exp (\ln x)=x$ voir 5.2 )

$\ \ \ \ \ \ \ \ \ \ \ \ \ \ \ \ \ \ \ \ \ \ \ \ \ \ \Leftrightarrow x=\frac{%
5}{2}$\ \ \ 

Comme \ $\frac{5}{2}\in \left] \frac{9}{4},+\infty \right[ $ ( solution
compatible avec les conditions initiales )

$S=\left\{ \frac{5}{2}\right\} $\ \ \ \ \ \ \ \ \ \ \ \ \ \ \ \ \ \ \ \ \ \
\ \ \ \ \ 

3. $\exp (x^{2}+1)=\exp (5(x-1))$

Soit $x\in 
%TCIMACRO{\U{211d} }%
%BeginExpansion
\mathbb{R}
%EndExpansion
$ . ( la fonction exp \'{e}tant d\'{e}finie sur $%
%TCIMACRO{\U{211d} }%
%BeginExpansion
\mathbb{R}
%EndExpansion
$ )

$\exp (x^{2}+1)=\exp (5(x-1))\Leftrightarrow \ln \left( \exp
(x^{2}+1)\right) =\ln \left( \exp (5(x-1))\right) \Leftrightarrow
x^{2}+1=5x-5$

\ \ \ \ \ \ \ \ \ \ \ \ \ \ \ \ \ \ \ \ \ \ \ \ \ \ \ \ \ \ \ \ \ \ \ \ \ \
\ \ \ $\Leftrightarrow x^{2}-5x+6=0\Leftrightarrow x=2$ ou $x=3$

Conclusion : $S=\left\{ 2,3\right\} $

4. $\ln (x+3)=\ln (x^{2}+3x)$\ \ \ 

Cette \'{e}galit\'{e} est d\'{e}finie ssi $x^{2}+3x>0$ et $x+3>0$

\ \ \ \ \ \ \ \ \ \ \ \ \ \ \ \ \ \ \ \ \ \ \ \ \ \ \ \ \ \ \ \ \ \ \ ssi $%
x\in \left( \left] -\infty ,-3\right[ \cup \left] 0,+\infty \right[ \right)
\cap \left( \left] -3,+\infty \right[ \right) )$

\ \ \ \ \ \ \ \ \ \ \ \ \ \ \ \ \ \ \ \ \ \ \ \ \ \ \ \ \ \ \ \ \ \ \ ssi $%
x\in \left] 0,+\infty \right[ $\ \ \ \ \ \ 

Soit \ \ $x\in \left] 0,+\infty \right[ .$

$\ln (x+3)=\ln (x^{2}+3x)\Leftrightarrow x^{2}+2x-3=0\Leftrightarrow x=1$ \
ou $x=-3$

Comme $1\in $\ \ $\left] 0,+\infty \right[ $ \ et \ $-3\notin \left]
0,+\infty \right[ ,$ \ $S=\left\{ 1\right\} .$

5. $\ln (x^{2}+5x+6)=\ln (x+11)$ \ \ 

Cette \'{e}galit\'{e} est d\'{e}finie ssi $x^{2}+5x+6>0$ et $x+11>0$

\ \ \ \ \ \ \ \ \ \ \ \ \ \ \ \ \ \ \ \ \ \ \ \ \ \ \ \ \ \ \ \ \ \ \ ssi $%
x\in \left( \left] -\infty ,-3\right[ \cup \left] -2,+\infty \right[ \right)
\cap \left( \left] -11,+\infty \right[ \right) )$

\ \ \ \ \ \ \ \ \ \ \ \ \ \ \ \ \ \ \ \ \ \ \ \ \ \ \ \ \ \ \ \ \ \ \ ssi $%
x\in \left( \left] -11,-3\right[ \cup \left] -2,+\infty \right[ \right) .$

soit $x\in \left( \left] -11,-3\right[ \cup \left] -2,+\infty \right[
\right) .$

$\ln (x^{2}+5x+6)=\ln (x+11)\Leftrightarrow
(x^{2}+5x+6)=(x+11)\Leftrightarrow x^{2}+4x-5=0\Leftrightarrow x=-5$ ou $%
x=1. $

Conclusion : ces solutions \'{e}tant compatibles avec les conditions
initiales, $S=\left\{ -5,1\right\} .$\ 

6. $2\ln (3-x)\leq \ln (x+1)+\ln (x-2).$\ \ \ \ 

Cette in\'{e}galit\'{e} est d\'{e}finie pour $x\in \left] 2,3\right[ .$

$2\ln (3-x)\leq \ln (x+1)+\ln (x-2)\Leftrightarrow \ln ((3-x)^{2})\leq \ln
\left( (x+1)(x-2)\right) \Leftrightarrow (3-x)^{2}\leq
(x+1)(x-2)\Leftrightarrow x\geq \frac{11}{5}.$

Conclusion : $S=\left[ \frac{11}{5},3\right[ .$\ \ \ \ \ \ \ \ \ \ \ \ \ \ \
\ \ \ \ \ \ \ \ \ \ \ 

7. $\exp (2x)-5\exp (x)+6<0$ \ \ \ \ \ \ \ \ \ 

Soit $x\in 
%TCIMACRO{\U{211d} }%
%BeginExpansion
\mathbb{R}
%EndExpansion
.$

\bigskip $\exp (2x)-5\exp (x)+6<0\Leftrightarrow $ $\left\{ 
\begin{array}{c}
X=e^{x} \\ 
6-5X+X^{2}<0%
\end{array}%
\right. .$

On d\'{e}termine les racines du polyn\^{o}me $P=6-5X+X^{2}=(X-2)(X-3)$

Donc $6-5X+X^{2}<0\Leftrightarrow X\in \left] 2;3\right[ $

\bigskip $\exp (2x)-5\exp (x)+6<0\Leftrightarrow $ $\left\{ 
\begin{array}{c}
X=e^{x} \\ 
X\in \left] 2;3\right[%
\end{array}%
\right. \Leftrightarrow \ 2<e^{x}<3$

$\exp (2x)-5\exp (x)+6<0\Leftrightarrow \ln 2<x<\ln 3$ \ ( en utilisant la
stricte croissance de $\ln $ sur $%
%TCIMACRO{\U{211d} }%
%BeginExpansion
\mathbb{R}
%EndExpansion
^{+\ast })$

Conclusion : $S=\left] \ln 2,\ln 3\right[ .$

\bigskip

\end{document}
