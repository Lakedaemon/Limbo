\documentclass[10pt,a4paper]{article}
\setlength{\textwidth}{17cm}
\setlength{\marginparwidth}{0cm}
\setlength{\oddsidemargin}{0cm}
\setlength{\textheight}{26cm}
\setlength{\topmargin}{-1.5cm}
\setlength{\headheight}{0.5cm}
\setlength{\headsep}{0.5cm}
\setlength{\hoffset}{-0.5cm}
\setlength{\voffset}{0cm}

\usepackage[francais]{babel}
\usepackage{amsmath}
\usepackage{amssymb}
\usepackage[latin1]{inputenc}
\usepackage{fancyhdr}
\usepackage{lastpage}
\usepackage{fancybox}
\usepackage{graphicx}
\usepackage{pstricks,pst-plot}

\newcommand{\R}{\ensuremath{\mathbb{R}}}
\newcommand{\N}{\ensuremath{\mathbb{N}}}
\newcommand{\Z}{\ensuremath{\mathbb{Z}}}
\newcommand{\C}{\ensuremath{\mathbb{C}}}
\newcommand{\Q}{\ensuremath{\mathbb{Q}}}
\newcommand{\K}{\ensuremath{\mathbb{K}}}
\newcommand{\ch}{\ensuremath{\text{ch\,}}}
\newcommand{\sh}{\ensuremath{\text{sh\,}}}
\newcommand{\tnh}{\ensuremath{\text{th\,}}}
\newcommand{\ach}{\ensuremath{\text{argch\,}}}
\newcommand{\ash}{\ensuremath{\text{argsh\,}}}
\newcommand{\ath}{\ensuremath{\text{argth\,}}}
\newcommand{\re}{\ensuremath{\text{Re\,}}}
\newcommand{\Id}{\ensuremath{\text{Id}}}
\newcommand{\im}{\ensuremath{\text{Im\,}}}
\newcommand{\Ker}{\ensuremath{\text{Ker\,}}}
\newcommand{\Vect}{\ensuremath{\text{Vect}}}
\newcommand{\Card}{\ensuremath{\text{Card}}}
\newcommand{\rg}{\ensuremath{\text{rg}}}

\begin{document}

\lhead{Correction du DNS n�1}
\rhead{Page \thepage{} sur \pageref{LastPage}}
\cfoot{isabelle.chauvin@free.fr}
\lfoot{PTSI, Lyc�e Newton, Clichy}
\rfoot{Ann�e 2011-2012}
\renewcommand{\footrulewidth}{0.4pt}

\pagestyle{fancy}

\begin{center}
	\shadowbox{\Large{\textbf{CORRECTION DU DNS n�1}}}
\end{center}

%\textbf{Partie A}

	\begin{enumerate}
		\item On sait que $\lim \limits_{x \to 0} x \ln(x) = 0$ ; 
			de plus, $\lim \limits_{x \to 0} x + 1 = 1$, 
			donc $\lim \limits_{x \to 0} f(x) = 0 = f(0)$, 
			et \fbox{$f$ est continue en 0}.\\
			Evaluons maintenant 
			$\lim \limits_{x \to 0} \frac{f(x) - f(0)}{x-0}$ : \quad 
			pour $x > 0,\; \frac{f(x) - f(0)}{x-0}
			= \frac{\ln x}{x+1}$ ; 
			or $\lim \limits_{x \to 0} \ln(x) = -\infty$ 
			et $\lim \limits_{x \to 0} x + 1 = 1$, 
			donc $\lim \limits_{x \to 0} \frac{f(x) - f(0)}{x-0} = -\infty$, 
			et \fbox{$f$ n'est pas d�rivable en 0}.
		\item 
			\begin{enumerate}
				\item $\varphi$ est d�rivable sur $\R_+^*$, 
					et $\forall x \in \R_+^*,\; 
					\varphi'(x) = \frac{1}{x} + 1 = \frac{1 + x}{x} > 0$, 
					donc \fbox{$\varphi$ cro�t strictement sur $\R_+^*$}.
				\item Par op�rations usuelles sur les limites, 
					$\lim \limits_{x \to 0} \varphi(x) = -\infty$ 
					et $\lim \limits_{x \to +\infty} \varphi(x) = +\infty$.\\
					Comme $\varphi$ est continue et strictement 
					croissante sur $\R_+^*$, alors elle r�alise une bijection 
					de $\R_+^*$ dans $]-\infty ; +\infty[\ = \R$ d'apr�s les 
					limites pr�c�dentes.\\
					Enfin, $0 \in \R$, donc 0 admet un unique ant�c�dent 
					par $\varphi$ dans $\R_+^*$ \\
					ou encore : \fbox{l'�quation $\varphi(x) = 0$ admet une 
					unique solution not�e $\beta$ dans $\R_+^*$}.
				\item On utilise maintenant une calculatrice, 
					et on constate que 
					$\varphi(0,27) < 0$ et $\varphi(0,28) > 0$.\\
					$\varphi$ est continue sur $\R_+^*$, 
					et on peut lui appliquer le th�or�me des valeurs 
					interm�diaires : $\varphi$ s'annule sur $[0,27 ; 0,28]$.
					Par ce qui pr�c�de, 
					forc�ment \fbox{$\beta \in [0,27 ; 0,28]$}.
				\item $f$ est d�rivable sur $\R_+^*$ comme quotient de 
					fonctions d�rivables sur $\R_+^*$, \\
					et $\forall x \in \R_+^*,\; 
					f'(x) = \frac{(1 + \ln x)(x+1) - x \ln x}{(x+1)^2}
					= \frac{x + 1 + \ln x}{(x+1)^2}$
					donc \fbox{$\forall x \in \R_+^*,\; 
					f'(x) = \frac{\varphi(x)}{(x+1)^2}$}.
				\item D'apr�s 2.(a) et 2.(b), on obtient le signe 
					de $\varphi$, 
					c'est-�-dire celui de $f'$ par 2.(d).\\
					$\begin{array}{l|ccccc}
					x & 0 & & \beta & & +\infty \\ \hline
					\varphi(x) & \| & - & 0 & + \\ \hline
					f'(x) & \| & - & 0 & + \\ \hline
					f & 0 & \searrow & | & \nearrow
					\end{array}$
			\end{enumerate}
		\item 
			\begin{enumerate}
				\item Soit $x > 0$ ; on simplifie $f(x) = \frac{x \ln x}{x+1}
					= \frac{\ln x}{1 + \frac{1}{x}}$\\
					Or $\lim \limits_{x \to +\infty} 1 + \frac{1}{x} = 1$ 
					et $\lim \limits_{x \to \infty} \ln x = +\infty$, 
					donc 
					\fbox{$\lim \limits_{x \to +\infty} f(x) = +\infty$}.
				\item Soit $x > 0$ ; on �value 
					$\ln(x) - f(x) = \frac{(x+1) \ln x - x \ln x}{x+1}
					= \frac{\ln x}{x+1}$\\
					Par croissance compar�e, 
					$\lim \limits_{x \to +\infty} \frac{\ln x}{x+1} = 0$, 
					donc 
					\fbox{$\lim \limits_{x \to +\infty} \ln(x) - f(x) = 0$}.
			\end{enumerate}
		\item 
			\begin{enumerate}
				\item Annulations de $f$ (on a d�j� $f(0) = 0$ par 
					l'�nonc�):\\
					Pour $x > 0,\quad 
					f(x) = 0 
					\iff x \ln x = 0 
					\iff \ln x = 0 
					\iff x = 1$\\
					donc \fbox{$f$ s'annule seulement en 0 et en 1}.
				\item Calcul de $f(\beta)$ :\\
					Par d�finition de $\beta$, 
					$\varphi(\beta) = 0$, donc $\ln(\beta) + \beta + 1 = 0$, 
					et donc $\ln(\beta) = - 1 - \beta$.\\
					Alors $f(\beta) = \frac{\beta \ln \beta}{\beta + 1}
					= \frac{\beta (-1-\beta)}{\beta + 1}
					= -\beta$
					et \fbox{$f(\beta) = -\beta$}.
				\item Courbes :\\
					\begin{center}
					\psset{unit=1.5}
					\begin{pspicture}(0,-2)(10,3)
					\psline[linewidth=0.04cm]{->}(0,0)(10,0)
					\psline[linewidth=0.04cm]{->}(0,-2)(0,3)
					\psgrid[gridcolor=green, gridwidth=0.2pt, subgriddots=1, 
						subgridwidth=0, subgriddiv=0, gridlabels=7pt]
						(0,0)(0,-2)(10,3)
					\psplot[plotstyle=curve,linestyle=dashed]{0.1}{10}{x ln}
					\psplot[plotstyle=curve]{0.01}{10}
						{x ln x mul x 1 add div}
					\psline{<->}(0.05,-0.278)(0.65,-0.278)
					\psline[linestyle=dashed](0.278,0.05)(0.278,-0.278)
					\uput{0}[0](0.3,0.2){$\beta$}
					\uput{0}[0](-0.4,-0.3){$-\beta$}
					\uput{0}[0](10.2,2){${\cal C}$}
					\uput{0}[0](10.2,2.4){$\Gamma$}
					\end{pspicture}
					\end{center}
			\end{enumerate}
	\end{enumerate}

\end{document}