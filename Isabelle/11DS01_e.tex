\documentclass[11pt,a4paper]{article}
\setlength{\textwidth}{17cm}
\setlength{\marginparwidth}{0cm}
\setlength{\oddsidemargin}{0cm}
\setlength{\textheight}{26cm}
\setlength{\topmargin}{-1.5cm}
\setlength{\headheight}{0.5cm}
\setlength{\headsep}{0.5cm}
\setlength{\hoffset}{-0.5cm}
\setlength{\voffset}{0cm}

\usepackage[francais]{babel}
\usepackage{amsmath}
\usepackage{amssymb}
\usepackage[latin1]{inputenc}
\usepackage{fancyhdr}
\usepackage{lastpage}
\usepackage{fancybox}
\usepackage{graphicx}
%\usepackage{pstricks,pst-plot}

\newcommand{\R}{\ensuremath{\mathbb{R}}}
\newcommand{\N}{\ensuremath{\mathbb{N}}}
\newcommand{\Z}{\ensuremath{\mathbb{Z}}}
\newcommand{\C}{\ensuremath{\mathbb{C}}}
\newcommand{\Q}{\ensuremath{\mathbb{Q}}}
\newcommand{\K}{\ensuremath{\mathbb{K}}}
\newcommand{\ch}{\ensuremath{\text{ch\,}}}
\newcommand{\sh}{\ensuremath{\text{sh\,}}}
\newcommand{\tnh}{\ensuremath{\text{th\,}}}
\newcommand{\ach}{\ensuremath{\text{argch\,}}}
\newcommand{\ash}{\ensuremath{\text{argsh\,}}}
\newcommand{\ath}{\ensuremath{\text{argth\,}}}
\newcommand{\re}{\ensuremath{\text{Re\,}}}
\newcommand{\Id}{\ensuremath{\text{Id}}}
\newcommand{\im}{\ensuremath{\text{Im\,}}}
\newcommand{\Ker}{\ensuremath{\text{Ker\,}}}
\newcommand{\Vect}{\ensuremath{\text{Vect}}}
\newcommand{\Card}{\ensuremath{\text{Card}}}
\newcommand{\rg}{\ensuremath{\text{rg}}}

\begin{document}

\lhead{DS n�1}
\rhead{Page \thepage{} sur \pageref{LastPage}}
\cfoot{isabelle.chauvin@free.fr}
\lfoot{PTSI, Lyc�e Newton, Clichy}
\rfoot{Ann�e 2011-2012}
\renewcommand{\footrulewidth}{0.4pt}

\pagestyle{fancy}

\begin{center}
	\shadowbox{\Large{\textbf{DEVOIR SURVEILLE n�1}}}\\
	Vendredi 23 septembre 2011 -- 13h-17h\\
	\textbf{L'usage de calculatrices est interdit}.\\
	La pr�sentation, la lisibilit�, l'orthographe, la qualit� de la 
	r�daction, la clart� et la pr�cision des raisonnements entreront 
	pour une part importante dans l'appr�ciation des copies. \\
	En particulier, les r�sultats non justifi�s ne seront pas pris 
	en compte.\\
	Les candidats sont invit�s � \textbf{encadrer} les r�sultats de leurs 
	calculs.\\
\end{center}

\begin{center}
	\fbox{\large{\textbf{Exercice I}}}
	%% Pr�paration 2006-2007, 01logique.pdf, exo 6
\end{center}

	\noindent
	Pour chacune des questions suivantes, �tudier les deux implications 
	(a) $\Rightarrow$ (b) et (b) $\Rightarrow$ (a). \\
	En d'autres termes, pr�ciser si ces implications sont vraies, et prouver le r�sultat.
	\begin{enumerate}
		\item $n$ est un entier naturel.
			\begin{enumerate}
				\item $n$ est multiple de 2.
				\item $n$ est multiple de 4 ou $n$ est multiple de 6.
			\end{enumerate}
		\item $x$ est un r�el.
			\begin{enumerate}
				\item $x-3 = x^2 + 2x$
				\item $(x-3)^2 = (x^2 + 2x)^2$
			\end{enumerate}
		\item 
			\begin{enumerate}
				\item $z \in \C$
				\item $\exists r \in \R_+^*,\; \exists \theta \in \R,\; 
					z = r e^{i \theta}$\\
			\end{enumerate}
	\end{enumerate}
				

\begin{center}
	\fbox{\large{\textbf{Exercice II}}}
	%% Pr�paration 2006-2007, 02rec.pdf, exo 1
\end{center}

	\noindent
	Montrer la propri�t� : 
	$$\forall n \in \N,\quad n \geqslant 2,\quad 
	\sum \limits_{k=1}^n \dfrac{1}{k^2} > \dfrac{3n}{2n+1}$$\\
		
	
\begin{center}
	\fbox{\large{\textbf{Exercice III}}}
	%% Pr�paration 2006-2007 -> Complexes -> trigo0607.pdf
\end{center}

	\begin{enumerate}
		\item R�soudre dans $\R$ l'�quation \quad $(E_1) : \cos x - \sqrt 3 \sin x = 1$
		\item R�soudre dans $\R$ l'�quation \quad $(E_2) : \sin(5x) = \frac{1}{2}$
		\item R�soudre dans $\R$ l'�quation 
			\quad $(E_3) : \cos x + \sin x + 2 \sqrt 2 \sin x \cos x = 0$
		\item R�soudre dans $\R$ l'in�quation \quad $(I_1) : \cos x > \frac{\sqrt 2}{2}$
		\item R�soudre dans $\R$ l'in�quation \quad $(I_2) : \sin x \leqslant \frac{1}{2}$
		\item Lin�ariser l'expression \quad 
			$A = \cos^3(x) \sin^2(x)$ pour $x \in \R$.
		\item R�soudre dans $\C$ l'�quation \quad $(E_4) : z^2 + iz + 1 - 3i = 0$
	\end{enumerate}

\newpage	
	
\begin{center}
	\fbox{\large{\textbf{Exercice IV}}}
	%% Complexes
\end{center}

	\noindent
	Le but de cet exercice est l'�tude de la somme 
	$C_n(\theta)$ d�finie pour tout entier naturel non nul $n$ et 
	tout r�el $\theta$ par : 
	$$C_n(\theta) = \sum\limits_{k=1}^n (\cos \theta)^k \cdot \cos (k\theta)$$
	\begin{enumerate}
		\item Montrer que $C_1(\theta)$ et $C_2(\theta)$ sont positifs ou 
			nuls pour toute valeur de $\theta$.
		\item 
			\begin{enumerate}
				\item Justifier que, pour tout entier relatif $p$, on a : 
					\quad $\cos (p\pi) = (-1)^p$.
				\item En d�duire la valeur de $C_n(\theta)$ lorsqu'il existe 
					un entier relatif $p$ tel que $\theta = p\pi$.
			\end{enumerate}
		\item 
			\begin{enumerate}
				\item Montrer que, pour tout r�el $\theta$, on a : \quad 
					$1 - (\cos \theta) e^{i\theta} 
					= - i (\sin \theta) e^{i\theta}$.
				\item On pose 
					$$S_n(\theta) = \sum\limits_{k=1}^n 
					(\cos \theta)^k \cdot \sin (k\theta)$$
					Montrer que $C_n(\theta) + i S_n(\theta)$ est 
					une somme de termes cons�cutifs d'une suite g�om�trique, 
					dont on pr�cisera le premier terme et la raison.
				\item En d�duire que, lorsque $\theta$ n'est pas multiple 
					de $\pi$, on a pour tout entier naturel non nul $n$ :
					$$C_n(\theta) = \dfrac{(\cos \theta)^{n+1} \cdot 
					\sin (n \theta)}{\sin \theta}$$
			\end{enumerate}
		\item Etudier en fonction des valeurs de $\theta$ le signe de 
			$C_3(\theta)$.
		\item 
			\begin{enumerate}
				\item Montrer que lorsque $\theta$ n'est pas multiple de $\pi$, 
					on a pour tout entier naturel $n$ non nul :
					$$|C_n(\theta)| \leqslant 
					\dfrac{|\cos \theta|^{n+1}}{|\sin \theta|}$$
				\item En d�duire, selon les valeurs de $\theta$, la limite de 
					$C_n(\theta)$ lorsque $n$ tend vers $+\infty$.
			\end{enumerate}
	\end{enumerate}
	
		
\end{document}