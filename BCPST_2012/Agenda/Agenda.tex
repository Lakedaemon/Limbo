\version[concept]
\mainlanguage[fr]
\setuppapersize[A4]
\setupbodyfont[12pt]
\setupheader[state=none]
\setupfooter[state=none]
\setupheads[alternative=middle]
\setuptextrules[style=italic, width=5cm]
\setupindenting[none]%medium]
%\setuppagenumber[state=stop]
\packed
%\showframe[margin,edge, text]
%\setupinmargin[distance=-5cm]



\def\Semaine #1, #2. #3\par{\page[preference]\textrule[top]{#2}\inleft{\framed{#1}}#3}%
\def\pn{\par\noindent}
\def\Vacances#1.{\subject{\bf\tfb Vacances #1}}

\def\TD#1. #2.{{TD #1. #2}}
\def\td#1. #2.{{\it TD #1'. #2}}
\def\Maple#1. {{\it Maple #1. }}
\def\ds #1. {{Devoir surveillé. }}
\def\Problème#1. {Problème. }




\starttext
\title{Agenda 2012-2013}


\subject{Analyse}

\subsubject{Suites}

\subsubject{Limites, continuité}

\subsubject{Dérivation, développements limités}

\subsubject{Rquations différentielles}

\subsubject{Fonctions de plusieurs variables}



\subject{Algèbre}


\subsubject{Nombres complexes}

\subsubject{Polynômes}

\subsubject{Systèmes d'équations linéaires}

\subsubject{Matrices}

\subsubject{Expaces vectoriels}

\subsubject{Applications linéaires}


\subject{Géométrie}


\subsubject{Géométrie euclidienne du plan}

\subsubject{Géométrie euclidienne de l'espace}


\subject{Probabilités}

\subsubject{Ensembles}

\subsubject{Dénombrement}

\subsubject{Probabilités}

\subsubject{Variables aléatoires}



\Semaine 1, 3--8 septembre. 
{\bf Révisions, développements limités}\pn
\TD 1. Développements limités, limites et équivalents.\pn
\td 1. Trigonométrie.

\Semaine 2, 10--15 septembre. {\bf Séries numériques I.}\pn
\TD 2. Séries numériques I.\pn
\td 1. Trigonométrie.\pn
\Problème a. Etude de série

\Semaine 3, 17--22 septembre. {\bf Séries numériques II.}\pn
\TD 3. Séries numériques II.\pn
\td 2. Courbes paramétrées.\pn
\ds 1. Reste de séries et constante d'Euler

\Semaine 4, 24--29 septembre. {\bf Métrique des courbes.}\pn
\TD 4. Métrique des courbes.\pn
\td 2. Courbes paramétrées.\pn
\Problème b. Courbes, courbure, longueur, repère de Frenet. 

\Semaine 5, 1--6 octobre. {\bf Séries entières I.}\pn
\TD 5. Séries entières I (rayon de convergence).\pn
\td 3. Opérations élémentaires : rang, déterminant, inverse, système.

\Semaine 6, 8--13 octobre. {\bf Séries entières II.}\pn
\TD 6. Séries entières (calcul, études).\pn
\td 3. Opérations élémentaires : rang, déterminant, inverse, système.\pn
\Maple 1. Programmation (tests, boucles, procédures...)

\Semaine 7, 15--20 octobre. {\bf Formes n-linéaires alternées, déterminant}\pn
\TD 7. Déterminant.\pn
\td 4. Algèbre linéaire : familles libres, génératrices, bases, noyau, image, matrices.\pn 
\Maple 1. Programmation (tests, boucles, procédures...).\pn
\ds 2. Permutations avec points fixes (sup/séries numériques/entières).\pn 
\Problème c. Suites adjacentes, séries et intégrales

\Semaine 8, 22--26 octobre.


\Vacances de la Toussaint.

\Semaine 9, 12--17 novembre. 
{\bf Polynôme caractéristique, vecteurs et espaces propres}\pn
\TD 8. Polynôme caractéristique et valeurs propres. .\pn
\td 4. Algèbre linéaire : familles libres, génératrices, bases, noyau, image, matrices. \pn
\Problème d. Matrice, puissance et système différentiel

\Semaine 10, 19--24 novembre. 
{\bf Diagonalisation}\pn
\TD 9. Diagonalisation.\pn
\td 5. Primitives.

\Semaine 11, 26 novembre--1 décembre. 
{\bf Trigonalisation}\pn
\TD 10. Trigonalisation.\pn
\td 5. Primitives.\pn
\ds 3. Exponentielle de matrice

\Semaine 12, 3--8 décembre. 
{\bf Intégrales généralisées}\pn
\TD 11. Intégrales généralisées.\pn
\td 6. Décomposition en éléments simples .\pn
\Maple 2. Algèbre linéaire, diagonalisation 

\Semaine 13, 10--15 décembre. 
{\bf Fonctions définies par une intégrale à un paramètre. }\pn
\TD 12. Intégrales à un paramètre.\pn 
\td 6. Décomposition en éléments simples.\pn
\Maple 2. Algèbre linéaire, diagonalisation\pn
\Problème e. Deux intégrales à un paramètre.  

\Semaine 14, 17--21 décembre. 
{\bf Concours Blanc}
\ds 4. Transformée de Laplace.\pn
\ds 5. Applications linéaires et équations différentielles\pn
\Problème f. Limite de Cesaro et séries entières.

\Vacances de Noël. 

\Semaine 15, 7--12 janvier. 
{\bf Surfaces, plan tangent, surfaces usuelles}\pn
\TD 13. Surfaces et nappes paramétrées.\pn
\td 7. Surfaces, volumes, Fubini et changement de variable.

\Semaine 16, 14--19 janvier. 
{\bf Quadriques. }\pn
\TD 14. Réduction des quadriques.\pn
\td 7. Surfaces, volumes, Fubini et changement de variable.

\Semaine 17, 21--26 janvier. 
{\bf Enveloppes, développées, développantes.}\pn 
\TD 15. Enveloppes, développées, développantes.\pn 
\td 8. Intersection, projection, droites incluses dans une surface. \pn
\Problème g. Etude d'une surface cubique. 

\Semaine 18, 28 janvier--2 février. 
{\bf Séries de Fourier}\pn
\TD 16. Séries de Fourier.\pn
\td 8. Intersection, projection, droites incluses dans une surface. \pn
\Maple 3. Séries de Fourier, courbes, surfaces. 
\Problème h. Surface et développante.

\Semaine 19, 4--9 février. 
{\bf Espaces préhilbertiens.} \pn
\TD 17. Produit scalaire, orthogonalité.\pn
\td 9. Orthonormalisation, projections orthogonales et distances.\pn
\Maple 3. Séries de Fourier, courbes, surfaces.\pn
\ds 6. Surfaces et courbes. 

\Semaine 20, 11--16 février.
{\bf Endomorphismes symétriques, orthogonaux, formes quadratiques} \pn
\TD 18. Isométries, matrices symétriques et formes quadratiques.\pn
\td 9. Orthonormalisation, projections orthogonales et distances.\pn

\Semaine 21, 18--22 février. 
{\bf Fonctions de plusieurs variables I}\pn
\TD 19. Limites, continuité et dérivation.\pn
\td 10. Equations aux dérivées partielles.\pn

\Vacances de février. 

\Semaine 22, 11--16 mars.  dd
{\bf Fonctions de plusieurs variables II}\pn
\TD 20. Recherche d'extrema.\pn
\td 10. Equations aux dérivées partielles.\pn

\Semaine 23, 18--23 mars. 
{\bf Equations différentielles et analyse vectorielle}\pn
\TD 21. Equations et systèmes différentiels.\pn
\td 11. .\pn
\Maple 4. \pn
\ds 7.

\Semaine 24, 25--30 mars. 
{\bf Géométrie : 2 épreuves de la banque PT}\pn
\TD 22. révisions et pratique (géométrie).\pn
\td 11. .\pn
\Maple 4. 

\Semaine 25, 2--6 avril. 
{\bf Analyse : 2 épreuves de la banque PT}\pn
\TD 23. révisions et pratique (analyse).\pn
\td 12. .\pn

\Semaine 26, 8--13 avril.
{\bf Algèbre linéaire : 2 épreuves de la banque PT}\pn
\TD 24. révisions et pratique (algèbre).\pn
\td 12. .\pn

\Semaine 27, 15--19 avril. 
{\bf Révisions de PTSI : 2 épreuves des petites mines}\pn
\ds 8. 

\Vacances de printemps. 

\Semaine 28, {\it 6--11 mai}. 

\Semaine 29, 13--18 mai. 

\Semaine 30, {\it 21--25 mai}. 

\Semaine 31, 27 mai--1 juin. 


\Semaine 32, 3 juin--8 juin.

\Semaine 33, 10--15 juin. 

\Semaine 34, 17--22 juin.

\Semaine 35, 24--29 juin. 

\Vacances d'été. 
\stoptext
\bye
